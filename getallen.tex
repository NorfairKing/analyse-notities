\documentclass[main.tex]{subfiles}
\begin{document}

\chapter{Re\"ele en complexe getallen}
\label{cha:reele-en-complexe-getallen}

\section{De rationale getallen en hun structuur}
\label{sec:de-rati-getall}

\begin{de}
  De natuurlijke getallen, genoteerd als $\mathbb{N}$ zijn inductief gedefini\"eerd als volgt.
  \begin{itemize}
  \item $\mathbb{N}$ bevat een neutraal element $0$.
  \item $\mathbb{N}$ bevat een eenheidselement $1$.
  \item Voor elk aantal $n$ bevat $\mathbb{N}$ ook $0+1n$.
  \end{itemize}
\end{de}

\begin{de}
  De \term{gehele getallen}, genoteerd als $\mathbb{Z}$.
  \[ \mathbb{Z} = \mathbb{N} \cup \{ -n \mid n\in \mathbb{N} \}  \]
\end{de}

\begin{de}
  De \term{rationale getallen}, genoteerd als $\mathbb{Q}$.
  \[ \mathbb{Q} = \{ \nicefrac{n}{m} \mid n\in \mathbb{Z}, m\in \mathbb{Z}, m \neq 0 \} \]
\end{de}

%\begin{opm}
%  In feite is $\mathbb{Q}$ het breukenveld van het integriteitsdomein $\mathbb{N}$, maar daar komen we later nog op terug.   
%\end{opm}

\begin{de}
  Een \term{veld} $\mathbb{F},+,\cdot$ is een verzameling $\mathbb{F}$ met twee bewerkingen $+$ en $\cdot$ met de volgende eigenschappen.
  \begin{itemize}
  \item $+$ is associatief:
    \[ \forall f,g,h \in \mathbb{F}:\ (f+g)+h = f+(g+h) \]
  \item Er bestaat een (uniek) neutraal element $0$ voor $+$:
    \[ \forall f\in \mathbb{F}:\ f + 0 = f = 0 + f \]
  \item Elk element $x$ heeft een (uniek) invers element $-x$ voor $+$:
    \[ \forall f\in \mathbb{F}: \exists (-f) \in \mathbb{F}: \ f + (-f) = 0 = (-f) + f \]
  \item $+$ is commutatief:
    \[ \forall f,g \in \mathbb{F}:\ f+g = g+f \]
  \item $\cdot$ is associatief:
    \[ \forall f,g,h \in \mathbb{F}:\ (f\cdot g) \cdot h = f\cdot (g\cdot h) \]
  \item Er bestaat een (uniek) neutraal element $1$ voor $\cdot$:
    \[ \forall f\in \mathbb{F}:\ f \cdot 1 = f = 1 \cdot f \]
  \item Elk element $x$ heeft een (uniek) invers element $x^{-1}$ voor $\cdot$:
    \[ \forall f\in \mathbb{F}: \exists f^{-1} \in \mathbb{F}: \ f \cdot f^{-1} = 0 = f^{-1} \cdot f \]
  \item $\cdot$ is commutatief:
    \[ \forall f,g \in \mathbb{F}:\ f\cdot g = g \cdot f \]
  \item $\cdot$ is distributief ten opzichte van $+$:
    \[ \forall f,g,h \in \mathbb{F}:\ f \cdot (g+h) = (f \cdot g) + (f \cdot h) \]
  \end{itemize}
\end{de}

\begin{de}
  Zij $\mathbb{F},+,\cdot$ een veld met een totale orderelatie $\le$, dan noemen we het een \term{totaal geordend veld} als aan de volgende eigenschappen voldaan is.
  \begin{itemize}
  \item $\forall x,y,z \in \mathbb{F}:\ x \le y \Rightarrow (x+z) \le (y+z)$
  \item $\forall x,y,z \in \mathbb{F}:\ x \le y \wedge 0 \le z \Rightarrow x\cdot z \le y\cdot z$
  \end{itemize}
\end{de}

\begin{pr}
  $\mathbb{Q}$ is een totaal geordend veld.

  \begin{proof}
    We bewijzen elk deel appart.
    \begin{itemize}
    \item $\mathbb{Q}$ is een veld.
      \begin{itemize}
      \item $+$ is associatief:
        \[ 
        \begin{array}{rrl}
          \forall \frac{f_{t}}{f_{n}},\frac{g_{t}}{g_{n}},\frac{h_{t}}{h_{n}} \in \mathbb{Q}:\
          &\left(\frac{f_{t}}{f_{n}} + \frac{g_{t}}{g_{n}}\right) + \frac{h_{t}}{h_{n}}
          &= \frac{f_{t}g_{n} + g_{t}f_{n}}{f_{n}g_{n}} + \frac{h_{t}}{h_{n}}\\
          &&= \frac{f_{t}g_{n}h_{n} + g_{t}f_{n}h_{n} + h_{t}f_{n}g_{n}}{f_{n}g_{n}h_{n}}\\
          &&=  \frac{f_{t}}{f_{n}} + \frac{g_{t}h_{n} + h_{t}g_{n}}{g_{n}h_{n}}\\
          &&= \frac{f_{t}}{f_{n}} + \left(\frac{g_{t}}{g_{n}} + \frac{h_{t}}{h_{n}}\right)\\
        \end{array}
        \]
        We gebruiken hier dat $+$ en $\cdot$ associatief en commutatief zijn in $\mathbb{Z}$
      \item Er bestaat een (uniek) neutraal element $0$ voor $+$:
        \[
        \begin{array}{rrl}
          \forall \frac{f_{t}}{f_{n}}\in \mathbb{Q}:\
          &\frac{f_{t}}{f_{n}} + \frac{0}{1}
          &= \frac{f_{t} + 0}{1}\\
          &&= \frac{f_{t}}{f_{n}}\\
          &&= \frac{0 + f_{t}}{1}\\
          &&= 0 +  \frac{f_{t}}{f_{n}}\\
        \end{array}
        \]
        We gebruiken hier dat $0$ het neutraal element is voor $+$ in $\mathbb{Z}$.
      \item Elk element $x$ heeft een (uniek) invers element $-x$ voor $+$:
        \[
        \begin{array}{rrl}
          \forall \frac{f_{t}}{f_{n}}\in \mathbb{Q}: \exists (-f) = \frac{-f_{t}}{f_{n}} \in \mathbb{F}: \ 
          &\frac{f_{t}}{f_{n}} + \frac{-f_{t}}{f_{n}}
          &= \frac{f_{t}f_{n} + (-f_{t}f_{n})}{f_{n}^{2}}\\
          &&= \frac{0}{f_{n}^{2}}\\
          &&= 0 \\
          &&= \frac{0}{f_{n}^{2}}\\
          &&= \frac{-f_{t}f_{n} + f_{t}f_{n}}{f_{n}^{2}}\\
          &&= \frac{-f_{t}}{f_{n}} + \frac{f_{t}}{f_{n}}\\
        \end{array}
        \]
        We gebruiken hier dat $-x$ het invers element is van $x$ voor $+$ in $\mathbb{Z}$. 
      \item $+$ is commutatief:
        \[
        \forall \frac{f_{t}}{f_{n}},\frac{g_{t}}{g_{n}} \in \mathbb{Q}:\
        \frac{f_{t}}{f_{n}} + \frac{g_{t}}{g_{n}}
        = \frac{f_{t}g_{n}+g_{t}f_{n}}{f_{n}g_{n}}\\ 
        = \frac{g_{t}}{g_{n}} + \frac{f_{t}}{f_{n}}\\
        \]
        We gebruiken hier dat $\cdot$ commutatief is en $+$ associatief in $\mathbb{Z}$.
      \item $\cdot$ is associatief:
        \[
        \forall \frac{f_{t}}{f_{n}},\frac{g_{t}}{g_{n}},\frac{h_{t}}{h_{n}} \in \mathbb{Q}:\
        \left(\frac{f_{t}}{f_{n}} \cdot \frac{g_{t}}{g_{n}}\right) \cdot \frac{h_{t}}{h_{n}}
        =\frac{f_{t}g_{t}}{f_{n}g_{n}} \cdot \frac{h_{t}}{h_{n}}
        =\frac{f_{t}g_{t}h_{t}}{f_{n}g_{n}h_{n}}
        =\frac{f_{t}}{f_{n}} \cdot \frac{g_{t}h_{t}}{g_{n}h_{n}}
        \]
       We gebruiken hier dat $\cdot$ commutatief is in $\mathbb{Z}$.
      \item Er bestaat een (uniek) neutraal element $1$ voor $\cdot$:
        \[
        \begin{array}{rrl}
          \forall \frac{f_{t}}{f_{n}}\in \mathbb{Q}:\
          &\frac{f_{t}}{f_{n}} \cdot \frac{1}{1}
          &= \frac{f_{t}\cdot 1}{1 \cdot 1}\\
          &&= \frac{f_{t}}{f_{n}}\\
          &&= \frac{1\cdot f_{t}}{1 \cdot 1}\\
          &&= 1 \cdot \frac{f_{t}}{f_{n}}\\
        \end{array}
        \]
        We gebruiken hier dat $1$ het neutraal element is voor $\cdot$ in $\mathbb{Z}$.
      \item Elk element $x$ heeft een (uniek) invers element $x^{-1}$ voor $\cdot$:
        \[
        \forall \frac{f_{t}}{f_{n}}\in \mathbb{Q}: \exists \frac{f_{t}}{f_{n}}^{-1} = \frac{f_{n}}{f_{t}} \in \mathbb{F}: \
        = \frac{f_{t}}{f_{n}}\frac{f_{n}}{f_{t}}
        = \frac{f_{t}f_{n}}{f_{t}f_{n}}
        = 1
        = \frac{f_{n}f_{t}}{f_{t}f_{n}}
        = \frac{f_{n}}{f_{t}}\frac{f_{t}}{f_{n}}
        \]
        We gebruiken hier dat $\cdot$ commutatief is in $\mathbb{Z}$.
      \item $\cdot$ is commutatief:
        \[
        \forall \frac{f_{t}}{f_{n}},\frac{g_{t}}{g_{n}} \in \mathbb{Q}:\
        \frac{f_{t}}{f_{n}} \cdot \frac{g_{t}}{g_{n}}
        = \frac{f_{t}g_{t}}{f_{n}g_{n}}
        = \frac{g_{t}}{g_{n}} \cdot \frac{f_{t}}{f_{n}}
        \]
        We gebruiken hier dat $\cdot$ commutatief is in $\mathbb{Z}$.
      \item $\cdot$ is distributief ten opzichte van $+$:
        \[
        \begin{array}{rrl}
          \forall \frac{f_{t}}{f_{n}},\frac{g_{t}}{g_{n}},\frac{h_{t}}{h_{n}} \in \mathbb{Q}:\ 
          &\frac{f_{t}}{f_{n}} \cdot \left(\frac{g_{t}}{g_{n}} + \frac{h_{t}}{h_{n}} \right)
          &= \frac{f_{t}}{f_{n}} \cdot \frac{g_{t}h_{n}+h_{t}g_{n}}{g_{n}h_{n}}\\
          &&= \frac{f_{t}\cdot (g_{t}h_{n}+h_{t}g_{n})}{f_{n}g_{n}h_{n}}\\
          &&= \frac{(f_{t} \cdot g_{t}h_{n}) + (f_{t}\cdot h_{t}g_{n})}{f_{n}g_{n}h_{n}}\\
          &&= \frac{f_{t}g_{t}}{f_{n}g_{n}} + \frac{f_{t}h_{t}}{f_{n}h_{n}}\\
        \end{array}
        \]
      \end{itemize}
    \item $\mathbb{Q}$ is totaal geordend veld.
      \begin{itemize}
      \item
        \[
        \forall \frac{f_{t}}{f_{n}},\frac{g_{t}}{g_{n}},\frac{h_{t}}{h_{n}} \in \mathbb{Q}:\ 
        \frac{f_{t}}{f_{n}} \le \frac{g_{t}}{g_{n}}
        \Rightarrow
        \frac{f_{t}}{f_{n}}+\frac{h_{t}}{h_{n}} \le \frac{g_{t}}{g_{n}}+\frac{h_{t}}{h_{n}}
        \]
\extra{vanuit welke axioma's bewijzen we dit?}
      \item $\forall x,y,z \in \mathbb{F}:\ x \le y \wedge 0 \le z \Rightarrow x\cdot z \le y\cdot z$
\extra{vanuit welke axioma's bewijzen we dit?}
      \end{itemize}
    \end{itemize}
  \end{proof}
\end{pr}

\begin{pr}
  \label{pr:geordend-veld-optelling-ongelijkheden}
  Zij $\mathbb{F},+,\cdot,\le$ een totaal geordend veld.
  \[ \forall a,b,c,d \in \mathbb{F}:\ a\le b \wedge c \le d \Rightarrow (a+c) \le (b+d) \]

  \begin{proof}
    We gebruiken enkel de eerste eigenschap in de definitie van een totaal geordend veld.
    Omdat $a\le b$ geldt geldt ook $a + c \le b+c$.
    Bovendien geldt $c+b \le d+b$ omdat $c\le d$ geldt.
    Omdat $+$ commutatief is in $\mathbb{F}$, geldt dan ook het volgende:
    \[ a+c \le b+c \le b+d\]
  \end{proof}
\end{pr}

\begin{pr}
  \label{pr:geordend-veld-ongelijkheid-maal-min-een}
  Zij $\mathbb{F},+,\cdot,\le$ een totaal geordend veld.
  \[ \forall a,b\in \mathbb{F}:\ a \le b \Rightarrow -b \le -a \]

  \begin{proof}
    Bewijs uit het ongerijmbde\\
    Stel $a \le b$ en $-b > -a$ (dus $-a \le b$).
    Omdat $\mathbb{F}$ een totaal geordend veld is, volgt uit $-a \le -b$ zowel $a \le -b+2a$ en $-a+2b \le -b$.
    Tel deze ongelijkheden op\prref{pr:geordend-veld-optelling-ongelijkheden} om $2b \le 2a$ te bekomen.
    Uit $a\le b$ volgt echter dat $2a \le 2b$ geldt (tel immers $a\le b$ bij zichzelf op).
    Contradictie.
  \end{proof}
\end{pr}

\begin{pr}
  Zij $\mathbb{F},+,\cdot,\le$ een totaal geordend veld.
  \[ \forall b \in \mathbb{F}:\ 0 \le b \Leftrightarrow -a \le 0 \]

  \begin{proof}
    Gebruik in propositie \ref{pr:geordend-veld-ongelijkheid-maal-min-een} $a=0$.
  \end{proof}
\end{pr}

\begin{pr}
  Zij $\mathbb{F},+,\cdot,\le$ een totaal geordend veld.
  \[ \forall a,b \in \mathbb{F}:\ 0 \le a \wedge 0 \le b \Rightarrow 0 \le ab \]

  \begin{proof}
    Omdat $\mathbb{F}$ een geordend veld is, volgt uit $0\le a$ dat $0b \le ab$ geldt vanwege $0 \le b$.
    Omdat $0$ het nulelement is van $\mathbb{F}$ geldt $0b = 0$.
  \end{proof}
\end{pr}

\begin{pr}
  \label{pr:nul-kleiner-dan-een}
  Zij $\mathbb{F},+,\cdot,\le$ een totaal geordend veld.
  \[ 0 < 1 \]

  \begin{proof}
    Bewijs uit het ongerijmde\\
    Stel dat $1 \le 0$ geldt, dan moet $1<0$ gelden omdat in een veld $1$ verschilt van $0$.
    Tel hierbij $-1$ op, dan bekomen we $0 \le -1$.
    Vanwege de tweede definierende eigenschap van een totaal geordend veld moet dan uit $1 \le 0$ ook $-1 \le 0$ volgen, maar dat is in strijd met $0 \le -1$ omdat $0$ verschilt van $1$.
  \end{proof}
\end{pr}

\begin{pr}
  \label{pr:geordend-veld-inverse-zelfde-teken}
  Zij $\mathbb{F},+,\cdot,\le$ een totaal geordend veld.
  \[ \forall a \in \mathbb{F}_{0}:\ 0 \le a \Rightarrow 0 \le a^{-1}\]

  \begin{proof}
    Bewijs uit het ongerijmde:
    Stel $0 \le a$ maar ook $a^{-1} < 0$ geldt.
    We mogen die tweede ongelijkheid dan vermenigvuldigen met $a$ om $aa^{-1}< 0$ te bekomen.
    Dit zou $1<0$ betekenen en dat is in contradictie met propositie \ref{pr:nul-kleiner-dan-een}.
  \end{proof}
\end{pr}

\begin{pr}
  Zij $\mathbb{F},+,\cdot,\le$ een totaal geordend veld.
  \[ \forall a,b \in \mathbb{F}_{0}^{+}:\  a \le b \Rightarrow b^{-1} \le a^{-1}\]

  \begin{proof}
    Omdat zowel $0 \le a$ en $0 \le b$ geld, mogen we $a\le b$ vermenigvuldigen met $a^{-1}$ en $b^{-1}$\prref{pr:geordend-veld-inverse-zelfde-teken} om $b^{-1} \le a^{-1}$ te bekomen.
  \end{proof}
\end{pr}

\begin{st}
  $\sqrt{2}$ is geen element van $\mathbb{Q}$.
\extra{bewijs}
\end{st}

\begin{st}
  $\mathbb{Q}$ heeft de supremumeigenschap niet.
\end{st}

\begin{st}
  De stelling van rolle geldt niet in $\mathbb{Q}$.
\extra{verwijzen naar een plaats met betere uitleg.}
\end{st}

\extra{nog een gebrek van $\mathbb{Q}$.}

\subsection{Axiomatische beschrijving van $\mathbb{R}$}
\label{sec:axiom-beschr-van}

\begin{st}
  \label{st:supremumeigenschap-R}
  De \term{supremumeigenschap}\\
  In $\mathbb{R}$ heeft elke niet-lege, naar boven begrensde deelverzameling een supremum.
  \extra{bewijs verder rigoreuzer}
\end{st}

\begin{st}
  Er bestaat, op isomorfisme na, maar \'e\'en totaal geordend veld met de supremumeigenschap.
  \extra{bewijs later}
\end{st}

\begin{pr}
  Er bestaat een unieke afbeelding $i:\ \mathbb{Q} \rightarrow \mathbb{R}$ met de volgende eigenschappen.
  \begin{itemize}
  \item $i(0_{\mathbb{Q}}) = 0_{\mathbb{R}}$
  \item $i(1_{\mathbb{Q}}) = 1_{\mathbb{R}}$
  \item $\forall p,q \in \mathbb{Q}: i(p+_{\mathbb{Q}}q) = i(p) +_{\mathbb{R}} i(q)$
  \item $\forall p,q \in \mathbb{Q}: i(p\cdot_{\mathbb{Q}} q) = i(p) \cdot_{\mathbb{R}} i(q)$
  \item $\forall p,q \in \mathbb{Q}: p \le_{\mathbb{Q}} q \Rightarrow i(p) \le_{\mathbb{R}} i(q)$
  \end{itemize}
  Bovendien is deze afbeelding injectief.

  \begin{proof}
    \begin{itemize}
    \item Uniciteit\\
      We bewijzen dat er slechts \'e\'en $i$ kan bestaan, eerst over $\mathbb{N}$, dan over $\mathbb{Z}$ en tenslotte over $\mathbb{Q}$.
      \begin{itemize}
      \item Beschouw een $n\in \mathbb{N}$.
        Er zijn dan twee gevallen:
        \begin{itemize}
        \item $n = 0_{\mathbb{O}}$:
          Dan moet $i(n)$ $0_{\mathbb{R}}$ zijn vanwege de eerste eigenschap.
        \item $n \neq 0$:
          $n$ is dan als de som van $n$ $1_{\mathbb{Q}}$-tjes te schrijven:\needed
          \[ n = n 1_{\mathbb{Q}} \]
          Vanwege de derde eigenschap is $i(n)$ dan de som van $n$ $1_{\mathbb{R}}$-jes te schrijven.
        \end{itemize}
        Hierdoor ligt $i$ al vast op $\mathbb{N}$. \waarom
      \item Beschouw vervolgens een getal $z\in \mathbb{Z}\setminus \mathbb{N}$.
        Er bestaat dan een getal $n \in \mathbb{N}$ zodat $z$ het tegengestelde is van $n$: $z = -n$.\needed
        De vierde eigenschap zegt ons dan het volgende:
        \[ i(-n) = -i(n) \]
        Hierdoor ligt $i$ al vast op $\mathbb{Z}$. \waarom
      \item Beschouw tenslotte een $q\in \mathbb{Q}$, dan valt $q$ te schrijven als $\nicefrac{n}{m}$ met $n\in \mathbb{Z}$ en $m\in \mathbb{N}_{0}$.
        Uit de derde eigenschap volgt dan dat het volgende moet gelden:
        \[ i(q) = i(n)i(m)^{-1} \]
        Dit vervolledigt de uniciteit van $i$.
      \end{itemize}
    \item Bestaan\\
      Verdergaand op het bewijs van de uniciteit construeren we $i$ achtereenvolgens op $\mathbb{N}$, dan op $\mathbb{Z}$ en dan op $\mathbb{Q}$.
      \extra{bewijzen dat $i$ door $+$ en $\cdot$ gaat}
      \extra{bewijzen dat $i$ goed gedefinieerd is voor $\mathbb{Q}$ (onafhankelijk van de gekozen $m$ en $n$).}
      \extra{bewijzen dat $i$ injectief en stijgend is}
    \end{itemize}
  \end{proof}
\end{pr}

\begin{opm}
  Onder dit morfisme beschouwen we $\mathbb{Q}$ als een deelverzameling van $\mathbb{R}$.
\end{opm}


\begin{lem}
  \label{lem:lemma-van-archimedes}
  Het \term{lemma van Archimedes}\\
  Voor elke $x\in \mathbb{R}$ bestaat er een $n\in \mathbb{N}$ zodat $x < n$ geldt.

  \begin{proof}
    Bewijs uit het ongerijmde\\
    Stel dat er een $x\in \mathbb{R}$ zou bestaan zodat $\forall n \in \mathbb{N}:\ x \ge n$ geldt, dan zou $\mathbb{N}$ naar boven begrensd zijn door die $x$.
    Noem $s$ dan het supremum van $\mathbb{N}$, dat bestaat immers zeker.\needed
    Omdat $s$ de kleinste bovengrens is van $\mathbb{N}$ is $s-1$ zeker geen bovengrens.
    We kunnen dus een $k\in \mathbb{N}$ vinden zodat $s-1$ kleiner is dan $k$.
    $s$ is dan echter kleiner dan $k+1$ en dus geen bovengrens.
  \end{proof}
\end{lem}

\begin{gev}
  $\forall a \in \mathbb{R}_{0}^{+},\ \forall b\in \mathbb{R}:\ \exists n\in N:\ na > b$

  \begin{proof}
    Als $b$ kleiner is dan $a$ is de stelling evident met $n=1$.
    Als $b$ groter is dan of gelijk aan $a$, ga dan als volgt te werk:
    $a^{-1}$ is positief (want $a$ is positief)\prref{pr:geordend-veld-inverse-zelfde-teken}, uit $a \le b$ volgt dus $1 \le \frac{b}{a}$.
    Kies dan een $n\in \mathbb{N}$ groter dan $\frac{b}{a}$, dit kan immers altijd.\lemref{lem:lemma-van-archimedes}
    Vermenigvuldig tenslotte $\frac{b}{a} < n$ met $a$ om de stelling te bekomen.
  \end{proof}
\end{gev}

\begin{gev}
  $\forall \epsilon \in \mathbb{R}_{0}^{+},\ \exists n\in N:\ \frac{1}{n} < \epsilon$
\extra{bewijs}
\end{gev}

\begin{gev}
  $\forall x\in \mathbb{R}:\ \exists m \in \mathbb{Z}:\ m-1 \le x < m$
\TODO{bewijs p 12}
\end{gev}

\begin{pr}
  $\forall x,y \in \mathbb{R}: (x<y \Rightarrow \exists q\in \mathbb{Q}:\ x<q<y$
\TODO{bewijs p 12}
\end{pr}

\begin{opm}
  We zeggen dat $\mathbb{Q}$ dicht ligt in $\mathbb{R}$.
\end{opm}


\begin{st}
  $\forall x\in \mathbb{R}^{+},\ \forall n\in \mathbb{N}:\ (n\ge 2 \Rightarrow \exists!\ y\in \mathbb{R}^{+}:\ y^{n}=x)$
\TODO{bewijs p 13} 
\end{st}

\subsection{Intervallen in $\mathbb{R}$}
\label{sec:intervallen-in-R}

\begin{de}
  Een \term{interval} in een totaal geordende verzameling $F,\le$ is een niet-lege deelverzameling $I$ van $F$ waarvoor elk element van $\mathbb{R}$ dat tussen twee elementen in $I$ ligt, tot $I$ behoort.
  \[ \forall x,y \in I,\ \forall z\in \mathbb{R}:\ x \le z \le y \Rightarrow z\in I \]
\end{de}

\subsubsection{Classificatie van intervallen}
\label{sec:class-van-interv}

\begin{st}
  De \term{classificatie van intervallen in $\mathbb{R}$}.

  Beschouw een willekeurig interval $I \subseteq \mathbb{R}$, dan zijn er een aantal mogelijkheden:
  \begin{itemize}
  \item $I$ is zowel naar boven als naar onder begrensd.\\
    $I$ heeft dan zowel een supremum $b$ als een infimum $a$.\stref{st:supremumeigenschap-R}.
    \begin{itemize}
    \item $a=b$: $I=\{a\} = [a,a]$
    \item $a<b$: 
      \begin{itemize}
      \item $a\in I \wedge b\in I$: $I = \{ x\in \mathbb{R} \mid a\le x \le b\} = [a,b]$ : ``het gesloten interval $a,b$''.
      \item $a\in I \wedge b\not\in I$: $I = \{ x\in \mathbb{R} \mid a\le x < b\} = [a,b[$ : ``het halfopen interval $a,b$, open in $b$''.
      \item $a\not\in I \wedge b\in I$: $I = \{ x\in \mathbb{R} \mid a< x \le b\} = ]a,b]$ : ``het halfopen interval $a,b$, open in $a$''.
      \item $a\not\in I \wedge b\not\in I$: $I = \{ x\in \mathbb{R} \mid a< x < b\} = ]a,b[$ : ``het open interval $a,b$''.
      \end{itemize}
    \end{itemize}
  \item $I$ is naar onder begrensd.
    $I$ heeft dan een infimum $a$.
    \begin{itemize}
    \item $a\in I$: $I = \{ x\in \mathbb{R} \mid x \ge a\} = [a,+\infty[$ : ``het gesloten interval $a, +\infty$''.
    \item $a\not\in I$: $I = \{ x\in \mathbb{R} \mid x > a\} = ]a,+\infty[$ : ``het open interval $a, +\infty$''.
    \end{itemize}
  \item $I$ is naar boven begrensd.
    $I$ heeft dan een supremum $b$.
    \begin{itemize}
    \item $a\in I$: $I = \{ x\in \mathbb{R} \mid x \le b \} = ]-\infty,b]$ : ``het gesloten interval $-\infty, b$''.
    \item $a\not\in I$: $I = \{ x\in \mathbb{R} \mid x < b \} = ]-\infty,b[$ : ``het open interval $-\infty,b$''. 
    \end{itemize}
  \item $I$ is niet begrensd. $I$ is dan gelijk aan $\mathbb{R}$.
  \end{itemize}
\zb
\end{st}


\subsection{Absolute waarde}
\label{sec:absolute-waarde}

\begin{de}
  De \term{absolute waarde} van een element $a$ van een totaal geordend veld $F,+,\cdot,\le$ defeni\"eren we als $|a|$:
  \[ 
  |a| = 
  \left\{
    \begin{array}{cl}
      a &\text{ als } a\ge 0\\
      a &\text{ als } a< 0\\
    \end{array}
  \right.
  \]
\end{de}

\begin{pr}
  $\forall a\in F: |a| \ge 0$
\extra{bewijs}
\end{pr}

\begin{pr}
  $\forall a\in F: |a| = 0 \Leftrightarrow a = 0$
\extra{bewijs}
\end{pr}

\begin{pr}
  $\forall a\in F: |a| = |-a|$
\extra{bewijs}
\end{pr}

\begin{pr}
  $\forall a\in F: -|a| \le a \le |a|$
\extra{bewijs}
\end{pr}

\begin{pr}
  $\forall a,b\in F: |a| \le b \Leftrightarrow -b \le a \le b$.
\extra{bewijs}
\end{pr}

\begin{de}
  De \term{afstand} tussen twee elementen van een totaal geordend veld $F,+,\cdot,\le$ defini\"eren we als $|x-y|$.
\end{de}

\begin{pr}
  De \term{driehoeksongelijkheid}\\
  \begin{itemize}
  \item $\forall a,b\in F:\ |a+b| \le |a| + |b|$
  \item $\forall x,y,z\in F:\ |x-y| \le |x-z| + |z-y|$
  \end{itemize}

\extra{bewijs}
\end{pr}

\begin{pr}
  De \term{tweede driehoeksongelijkheid}\\
  $\forall a,b\in F: ||a|-|b|| \le |a-b|$
\extra{bewijs}
\end{pr}

\section{Rijen in $\mathbb{R}$}
\label{sec:rijen-mathbbr}

\begin{de}
  Een \term{rij} in een verzameling $V$ is een functie als volgt:
  \[ x: \mathbb{N} \rightarrow V:\ n\mapsto x_{n} \]
  De functiewaarden noemen we \term{termen} en de rij wordt genoteerd met $(x_{n})_{n}$.
  Hierin noemen we de $n$ de \term{index} van de term.
\end{de}

\begin{de}
  We zeggen dat een rij $(x_{n})_{n}$ in een totaal geordend veld $F,+,\cdot,\le$ \term{convergeert} naar $a\in F$ als en slechts als het volgende geldt:
  \[ \forall \epsilon \in F_{0}^{+},\ \exists n_{0}\in \mathbb{N},\ \forall n\in \mathbb{N}:\ n \ge n_{0} \Rightarrow |x_{n}-a| < \epsilon \]
  We noemen $a$ de \term{limiet} van de rij $(x_{n})_{n}$:
  \[ a = \lim_{n\rightarrow \infty}x_{n} \]
  Een rij die convergeert noemen we een \term{convergente rij}.
  Een rij die niet convergeert noemen we een \term{divergente rij}.
\end{de}

\begin{de}
  We zeggen dat een rij $(r_{n})_{n}$ in een totaal geordend veld $F,+,\cdot,\le$ divergeert naar plus oneindig als en slechts het volgende geldt:
  \[ \forall M\in F,\ \exists n_{0}\in \mathbb{N},\ \forall n\in \mathbb{N}:\ n \ge n_{0} \Rightarrow x_{n} > M \]
  \[ + \infty = \lim_{n\rightarrow \infty}x_{n}\]
\end{de}

\begin{de}
  We zeggen dat een rij $(r_{n})_{n}$ in een totaal geordend veld $F,+,\cdot,\le$ divergeert naar min oneindig als en slechts het volgende geldt:
  \[ \forall M\in F,\ \exists n_{0}\in \mathbb{N},\ \forall n\in \mathbb{N}:\ n \ge n_{0} \Rightarrow x_{n} < M \]
  \[ - \infty = \lim_{n\rightarrow \infty}x_{n}\]
\end{de}


\begin{pr}
  Zij $p\in \mathbb{Z}$, dan heeft de rij $(n^{p})_{n}$ een limiet:
  \begin{itemize}
  \item $p>0 \rightarrow \lim_{n\rightarrow \infty}n^{p} = + \infty$
  \item $p=0 \rightarrow \lim_{n\rightarrow \infty}n^{p} = 1$
  \item $p<0 \rightarrow \lim_{n\rightarrow \infty}n^{p} = 0$
  \end{itemize}
\extra{bewijs}
\end{pr}

\begin{pr}
  Als een rij een limiet heeft, dan is deze uniek.
\TODO{bewijs p 31}
\end{pr}

\begin{pr}
  Een convergente rij $(x_{n})_{n}$ is begrensd.
  \[ \exists M \in F^{+}: \forall n\in \mathbb{N}: |x_{n}| \le M \]
\TODO{bewijs p 32}
\end{pr}

\begin{opm}
  Merk op dat het omgekeerde niet geldt.
\extra{tegenvoorbeeld}
\end{opm}

\begin{pr}
  Beschouw twee rijen $(x_{n})_{n}$ en $(y_{n})_{n}$.
  Stel dat de staart van de rijen overeen komt, dus dat er een $k\in \mathbb{N}$ bestaat zodat $x_{n}$ gelijk is aan $y_{n}$ voor alle $n$ groter dan $k$.
  De rijen vertonen dan hetzelfde asymptotisch gedrag.
\TODO{bewijs: oefening}
\end{pr}

\begin{pr}
  Voor elk re\"eel getal $x\in \mathbb{R}$ bestaat er een rij $(q_{n})_{n}$ in $\mathbb{Q}$ die convergeert naar $x$.
\TODO{bewijs p 33}
\end{pr}

\subsection{Limieten en orde}
\label{sec:limieten-en-orde}


\begin{de}
  We noemen een rij $(x_{n})_{n}$ in $\mathbb{R}$ ...
  \begin{itemize}
  \item ... \term{stijgend} als en slechts als $x_{n+1} \ge x_{n}$ geldt voor alle $n\in \mathbb{N}$.  
  \item ... \term{strikt stijgend} als en slechts als $x_{n+1} > x_{n}$ geldt voor alle $n\in \mathbb{N}$.  
  \item ... \term{dalend} als en slechts als $x_{n+1} \le x_{n}$ geldt voor alle $n\in \mathbb{N}$.  
  \item ... \term{strikt dalend} als en slechts als $x_{n+1} < x_{n}$ geldt voor alle $n\in \mathbb{N}$.  
  \end{itemize}
\end{de}

\begin{st}
  Een stijgende rij heeft atijd een limiet.
  Deze limiet is bovendien eindig als en slechts als de rij naar boven begrensd is.
  Die limiet is dan het supremum van de rij.
\TODO{bewijs p 34}
\end{st}
\begin{st}
  Een dalende rij heeft atijd een limiet.
  Deze limiet is bovendien eindig als en slechts als de rij naar onder begrensd is.
  Die limiet is dan het infimum van de rij.
\TODO{bewijs: oefening}
\end{st}

\begin{pr}
  Zij $r\in \mathbb{R}$, dan heeft de rij $(r^{n})_{n}$ ...
  \begin{itemize}
  \item ... limiet plus oneindig als $r>1$ geldt.
  \item ... limiet $1$ als $r=1$ geldt.
  \item ... limiet $0$ als $|r|<1$ geldt.
  \item ... geen limiet als $r<-1$ geldt.
  \end{itemize}
\TODO{bewijs p 35}
\end{pr}

\begin{pr}
  Zij $A$ een niet-leeg, naar boven begrensd deel van $\mathbb{R}$, dan bestaat er een stijgende rij in $A$ die convergeert naar het supremum van $A$.
  
\TODO{bewijs p 36}
\end{pr}

\begin{de}
  De \term{uitgebreidde orde in een totaal geordend veld} $F,+,\cdot,\le$.\\
  We moeten soms de orde van een totaal geordend veld uitbreiden over $F \cup \{ -\infty,+\infty\}$.
  De notatie blijft dan hetzelfde maar we voegen het volgende toe.
  \[ \forall a\in F: -\infty \le a \quad\text{ en }\quad \forall a \in F:\ a \le +\infty \]
\end{de}
\TODO{oneindigheden zelf ook eens definieren?}

\begin{pr}
  Zij $(x_{n})_{n}$ en $(y_{n})_{n}$ twee rijen zodat voor alle $n\in \mathbb{N}$ $x_{n}\le y_{n}$ geldt, dan geldt het volgende:
  \[ \lim_{n\rightarrow \infty}x_{n} \le \lim_{n\rightarrow \infty}y_{n} \]
\TODO{bewijs p 37}
\end{pr}

\begin{st}
  De \term{insluitstelling}\\
  Beschouw drie rijen $(x_{n})_{n}$, $(y_{n})_{n}$ en $(z_{n})_{n}$ zodat het volgende geldt:
  \[ \forall n\in \mathbb{N}: x_{n}\le y_{n}\le z_{n} \]
  Als $(x_{n})_{n}$ en $(z_{n})_{n}$ elk een limiet hebben, en die limiet hetzelfde is, dan is dit ook de limiet van $(y_{n})_{n}$.
  \[ \lim_{n\rightarrow \infty}x_{n} = \lim_{n\rightarrow \infty}y_{n} = \lim_{n\rightarrow \infty}z_{n} \]
\TODO{bewijs p 38}
\end{st}

\subsection{Limieten en bewerkingen}
\label{sec:limi-en-bewerk}

\begin{st}
  Zij $(x_{n})_{n}$ een convergente rij en $\lambda\in \mathbb{R}$ een re\"eel getal.
  \[ \lim_{n \rightarrow \infty}(\lambda x_{n}) = \lambda \lim_{n\rightarrow \infty}x_{n} \]
\TODO{bewijs: oefening}
\end{st}

\begin{st}
  Zij $(x_{n})_{n}$ en $(y_{n})_{n}$ twee convergente rijen.
  \[ \lim_{n \rightarrow \infty}(x_{n}+y_{n}) = \lim_{n\rightarrow \infty}x_{n} + \lim_{n\rightarrow \infty}y_{n} \]
\TODO{bewijs: oefening}
\end{st}

\begin{st}
  Zij $(x_{n})_{n}$ en $(y_{n})_{n}$ twee convergente rijen.
  \[ \lim_{n \rightarrow \infty}(x_{n}y_{n}) = \lim_{n\rightarrow \infty}x_{n} \cdot \lim_{n\rightarrow \infty}y_{n} \]
\TODO{bewijs: p 40}
\end{st}

\begin{st}
  Zij $(x_{n})_{n}$ en $(y_{n})_{n}$ twee convergente rijen zodat $\forall n\in \mathbb{N}:\ y_{n}\neq 0$ geldt
  \[ \lim_{n \rightarrow \infty}\left(\frac{x_{n}}{y_{n}}\right) = \frac{\lim_{n\rightarrow \infty}x_{n}}{\lim_{n\rightarrow \infty}y_{n}} \]
\TODO{bewijs: p 40}
\end{st}

\begin{st}
  We kunnen bovenstaande stellingen uitbreiden om te gelden over $F\cup\{ -\infty,+\infty\}$ als we de volgende rekenregels toevoegen.
  \[
  \begin{array}{rccccl}
                             & (+\infty) &+    & (+\infty) &= +\infty\\
                             & (-\infty) &+    & (-\infty) &= -\infty\\
                             & (+\infty) &+    & (-\infty) & & \text{ is onbepaald.} \\
                             & (-\infty) &+    & (+\infty) & & \text{ is onbepaald.} \\\\

    \forall a \in F:\        & a         &+    & (+\infty) &= + \infty \\
    \forall a \in F:\        & (+\infty) &+    & a         &= + \infty \\
    \forall a \in F:\        & a         &+    & (-\infty) &= - \infty \\
    \forall a \in F:\        & (-\infty) &+    & a         &= - \infty \\\\

                             & (+\infty) &\cdot& (+\infty) &= +\infty\\
                             & (-\infty) &\cdot& (-\infty) &= +\infty\\
                             & (+\infty) &\cdot& (-\infty) &= -\infty\\
                             & (-\infty) &\cdot& (+\infty) &= -\infty\\\\

                             & 0         &\cdot& (+\infty) & & \text{ is onbepaald.} \\
                             & 0         &\cdot& (-\infty) & & \text{ is onbepaald.} \\
                             & (+\infty) &\cdot& 0         & & \text{ is onbepaald.} \\
                             & (-\infty) &\cdot& 0         & & \text{ is onbepaald.} \\
    \forall a \in F_{0}^{+}:\ & a         &\cdot& (+\infty) &= + \infty \\
    \forall a \in F_{0}^{+}:\ & (+\infty) &\cdot& a         &= + \infty \\
    \forall a \in F_{0}^{+}:\ & a         &\cdot& (-\infty) &= - \infty \\
    \forall a \in F_{0}^{+}:\ & (-\infty) &\cdot& a         &= - \infty \\\\

    \forall a \in F_{0}^{-}:\ & a         &\cdot& (+\infty) &= + \infty \\
    \forall a \in F_{0}^{-}:\ & (+\infty) &\cdot& a         &= + \infty \\
    \forall a \in F_{0}^{-}:\ & a         &\cdot& (-\infty) &= - \infty \\
    \forall a \in F_{0}^{-}:\ & (-\infty) &\cdot& a         &= - \infty \\\\

    \forall a \in F:\        & \frac{a}{+\infty}      &= 0 \\
    \forall a \in F:\        & \frac{a}{-\infty}      &= 0 \\
    \forall a \in F_{0}^{+}:\ & \frac{+\infty}{a}      &= +\infty \\
    \forall a \in F_{0}^{+}:\ & \frac{-\infty}{a}      &= -\infty \\
    \forall a \in F_{0}^{-}:\ & \frac{+\infty}{a}      &= -\infty \\
    \forall a \in F_{0}^{-}:\ & \frac{-\infty}{a}      &= +\infty \\
                             & \frac{+\infty}{-\infty} &&&& \text{ is onbepaald.}\\
                             & \frac{+\infty}{+\infty} &&&& \text{ is onbepaald.}\\
                             & \frac{-\infty}{+\infty} &&&& \text{ is onbepaald.}\\
                             & \frac{-\infty}{-\infty} &&&& \text{ is onbepaald.}\\\\
                             & \frac{0}{0}             &&&& \text{ is onbepaald.}\\
  \end{array}
  \]
\extra{dit kan mooier?}
\TODO{bewiqs p 42}
\end{st}

\subsection{Deelrijen}
\label{sec:deelrijen}

\begin{de}
  Zij $(x_{n})_{n}$ een rij over een verzameling $V$ en $(n_{k})_{k}$ een strikt stijgende rij over $\mathbb{N}$, dan is de rij $(x_{n_{k}})_{k}$ een \term{deelrij} van $(x_{n})_{n}$.
\end{de}

\begin{pr}
  Zij $(x_{n})_{n}$ een rij in $F$ met een limiet in $F\cup\{+\infty,-\infty\}$., dan heeft elke deelrij ervan dezelfde limiet.
\TODO{bewijs p 50}
\end{pr}

\begin{st}
  De \term{stelling van Bolzano-Weierstra\ss}\\
  Elke begrensde rij heeft een convergente deelrij.
\TODO{bewijs p 51}
\end{st}
\begin{opm}
  Deze stelling gaat niet op in $\mathbb{Q}$.
\extra{bewijs}
\end{opm}

\begin{de}
  Zij $(x_{n})_{n}$ een begrensde rij.
  We defini\"eren de \term{limes superior} of \term{lim sup} en de \term{limes inferior} of \term{lim inf} van de rij $(x_{n})_{n}$ als volgt:
  \[ \limsup_{n\rightarrow \infty} x_{n} = \lim_{n\rightarrow \infty} sup\{x_{k}\mid k\ge n\} \quad\text{ en }\quad \liminf_{n\rightarrow \infty} x_{n} = \lim_{n\rightarrow \infty} inf\{ x_{k}\mid k\ge n\} \]
\end{de}

\TODO{de volgende 4 stellingen werken ook voor onbegrensde rijen, pas de bewijzen aan.}
\begin{pr}
  Zij $(x_{n})_{n}$ een begrensde rij.
  \[ \liminf_{n\rightarrow \infty} x_{n} \le \limsup_{n\rightarrow \infty} x_{n} \]
\TODO{bewijs p 53}
\end{pr}

\begin{pr}
  Zij $(x_{n})_{n}$ een begrensde rij.
  Als en slechts als $(x_{n})_{n}$ convergeert geldt hetvolgende.
  \[ \limsup_{n\rightarrow \infty} x_{n} = \liminf_{n\rightarrow \infty} x_{n} \]
  De limiet van $(x_{n})_{n}$ is dan ook de gemeenschappelijke woorde van limsup en liminf.
\TODO{bewijs p 53}
\end{pr}

\begin{pr}
  Zij $(x_{n})_{n}$ een begrensde rij, dan bestaat er een deelrij van $(x_{n})_{n}$ die convergeert naar $\limsup_{n\rightarrow \infty} x_{n}$ en een deelrij die convergeert naar $\liminf_{n\rightarrow \infty} x_{n}$.

\TODO{bewijs p 53}
\end{pr}

\begin{pr}
  Zij $(x_{k})_{k}$ een convergente deelrij van een begrensde rij $(x_{n})_{n}$.
  \[ \liminf_{n\rightarrow \infty} x_{n} \le \lim_{n\rightarrow \infty} x_{n} \le \limsup_{n\rightarrow \infty} x_{n} \]

\TODO{bewijs p 53}
\end{pr}

\subsection{Cauchyrijen en de volledigheid van $\mathbb{R}$}
\label{sec:cauchyrijen-en-de}

\begin{de}
  We noemen een rij $(x_{n})_{n}$ over een totaal georden veld $F,+,\cdot,\le$ een \term{Cauchyrij} als en slechts als het volgende geldt:
  \[ \forall \epsilon \in F_{0}^{+},\ \exists n\in \mathbb{N},\ forall n,m \in \mathbb{N}_{0}:\ n,m \ge n_{0} \Rightarrow |x_{n}-x_{m}| < \epsilon \]
\end{de}

\begin{pr}
  Elke convergente rij is een Cauchyrij.
\TODO{bewijs p 56}
\end{pr}

\begin{pr}
  In $\mathbb{Q}$ bestaan er Cauchyren die niet convergeren (in $\mathbb{Q}$).
\TODO{bewijs p 57}
\end{pr}

\begin{pr}
  Elke Cauchyrij over een totaal geordend veld $F,+,\cdot,\le$ is begrensd.
\TODO{bewijs p 58}
\end{pr} 

\begin{pr}
  Een Cauchyrij over een totaal geordend veld $F,+,\cdot,\le$ met een convergente deelrij convergeert naar dezelfde limiet als die deelrij.
\TODO{bewijs p 59}
\end{pr}

\begin{pr}
  Elke Cauchyrij over een totaal geordend veld $F,+,\cdot,\le$ convergeert.
\end{pr}

\begin{de}
  We noemen een totaal geordend veld $F,+,\cdot,\le$ \term{volledig} als elke Cauchyrij in $F$ een limiet heeft in $F$.
\end{de}

\begin{st}
  $\mathbb{R}$ is volledig.
\extra{bewijs}
\end{st}

\extra{tekstje op blz 60 toch nog eens bekijken}

\section{Topologie in $\mathbb{R}$}
\label{sec:topologie-mathbbr}

\begin{de}
  We noemen een deelzverzameling $A$ van $\mathbb{R}$ \term{open} als het volgende geldt:
  \[ \forall x\in A: \exists \delta \in \mathbb{R}_{0}^{+}, \forall y\in \mathbb{R}:\ |y-x| \le \delta \Rightarrow y\in A \]
\end{de}

\begin{st}
  Een $A$ van $\mathbb{R}$ is open als en slechts als het volgende geldt:
  \[ \forall x\in A:\ \exists \delta \in \mathbb{R}_{0}^{+}:\ ]x-\delta,x+\delta[ \subseteq A \]
\extra{bewijs}
\end{st}

\begin{de}
  Een deelverzameling $B$ van $\mathbb{R}$ noemen we \term{gesloten} als het complement ervan (in $\mathbb{R}$) open is.
\end{de}

\begin{opm}
  ``open'' en ``gesloten'' zijn geen complementaire begrippen.
\end{opm}

\begin{st}
  Open intervallen zijn inderdaad open deelverzamelingen.
\extra{bewijs p 67}
\end{st}
\begin{st}
  Gesloten intervallen zijn inderdaad gesloten deelverzamelingen.
\extra{bewijs p 67}
\end{st}

\begin{st}
  Halfopen intervallen zijn noch open, noch gesloten.
\extra{bewijs p 67}
\end{st}

\begin{st}
  $\mathbb{R}$ is open \'en gesloten.
\extra{bewijs p 67}
\end{st}

\begin{st}
  $\mathbb{Q}$ is niet gesloten.
\extra{bewijs p 67}
\end{st}

\begin{pr}
  De unie van open verzamelingen is open.
\TODO{bewijs p 69}
\end{pr}

\begin{pr}
  De doorsnede van \textbf{een eindig aantal} open verzamelingen is open.
\TODO{bewijs p 69}
\end{pr}
\TODO{tegenvoorbeeld voor oneindig aantal verzamelingen}

\begin{pr}
  Een doorsnede van gesloten verzamelingen is gesloten.
\TODO{bewijs: oefening}
\end{pr}

\begin{pr}
  De unie van \textbf{een eindig aantal} gesloten verzamelingen is gesloten.
\TODO{bewijs: oefening}
\end{pr}

\begin{pr}
  Zij $A$ een niet-leeg deel van $\mathbb{R}$, dan is $A$ gesloten als en slechts als elke convergente rij $(x_{n})_{n}$ in $A$ een limiet heeft in $A$.
\TODO{bewijs p 70}
\end{pr}

\begin{pr}
  Zij $A$ een niet-leeg deel van $\mathbb{R}$, dan is $A$ gesloten en begrensd als en slechs als elke rij in $A$ een convergente deelrij heeft met limiet in $A$.
\TODO{bewijs p 71}
\end{pr}

\subsection{Sluiting en inwendige}
\label{sec:sluit-en-inwend}

\begin{de}
  Zij $A$ een deelverzameling van $\mathbb{R}$.
  De grootste open deelverzameling $\mathring{A}$ van $A$ noemen we het \term{inwendige} van $A$.
\end{de}

\begin{de}
  Zij $A$ een deelverzameling van $\mathbb{R}$.
  De kleinste gesloten oververzameling $\bar{A}$ van $A$ noemen we de \term{sluiting} van $A$.
\end{de}

\TODO{definieer oververzameling?}

\TODO{bewijs dat deze definities nuttig zijn p 72}
 
\begin{pr}
  Zij $A$ een deelverzameling van $\mathbb{R}$, dan ziet $\mathring{A}$ er als volgt uit:
  \[ \mathring{A} = \{ x\in A\mid \exists \delta \in \mathbb{R}_{0}^{+}:\ ]x-\delta,x+\delta[ \subseteq A \} \]
\TODO{bewijs p 72}
\end{pr}

\begin{pr}
  Zij $A$ een deelverzameling van $\mathbb{R}$, dan ziet $\bar{A}$ er als volgt uit:
  \[ \mathring{A} = \{ x\in \mathbb{R} \mid \forall \delta \in \mathbb{R}_{0}^{+}:\ ]x-\delta, x+\delta[ \cap A \neq \emptyset \} \]
\TODO{bewijs p 72}
\end{pr}

\begin{de}
  We noemen een punt $x$ van niet-lege deelverzameling $A$ van $\mathbb{R}$ een \term{inwendig punt} van $A$ als het tot $\mathring{A}$ behoort.
  \[ \exists \delta \in \mathbb{R}_{0}^{+} \]
\end{de}

\begin{de}
  We noemen een punt $x$ van niet-lege deelverzameling $A$ van $\mathbb{R}$ een \term{adherent punt} aan $A$ als het tot $\bar{A}$ behoort.
  \[ \forall \delta \in \mathbb{R}_{0}^{+}:\ ]x-\delta, x+\delta[ \cap A \neq \emptyset \]
\end{de}

\begin{pr}
  Zij $A$ een deel van $\mathbb{R}$ en $x\in \mathbb{R}$, dan behoort $x$ tot $\bar{A}$ als en slechs als er een rij $(x_{n})_{n}$ bestaat met $x$ als limiet.
\TODO{bewijs p 74}
\end{pr}

\subsection{Randpunten, ge\"isoleerde punten en ophopingspunten.}
\label{sec:randp-geis-punt}

\begin{de}
  De \term{rand} $\partial A$ van een deelverzameling $A$ van $\mathbb{R}$ definieren we als volgt:
  \[ \partial A = \bar{A} \setminus \mathring{A} \]
\end{de}

\begin{de}
  Een \term{randpunt} van een deelverzameling $A$ van $\mathbb{R}$ is een element van de rand $\partial A$ van $A$.
\end{de}

\begin{st}
  Een punt $x\in \mathbb{R}$ is een randpunt van een deelverzameling $A$ van $\mathbb{R}$ als en slechts het volgende geldt:
  \[ \forall \delta \in \mathbb{R}_{0}^{+}:\ ]x-\delta,x+\delta[ \cap A \neq \emptyset\ \wedge\  ]x-\delta,x+\delta[ \cap A^{c} \neq \emptyset \]
\TODO{bewijs: oefening}
\end{st}

\begin{de}
  Zij $A$ een niet-lege deelverzameling van $\mathbb{R}$, dan noemen we een punt $x\in A$ een \term{ge\"isoleerd punt} van $A$ als het volgende geldt:
  \[ \exists \delta \in \mathbb{R}_{0}^{+}:\ ]x-\delta,x+\delta[ \cap A = \{ x \} \]
\end{de}

\begin{de}
  Zij $A$ een niet-lege deelverzameling van $\mathbb{R}$, dan noemen we een punt $x\in A$ een \term{ophopingspunt} of \term{accumulatiepunt} van $A$ als het volgende geldt:
  \[ ]x-\delta,x+\delta[ \cap (A \setminus \{x\}) \neq \emptyset \]
\end{de}

\begin{st}
  Een punt $x$ in een niet-lege deelverzameling $A$ van $\mathbb{R}$, dan is $x$ ofwel een ge\"isoleerd punt, ofwel een ophopingspunt.
\TODO{verifieer}
\extra{bewijs}
\end{st}

\begin{pr}
  Zij $A$ een niet een niet-lege deelverzameling van $\mathbb{R}$, dan zijn volgende uitspraken equivalent.
  \begin{enumerate}
  \item $x$ is een ophopingspunt van $A$.
  \item Voor alle $\delta > 0$ bevat $]x-\delta,x+\delta[ \cap A$ oneindig veel punten.
  \item Er bestaat een rij $(x_{n})_{n}$ in $A\setminus \{x\}$ die naar $x$ convergeert.  
  \end{enumerate}
\TODO{bewijs: oefening}
\end{pr}

\begin{st}
  De \term{stelling van Bolzano-Weierstra\ss} (ophopingspuntversie).
  Elk oneindig begrensd deel van $\mathbb{R}$ heeft minstens \'e\'en ophopingspunt.
\TODO{bewijs p 76}
\end{st}

\subsection{Relatieve topologie}
\label{sec:relatieve-topologie}

\begin{de}
  We noemen een deelverzameling $A$ van $X\subseteq \mathbb{R}$ \term{relatief open} in $X$ als het volgende geldt:
  \[ \forall x\in A:\ \exists \delta \in \mathbb{R}_{0}^{+}:\ \forall y\in X:\ |y-x| < \delta \Rightarrow y \in A \]
\end{de}

\begin{de}
  We noemen een deelverzameling $A$ van $X\subseteq \mathbb{R}$ \term{relatief gesloten} in $X$ als het relatief complement $X\setminus B$ van $B$ relatief open is in $X$.
\end{de}

\begin{st}
  Een  deelverzameling $A$ van $X\subseteq \mathbb{R}$  is relatief open in $X$ als en slechts als het volgende geldt:
  \[ \forall x\in A:\ \exists \delta\in \mathbb{R}_{0}^{+}:\ ]x-\delta,x+\delta[ \subseteq X \]
\TODO{bewijs}
\end{st}

\begin{pr}
  Zij $X$ een niet-leeg deel van $\mathbb{R}$ en $A \subseteq X$.
  \begin{itemize}
  \item $A$ is relatief open in $X$ als en slechts als er een open $V\subseteq \mathbb{R}$ bestaat zodat $A=V \cap X$ geldt.
  \item $A$ is relatief gesloten in $X$ als en slechs als er een gesloten $F \subseteq \mathbb{R}$ bestaat zodat $A=F \cap X$ geldt.
  \end{itemize}

\TODO{bewijs p 77}
\end{pr}

\section{De complexe getallen}
\label{sec:de-complexe-getallen}

\subsection{Het veld van de complexe getallen}
\label{sec:het-veld-van}



\begin{de}
  De verzameling $\mathbb{C}$ van \term{complexe getallen} defini\"eren we als volgt, samen met de optelling $(+)$ en vermenigvuldiging $(\cdot)$.
  \[ \mathbb{C} = \mathbb{R}^{2} = \{ (a,b) \mid a,b\in \mathbb{R} \} \]
  \[ (+):\ \mathbb{C}^{2} \rightarrow \mathbb{C}:\ (a,b) + (c,d) = (a+c,b+d) \]
  \[ (\cdot):\ \mathbb{C}^{2} \rightarrow \mathbb{C}:\ (a,b) \cdot (c,d) = (ac-bd, ad+bc) \]
\end{de}

\begin{st}
  $\mathbb{C},+,\cdot$ is een veld.
\TODO{bewijs p 81}
\end{st}

\begin{pr}
  We kunnen $\mathbb{R}$ inbedden in $\mathbb{C}$ met de volgende injectieve afbeelding:
  \[ \phi:\ \mathbb{R} \rightarrow \mathbb{C}:\ a \mapsto (a,0) \]
  Deze afbeelding is bovendien een ringmorfisme:
  \[ \forall a,b \in \mathbb{R}:\ \phi(a+b) = \phi(a) + \phi(b) \]
  \[ \forall a,b \in \mathbb{R}:\ \phi(ab) = \phi(a)\phi(b) \] 
\TODO{bewijs: oefening}
\end{pr}

\begin{de}
  We noteren een element $(a,b)$ van $\mathbb{C}$ vaak als $a+bi$.
  We noemen dit de \term{carthesiaanse vorm van een complex getal}.
\end{de}

\begin{opm}
  Deze notatie komt van pas om rekenregels binnen $\mathbb{C}$ eenvoudig te houden als we $i^{2}=1$ als regel in het achterhoofd houden. 
\extra{bewijs!}
\end{opm}

\begin{de}
  We noemen in een element $a+bi$ van $\mathbb{C}$ $a$ het \term{re\"eel} deel en $b$ het \term{imaginair} deel.
  We voeren daarom twee afbeeldingen in:
  \[ Re:\ \mathbb{C} \rightarrow \mathbb{R}:\ (a+bi) \mapsto a \]
  \[ Im:\ \mathbb{C} \rightarrow \mathbb{R}:\ (a+bi) \mapsto b \]
\end{de}

\begin{st}
  De \term{hoofdstelling van de algebra}\\
  $\mathbb{C}$ is algebra\"isch gesloten: Een $n$-de graadsveelterm over $\mathbb{C}$ heeft precies $n$ wortels in $\mathbb{C}$.
\zb
\end{st}

\begin{pr}
  Er bestaat geen orde op $\mathbb{C}$ die van $\mathbb{C},+,\cdot$ een totaal geordend veld maakt.
\TODO{bewijs p 84}
\end{pr}

\begin{de}
  Het \term{complex toegevoegde} $\overline{a+bi}$ van een complex getal $a+bi$ definieren we als volgt:
  \[ \overline{a+bi} = a-bi \]
  \[ \overline{\, \cdot\ }:\ \mathbb{C} \rightarrow \mathbb{C}:\ a+bi \mapsto \overline{a+bi} = a-bi \]
\end{de}

\begin{de}
  De \term{modulus} $|a+bi|$ van een complex getal $a+bi$ definieren we als volgt:
  \[ |a+bi| = \sqrt{a^{2}+b^{2}} \]
  \[ |\cdot|:\ \mathbb{C} \rightarrow \mathbb{R}:\ a+bi \mapsto |a+bi| = \sqrt{a^{2}+b^{2}} \]
\end{de}

\begin{ei}
  De modulus van een complex getal is de tegenhanger van de absolute waarde van een re\"eel getal.
  \[ \forall a \in \mathbb{R}:\ |a| = |\phi(a)| \]
\extra{bewijs}
\end{ei}

\begin{pr}
  \[ \forall z\in \mathbb{C}:\ \bar{\bar{z}} = z \]
\extra{bewijs}
\end{pr}

\begin{pr}
  \[ \forall z_{1},z_{2}\in \mathbb{C}:\ \overline{z_{1}+z_{2}} = \overline{z_{1}} + \overline{z_{2}} \]
\extra{bewijs}
\end{pr}

\begin{pr}
  \[ \forall z_{1},z_{2}\in \mathbb{C}:\ \overline{z_{1}z_{2}} = \overline{z_{1}}  \overline{z_{2}} \]
\extra{bewijs}
\end{pr}

\begin{pr}
  \[ \forall z\in \mathbb{C}:\ Re(z) = \frac{z+\bar{z}}{2} \]
\extra{bewijs}
\end{pr}

\begin{pr}
  \[ \forall z\in \mathbb{C}:\ Im(z) = \frac{z-\bar{z}}{2i} \]
\extra{bewijs}
\end{pr}

\begin{pr}
  \[ \forall z\in \mathbb{C}:\ |\bar{z}| = |z| \]
\extra{bewijs}
\end{pr}

\begin{pr}
  \[ \forall z\in \mathbb{C}:\ |Re(z)| \le |z| \]
\extra{bewijs}
\end{pr}

\begin{pr}
  \[ \forall z\in \mathbb{C}:\ |Im(z)| \le |z|\]
\extra{bewijs}
\end{pr}

\begin{pr}
  \[ \forall z\in \mathbb{C}:\ \bar{z}z\in \mathbb{R} \]
\extra{bewijs}
\end{pr}

\begin{pr}
  \[ \forall z\in \mathbb{C}:\ |z| = \sqrt{\bar{z}z} \]
\extra{bewijs}
\end{pr}

\begin{pr}
  \[ \forall z\in \mathbb{C}:\ \forall z \in \mathbb{C}_{0}: \frac{1}{z} = \frac{\bar{z}}{|z|^{2}} \]
\extra{bewijs}
\end{pr}

\begin{pr}
  \[ \forall z_{1},z_{2}\in \mathbb{C}:\ |z_{1}z_{2}| = |z_{1}||z_{2}| \]
\extra{bewijs}
\end{pr}

\begin{pr}
  \[ \forall z_{1},z_{2}\in \mathbb{C}:\ |z_{1}+z_{2}| \le |z_{1}|+|z_{2}| \]
\TODO{bewijs p 85}
\end{pr}

\begin{pr}
  \[ \forall x,y,z\in \mathbb{C}:\ |x-z| \le |x-y| + |y-z| \]
\extra{bewijs}
\end{pr}

\begin{pr}
  \[ \forall z_{1},z_{2}\in \mathbb{C}:\ ||z_{1}|-|z_{2}|| \le |z_{1}-z_{2}| \]
\extra{bewijs}
\end{pr}

\begin{de}
  Zij $z = a+bi$ de carthesiaanse co\"ordinaten van een complex getal, dan definieren we de \term{poolcoordinaten} van dat getal als $(r,\theta)$:
  \[ a = r \cos \theta \text{ en } b = r\sin \theta \]
  \[ z = r(\cos \theta + i \sin \theta)  \]
\end{de}

\begin{st}
  De \term{formule van de Moivre}\\
  \[ \forall z = r(\cos \theta + i \sin \theta) \in \mathbb{C}, n\in \mathbb{Z}:\ (r(\cos \theta + i \sin \theta))^{n} = r^{n}(\cos n\theta + i \sin n\theta)\]
\TODO{bewijs p 88}
\end{st}

\TODO{eenheidswortels definieren?}


\subsection{Rijen in $\mathbb{C}$}
\label{sec:rijen-mathbbc}

\begin{de}
  We zeggen dat een rij $(x_{n})_{n}$ in $\mathbb{C}$ \term{convergeert} naar een element $c\in \mathbb{C}$ als en slechts als het volgende geldt:
  \[ \forall \epsilon \in \mathbb{R}_{0}^{+}:\ \exists n_{0}\in \mathbb{N}:\ \forall n\in \mathbb{N}:\ n \ge n_{0} \Rightarrow |z_{n}-c| < \epsilon \]
  We noemen $c$ de \term{limiet} van de rij $(x_{n})_{n}$.
  \[ \lim_{n\rightarrow +\infty}z_{n} = c \]
\end{de}

\begin{de}
  Een rij die convergeert noemen we een \term{convergente}.
\end{de}

\begin{pr}
  Zij $(z_{n})_{n}$ een rij in $\mathbb{C}$.
  Noem $x_{n} = Re(z_{n})$ en $y_{n} = Im(z_{n})$.
  Noem bovendien $a=Re(c)$ en $b=Im(c)$.
  $(z_{n})_{n}$ convergeert naar $c$ als en slechts als $(x_{n})_{n}$ naar $a$ en $(y_{n})_{n}$ naar $b$ convergeert.
\TODO{bewijs p 91}
\end{pr}

\begin{st}
  De \term{stelling van Bolzano-Weierstra\ss} voor complexe rijen\\
  Elke begrensde rij in $\mathbb{C}$ heeft een convergente deelrij.
  \TODO{bewijs p 92}
\end{st}

\begin{de}
  We noemen een rij $(z_{n})_{n}$ in $\mathbb{C}$ een \term{Cauchyrij} als en slechts als het volgende geldt:
  \[ \forall \epsilon \in \mathbb{R}_{0}^{+}:\ \exists n_{0}\in \mathbb{N}:\ \forall n,m\in \mathbb{N}_{0}: n,m\ge n_{0} \Rightarrow |z_{n}-z_{m}| < \epsilon \]
\end{de}

\begin{pr}
  Zij $(z_{n})_{n}$ een rij in $\mathbb{C}$.
  Noem $x_{n} = Re(z_{n})$ en $y_{n} = Im(z_{n})$.
  $(z_{n})_{n}$ is een Cauchyrij als en slechts als $(x_{n})_{n}$ en $(y_{n})_{n}$ Cauchyrijen zijn.
  \TODO{bewijs: oefening p 93}
\end{pr}

\begin{st}
  Een rij $(z_{n})_{n}$ in $\mathbb{C}$ is convergent als en slechts als het een Cauchyrij is.
  \TODO{bewijs p 93}
\end{st}

\subsection{Topologie in $\mathbb{C}$}
\label{sec:topologie-mathbbc}

\begin{de}
  We noemen een deelverzameling $A$ van $\mathbb{C}$ \term{open} als en slechts als het volgende geldt:
  \[ \forall x\in A:\ \exists \delta \in \mathbb{R}_{0}^{+}:\ \forall y\in \mathbb{C}:\ |y-x| < \delta \Rightarrow y \in A \]
\end{de}

























\end{document}
