\documentclass[main.tex]{subfiles}
\begin{document}

\chapter{Getallen}
\label{cha:reele-en-complexe-getallen}

\section{Natuurlijke getallen}
\label{sec:natuurlijke-getallen}

\begin{de}
  De \term{natuurlijke getallen} zijn inductief gedefinieerd met de volgende axioma's.
  \begin{itemize}
  \item $0$ is een natuurlijk getal: $0 \in \mathbb{N}$.
  \item $\forall a \in \mathbb{N}: S(a) \in \mathbb{N}$.
  \item $\forall a,b\in \mathbb{N}: S(a) = S(b) \Rightarrow a=b$
  \item Als een verzameling $V$ $0$ bevat alsook de successor van elk getal in $V$, dan geldt $V=\mathbb{N}$.
  \end{itemize}
\end{de}

\begin{de}
  De \term{optellling} is gedefinieerd als volgt:
  \[ a + S(b) = s(a+b) \]
\end{de}

\begin{de}
  De \term{vermenigvuldiging} is gedefinieerd als volgt:
  \[ a \cdot S(b) = a+ s(a\cdot b) \]
\end{de}

\begin{de}
  De \term{orde} in $\mathbb{N}$ is gedefinieerd als volgt:
  \[ n \le S(m) \Leftrightarrow n = S(m) \vee n \le m \]
\end{de}

\begin{de}
  De \term{natuurlijke getallen} kunnen ook concreet gedefineerd worden:
  \[
  \left\{
  \begin{array}{rl}
    0 &= \emptyset\\
    s(n) &= n \cup \{ n \}\\
  \end{array}
  \right.
  \]
\end{de}

\section{Gehele getallen}
\label{sec:gehele-getallen}


\begin{de}
  De \term{gehele getallen}, genoteerd als $\mathbb{Z}$, definieren we intu\"itief als volgt:
  \[ \mathbb{Z} = \mathbb{N} \cup \{ -n \mid n\in \mathbb{N} \}  \]
\end{de}

\begin{de}
  We kunnen de \term{gehele getallen} ook concreet definieren als volgt:
  $\mathbb{Z}$ is partitie van $\mathbb{N}^{2}$ onder de equivalentierelatie $\sim$ als volgt:
  \[ \sim:\ \mathbb{N}^{2} \times \mathbb{N}^{2}:\ (a,b) \sim (c,d) \Leftrightarrow b+c = d+a \]
\end{de}

\begin{opm}
  In deze representatie stelt $(a,b)$ $b+(-a)$ voor in de intu\"itieve notatie.
\end{opm}

\begin{de}
  De \term{optelling} definieren we dan als volgt:
  \[ (a,b) + (c,d) = (a+c,b+d) \]
\end{de}
\extra{nagaan dat deze definitie onafhankelijk is van de gekozen representanten uit de equivalentieklassen}

\begin{de}
  De \term{vermenigvuldiging} definieren we als volgt:
  \[ (a,b) \cdot (c,d) = (bd+ac,da+bc) \]
\end{de}
\extra{nagaan dat deze definitie onafhankelijk is van de gekozen representanten uit de equivalentieklassen}

\begin{de}
  De \term{orde} definieren we in termen van de orde op $\mathbb{N}$ als volgt:
  \[ (a,b) \le (c,d) \Leftrightarrow b+c \le d+a \]
\end{de}



\section{De rationale getallen en hun structuur}
\label{sec:de-rati-getall}

\begin{de}
  De \term{rationale getallen}, genoteerd als $\mathbb{Q}$.
  \[ \mathbb{Q} = \{ \nicefrac{n}{m} \mid n\in \mathbb{Z}, m\in \mathbb{Z}, m \neq 0 \} \]
\end{de}

\begin{de}
  Concreet is $\mathbb{Q}$ het breukenveld van $\mathbb{Z}$, maar daarover later meer.
\end{de}

\section{Velden}
\label{sec:velden}

\begin{de}
  Een \term{veld} $\mathbb{F},+,\cdot$ is een verzameling $\mathbb{F}$ met twee bewerkingen $+$ en $\cdot$ met de volgende eigenschappen.
  \begin{itemize}
  \item $+$ is associatief:
    \[ \forall f,g,h \in \mathbb{F}:\ (f+g)+h = f+(g+h) \]
  \item Er bestaat een (uniek) neutraal element $0$ voor $+$:
    \[ \forall f\in \mathbb{F}:\ f + 0 = f = 0 + f \]
  \item Elk element $x$ heeft een (uniek) invers element $-x$ voor $+$:
    \[ \forall f\in \mathbb{F}: \exists (-f) \in \mathbb{F}: \ f + (-f) = 0 = (-f) + f \]
  \item $+$ is commutatief:
    \[ \forall f,g \in \mathbb{F}:\ f+g = g+f \]
  \item $\cdot$ is associatief:
    \[ \forall f,g,h \in \mathbb{F}:\ (f\cdot g) \cdot h = f\cdot (g\cdot h) \]
  \item Er bestaat een (uniek) neutraal element $1$ voor $\cdot$:
    \[ \forall f\in \mathbb{F}:\ f \cdot 1 = f = 1 \cdot f \]
  \item Elk element $x$ heeft een (uniek) invers element $x^{-1}$ voor $\cdot$:
    \[ \forall f\in \mathbb{F}: \exists f^{-1} \in \mathbb{F}: \ f \cdot f^{-1} = 0 = f^{-1} \cdot f \]
  \item $\cdot$ is commutatief:
    \[ \forall f,g \in \mathbb{F}:\ f\cdot g = g \cdot f \]
  \item $\cdot$ is distributief ten opzichte van $+$:
    \[ \forall f,g,h \in \mathbb{F}:\ f \cdot (g+h) = (f \cdot g) + (f \cdot h) \]
  \end{itemize}
\end{de}

\begin{de}
  \label{de:totaal-geordend-veld}
  Zij $\mathbb{F},+,\cdot$ een veld met een totale orderelatie $\le$, dan noemen we $\mathbb{F},+,\cdot,\le$ een \term{totaal geordend veld} als aan de volgende eigenschappen voldaan is.
  \begin{itemize}
  \item $\forall x,y,z \in \mathbb{F}:\ x \le y \Rightarrow (x+z) \le (y+z)$
  \item $\forall x,y,z \in \mathbb{F}:\ x \le y \wedge 0 \le z \Rightarrow x\cdot z \le y\cdot z$
  \end{itemize}
\end{de}

\begin{pr}
  $\mathbb{Q}$ is een totaal geordend veld.

  \begin{proof}
    We bewijzen elk deel appart.
    \begin{itemize}
    \item $\mathbb{Q}$ is een veld.
      \begin{itemize}
      \item $+$ is associatief:
        \[ 
        \begin{array}{rrl}
          \forall \frac{f_{t}}{f_{n}},\frac{g_{t}}{g_{n}},\frac{h_{t}}{h_{n}} \in \mathbb{Q}:\
          &\left(\frac{f_{t}}{f_{n}} + \frac{g_{t}}{g_{n}}\right) + \frac{h_{t}}{h_{n}}
          &= \frac{f_{t}g_{n} + g_{t}f_{n}}{f_{n}g_{n}} + \frac{h_{t}}{h_{n}}\\
          &&= \frac{f_{t}g_{n}h_{n} + g_{t}f_{n}h_{n} + h_{t}f_{n}g_{n}}{f_{n}g_{n}h_{n}}\\
          &&=  \frac{f_{t}}{f_{n}} + \frac{g_{t}h_{n} + h_{t}g_{n}}{g_{n}h_{n}}\\
          &&= \frac{f_{t}}{f_{n}} + \left(\frac{g_{t}}{g_{n}} + \frac{h_{t}}{h_{n}}\right)\\
        \end{array}
        \]
        We gebruiken hier dat $+$ en $\cdot$ associatief en commutatief zijn in $\mathbb{Z}$
      \item Er bestaat een (uniek) neutraal element $0$ voor $+$:
        \[
        \begin{array}{rrl}
          \forall \frac{f_{t}}{f_{n}}\in \mathbb{Q}:\
          &\frac{f_{t}}{f_{n}} + \frac{0}{1}
          &= \frac{f_{t} + 0}{1}\\
          &&= \frac{f_{t}}{f_{n}}\\
          &&= \frac{0 + f_{t}}{1}\\
          &&= 0 +  \frac{f_{t}}{f_{n}}\\
        \end{array}
        \]
        We gebruiken hier dat $0$ het neutraal element is voor $+$ in $\mathbb{Z}$.
      \item Elk element $x$ heeft een (uniek) invers element $-x$ voor $+$:
        \[
        \begin{array}{rrl}
          \forall \frac{f_{t}}{f_{n}}\in \mathbb{Q}: \exists (-f) = \frac{-f_{t}}{f_{n}} \in \mathbb{F}: \ 
          &\frac{f_{t}}{f_{n}} + \frac{-f_{t}}{f_{n}}
          &= \frac{f_{t}f_{n} + (-f_{t}f_{n})}{f_{n}^{2}}\\
          &&= \frac{0}{f_{n}^{2}}\\
          &&= 0 \\
          &&= \frac{0}{f_{n}^{2}}\\
          &&= \frac{-f_{t}f_{n} + f_{t}f_{n}}{f_{n}^{2}}\\
          &&= \frac{-f_{t}}{f_{n}} + \frac{f_{t}}{f_{n}}\\
        \end{array}
        \]
        We gebruiken hier dat $-x$ het invers element is van $x$ voor $+$ in $\mathbb{Z}$. 
      \item $+$ is commutatief:
        \[
        \forall \frac{f_{t}}{f_{n}},\frac{g_{t}}{g_{n}} \in \mathbb{Q}:\
        \frac{f_{t}}{f_{n}} + \frac{g_{t}}{g_{n}}
        = \frac{f_{t}g_{n}+g_{t}f_{n}}{f_{n}g_{n}}\\ 
        = \frac{g_{t}}{g_{n}} + \frac{f_{t}}{f_{n}}\\
        \]
        We gebruiken hier dat $\cdot$ commutatief is en $+$ associatief in $\mathbb{Z}$.
      \item $\cdot$ is associatief:
        \[
        \forall \frac{f_{t}}{f_{n}},\frac{g_{t}}{g_{n}},\frac{h_{t}}{h_{n}} \in \mathbb{Q}:\
        \left(\frac{f_{t}}{f_{n}} \cdot \frac{g_{t}}{g_{n}}\right) \cdot \frac{h_{t}}{h_{n}}
        =\frac{f_{t}g_{t}}{f_{n}g_{n}} \cdot \frac{h_{t}}{h_{n}}
        =\frac{f_{t}g_{t}h_{t}}{f_{n}g_{n}h_{n}}
        =\frac{f_{t}}{f_{n}} \cdot \frac{g_{t}h_{t}}{g_{n}h_{n}}
        \]
        We gebruiken hier dat $\cdot$ commutatief is in $\mathbb{Z}$.
      \item Er bestaat een (uniek) neutraal element $1$ voor $\cdot$:
        \[
        \begin{array}{rrl}
          \forall \frac{f_{t}}{f_{n}}\in \mathbb{Q}:\
          &\frac{f_{t}}{f_{n}} \cdot \frac{1}{1}
          &= \frac{f_{t}\cdot 1}{1 \cdot 1}\\
          &&= \frac{f_{t}}{f_{n}}\\
          &&= \frac{1\cdot f_{t}}{1 \cdot 1}\\
          &&= 1 \cdot \frac{f_{t}}{f_{n}}\\
        \end{array}
        \]
        We gebruiken hier dat $1$ het neutraal element is voor $\cdot$ in $\mathbb{Z}$.
      \item Elk element $x$ heeft een (uniek) invers element $x^{-1}$ voor $\cdot$:
        \[
        \forall \frac{f_{t}}{f_{n}}\in \mathbb{Q}: \exists \frac{f_{t}}{f_{n}}^{-1} = \frac{f_{n}}{f_{t}} \in \mathbb{F}: \
        = \frac{f_{t}}{f_{n}}\frac{f_{n}}{f_{t}}
        = \frac{f_{t}f_{n}}{f_{t}f_{n}}
        = 1
        = \frac{f_{n}f_{t}}{f_{t}f_{n}}
        = \frac{f_{n}}{f_{t}}\frac{f_{t}}{f_{n}}
        \]
        We gebruiken hier dat $\cdot$ commutatief is in $\mathbb{Z}$.
      \item $\cdot$ is commutatief:
        \[
        \forall \frac{f_{t}}{f_{n}},\frac{g_{t}}{g_{n}} \in \mathbb{Q}:\
        \frac{f_{t}}{f_{n}} \cdot \frac{g_{t}}{g_{n}}
        = \frac{f_{t}g_{t}}{f_{n}g_{n}}
        = \frac{g_{t}}{g_{n}} \cdot \frac{f_{t}}{f_{n}}
        \]
        We gebruiken hier dat $\cdot$ commutatief is in $\mathbb{Z}$.
      \item $\cdot$ is distributief ten opzichte van $+$:
        \[
        \begin{array}{rrl}
          \forall \frac{f_{t}}{f_{n}},\frac{g_{t}}{g_{n}},\frac{h_{t}}{h_{n}} \in \mathbb{Q}:\ 
          &\frac{f_{t}}{f_{n}} \cdot \left(\frac{g_{t}}{g_{n}} + \frac{h_{t}}{h_{n}} \right)
          &= \frac{f_{t}}{f_{n}} \cdot \frac{g_{t}h_{n}+h_{t}g_{n}}{g_{n}h_{n}}\\
          &&= \frac{f_{t}\cdot (g_{t}h_{n}+h_{t}g_{n})}{f_{n}g_{n}h_{n}}\\
          &&= \frac{(f_{t} \cdot g_{t}h_{n}) + (f_{t}\cdot h_{t}g_{n})}{f_{n}g_{n}h_{n}}\\
          &&= \frac{f_{t}g_{t}}{f_{n}g_{n}} + \frac{f_{t}h_{t}}{f_{n}h_{n}}\\
        \end{array}
        \]
      \end{itemize}
    \item $\mathbb{Q}$ is totaal geordend veld.
      \begin{itemize}
      \item
        \[
        \forall \frac{f_{t}}{f_{n}},\frac{g_{t}}{g_{n}},\frac{h_{t}}{h_{n}} \in \mathbb{Q}:\ 
        \frac{f_{t}}{f_{n}} \le \frac{g_{t}}{g_{n}}
        \Rightarrow
        \frac{f_{t}}{f_{n}}+\frac{h_{t}}{h_{n}} \le \frac{g_{t}}{g_{n}}+\frac{h_{t}}{h_{n}}
        \]
        \extra{vanuit welke axioma's bewijzen we dit?}
      \item $\forall x,y,z \in \mathbb{F}:\ x \le y \wedge 0 \le z \Rightarrow x\cdot z \le y\cdot z$
        \extra{vanuit welke axioma's bewijzen we dit?}
      \end{itemize}
    \end{itemize}
  \end{proof}
\end{pr}

\begin{pr}
  \label{pr:geordend-veld-optelling-ongelijkheden}
  Zij $\mathbb{F},+,\cdot,\le$ een totaal geordend veld.
  \[ \forall a,b,c,d \in \mathbb{F}:\ a\le b \wedge c \le d \Rightarrow (a+c) \le (b+d) \]

  \begin{proof}
    We gebruiken enkel de eerste eigenschap in de definitie van een totaal geordend veld.
    Omdat $a\le b$ geldt geldt ook $a + c \le b+c$.
    Bovendien geldt $c+b \le d+b$ omdat $c\le d$ geldt.
    Omdat $+$ commutatief is in $\mathbb{F}$, geldt dan ook het volgende:
    \[ a+c \le b+c \le b+d\]
  \end{proof}
\end{pr}

\begin{pr}
  \label{pr:geordend-veld-ongelijkheid-maal-min-een}
  Zij $\mathbb{F},+,\cdot,\le$ een totaal geordend veld.
  \[ \forall a,b\in \mathbb{F}:\ a \le b \Rightarrow -b \le -a \]

  \begin{proof}
    Bewijs uit het ongerijmbde\\
    Stel $a \le b$ en $-b > -a$ (dus $-a \le b$).
    Omdat $\mathbb{F}$ een totaal geordend veld is, volgt uit $-a \le -b$ zowel $a \le -b+2a$ en $-a+2b \le -b$.
    Tel deze ongelijkheden op\prref{pr:geordend-veld-optelling-ongelijkheden} om $2b \le 2a$ te bekomen.
    Uit $a\le b$ volgt echter dat $2a \le 2b$ geldt (tel immers $a\le b$ bij zichzelf op).
    Contradictie.
  \end{proof}
\end{pr}

\begin{pr}
  \label{pr:geordend-veld-tegengestelde-wisselt-teken}
  Zij $\mathbb{F},+,\cdot,\le$ een totaal geordend veld.
  \[ \forall b \in \mathbb{F}:\ 0 \le b \Leftrightarrow -a \le 0 \]

  \begin{proof}
    Gebruik in propositie \ref{pr:geordend-veld-ongelijkheid-maal-min-een} $a=0$.
  \end{proof}
\end{pr}

\begin{pr}
  \label{pr:geordend-veld-ongelijkheid-vermenigvuldiging}
  Zij $\mathbb{F},+,\cdot,\le$ een totaal geordend veld.
  \[ \forall a,b \in \mathbb{F}:\ 0 \le a \wedge 0 \le b \Rightarrow 0 \le ab \]

  \begin{proof}
    Omdat $\mathbb{F}$ een geordend veld is, volgt uit $0\le a$ dat $0b \le ab$ geldt vanwege $0 \le b$.
    Omdat $0$ het nulelement is van $\mathbb{F}$ geldt $0b = 0$.
  \end{proof}
\end{pr}

\begin{pr}
  \label{pr:nul-kleiner-dan-een}
  Zij $\mathbb{F},+,\cdot,\le$ een totaal geordend veld.
  \[ 0 < 1 \]

  \begin{proof}
    Bewijs uit het ongerijmde\\
    Stel dat $1 \le 0$ geldt, dan moet $1<0$ gelden omdat in een veld $1$ verschilt van $0$.
    Tel hierbij $-1$ op, dan bekomen we $0 \le -1$.
    Vanwege de tweede definierende eigenschap van een totaal geordend veld moet dan uit $1 \le 0$ ook $-1 \le 0$ volgen, maar dat is in strijd met $0 \le -1$ omdat $0$ verschilt van $1$.
  \end{proof}
\end{pr}

\begin{pr}
  \label{pr:geordend-veld-inverse-zelfde-teken}
  Zij $\mathbb{F},+,\cdot,\le$ een totaal geordend veld.
  \[ \forall a \in \mathbb{F}_{0}:\ 0 \le a \Rightarrow 0 \le a^{-1}\]

  \begin{proof}
    Bewijs uit het ongerijmde:
    Stel $0 \le a$ maar ook $a^{-1} < 0$ geldt.
    We mogen die tweede ongelijkheid dan vermenigvuldigen met $a$ om $aa^{-1}< 0$ te bekomen.
    Dit zou $1<0$ betekenen en dat is in contradictie met propositie \ref{pr:nul-kleiner-dan-een}.
  \end{proof}
\end{pr}

\begin{pr}
  \label{pr:geordend-veld-inverse-ongelijkheid-rekenregel}
  Zij $\mathbb{F},+,\cdot,\le$ een totaal geordend veld.
  \[ \forall a,b \in \mathbb{F}_{0}^{+}:\  a \le b \Rightarrow b^{-1} \le a^{-1}\]

  \begin{proof}
    Omdat zowel $0 \le a$ en $0 \le b$ geld, mogen we $a\le b$ vermenigvuldigen met $a^{-1}$ en $b^{-1}$\prref{pr:geordend-veld-inverse-zelfde-teken} om $b^{-1} \le a^{-1}$ te bekomen.
  \end{proof}
\end{pr}

\begin{st}
  $\sqrt{2}$ is geen element van $\mathbb{Q}$.
  \extra{bewijs}
\end{st}

\begin{st}
  $\mathbb{Q}$ heeft de supremumeigenschap niet.
\end{st}

\begin{st}
  De stelling van rolle geldt niet in $\mathbb{Q}$.
  \extra{verwijzen naar een plaats met betere uitleg.}
\end{st}

\extra{nog een gebrek van $\mathbb{Q}$.}

\subsection{Axiomatische beschrijving van $\mathbb{R}$}
\label{sec:axiom-beschr-van}

\begin{st}
  \label{st:supremumeigenschap-R}
  De \term{supremumeigenschap}\\
  In $\mathbb{R}$ heeft elke niet-lege, naar boven begrensde deelverzameling een supremum.
  \extra{bewijs verder rigoreuzer}
\end{st}

\begin{st}
  Er bestaat, op isomorfisme na, maar \'e\'en totaal geordend veld met de supremumeigenschap.
  \extra{bewijs later}
\end{st}

\begin{pr}
  Er bestaat een unieke afbeelding $i:\ \mathbb{Q} \rightarrow \mathbb{R}$ met de volgende eigenschappen.
  \begin{itemize}
  \item $i(0_{\mathbb{Q}}) = 0_{\mathbb{R}}$
  \item $i(1_{\mathbb{Q}}) = 1_{\mathbb{R}}$
  \item $\forall p,q \in \mathbb{Q}: i(p+_{\mathbb{Q}}q) = i(p) +_{\mathbb{R}} i(q)$
  \item $\forall p,q \in \mathbb{Q}: i(p\cdot_{\mathbb{Q}} q) = i(p) \cdot_{\mathbb{R}} i(q)$
  \item $\forall p,q \in \mathbb{Q}: p \le_{\mathbb{Q}} q \Rightarrow i(p) \le_{\mathbb{R}} i(q)$
  \end{itemize}
  Bovendien is deze afbeelding injectief.

  \begin{proof}
    \begin{itemize}
    \item Uniciteit\\
      We bewijzen dat er slechts \'e\'en $i$ kan bestaan, eerst over $\mathbb{N}$, dan over $\mathbb{Z}$ en tenslotte over $\mathbb{Q}$.
      \begin{itemize}
      \item Beschouw een $n\in \mathbb{N}$.
        Er zijn dan twee gevallen:
        \begin{itemize}
        \item $n = 0_{\mathbb{O}}$:
          Dan moet $i(n)$ $0_{\mathbb{R}}$ zijn vanwege de eerste eigenschap.
        \item $n \neq 0$:
          $n$ is dan als de som van $n$ $1_{\mathbb{Q}}$-tjes te schrijven:\needed
          \[ n = n 1_{\mathbb{Q}} \]
          Vanwege de derde eigenschap is $i(n)$ dan de som van $n$ $1_{\mathbb{R}}$-jes te schrijven.
        \end{itemize}
        Hierdoor ligt $i$ al vast op $\mathbb{N}$. \waarom
      \item Beschouw vervolgens een getal $z\in \mathbb{Z}\setminus \mathbb{N}$.
        Er bestaat dan een getal $n \in \mathbb{N}$ zodat $z$ het tegengestelde is van $n$: $z = -n$.\needed
        De vierde eigenschap zegt ons dan het volgende:
        \[ i(-n) = -i(n) \]
        Hierdoor ligt $i$ al vast op $\mathbb{Z}$. \waarom
      \item Beschouw tenslotte een $q\in \mathbb{Q}$, dan valt $q$ te schrijven als $\nicefrac{n}{m}$ met $n\in \mathbb{Z}$ en $m\in \mathbb{N}_{0}$.
        Uit de derde eigenschap volgt dan dat het volgende moet gelden:
        \[ i(q) = i(n)i(m)^{-1} \]
        Dit vervolledigt de uniciteit van $i$.
      \end{itemize}
    \item Bestaan\\
      Verdergaand op het bewijs van de uniciteit construeren we $i$ achtereenvolgens op $\mathbb{N}$, dan op $\mathbb{Z}$ en dan op $\mathbb{Q}$.
      \extra{bewijzen dat $i$ door $+$ en $\cdot$ gaat}
      \extra{bewijzen dat $i$ goed gedefinieerd is voor $\mathbb{Q}$ (onafhankelijk van de gekozen $m$ en $n$).}
      \extra{bewijzen dat $i$ injectief en stijgend is}
    \end{itemize}
  \end{proof}
\end{pr}

\begin{opm}
  Onder dit morfisme beschouwen we $\mathbb{Q}$ als een deelverzameling van $\mathbb{R}$.
\end{opm}


\begin{lem}
  \label{lem:lemma-van-archimedes}
  Het \term{lemma van Archimedes}\\
  Voor elke $x\in \mathbb{R}$ bestaat er een $n\in \mathbb{N}$ zodat $x < n$ geldt.

  \begin{proof}
    Bewijs uit het ongerijmde\\
    Stel dat er een $x\in \mathbb{R}$ zou bestaan zodat $\forall n \in \mathbb{N}:\ x \ge n$ geldt, dan zou $\mathbb{N}$ naar boven begrensd zijn door die $x$.
    Noem $s$ dan het supremum van $\mathbb{N}$, dat bestaat immers zeker.\needed
    Omdat $s$ de kleinste bovengrens is van $\mathbb{N}$ is $s-1$ zeker geen bovengrens.
    We kunnen dus een $k\in \mathbb{N}$ vinden zodat $s-1$ kleiner is dan $k$.
    $s$ is dan echter kleiner dan $k+1$ en dus geen bovengrens.
  \end{proof}
\end{lem}

\begin{gev}
  $\forall a \in \mathbb{R}_{0}^{+},\ \forall b\in \mathbb{R}:\ \exists n\in N:\ na > b$

  \begin{proof}
    Als $b$ kleiner is dan $a$ is de stelling evident met $n=1$.
    Als $b$ groter is dan of gelijk aan $a$, ga dan als volgt te werk:
    $a^{-1}$ is positief (want $a$ is positief)\prref{pr:geordend-veld-inverse-zelfde-teken}, uit $a \le b$ volgt dus $1 \le \frac{b}{a}$.
    Kies dan een $n\in \mathbb{N}$ groter dan $\frac{b}{a}$, dit kan immers altijd.\lemref{lem:lemma-van-archimedes}
    Vermenigvuldig tenslotte $\frac{b}{a} < n$ met $a$ om de stelling te bekomen.\deref{de:geordend-veld}
  \end{proof}
\end{gev}

\begin{gev}
  \label{gev:er-bestaat-alijd-iets-rationaal-kleiner}
  $\forall \epsilon \in \mathbb{R}_{0}^{+},\ \exists n\in \mathbb{N}_{0}:\ \frac{1}{n} < \epsilon$

  \begin{proof}
    Gegeven een $\epsilon \in \mathbb{R}_{0}^{+}$, bestaat er een $n \in \mathbb{N}$ groter dan $\frac{1}{\epsilon}$. \lemref{lem:lemma-van-archimedes}
    Voor deze $n$ geldt $\frac{1}{n} < \epsilon$. \prref{pr:geordend-veld-inverse-ongelijkheid-rekenregel}
  \end{proof}
\end{gev}

\begin{gev}
  \label{gev:z-omsluit-elk-r}
  $\forall x\in \mathbb{R}:\ \exists m \in \mathbb{Z}:\ m-1 \le x < m$

  \begin{proof}
    Gevalsonderscheid:
    \begin{itemize}
    \item $x = 0$: triviaal, kies $m=1$.
    \item $x > 0$\\
      Noem $X$ de verzameling van natuurlijke getallen groter dan $x$.
      $X$ is zeker niet leeg.\lemref{lem:lemma-van-archimedes}
      Kies nu voor $m$ het minimum van $X$.(Dat bestaat zeker.) \waarom
      Omdat $m$ het kleinste element is van $X$ is $(m-1)$ geen element van $X$.
    \item $x < 0$\\
      Vindt zoals hierboven beschreven de $m$ zodat $m-1 \le -x < m$ geldt.
      Voor $x$ $-m+1$ dan het gezochte getal.\waarom
    \end{itemize}
  \end{proof}
\end{gev}

\begin{pr}
  \label{pr:q-dicht-in-r}
  $\forall x,y \in \mathbb{R}: (x<y \Rightarrow \exists q\in \mathbb{Q}:\ x<q<y)$

  \begin{proof}
    Kies twee elementen $x$ en $y$ in $\mathbb{R}$ met $x<y$, dan kunnen we een $n\in \mathbb{N}_{0}$ nemen zodat $\frac{1}{n}< (y-x)$ geldt.\gevref{gev:er-bestaat-alijd-iets-rationaal-kleiner}
    Tel bij beide kanten van de ongelijkheid $x$ op om $x<\left(x+\frac{1}{n}\right)<y$ te krijgen.\deref{de:totaal-geordend-veld}
    Neem nu een $m\in \mathbb{Z}$ zodat $m-1<nx<m$ geldt.\gevref{gev:z-omsluit-elk-r}
    Vermenigvuldig de rechtse ongelijkheid met $n^{-1}$ om $x< \frac{m}{n}$ te bekomen.\prref{pr:geordend-veld-ongelijkheid-vermenigvuldiging}\prref{pr:geordend-veld-inverse-zelfde-teken}
    Tel bovendien bij de linkse ongelijkheid $1$ op om $m \le nx+1$ te bekomen.\deref{de:totaal-geordend-veld}
    Zet de laatste twee resultaten om $x < \frac{m}{n} < y$ te bekomen.
    Deze $\frac{m}{n}$ is dan de gezochte $q$.
  \end{proof}
\end{pr}

\begin{opm}
  We zeggen dat $\mathbb{Q}$ dicht ligt in $\mathbb{R}$.
\end{opm}

\begin{st}
  \[ \forall x\in \mathbb{R}^{+},\ \forall n\in \mathbb{N}:\ (n\ge 2 \Rightarrow \exists!\ y\in \mathbb{R}^{+}:\ y^{n}=x) \]
  
  \begin{proof}
    Voor $x=0$ is de stelling triviaal met $y=0$.
    Beschouw daarom $x>0$.
    \begin{itemize}
    \item Bestaan\\
      Beschouw de verzameling $A$:
      \[ A = \{ a\in \mathbb{R}^{+}\mid a^{n}<x \} \]
      \begin{itemize}
      \item $A$ is niet leeg:\\
        Beschouw $a=\frac{x}{x+1}$, dan geldt $a^{n} < a$ omdat $a \le 1$ geldt (want $x+1$ is groter dan $x$).
        $a$ is bovendien kleiner dan $x$ (want $x+1$ is groter dan $1$, $x$ is immers positief).
        $a$ behoort dus tot $A$.
      \item $A$ is naar boven begrensd:\\
        We beweren dat $x+1$ een bovengrens is voor $A$.
        Neem daartoe een $a\in A$.
        Als $x+1$ immers kleiner zijn dan $A$, zou $x$ kleiner zijn dan $a^{n}$, en dat is een tegenspraak.
        \[ 
        \begin{array}{c}
          x+1 < a\\
          (x+1)^{n} < a^{n}\\
          x < x+1 \le (x+1)^{n} < a^{n}\\
        \end{array}
        \]
      \item $y = sup A$ zodat $y^{n}=x$:\\
        We maken een gevalsonderscheid om de tegenstelling tegen te spreken:
        \begin{itemize}
        \item $y^{n}<n$\\
          Moest dit gelden, dan zou er een $h\in \mathbb{R}_{0}^{+}$ bestaan zodat $(y+h)^{n}<x$ geldt (zie hierna), maar dat zou betekenen dat $y+h$ ook in $A$ zou zitten en dat kan niet omdat $y$ een bovengrens is.
          \[ (y+h)^{n}-y^{n} = h\left( \sum^{n-1}_{i=0}(y+h)^{n-1-i}y^{i}\right) \]
          Als $h$ kleiner is dan $1$ is het rechterlid kleiner dan $nh(y+1)^{n-1}$.\waarom
          Als $h$ kleiner is dan $\min\left\{ 1,\frac{x-y}{h(y+1)^{n-1}}\right\}$ is dit kleiner dan $x-y^{n}$\waarom.
          Hieruit volgt tenslotte $(y+h)^{n}<x$.
        \item $y^{n}>k$\\
          Analoog vinden we een $h\in \mathbb{R}_{0}^{+}$ zodat $y-h>0$ en $(y-h)^{n}>x$ gelden.\question{hoe precies?}
          Omdat $y-h$ hierdoor geen bovengrens is voor $A$\waarom, bestaat er dan een $a\in A$ groter dan $y-h$.
          Daaruit volgt dan dat $x<a^{n}$ geldt en dat is opnieuw een tegenspraak.
        \end{itemize}
      \end{itemize}
    \item Uniciteit\\
      Uit het ongerijmde:
      Stel dat er twee verschillende getallen $y$ en $y'$ bestonden met $y^{n}=x$ en $y'^{n}=x$, stel met $y<y'$, dan zou $y^{n}$ kleiner zijn dan $y'^{n}$ en dat is in tegenspraak met $y^{n}=x$:\waarom
      \[ y^{n}-y'^{n} = (y-y')\left( \sum^{n-1}_{i=0}y^{n-1-i}y'^{i}\right) \]
    \end{itemize}
  \end{proof}
\end{st}

\subsection{Intervallen in $\mathbb{R}$}
\label{sec:intervallen-in-R}

\begin{de}
  Een \term{interval} in een totaal geordende verzameling $F,\le$ is een niet-lege deelverzameling $I$ van $F$ waarvoor elk element van $F$ dat tussen twee elementen in $I$ ligt, tot $I$ behoort.
  \[ \forall x,y \in I,\ \forall z\in F:\ x \le z \le y \Rightarrow z\in I \]
\end{de}

\begin{vb}
  $\{x\in \mathbb{R} \mid i \le x \le 2\}$ is een interval.
\extra{bewijs}
\end{vb}

\begin{vb}
  $\{1,2\}$ is geen interval want $1 \le \nicefrac{3}{2} \le 2$ geldt maar $\nicefrac{3}{2}$ zit niet in $\{1,2\}$.  
\extra{bewijs}
\end{vb}

\begin{vb}
  $\{x\in \mathbb{R} \mid x > 5\}$ is een interval.
\extra{bewijs}
\end{vb}

\begin{vb}
  $\mathbb{R}_{0}$ is geen interval.
\extra{bewijs}
\end{vb}


\subsubsection{Classificatie van intervallen}
\label{sec:class-van-interv}

\begin{st}
  De \term{classificatie van intervallen in $\mathbb{R}$}.

  Beschouw een willekeurig interval $I \subseteq \mathbb{R}$, dan zijn er een aantal mogelijkheden:
  \begin{itemize}
  \item $I$ is zowel naar boven als naar onder begrensd.\\
    $I$ heeft dan zowel een supremum $b$ als een infimum $a$.\stref{st:supremumeigenschap-R}.
    \begin{itemize}
    \item $a=b$: $I=\{a\} = [a,a]$
    \item $a<b$: 
      \begin{itemize}
      \item $a\in I \wedge b\in I$: $I = \{ x\in \mathbb{R} \mid a\le x \le b\} = [a,b]$ : ``het gesloten interval $a,b$''.
      \item $a\in I \wedge b\not\in I$: $I = \{ x\in \mathbb{R} \mid a\le x < b\} = [a,b[$ : ``het halfopen interval $a,b$, open in $b$''.
      \item $a\not\in I \wedge b\in I$: $I = \{ x\in \mathbb{R} \mid a< x \le b\} = ]a,b]$ : ``het halfopen interval $a,b$, open in $a$''.
      \item $a\not\in I \wedge b\not\in I$: $I = \{ x\in \mathbb{R} \mid a< x < b\} = ]a,b[$ : ``het open interval $a,b$''.
      \end{itemize}
    \end{itemize}
  \item $I$ is naar onder begrensd.
    $I$ heeft dan een infimum $a$.
    \begin{itemize}
    \item $a\in I$: $I = \{ x\in \mathbb{R} \mid x \ge a\} = [a,+\infty[$ : ``het gesloten interval $a, +\infty$''.
    \item $a\not\in I$: $I = \{ x\in \mathbb{R} \mid x > a\} = ]a,+\infty[$ : ``het open interval $a, +\infty$''.
    \end{itemize}
  \item $I$ is naar boven begrensd.
    $I$ heeft dan een supremum $b$.
    \begin{itemize}
    \item $a\in I$: $I = \{ x\in \mathbb{R} \mid x \le b \} = ]-\infty,b]$ : ``het gesloten interval $-\infty, b$''.
    \item $a\not\in I$: $I = \{ x\in \mathbb{R} \mid x < b \} = ]-\infty,b[$ : ``het open interval $-\infty,b$''. 
    \end{itemize}
  \item $I$ is niet begrensd. $I$ is dan gelijk aan $\mathbb{R}$.
  \end{itemize}
  \zb
\end{st}


\subsection{Absolute waarde}
\label{sec:absolute-waarde}

\begin{de}
  De \term{absolute waarde} van een element $a$ van een totaal geordend veld $F,+,\cdot,\le$ defeni\"eren we als $|a|$:
  \[ 
  |a| = 
  \left\{
    \begin{array}{cl}
      a &\text{ als } a\ge 0\\
      -a &\text{ als } a < 0\\
    \end{array}
  \right.
  \]
\end{de}

\begin{pr}
  \label{pr:absolute-waarde-positief}
  $\forall a\in F: |a| \ge 0$

  \begin{proof}
    Gevalsonderscheid:
    \begin{itemize}
    \item $a \ge 0$: $|a| = a \ge 0$
    \item $a < 0$: $|a| = -a \ge 0$\prref{pr:geordend-veld-tegengestelde-wisselt-teken}
    \end{itemize}
  \end{proof}
\end{pr}

\begin{pr}
  $\forall a\in F: |a| = 0 \Leftrightarrow a = 0$
  \question{hoe bewijzen we dit?}
\end{pr}

\begin{pr}
  $\forall a\in F: |a| = |-a|$

  \begin{proof}
    Ofwel $a$, ofwel $-a$ is negatief\needed, de absolute waarde daarvan is het tegengestelde, dus de andere en die blijft gelijk.
  \end{proof}
\end{pr}

\begin{pr}
  $\forall a\in F: -|a| \le a \le |a|$
  \begin{proof}
    Gevalsonderscheid.
    \begin{itemize}
    \item $a \ge 0$: $a = |a|$ en zeker $a \ge -a = -|a| \le 0$\prref{pr:absolute-waarde-positief}\prref{pr:geordend-veld-tegengestelde-wisselt-teken}
    \item $a <0$: $a=-|a|$ en zeker $a \le |a| \ge 0$\prref{pr:absolute-waarde-positief}
    \end{itemize}
  \end{proof}
\end{pr}

\begin{pr}
  $\forall a,b\in F: |a| \le b \Leftrightarrow -b \le a \le b$.

  \begin{proof}
    Gevalsonderscheid.
    \begin{itemize}
    \item $a \ge 0$: $a \le b \Leftrightarrow -b \le -a \le a \le b$
    \item $a <0$: $-a \le b \Leftrightarrow -b \le a \le -a \le b$
      \extra{meer uitwerken?}
    \end{itemize}
  \end{proof}
\end{pr}

\begin{de}
  De \term{afstand} tussen twee elementen van een totaal geordend veld $F,+,\cdot,\le$ defini\"eren we als $|x-y|$.
\end{de}

\begin{pr}
  De \term{driehoeksongelijkheid}\\
  \begin{itemize}
  \item $\forall a,b\in F:\ |a+b| \le |a| + |b|$
  \item $\forall x,y,z\in F:\ |x-y| \le |x-z| + |z-y|$
  \end{itemize}
  \extra{bewijs}
\end{pr}

\begin{pr}
  De \term{tweede driehoeksongelijkheid}\\
  $\forall a,b\in F: ||a|-|b|| \le |a-b|$
  \extra{bewijs}
\end{pr}

\section{Complexe getallen}
\label{sec:complexe-getallen}



\begin{de}
  De verzameling $\mathbb{C}$ van \term{complexe getallen} defini\"eren we als volgt, samen met de optelling $(+)$ en vermenigvuldiging $(\cdot)$.
  \[ \mathbb{C} = \mathbb{R}^{2} = \{ (a,b) \mid a,b\in \mathbb{R} \} \]
  \[ (+):\ \mathbb{C}^{2} \rightarrow \mathbb{C}:\ (a,b) + (c,d) = (a+c,b+d) \]
  \[ (\cdot):\ \mathbb{C}^{2} \rightarrow \mathbb{C}:\ (a,b) \cdot (c,d) = (ac-bd, ad+bc) \]
\end{de}

\begin{st}
  $\mathbb{C},+,\cdot$ is een veld.
  \extra{bewijs p 81}
\end{st}

\begin{pr}
  We kunnen $\mathbb{R}$ inbedden in $\mathbb{C}$ met de volgende injectieve afbeelding:
  \[ \phi:\ \mathbb{R} \rightarrow \mathbb{C}:\ a \mapsto (a,0) \]
  Deze afbeelding is bovendien een ringmorfisme:
  \[ \forall a,b \in \mathbb{R}:\ \phi(a+b) = \phi(a) + \phi(b) \]
  \[ \forall a,b \in \mathbb{R}:\ \phi(ab) = \phi(a)\phi(b) \] 
  \extra{bewijs: oefening}
\end{pr}

\begin{opm}
  Onder deze injectie beschouwen we $\mathbb{R}$ als een deelveld van $\mathbb{C}$.
\end{opm}

\begin{de}
  We noteren een element $(a,b)$ van $\mathbb{C}$ vaak als $a+bi$.
  We noemen dit de \term{carthesiaanse vorm van een complex getal}.
\end{de}

\begin{opm}
  Deze notatie komt van pas om rekenregels binnen $\mathbb{C}$ eenvoudig te houden als we $i^{2}=1$ als regel in het achterhoofd houden. 
  \extra{bewijs!}
\end{opm}

\begin{de}
  We noemen in een element $a+bi$ van $\mathbb{C}$ $a$ het \term{re\"eel} deel en $b$ het \term{imaginair} deel.
  We voeren daarom twee afbeeldingen in:
  \[ Re:\ \mathbb{C} \rightarrow \mathbb{R}:\ (a+bi) \mapsto a \]
  \[ Im:\ \mathbb{C} \rightarrow \mathbb{R}:\ (a+bi) \mapsto b \]
\end{de}

\begin{st}
  De \term{hoofdstelling van de algebra}\\
  $\mathbb{C}$ is algebra\"isch gesloten: Een $n$-de graadsveelterm over $\mathbb{C}$ heeft precies $n$ wortels in $\mathbb{C}$.
  \zb
\end{st}

\begin{pr}
  Er bestaat geen orde op $\mathbb{C}$ die van $\mathbb{C},+,\cdot$ een totaal geordend veld maakt.

  \begin{proof}
    Bewijs uit het ongerijmde: stel dat er wel zo'n orde $\le$ bestaat.
    Dan moet $i$ ofwel groter dan $0$ zijn, ofwel kleiner dan nul.
    In beide gevallen volgt $-1 > 0$ of $1 < 0$,\prref{pr:geordend-veld-ongelijkheid-vermenigvuldiging} wat niet mogelijk is in een geordende veld.\prref{pr:nul-kleiner-dan-een}
  \end{proof}
\end{pr}

\begin{de}
  Het \term{complex toegevoegde} $\overline{a+bi}$ van een complex getal $a+bi$ definieren we als volgt:
  \[ \overline{a+bi} = a-bi \]
  \[ \overline{\, \cdot\ }:\ \mathbb{C} \rightarrow \mathbb{C}:\ a+bi \mapsto \overline{a+bi} = a-bi \]
\end{de}

\begin{de}
  De \term{modulus} $|a+bi|$ van een complex getal $a+bi$ definieren we als volgt:
  \[ |a+bi| = \sqrt{a^{2}+b^{2}} \]
  \[ |\cdot|:\ \mathbb{C} \rightarrow \mathbb{R}:\ a+bi \mapsto |a+bi| = \sqrt{a^{2}+b^{2}} \]
\end{de}

\begin{ei}
  De modulus van een complex getal is de tegenhanger van de absolute waarde van een re\"eel getal.
  \[ \forall a \in \mathbb{R}:\ |a| = |\phi(a)| \]
  \begin{proof}
    $\forall a\in \mathbb{R}:\ |\phi(a)| = |a+0i| = \sqrt{a^{2} + 0^{2}} = |a|$
  \end{proof}
\end{ei}

\begin{pr}
  \[ \forall z\in \mathbb{C}:\ \bar{\bar{z}} = z \]

  \begin{proof}
    $\forall a+bi\in \mathbb{C}:\ \bar{\bar{a+bi}} = \bar{a-bi} = a-(-bi) = a+bi$
  \end{proof}
\end{pr}

\begin{pr}
  \label{pr:modulus-door-optelling}
  \[ \forall z_{1},z_{2}\in \mathbb{C}:\ \overline{z_{1}+z_{2}} = \overline{z_{1}} + \overline{z_{2}} \]

  \begin{proof}
    $\forall a+bi,c+di \in \mathbb{C}:\ \overline{a+bi + c+di} = \overline{(a+c)+(b+d)i} =(a+c)-(b+d)i = a+c-bi-di = (a-bi) + (c-di)$
  \end{proof}
\end{pr}

\begin{pr}
  \[ \forall z_{1},z_{2}\in \mathbb{C}:\ \overline{z_{1}z_{2}} = \bar{z}_{1}  \bar{z}_{2} \]

  \begin{proof}
    $\forall a+bi,c+di \in \mathbb{C}:\ \overline{a+bi \cdot c+di} = \overline{ac-bd + (ad+bc)i} = ac-bd - adi - bci = ac-(-b)(-d) + (a(-d)+(-b)c)i = \overline{a+bi} \cdot \overline{c+di} $
  \end{proof}
\end{pr}

\begin{pr}
  \[ \forall z\in \mathbb{C}:\ Re(z) = \frac{z+\bar{z}}{2} \]

  \begin{proof}
    $\forall a+bi\in \mathbb{C}:\ \frac{(a+bi)+\overline{a+bi}}{2} = \frac{(a+bi)+(a-bi)}{2} = \frac{2a}{2} = a = Re(a+bi)$
  \end{proof}
\end{pr}

\begin{pr}
  \[ \forall z\in \mathbb{C}:\ Im(z) = \frac{z-\bar{z}}{2i} \]

  \begin{proof}
    $\forall a+bi\in \mathbb{C}:\ \frac{(a+bi)-\overline{a+bi}}{2i} = \frac{(a+bi)-(a-bi)}{2i} = \frac{(a+bi-a+bi)}{2i} = \frac{2bi}{2i} = b = Im(a+bi) $
  \end{proof}
\end{pr}

\begin{pr}
  \[ \forall z\in \mathbb{C}:\ |\bar{z}| = |z| \]

  \begin{proof}
    $\forall a+bi\in \mathbb{C}:\ |\bar{a+bi}| = |a-bi| = \sqrt{a^{2}+(-b)^{2}} = \sqrt{a^{2}+b^{2}} = |a+bi|$
  \end{proof}
\end{pr}

\begin{pr}
  \label{pr:reel-deel-kleiner}
  \[ \forall z\in \mathbb{C}:\ |Re(z)| \le |z| \]

  \begin{proof}
    $\forall a+bi\in \mathbb{C}:\  |Re(a+bi)| = |a| \le \sqrt{a^{2}+b^{2}} = |a+bi|$ 
  \end{proof}
\end{pr}

\begin{pr}
  \label{pr:imaginair-deel-kleiner}
  \[ \forall z\in \mathbb{C}:\ |Im(z)| \le |z|\]

  \begin{proof}
    $\forall a+bi\in \mathbb{C}:\  |Im(a+bi)| = |b| \le \sqrt{a^{2}+b^{2}} = |a+bi|$ 
  \end{proof}
\end{pr}

\begin{pr}
  \label{pr:normaal-maal-toegevoegde-in-r}
  \[ \forall z\in \mathbb{C}:\ \bar{z}z = Im(z)^{2}+Re(z)^{2} \in \mathbb{R} \]

  \begin{proof}
    $\forall a+bi\in \mathbb{C}:\  \overline{a+bi}\cdot(a+bi) = (a-bi)(a+bi) = a^{2} + (-bi)a + a(bi) + (-bi)(bi) = a^{2} + b^{2} \in \mathbb{R}$
  \end{proof}
\end{pr}

\begin{pr}
  \label{pr:modulus-in-termen-van-toegevoegde}
  \[ \forall z\in \mathbb{C}:\ |z| = \sqrt{\bar{z}z} \]

  \begin{proof}
    $\forall a+bi\in \mathbb{C}:\ \sqrt{\overline{a+bi}\cdot(a+bi)} = \sqrt{a^{2}+b^{2}} = |a+bi|$
  \end{proof}
\end{pr}

\begin{pr}
  \[ \forall z\in \mathbb{C}:\ \forall z \in \mathbb{C}_{0}: \frac{1}{z} = \frac{\bar{z}}{|z|^{2}} \]

  \begin{proof}
    $\forall a+bi\in \mathbb{C}:\ \frac{\overline{a+bi}}{|a+bi|^{2}} = \frac{a-bi}{\left(\sqrt{a^{2}+b^{2}}\right)^{2}} = \frac{a-bi}{a^{2}+b^{2}}= \frac{(a-bi)(a+bi)}{(a^{2}+b^{2})(a+bi)} = \frac{a^{2}+b^{2}}{(a^{2}+b^{2})(a+bi)} = \frac{1}{a+bi}$
  \end{proof}
\end{pr}

\begin{pr}
  \[ \forall z_{1},z_{2}\in \mathbb{C}:\ |z_{1}z_{2}| = |z_{1}||z_{2}| \]

  \begin{proof}
    \[
    \forall a+bi,c+di \in \mathbb{C}:\
    \begin{array}{rll}
      |(a+bi)(c+di)| &= |ac-bd + (ad+bc)i|\\
      &= \sqrt{(ac-bd)^{2} + (ad+bc)^{2}}\\
      &= \sqrt{a^{2}c^{2} +b^{2}d^{2}-2abcd + a^{2}d^{2} + b^{2}c^{2} +2abcd}\\
      &= \sqrt{a^{2}c^{2} +b^{2}d^{2} + a^{2}d^{2} + b^{2}c^{2}}\\
      &= \sqrt{a^{2}+b^{2}}\sqrt{c^{2}+d^{2}}
    \end{array}
    \]
  \end{proof}
\end{pr}

\begin{pr}
  \[ \forall z_{1},z_{2}\in \mathbb{C}:\ |z_{1}+z_{2}| \le |z_{1}|+|z_{2}| \]

  \begin{proof}
    \[
    \forall a+bi,c+di \in \mathbb{C}:\
    \begin{array}{rll}
      |(a+bi)+(c+di)|^{2} &= |(a+c)+(b+d)i|^{2}\\
      &= (a+c)^{2}+ (b+d)^{2}\\
      &= a^{2}+c^{2} + b^{2}+d^{2}+2(ac + bd)\\
      &= a^{2}+c^{2} + b^{2}+d^{2}+2Re((a-bi)(c+di))\\
      &\le a^{2}+b^{2} + c^{2}+d^{2} + 2|(a-bi)(c+di)|\\
      &= a^{2}+b^{2} + c^{2}+d^{2} + 2\sqrt{a^{2}c^{2} + a^{2}d^{2} +b^{2}c^{2}+ b^{2}d^{2}}\\
      &= a^{2}+b^{2} + c^{2}+d^{2} + 2\sqrt{(a^{2}+b^{2})(c^{2}+d^{2})}\\
      &= |a+bi|^{2} + |c+di|^{2} + 2|a+bi||c+di|\\
      &= \left(|a+bi|+|c+di|\right)^{2}\\
    \end{array}
    \]
  \end{proof}
\end{pr}

\begin{pr}
  \label{pr:tweede-driehoeksongelijkheid-C}
  \[ \forall x,y,z\in \mathbb{C}:\ |x-z| \le |x-y| + |y-z| \]

  \begin{proof}
    Gebruik de driehoeksongelijkheid op $|(x-y)+(y-z)| \le |x-y| + |y-z|$
  \end{proof}
\end{pr}

\begin{pr}
  \[ \forall z_{1},z_{2}\in \mathbb{C}:\ ||z_{1}|-|z_{2}|| \le |z_{1}-z_{2}| \]

  \begin{proof}
    \[ |x| + |y-x| \ge |x+y-x| = |y| \]
    \[ |y| + |x-y| \ge |y+x-y| = |x| \]
    Verplaats in beide ongelijkheden de linker term naar de rechterkant:
    \[ |y-x| \ge |y| - |x| \]
    \[ |x-y| \ge |x| - |y| \]
    $|y-x|$ is gelijk aan $|x-y|$. Hieruit, samen met $t \ge a \wedge t \ge -a \Rightarrow t \ge |a|$ volgt de stelling:
    \[ |x-y| \ge \left||x|-|y|\right| \]
    \extra{gebruikte stelling afsplitsen}
  \end{proof}
\end{pr}

\begin{de}
  Zij $z = a+bi$ de carthesiaanse co\"ordinaten van een complex getal, dan definieren we de \term{poolcoordinaten} van dat getal als $(r,\theta)$:
  \[ a = r \cos \theta \quad\text{ en }\quad b = r\sin \theta \]
  \[ z = r(\cos \theta + i \sin \theta)  \]
\end{de}

\begin{st}
  \label{st:vermenigvuldiging-poolcoordinaten}
  \[
  \forall z_{1},z_{2} \in \mathbb{C}:\ 
  r_{1}(\cos \theta_{1} + i \sin \theta_{1}) \cdot r_{2}(\cos \theta_{2} + i \sin \theta_{2})
  = r_{1}r_{2}\left(\cos(\theta_{1}+\theta_{2}) + i\sin(\theta_{1}+\theta_{2})\right)
  \]
  
  \[ 
  \begin{array}{l}
    \forall z_{1},z_{2} \in \mathbb{C}:\\
    r_{1}(\cos \theta_{1} + i \sin \theta_{1}) \cdot r_{2}(\cos \theta_{2} + i \sin \theta_{2})\\
    = r_{1}r_{2} (\cos \theta_{1} + i \sin \theta_{1})(\cos \theta_{2} + i \sin \theta_{2})\\
    = r_{1}r_{2} \left( \left(\cos \theta_{1}\cos \theta_{2} -\sin \theta_{1}\sin \theta_{2}\right)+ i\left( \sin \theta_{1}\cos \theta_{2} + \cos \theta_{1}\sin \theta_{2}\right)  \right)\\
    = r_{1}r_{2}\left(\cos(\theta_{1}+\theta_{2}) + i\sin(\theta_{1}+\theta_{2})\right)\\
  \end{array}
  \]
\end{st}

\begin{st}
  De \term{formule van de Moivre}\\
  \[ \forall z = r(\cos \theta + i \sin \theta) \in \mathbb{C}, n\in \mathbb{Z}:\ (r(\cos \theta + i \sin \theta))^{n} = r^{n}(\cos n\theta + i \sin n\theta)\]

  \begin{proof}
    Gevalsonderscheid
    \begin{itemize}
    \item Voor alle $n\in \mathbb{N}$ volgt dit meteen uit stelling
      \ref{st:vermenigvuldiging-poolcoordinaten}.
    \item Voor $n=-1$:
      \[
      \begin{array}{rl}
        (r(\cos \theta + i \sin \theta))^{-1}
        &= \frac{1}{r(\cos \theta + i \sin \theta)}\\
        &= \frac{\cos \theta - i \sin \theta}{r(\cos \theta + i \sin \theta)(\cos \theta - i \sin \theta)}\\
        &= \frac{\cos(\theta) - i\sin(\theta)}{r}\\
        &= r^{-1}\cos(-\theta) + i\sin(-\theta)\\
      \end{array}
      \]
    \item Voor $-n\in \mathbb{Z}^{-}$ komt een $-n$-de macht met een $n$-de macht en een $-1$-e macht na elkaar.
    \end{itemize}
  \end{proof}
\end{st}


\begin{de}
  We noemen $z\in \mathbb{C}$ een $n$-de \term{eenheidswortel} als het volgende geldt:
  \[ z^{n} = 1 \]
  Merk op dat er voor graad $n$ precies $n$ eenheidswortel zijn:
  \[ z = \cos\left(\frac{2k\pi}{n}\right) + i \sin \left( \frac{2k\pi}{n} \right) \]
  \extra{afsplitsen in stelling?}
\end{de}


\end{document}

%%% Local Variables:
%%% mode: latex
%%% TeX-master: t
%%% End:
