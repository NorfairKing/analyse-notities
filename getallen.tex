\documentclass[main.tex]{subfiles}
\begin{document}

\chapter{Re\"ele en complexe getallen}
\label{cha:reele-en-complexe-getallen}

\section{De rationale getallen en hun structuur}
\label{sec:de-rati-getall}

\begin{de}
  De natuurlijke getallen, genoteerd als $\mathbb{N}$ zijn inductief gedefini\"eerd als volgt.
  \begin{itemize}
  \item $\mathbb{N}$ bevat een neutraal element $0$.
  \item $\mathbb{N}$ bevat een eenheidselement $1$.
  \item Voor elk aantal $n$ bevat $\mathbb{N}$ ook $0+1n$.
  \end{itemize}
\end{de}

\begin{de}
  De \term{gehele getallen}, genoteerd als $\mathbb{Z}$.
  \[ \mathbb{Z} = \mathbb{N} \cup \{ -n \mid n\in \mathbb{N} \}  \]
\end{de}

\begin{de}
  De \term{rationale getallen}, genoteerd als $\mathbb{Q}$.
  \[ \mathbb{Q} = \{ \nicefrac{n}{m} \mid n\in \mathbb{Z}, m\in \mathbb{Z}, m \neq 0 \} \]
\end{de}

\begin{opm}
  In feite is $\mathbb{Q}$ het breukenveld van het integriteitsdomein $\mathbb{N}$, maar daar komen we later nog op terug.   
\end{opm}

\begin{de}
  Een \term{veld} $\mathbb{F},+,\cdot$ is een verzameling $\mathbb{F}$ met twee bewerkingen $+$ en $\cdot$ met de volgende eigenschappen.
  \begin{itemize}
  \item $+$ is associatief.
  \item Er bestaat een (uniek) neutraal element $0$ voor $+$.
  \item Elk element $x$ heeft een (uniek) invers element $-x$ voor $+$.
  \item $+$ is commutatief.
  \item $\cdot$ is associatief.
  \item Er bestaat een (uniek) neutraal element $1$ voor $\cdot$.
  \item Elk element $x$ heeft een (uniek) invers element $x^{-1}$ voor $\cdot$.
  \item $\cdot$ is commutatief.
  \item $\cdot$ is distributief ten opzichte van $+$.
  \end{itemize}
\end{de}

\begin{opm}
  Deze definitie van een veld is een samenraapsel van een hele boel concepten die pas later aan bod komen.
  We gaan daarom momenteel niet in op de meer rigoreuze benamingen en bewijzen ivm de definierende eigenschappen van velden.
\end{opm}
\extra{een veld goed uitwerken zonder andere algebra te gebruiken.}
                                 mw
\begin{de}
  Zij $\mathbb{F},+,\cdot$ een veld met een totale orderelatie $\le$, dan noemen we het een \term{totaal geordend veld} als aan de volgende eigenschappen voldaan is.
  \begin{itemize}
  \item $\forall x,y,z \in \mathbb{F}:\ x \le y \Rightarrow (x+z) \le (y+z)$
  \item $\forall x,y,z \in \mathbb{F}:\ x \le y \wedge 0 \le z \Rightarrow x\cdot z \le y\cdot z$
  \end{itemize}
\end{de}
\extra{definieer een orderelatie en totale orderelatie.}

\begin{pr}
  $\mathbb{Q}$ is een totaal geordend veld.

\extra{bewijs}
\end{pr}

\begin{pr}
  Zij $\mathbb{F},+,\cdot,\le$ een totaal geordend veld, dan geldt voor alle $a,b,c,d \in \mathbb{F}$ het volgende:
  \begin{itemize}
  \item $a\le b \wedge c \le d \Rightarrow (a+c) \le (b+d)$
  \item $a \le b \Rightarrow -b \le -a$
  \item $0 \le c \Leftrightarrow -c \le 0$.
  \item $0 \le a \wedge 0 \le b \Rightarrow 0 \le ab$
  \item $0 \le 1$
  \item $0 \le a \wedge a \neq 0 \Rightarrow a^{-1} \neq 0 \wedge 0 \le a^{-1}$.
  \item $a \neq 0 \wedge 0 \le a \wedge a \le b \Rightarrow b^{-1} \le a^{-1}$.
  \end{itemize}

  \extra{splitsen in meerdere proposities?}
\extra{bewijs p 5}
\end{pr}

\TODO{definieer naar boven en naar onder begrensde verzamelingen.}

\begin{st}
  $\sqrt{2}$ is geen element van $\mathbb{Q}$.
\extra{bewijs}
\end{st}

\begin{st}
  $\mathbb{Q}$ heeft de supremumeigenschap niet.
\end{st}

\begin{st}
  De stelling van rolle geldt niet in $\mathbb{Q}$.
\extra{verwijzen naar een plaats met betere uitleg.}
\end{st}

\extra{nog een gebrek van $\mathbb{Q}$.}

\section{Axiomatische beschrijving van $\mathbb{R}$}
\label{sec:axiom-beschr-van}

\begin{st}
  De \term{supremumeigenschap}\\
  In $\mathbb{R}$ heeft elke niet-lege, naar boven begrensde deelverzameling een supremum.
\end{st}

\begin{st}
  Er bestaat, op isomorfisme na, maar \'e\'en totaal geordend veld met de supremumeigenschap.
\end{st}

\begin{pr}
  Er bestaat een unieke afbeelding $i:\ \mathbb{Q} \rightarrow \mathbb{R}$ met de volgende eigenschappen.
  \begin{itemize}
  \item $i(0_{\mathbb{Q}}) = 0_{\mathbb{R}}$
  \item $i(1_{\mathbb{Q}}) = 1_{\mathbb{R}}$
  \item $\forall p,q \in \mathbb{Q}: i(p+_{\mathbb{Q}}q) = i(p) +_{\mathbb{R}} i(q)$
  \item $\forall p,q \in \mathbb{Q}: i(p\cdot_{\mathbb{Q}} q) = i(p) \cdot_{\mathbb{R}} i(q)$
  \item $\forall p,q \in \mathbb{Q}: p \le_{\mathbb{Q}} q \Rightarrow i(p) \le_{\mathbb{R}} i(q)$
  \end{itemize}
  Bovendien is deze afbeelding injectief.
\TODO{bewijs p 10}
\end{pr}

\begin{lem}
  Het \term{lemma van Archimedes}\\
  Voor elke $r\in \mathbb{R}$ bestaat er een $n\in \mathbb{N}$ zodat $x < n$ geldt.
\TODO{bewijs p 11}
\end{lem}


\end{document}
