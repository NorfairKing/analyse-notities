\documentclass[main.tex]{subfiles}
\begin{document}

\chapter{Re\"ele en complexe getallen}
\label{cha:reele-en-complexe-getallen}

\section{De rationale getallen en hun structuur}
\label{sec:de-rati-getall}

\begin{de}
  De natuurlijke getallen, genoteerd als $\mathbb{N}$ zijn inductief gedefini\"eerd als volgt.
  \begin{itemize}
  \item $\mathbb{N}$ bevat een neutraal element $0$.
  \item $\mathbb{N}$ bevat een eenheidselement $1$.
  \item Voor elk aantal $n$ bevat $\mathbb{N}$ ook $0+1n$.
  \end{itemize}
\end{de}

\begin{de}
  De \term{gehele getallen}, genoteerd als $\mathbb{Z}$.
  \[ \mathbb{Z} = \mathbb{N} \cup \{ -n \mid n\in \mathbb{N} \}  \]
\end{de}

\begin{de}
  De \term{rationale getallen}, genoteerd als $\mathbb{Q}$.
  \[ \mathbb{Q} = \{ \nicefrac{n}{m} \mid n\in \mathbb{Z}, m\in \mathbb{Z}, m \neq 0 \} \]
\end{de}

\begin{opm}
  In feite is $\mathbb{Q}$ het breukenveld van het integriteitsdomein $\mathbb{N}$, maar daar komen we later nog op terug.   
\end{opm}

\begin{de}
  Een \term{veld} $\mathbb{F},+,\cdot$ is een verzameling $\mathbb{F}$ met twee bewerkingen $+$ en $\cdot$ met de volgende eigenschappen.
  \begin{itemize}
  \item $+$ is associatief.
  \item Er bestaat een (uniek) neutraal element $0$ voor $+$.
  \item Elk element $x$ heeft een (uniek) invers element $-x$ voor $+$.
  \item $+$ is commutatief.
  \item $\cdot$ is associatief.
  \item Er bestaat een (uniek) neutraal element $1$ voor $\cdot$.
  \item Elk element $x$ heeft een (uniek) invers element $x^{-1}$ voor $\cdot$.
  \item $\cdot$ is commutatief.
  \item $\cdot$ is distributief ten opzichte van $+$.
  \end{itemize}
\end{de}

\begin{opm}
  Deze definitie van een veld is een samenraapsel van een hele boel concepten die pas later aan bod komen.
  We gaan daarom momenteel niet in op de meer rigoreuze benamingen en bewijzen ivm de definierende eigenschappen van velden.
\end{opm}
\extra{een veld goed uitwerken zonder andere algebra te gebruiken.}
                                 mw
\begin{de}
  Zij $\mathbb{F},+,\cdot$ een veld met een totale orderelatie $\le$, dan noemen we het een \term{totaal geordend veld} als aan de volgende eigenschappen voldaan is.
  \begin{itemize}
  \item $\forall x,y,z \in \mathbb{F}:\ x \le y \Rightarrow (x+z) \le (y+z)$
  \item $\forall x,y,z \in \mathbb{F}:\ x \le y \wedge 0 \le z \Rightarrow x\cdot z \le y\cdot z$
  \end{itemize}
\end{de}
\extra{definieer een orderelatie en totale orderelatie.}

\begin{pr}
  $\mathbb{Q}$ is een totaal geordend veld.

\extra{bewijs}
\end{pr}

\begin{pr}
  Zij $\mathbb{F},+,\cdot,\le$ een totaal geordend veld, dan geldt voor alle $a,b,c,d \in \mathbb{F}$ het volgende:
  \begin{itemize}
  \item $a\le b \wedge c \le d \Rightarrow (a+c) \le (b+d)$
  \item $a \le b \Rightarrow -b \le -a$
  \item $0 \le c \Leftrightarrow -c \le 0$.
  \item $0 \le a \wedge 0 \le b \Rightarrow 0 \le ab$
  \item $0 \le 1$
  \item $0 \le a \wedge a \neq 0 \Rightarrow a^{-1} \neq 0 \wedge 0 \le a^{-1}$.
  \item $a \neq 0 \wedge 0 \le a \wedge a \le b \Rightarrow b^{-1} \le a^{-1}$.
  \end{itemize}

  \extra{splitsen in meerdere proposities?}
\extra{bewijs p 5}
\end{pr}

\TODO{definieer naar boven en naar onder begrensde verzamelingen.}

\begin{st}
  $\sqrt{2}$ is geen element van $\mathbb{Q}$.
\extra{bewijs}
\end{st}

\begin{st}
  $\mathbb{Q}$ heeft de supremumeigenschap niet.
\end{st}

\begin{st}
  De stelling van rolle geldt niet in $\mathbb{Q}$.
\extra{verwijzen naar een plaats met betere uitleg.}
\end{st}

\extra{nog een gebrek van $\mathbb{Q}$.}

\subsection{Axiomatische beschrijving van $\mathbb{R}$}
\label{sec:axiom-beschr-van}

\begin{st}
  \label{st:supremumeigenschap-R}
  De \term{supremumeigenschap}\\
  In $\mathbb{R}$ heeft elke niet-lege, naar boven begrensde deelverzameling een supremum.
\end{st}

\begin{st}
  Er bestaat, op isomorfisme na, maar \'e\'en totaal geordend veld met de supremumeigenschap.
\end{st}

\begin{pr}
  Er bestaat een unieke afbeelding $i:\ \mathbb{Q} \rightarrow \mathbb{R}$ met de volgende eigenschappen.
  \begin{itemize}
  \item $i(0_{\mathbb{Q}}) = 0_{\mathbb{R}}$
  \item $i(1_{\mathbb{Q}}) = 1_{\mathbb{R}}$
  \item $\forall p,q \in \mathbb{Q}: i(p+_{\mathbb{Q}}q) = i(p) +_{\mathbb{R}} i(q)$
  \item $\forall p,q \in \mathbb{Q}: i(p\cdot_{\mathbb{Q}} q) = i(p) \cdot_{\mathbb{R}} i(q)$
  \item $\forall p,q \in \mathbb{Q}: p \le_{\mathbb{Q}} q \Rightarrow i(p) \le_{\mathbb{R}} i(q)$
  \end{itemize}
  Bovendien is deze afbeelding injectief.
\TODO{bewijs p 10}
\end{pr}

\begin{lem}
  Het \term{lemma van Archimedes}\\
  Voor elke $r\in \mathbb{R}$ bestaat er een $n\in \mathbb{N}$ zodat $x < n$ geldt.
\TODO{bewijs p 11}
\end{lem}

\begin{gev}
  $\forall a \in \mathbb{R}_{0}^{+},\ \forall b\in \mathbb{R}:\ \exists n\in N:\ na > b$
\extra{bewijs}
\end{gev}

\begin{gev}
  $\forall \epsilon \in \mathbb{R}_{0}^{+},\ \exists n\in N:\ \frac{1}{n} < \epsilon$
\extra{bewijs}
\end{gev}

\begin{gev}
  $\forall x\in \mathbb{R}:\ \exists m \in \mathbb{Z}:\ m-1 \le x < m$
\TODO{bewijs p 12}
\end{gev}

\begin{pr}
  $\forall x,y \in \mathbb{R}: (x<y \Rightarrow \exists q\in \mathbb{Q}:\ x<q<y$
\TODO{bewijs p 12}
\end{pr}

\begin{opm}
  We zeggen dat $\mathbb{Q}$ dicht ligt in $\mathbb{R}$.
\end{opm}


\begin{st}
  $\forall x\in \mathbb{R}^{+},\ \forall n\in \mathbb{N}:\ (n\ge 2 \Rightarrow \exists!\ y\in \mathbb{R}^{+}:\ y^{n}=x)$
\TODO{bewijs p 13} 
\end{st}

\subsection{Intervallen in $\mathbb{R}$}
\label{sec:intervallen-in-R}

\begin{de}
  Een \term{interval} in een totaal geordende verzameling $F,\le$ is een niet-lege deelverzameling $I$ van $F$ waarvoor elk element van $\mathbb{R}$ dat tussen twee elementen in $I$ ligt, tot $I$ behoort.
  \[ \forall x,y \in I,\ \forall z\in \mathbb{R}:\ x \le z \le y \Rightarrow z\in I \]
\end{de}

\subsubsection{Classificatie van intervallen}
\label{sec:class-van-interv}

\begin{st}
  De \term{classificatie van intervallen in $\mathbb{R}$}.

  Beschouw een willekeurig interval $I \subseteq \mathbb{R}$, dan zijn er een aantal mogelijkheden:
  \begin{itemize}
  \item $I$ is zowel naar boven als naar onder begrensd.\\
    $I$ heeft dan zowel een supremum $b$ als een infimum $a$.\stref{st:supremumeigenschap-R}.
    \begin{itemize}
    \item $a=b$: $I=\{a\} = [a,a]$
    \item $a<b$: 
      \begin{itemize}
      \item $a\in I \wedge b\in I$: $I = \{ x\in \mathbb{R} \mid a\le x \le b\} = [a,b]$ : ``het gesloten interval $a,b$''.
      \item $a\in I \wedge b\not\in I$: $I = \{ x\in \mathbb{R} \mid a\le x < b\} = [a,b[$ : ``het halfopen interval $a,b$, open in $b$''.
      \item $a\not\in I \wedge b\in I$: $I = \{ x\in \mathbb{R} \mid a< x \le b\} = ]a,b]$ : ``het halfopen interval $a,b$, open in $a$''.
      \item $a\not\in I \wedge b\not\in I$: $I = \{ x\in \mathbb{R} \mid a< x < b\} = ]a,b[$ : ``het open interval $a,b$''.
      \end{itemize}
    \end{itemize}
  \item $I$ is naar onder begrensd.
    $I$ heeft dan een infimum $a$.
    \begin{itemize}
    \item $a\in I$: $I = \{ x\in \mathbb{R} \mid x \ge a\} = [a,+\infty[$ : ``het gesloten interval $a, +\infty$''.
    \item $a\not\in I$: $I = \{ x\in \mathbb{R} \mid x > a\} = ]a,+\infty[$ : ``het open interval $a, +\infty$''.
    \end{itemize}
  \item $I$ is naar boven begrensd.
    $I$ heeft dan een supremum $b$.
    \begin{itemize}
    \item $a\in I$: $I = \{ x\in \mathbb{R} \mid x \le b \} = ]-\infty,b]$ : ``het gesloten interval $-\infty, b$''.
    \item $a\not\in I$: $I = \{ x\in \mathbb{R} \mid x < b \} = ]-\infty,b[$ : ``het open interval $-\infty,b$''. 
    \end{itemize}
  \item $I$ is niet begrensd. $I$ is dan gelijk aan $\mathbb{R}$.
  \end{itemize}
\zb
\end{st}


\subsection{Absolute waard}
\label{sec:absolute-waarde}

\begin{de}
  De \term{absolute waarde} van een element $a$ van een totaal geordend veld $F,+,\cdot,\le$ defeni\"eren we als $|a|$:
  \[ 
  |a| = 
  \left\{
    \begin{array}{cl}
      a &\text{ als } a\ge 0\\
      a &\text{ als } a< 0\\
    \end{array}
  \right.
  \]
\end{de}

\begin{pr}
  $\forall a\in F: |a| \ge 0$
\extra{bewijs}
\end{pr}

\begin{pr}
  $\forall a\in F: |a| = 0 \Leftrightarrow a = 0$
\extra{bewijs}
\end{pr}

\begin{pr}
  $\forall a\in F: |a| = |-a|$
\extra{bewijs}
\end{pr}

\begin{pr}
  $\forall a\in F: -|a| \le a \le |a|$
\extra{bewijs}
\end{pr}

\begin{pr}
  $\forall a,b\in F: |a| \le b \Leftrightarrow -b \le a \le b$.
\extra{bewijs}
\end{pr}

\begin{de}
  De \term{afstand} tussen twee elementen van een totaal geordend veld $F,+,\cdot,\le$ defini\"eren we als $|x-y|$.
\end{de}

\begin{pr}
  De \term{driehoeksongelijkheid}\\
  \begin{itemize}
  \item $\forall a,b\in F:\ |a+b| \le |a| + |b|$
  \item $\forall x,y,z\in F:\ |x-y| \le |x-z| + |z-y|$
  \end{itemize}

\extra{bewijs}
\end{pr}

\begin{pr}
  De \term{tweede driehoeksongelijkheid}\\
  $\forall a,b\in F: ||a|-|b|| \le |a-b|$
\extra{bewijs}
\end{pr}

\section{Rijen in $\mathbb{R}$}
\label{sec:rijen-mathbbr}

\begin{de}
  Een \term{rij} in een verzameling $V$ is een functie als volgt:
  \[ x: \mathbb{N} \rightarrow V:\ n\mapsto x_{n} \]
  De functiewaarden noemen we \term{termen} en de rij wordt genoteerd met $(x_{n})_{n}$.
  Hierin noemen we de $n$ de \term{index} van de term.
\end{de}

\begin{de}
  We zeggen dat een rij $(x_{n})_{n}$ in een totaal geordend veld $F,+,\cdot,\le$ \term{convergeert} naar $a\in F$ als en slechts als het volgende geldt:
  \[ \forall \epsilon \in F_{0}^{+},\ \exists n_{0}\in \mathbb{N},\ \forall n\in \mathbb{N}:\ n \ge n_{0} \Rightarrow |x_{n}-a| < \epsilon \]
  We noemen $a$ de \term{limiet} van de rij $(x_{n})_{n}$:
  \[ a = \lim_{n\rightarrow \infty}x_{n} \]
  Een rij die convergeert noemen we een \term{convergente rij}.
  Een rij die niet convergeert noemen we een \term{divergente rij}.
\end{de}

\begin{de}
  We zeggen dat een rij $(r_{n})_{n}$ in een totaal geordend veld $F,+,\cdot,\le$ divergeert naar plus oneindig als en slechts het volgende geldt:
  \[ \forall M\in F,\ \exists n_{0}\in \mathbb{N},\ \forall n\in \mathbb{N}:\ n \ge n_{0} \Rightarrow x_{n} > M \]
  \[ + \infty = \lim_{n\rightarrow \infty}x_{n}\]
\end{de}

\begin{de}
  We zeggen dat een rij $(r_{n})_{n}$ in een totaal geordend veld $F,+,\cdot,\le$ divergeert naar min oneindig als en slechts het volgende geldt:
  \[ \forall M\in F,\ \exists n_{0}\in \mathbb{N},\ \forall n\in \mathbb{N}:\ n \ge n_{0} \Rightarrow x_{n} < M \]
  \[ - \infty = \lim_{n\rightarrow \infty}x_{n}\]
\end{de}


\begin{pr}
  Zij $p\in \mathbb{Z}$, dan heeft de rij $(n^{p})_{n}$ een limiet:
  \begin{itemize}
  \item $p>0 \rightarrow \lim_{n\rightarrow \infty}n^{p} = + \infty$
  \item $p=0 \rightarrow \lim_{n\rightarrow \infty}n^{p} = 1$
  \item $p<0 \rightarrow \lim_{n\rightarrow \infty}n^{p} = 0$
  \end{itemize}
\extra{bewijs}
\end{pr}

\begin{pr}
  Als een rij een limiet heeft, dan is deze uniek.
\TODO{bewijs p 31}
\end{pr}

\begin{pr}
  Een convergente rij $(x_{n})_{n}$ is begrensd.
  \[ \exists M \in F^{+}: \forall n\in \mathbb{N}: |x_{n}| \le M \]
\TODO{bewijs p 32}
\end{pr}

\begin{opm}
  Merk op dat het omgekeerde niet geldt.
\extra{tegenvoorbeeld}
\end{opm}

\begin{pr}
  Beschouw twee rijen $(x_{n})_{n}$ en $(y_{n})_{n}$.
  Stel dat de staart van de rijen overeen komt, dus dat er een $k\in \mathbb{N}$ bestaat zodat $x_{n}$ gelijk is aan $y_{n}$ voor alle $n$ groter dan $k$.
  De rijen vertonen dan hetzelfde asymptotisch gedrag.
\TODO{bewijs: oefening}
\end{pr}

\begin{pr}
  Voor elk re\"eel getal $x\in \mathbb{R}$ bestaat er een rij $(q_{n})_{n}$ in $\mathbb{Q}$ die convergeert naar $x$.
\TODO{bewijs p 33}
\end{pr}

\subsection{Limieten en orde}
\label{sec:limieten-en-orde}


\begin{de}
  We noemen een rij $(x_{n})_{n}$ in $\mathbb{R}$ ...
  \begin{itemize}
  \item ... \term{stijgend} als en slechts als $x_{n+1} \ge x_{n}$ geldt voor alle $n\in \mathbb{N}$.  
  \item ... \term{strikt stijgend} als en slechts als $x_{n+1} > x_{n}$ geldt voor alle $n\in \mathbb{N}$.  
  \item ... \term{dalend} als en slechts als $x_{n+1} \le x_{n}$ geldt voor alle $n\in \mathbb{N}$.  
  \item ... \term{strikt dalend} als en slechts als $x_{n+1} < x_{n}$ geldt voor alle $n\in \mathbb{N}$.  
  \end{itemize}
\end{de}

\begin{st}
  Een stijgende rij heeft atijd een limiet.
  Deze limiet is bovendien eindig als en slechts als de rij naar boven begrensd is.
  Die limiet is dan het supremum van de rij.
\TODO{bewijs p 34}
\end{st}
\begin{st}
  Een dalende rij heeft atijd een limiet.
  Deze limiet is bovendien eindig als en slechts als de rij naar onder begrensd is.
  Die limiet is dan het infimum van de rij.
\TODO{bewijs: oefening}
\end{st}

\begin{pr}
  Zij $r\in \mathbb{R}$, dan heeft de rij $(r^{n})_{n}$ ...
  \begin{itemize}
  \item ... limiet plus oneindig als $r>1$ geldt.
  \item ... limiet $1$ als $r=1$ geldt.
  \item ... limiet $0$ als $|r|<1$ geldt.
  \item ... geen limiet als $r<-1$ geldt.
  \end{itemize}
\TODO{bewijs p 35}
\end{pr}

\begin{pr}
  Zij $A$ een niet-leeg, naar boven begrensd deel van $\mathbb{R}$, dan bestaat er een stijgende rij in $A$ die convergeert naar het supremum van $A$.
  
\TODO{bewijs p 36}
\end{pr}

\begin{de}
  De \term{uitgebreidde orde in een totaal geordend veld} $F,+,\cdot,\le$.\\
  We moeten soms de orde van een totaal geordend veld uitbreiden over $F \cup \{ -\infty,+\infty\}$.
  De notatie blijft dan hetzelfde maar we voegen het volgende toe.
  \[ \forall a\in F: -\infty \le a \quad\text{ en }\quad \forall a \in F:\ a \le +\infty \]
\end{de}
\TODO{oneindigheden zelf ook eens definieren?}

\begin{pr}
  Zij $(x_{n})_{n}$ en $(y_{n})_{n}$ twee rijen zodat voor alle $n\in \mathbb{N}$ $x_{n}\le y_{n}$ geldt, dan geldt het volgende:
  \[ \lim_{n\rightarrow \infty}x_{n} \le \lim_{n\rightarrow \infty}y_{n} \]
\TODO{bewijs p 37}
\end{pr}

\begin{st}
  De \term{insluitstelling}\\
  Beschouw drie rijen $(x_{n})_{n}$, $(y_{n})_{n}$ en $(z_{n})_{n}$ zodat het volgende geldt:
  \[ \forall n\in \mathbb{N}: x_{n}\le y_{n}\le z_{n} \]
  Als $(x_{n})_{n}$ en $(z_{n})_{n}$ elk een limiet hebben, en die limiet hetzelfde is, dan is dit ook de limiet van $(y_{n})_{n}$.
  \[ \lim_{n\rightarrow \infty}x_{n} = \lim_{n\rightarrow \infty}y_{n} = \lim_{n\rightarrow \infty}z_{n} \]
\TODO{bewijs p 38}
\end{st}

\subsection{Limieten en bewerkingen}
\label{sec:limi-en-bewerk}

\begin{st}
  Zij $(x_{n})_{n}$ een convergente rij en $\lambda\in \mathbb{R}$ een re\"eel getal.
  \[ \lim_{n \rightarrow \infty}(\lambda x_{n}) = \lambda \lim_{n\rightarrow \infty}x_{n} \]
\TODO{bewijs: oefening}
\end{st}

\begin{st}
  Zij $(x_{n})_{n}$ en $(y_{n})_{n}$ twee convergente rijen.
  \[ \lim_{n \rightarrow \infty}(x_{n}+y_{n}) = \lim_{n\rightarrow \infty}x_{n} + \lim_{n\rightarrow \infty}y_{n} \]
\TODO{bewijs: oefening}
\end{st}

\begin{st}
  Zij $(x_{n})_{n}$ en $(y_{n})_{n}$ twee convergente rijen.
  \[ \lim_{n \rightarrow \infty}(x_{n}y_{n}) = \lim_{n\rightarrow \infty}x_{n} \cdot \lim_{n\rightarrow \infty}y_{n} \]
\TODO{bewijs: p 40}
\end{st}

\begin{st}
  Zij $(x_{n})_{n}$ en $(y_{n})_{n}$ twee convergente rijen zodat $\forall n\in \mathbb{N}:\ y_{n}\neq 0$ geldt
  \[ \lim_{n \rightarrow \infty}\left(\frac{x_{n}}{y_{n}}\right) = \frac{\lim_{n\rightarrow \infty}x_{n}}{\lim_{n\rightarrow \infty}y_{n}} \]
\TODO{bewijs: p 40}
\end{st}

\begin{st}
  We kunnen bovenstaande stellingen uitbreiden om te gelden over $F\cup\{ -\infty,+\infty\}$ als we de volgende rekenregels toevoegen.
  \[
  \begin{array}{rccccl}
                             & (+\infty) &+    & (+\infty) &= +\infty\\
                             & (-\infty) &+    & (-\infty) &= -\infty\\
                             & (+\infty) &+    & (-\infty) & & \text{ is onbepaald.} \\
                             & (-\infty) &+    & (+\infty) & & \text{ is onbepaald.} \\\\

    \forall a \in F:\        & a         &+    & (+\infty) &= + \infty \\
    \forall a \in F:\        & (+\infty) &+    & a         &= + \infty \\
    \forall a \in F:\        & a         &+    & (-\infty) &= - \infty \\
    \forall a \in F:\        & (-\infty) &+    & a         &= - \infty \\\\

                             & (+\infty) &\cdot& (+\infty) &= +\infty\\
                             & (-\infty) &\cdot& (-\infty) &= +\infty\\
                             & (+\infty) &\cdot& (-\infty) &= -\infty\\
                             & (-\infty) &\cdot& (+\infty) &= -\infty\\\\

                             & 0         &\cdot& (+\infty) & & \text{ is onbepaald.} \\
                             & 0         &\cdot& (-\infty) & & \text{ is onbepaald.} \\
                             & (+\infty) &\cdot& 0         & & \text{ is onbepaald.} \\
                             & (-\infty) &\cdot& 0         & & \text{ is onbepaald.} \\
    \forall a \in F_{0}^{+}:\ & a         &\cdot& (+\infty) &= + \infty \\
    \forall a \in F_{0}^{+}:\ & (+\infty) &\cdot& a         &= + \infty \\
    \forall a \in F_{0}^{+}:\ & a         &\cdot& (-\infty) &= - \infty \\
    \forall a \in F_{0}^{+}:\ & (-\infty) &\cdot& a         &= - \infty \\\\

    \forall a \in F_{0}^{-}:\ & a         &\cdot& (+\infty) &= + \infty \\
    \forall a \in F_{0}^{-}:\ & (+\infty) &\cdot& a         &= + \infty \\
    \forall a \in F_{0}^{-}:\ & a         &\cdot& (-\infty) &= - \infty \\
    \forall a \in F_{0}^{-}:\ & (-\infty) &\cdot& a         &= - \infty \\\\

    \forall a \in F:\        & \frac{a}{+\infty}      &= 0 \\
    \forall a \in F:\        & \frac{a}{-\infty}      &= 0 \\
    \forall a \in F_{0}^{+}:\ & \frac{+\infty}{a}      &= +\infty \\
    \forall a \in F_{0}^{+}:\ & \frac{-\infty}{a}      &= -\infty \\
    \forall a \in F_{0}^{-}:\ & \frac{+\infty}{a}      &= -\infty \\
    \forall a \in F_{0}^{-}:\ & \frac{-\infty}{a}      &= +\infty \\
                             & \frac{+\infty}{-\infty} &&&& \text{ is onbepaald.}\\
                             & \frac{+\infty}{+\infty} &&&& \text{ is onbepaald.}\\
                             & \frac{-\infty}{+\infty} &&&& \text{ is onbepaald.}\\
                             & \frac{-\infty}{-\infty} &&&& \text{ is onbepaald.}\\\\
                             & \frac{0}{0}             &&&& \text{ is onbepaald.}\\
  \end{array}
  \]
\extra{dit kan mooier?}
\TODO{bewiqs p 42}
\end{st}

\subsection{Deelrijen}
\label{sec:deelrijen}

\begin{de}
  Zij $(x_{n})_{n}$ een rij over een verzameling $V$ en $(n_{k})_{k}$ een strikt stijgende rij over $\mathbb{N}$, dan is de rij $(x_{n_{k}})_{k}$ een \term{deelrij} van $(x_{n})_{n}$.
\end{de}

\begin{pr}
  Zij $(x_{n})_{n}$ een rij in $F$ met een limiet in $F\cup\{+\infty,-\infty\}$., dan heeft elke deelrij ervan dezelfde limiet.
\TODO{bewijs p 50}
\end{pr}

\begin{st}
  De \term{stelling van Bolzano-Weierstra\ss}\\
  Elke begrensde rij heeft een convergente deelrij.
\TODO{bewijs p 51}
\end{st}
\begin{opm}
  Deze stelling gaat niet op in $\mathbb{Q}$.
\extra{bewijs}
\end{opm}

\begin{de}
  Zij $(x_{n})_{n}$ een begrensde rij.
  We defini\"eren de \term{limes superior} of \term{lim sup} en de \term{limes inferior} of \term{lim inf} van de rij $(x_{n})_{n}$ als volgt:
  \[ \limsup_{n\rightarrow \infty} x_{n} = \lim_{n\rightarrow \infty} sup\{x_{k}\mid k\ge n\} \quad\text{ en }\quad \liminf_{n\rightarrow \infty} x_{n} = \lim_{n\rightarrow \infty} inf\{ x_{k}\mid k\ge n\} \]
\end{de}

\TODO{de volgende 4 stellingen werken ook voor onbegrensde rijen, pas de bewijzen aan.}
\begin{pr}
  Zij $(x_{n})_{n}$ een begrensde rij.
  \[ \liminf_{n\rightarrow \infty} x_{n} \le \limsup_{n\rightarrow \infty} x_{n} \]
\TODO{bewijs p 53}
\end{pr}

\begin{pr}
  Zij $(x_{n})_{n}$ een begrensde rij.
  Als en slechts als $(x_{n})_{n}$ convergeert geldt hetvolgende.
  \[ \limsup_{n\rightarrow \infty} x_{n} = \liminf_{n\rightarrow \infty} x_{n} \]
  De limiet van $(x_{n})_{n}$ is dan ook de gemeenschappelijke woorde van limsup en liminf.
\TODO{bewijs p 53}
\end{pr}

\begin{pr}
  Zij $(x_{n})_{n}$ een begrensde rij, dan bestaat er een deelrij van $(x_{n})_{n}$ die convergeert naar $\limsup_{n\rightarrow \infty} x_{n}$ en een deelrij die convergeert naar $\liminf_{n\rightarrow \infty} x_{n}$.

\TODO{bewijs p 53}
\end{pr}

\begin{pr}
  Zij $(x_{k})_{k}$ een convergente deelrij van een begrensde rij $(x_{n})_{n}$.
  \[ \liminf_{n\rightarrow \infty} x_{n} \le \lim_{n\rightarrow \infty} x_{n} \le \limsup_{n\rightarrow \infty} x_{n} \]

\TODO{bewijs p 53}
\end{pr}

\subsection{Cauchyrijen en de volledigheid van $\mathbb{R}$}
\label{sec:cauchyrijen-en-de}

\begin{de}
  We noemen een rij $(x_{n})_{n}$ over een totaal georden veld $F,+,\cdot,\le$ een \term{Cauchyrij} als en slechts als het volgende geldt:
  \[ \forall \epsilon \in F_{0}^{+},\ \exists n\in \mathbb{N},\ forall n,m \in \mathbb{N}_{0}:\ n,m \ge n_{0} \Rightarrow |x_{n}-x_{m}| < \epsilon \]
\end{de}

\begin{pr}
  Elke convergente rij is een Cauchyrij.
\TODO{bewijs p 56}
\end{pr}

\begin{pr}
  In $\mathbb{Q}$ bestaan er Cauchyren die niet convergeren (in $\mathbb{Q}$).
\TODO{bewijs p 57}
\end{pr}

\begin{pr}
  Elke Cauchyrij over een totaal geordend veld $F,+,\cdot,\le$ is begrensd.
\TODO{bewijs p 58}
\end{pr} 

\begin{pr}
  Een Cauchyrij over een totaal geordend veld $F,+,\cdot,\le$ met een convergente deelrij convergeert naar dezelfde limiet als die deelrij.
\TODO{bewijs p 59}
\end{pr}

\begin{pr}
  Elke Cauchyrij over een totaal geordend veld $F,+,\cdot,\le$ convergeert.
\end{pr}

\begin{de}
  We noemen een totaal geordend veld $F,+,\cdot,\le$ \term{volledig} als elke Cauchyrij in $F$ een limiet heeft in $F$.
\end{de}

\begin{st}
  $\mathbb{R}$ is volledig.
\extra{bewijs}
\end{st}

\extra{tekstje op blz 60 toch nog eens bekijken}

\section{Topologie in $\mathbb{R}$}
\label{sec:topologie-mathbbr}


\end{document}
