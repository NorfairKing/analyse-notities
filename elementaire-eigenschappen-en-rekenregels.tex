\documentclass[main.tex]{subfiles}
\begin{document}


\section{Elementaire eigenschappen en rekenregels}
\label{sec:elem-eigensch-en}

\begin{pr}
  Beschouw een functe $f:\ A \subseteq \mathbb{R} \rightarrow \mathbb{R}$ en een $a\in A$.
  Stel dat $a$ een ophopingspunt is van zowel $\interval[open left]{-\infty}{a} \cap A$ als $\interval[open right]{a}{+\infty} \cap A$.
  $f$ is dan afleidbaar in $a$ als en slechts $f$ zowel links- als rechtsafleidbaar is in $a$ en $f'(a^{-})=f'(a^{+})$.
  \[ f'(a) = f'(a^{-})=f'(a^{+}) \]
\TODO{bewijs: oefening}
\end{pr}

\begin{pr}
  \label{pr:afleidbaar-dan-continu}
  Beschouw een functie $f:\ A \subseteq \mathbb{R} \rightarrow \mathbb{R}$ en een $a\in A$ een ophopingspunt van $A$.
  Als $f$ afleidbaar is in $a$, dan is $f$ continu is $a$.

  \begin{proof}
    Zij $f$ een afleidbare functie.
    Om te bewijzen dat de limiet van $f$ in $a$ $f(a)$ is\prref{pr:functie-continu-asa-limiet-is-beeld} bewijzen we het volgende:
    \[  \lim_{x \rightarrow a}f(x)=\lim_{x \rightarrow a}f(a) \Leftrightarrow \lim_{x \rightarrow a}f(x)-\lim_{x \rightarrow a}f(a) = 0 \Leftrightarrow  \lim_{x \rightarrow a}\left(f(x)-f(a)\right) = 0 \]
    Deze limiet is eveneens gelijk aan de volgende:
    \[ = \lim_{x \rightarrow a}\left(f(x)-f(a)\right)\frac{x-a}{x-a} \]
    Vanwege de rekenregels voor limieten mogen we deze limiet splitsen.\prref{pr:rekenregels-limieten}
    \[ = \lim_{x \rightarrow a}\frac{f(x)-f(a)}{x-a}\lim_{x \rightarrow a}(x-a) \]
    Omdat $f$ afleidbaar is geldt dan het volgende:
    \[ = f'(a)\cdot 0 = 0 \]
  \end{proof}
\end{pr}

\begin{tvb}
  Als een functie $f:\ \mathbb{R} \rightarrow \mathbb{R}$ continu is in een punt $a$, betekent dit nog *niet* dat $f$ afleidbaar is in $a$.

  \begin{proof}
    Beschouw de functie $f:| \mathbb{R} \rightarrow \mathbb{R}^{+}: x \mapsto |x|$.
    $f$ is continu in $0$\vbref{vb:absolute-waardefunctie-continu}, maar niet afleidbaar in $0$.\tvbref{tvb:absolute-waarde-niet-afleidbaar-in-0}
  \end{proof}
\end{tvb}

\begin{st}
  Er bestaan functies van $\mathbb{R}$ naar $\mathbb{R}$ die overal continu maar nergens afleidbaar zijn.
\extra{bewijs p 10}
\end{st}

\begin{pr}
  Beschouw een functie $f: A \subseteq \mathbb{R} \rightarrow \mathbb{R}$ en een punt $a\in A$ dat een ophopingspunt is van $A$.
  Stel dat $f$ afleidbaar is en zij $\lambda \in \mathbb{R}$.
  \[ \lambda f \text{ is afleidbaar in } a \text{ en } (\lambda f)'(a) = \lambda f'(a) \]

  \begin{proof}
    We bewijzen dit in twee delen.
    \begin{itemize}
    \item $\lambda f$ is afleidbaar in $a$.
\extra{bewijs}
    \item $(\lambda f)'(a) = \lambda f'(a)$\\
      Zie hiervoor de rekenregels voor limieten.\prref{pr:rekenregels-limieten}
      \[ \lim_{x \rightarrow a}\frac{\lambda f(x)-\lambda f(a)}{x-a} \lim_{x \rightarrow a}\lambda \frac{\lambda f(x)-\lambda f(a)}{x-a} = \lambda \lim_{x \rightarrow a}\frac{f(x)-f(a)}{x-a} \]
    \end{itemize}
  \end{proof}
\end{pr}


\begin{pr}
  Beschouw functie $f,g: A \subseteq \mathbb{R} \rightarrow \mathbb{R}$ en een punt $a\in A$ dat een ophopingspunt is van $A$.
  Stel dat $f$ en $g$ afleidbaar zijn.
  \[ f+g \text{ is afleidbaar in } a \text{ en } (f+g)'(a) = f'(a) + g'(a) \]

  \begin{proof}
    We bewijzen dit in twee delen.
    \begin{itemize}
    \item $f+g$ is afleidbaar in $a$.
      \extra{bewijs}
    \item $(f+g)'(a) = f'(a) + g'(a)$\\
      Zie hiervoor de rekenregels voor limieten.\prref{pr:rekenregels-limieten}
      \[
      \begin{array}{rl}
        \lim_{x \rightarrow a}\frac{(f+g)(x)-(f+g)(a)}{x-a}
        &= \lim_{x \rightarrow a}\frac{f(x)+g(x)-f(a)-g(a)}{x-a}\\
        &= \lim_{x \rightarrow a}\frac{f(x)-f(a)+g(x)-g(a)}{x-a}\\
        &= \lim_{x \rightarrow a}\frac{f(x)-f(a)}{x-a}+\frac{g(x)-g(a)}{x-a}\\
        &= \lim_{x \rightarrow a}\frac{f(x)-f(a)}{x-a}+\lim_{x \rightarrow a}\frac{g(x)-g(a)}{x-a}\\
        &= f'(a) + g'(a)
      \end{array}
      \]
    \end{itemize}

  \end{proof}
\end{pr}

\begin{pr}
  Beschouw functie $f,g: A \subseteq \mathbb{R} \rightarrow \mathbb{R}$ en een punt $a\in A$ dat een ophopingspunt is van $A$.
  Stel dat $f$ en $g$ afleidbaar zijn.
  \[ fg \text{ is afleidbaar in } a \text{ en } (fg)'(a) = f(a) g'(a) + f'(a)g(a) \]

  \begin{proof}
    We bewijzen dit in twee delen.
    \begin{itemize}
    \item $fg$ is afleidbaar in $a$.
      \extra{bewijs}
    \item $(fg)'(a) = f(a) g'(a) + f'(a)g(a)$\\
      Zie hiervoor de rekenregels voor limieten.\prref{pr:rekenregels-limieten}
      \[
      \begin{array}{rl}
        \lim_{x \rightarrow a}\frac{(fg)(x)-(fg)(a)}{x-a}
        &= \lim_{x \rightarrow a}\frac{f(x)g(x)-f(a)g(a)}{x-a}\\
        &= \lim_{x \rightarrow a}\frac{f(x)g(x)-f(a)g(x)+f(a)g(x) -f(a)g(a)}{x-a}\\
        &= \lim_{x \rightarrow a}g(x)\frac{f(x)-f(a)}{x-a} + f(a)\frac{g(x) -g(a)}{x-a}\\
        &= \lim_{x \rightarrow a}g(x)\lim_{x \rightarrow a}\frac{f(x)-f(a)}{x-a} + f(a)\lim_{x \rightarrow a}\frac{g(x) -g(a)}{x-a}\\
        &= g(a)f'(a) + f(a)g'(a)
      \end{array}
      \]
    \end{itemize}
  \end{proof}
\end{pr}

\begin{pr}
  Beschouw functie $g: A \subseteq \mathbb{R} \rightarrow \mathbb{R}$ en een punt $a\in A$ dat een ophopingspunt is van $A$.
  Stel dat $g$ afleidbaar is, en $g(a) \neq 0$.
  \[ \frac{1}{g}:\ \{ x \in A \mid g(x) \neq 0\} \rightarrow \mathbb{R}: x \mapsto \frac{1}{g(x)} \]
  \[ \left(\frac{1}{g}\right)' \text{ is afleidbaar in } a \text{ en } \left(\frac{1}{g}\right)'(a) = -\frac{g'(a)}{g(a)^{2}} \]

  \begin{proof}
    We bewijzen dit in twee delen.
    \begin{itemize}
    \item $\frac{1}{g}$ is afleidbaar.\\
      Omdat $g$ afleidbaar is in $a$ is $g$ continu in $a$.\prref{pr:afleidbaar-dan-continu}
      Omdat $g(a)$ verschillend is van $0$ bestaat er dus een $\delta > 0$ zodat $g(x) \neq 0$ voor alle $x\in A$, dichter dan $\delta$ bij $a$.\waarom
      We vinden dus dat $A \cap \interval[open]{a-\delta}{a+\delta}$ een deel is van $A_{0} = \{ x \in A \mid g(x) \neq 0\}$.
      Omdat $a$ een ophopingspunt is volgt hier uit dat $a$ ook een ophopingspunt is van $A_{0}$.\waarom
      $0$ is dus een ophopingspunt van $(A_{0}-a) \setminus \{0\}= \{ x-a \mid x \in A, g(x) \neq 0\} \setminus \{0\}$.
      We beschouwen de limiet in $0$ van de functie $\Delta$:
      \[ \Delta:\ (A_{0}-a) \setminus \{0\} \rightarrow \mathbb{R}:\ h \mapsto \frac{1}{h}\left(\frac{1}{g(a+h)}-\frac{1}{g(a)} \right) \]
      Met behulp van de rekenregels voor limieten vinden we dan het volgende:
      \[
      \begin{array}{rl}
        \lim_{h \rightarrow 0}\Delta(h)
        &= \lim_{h \rightarrow 0}\frac{1}{h} \frac{g(a)-g(a+h)}{g(a+h)g(a)}\\
        &= -\lim_{h \rightarrow 0}\frac{g(a+h)-g(a)}{h} \cdot \lim_{h\rightarrow 0}\frac{1}{g(a+h)g(a)}\\
        &= -\frac{g'(a)}{g(a)^{2}}
      \end{array}
      \]
      \extra{bewijs meer uitleggen}
    \item $\left(\frac{1}{g}\right)'(a) = -\frac{g'(a)}{g(a)^{2}}$\\
      \[
      \begin{array}{rl}
        \lim_{x \rightarrow a}\frac{\left(\frac{1}{g}\right)(x)-\left(\frac{1}{g}\right)(a)}{x-a}
        &= \lim_{x \rightarrow a}\frac{\frac{1}{g(x)}-\frac{1}{g(a)} }{x-a}\\
        &= \lim_{x \rightarrow a}\frac{1}{(x-a)g(x)}-\frac{1}{(x-a)g(a)}\\
        &= \lim_{x \rightarrow a}\frac{g(a)-g(x)}{(x-a)g(a)g(x)}\\
        &= \lim_{x \rightarrow a}\frac{g(a)-g(x)}{(x-a)}\lim_{x\rightarrow a}\frac{1}{g(a)g(x)}\\
        &= -\frac{g'(a)}{g(a)^{2}}\\
      \end{array}
      \]
    \end{itemize}
  \end{proof}
\end{pr}

\begin{st}
  \label{st:kettingregel}
  De \term{kettingregel}\\
  Beschouw functies $f:\ A \subseteq \mathbb{R} \rightarrow B \subseteq \mathbb{R}$ en $g:\ B \subseteq \mathbb{R} \rightarrow \mathbb{R}$.
  Veronderstel dat $a\in A$ een ophopingspunt is van $A$ en dat $f$ afleidbaar is in $a$.
  Stel bovendien dat $f(a)$ een ophopingspunt is van $B$ en dat $g$ afleidbaar is in $f(a)$.
  $g\circ f$ is dan afleidbaar in $a$:
  \[ (g \circ f)'(a) = g'(f(a))f'(a) \]

  \begin{proof}
    Noem $h=g\circ f$.
    We moeten aantonen dat de volgende limiet bestaat en gelijk is aan $g'(f(a))f'(a)$.
    \[ \lim_{x\rightarrow a}\frac{h(x)-h(a)}{x-a} \]
    Noem $b=f(a)$ en beschouw de functie $phi: B \rightarrow \mathbb{R}$:
    \[
    \phi(y) =
    \left\{
      \begin{array}{cl}
        \frac{g(y)-g(b)}{y-b} & \text{ als } y\in B, y \neq b\\
        g'(x) &\text{ als } y=b\\
      \end{array}
    \right.
    \]
    \clarify{waarom deze functie nodig?}
    Omdat $g$ afleidbaar is in $b$ is de limiet van $\phi$ in $b$ gelijk aan de afgeleide van $g$ in $b$, wat het beeld is van $\phi$ in $b$:
    \[ \lim_{y\rightarrow b}\phi(y) = g'(b) = \phi(b) \]
    $\phi$ is dus continu in $b$.
    Omdat $f$ afleidbaar is in $a$, is $f$ continu in $a$.\prref{pr:afleidbaar-dan-continu}
    De functie $\phi \circ f: A \rightarrow \mathbb{R}$ is dus continu in $a$.
    Er geldt dan het volgende:
    \[ \lim_{x\rightarrow a}\phi(f(x)) = \phi(f(a)) = \phi(b) = g'(b) \]
    Voor elke $x\in a$ met een beeld verschillend van $f(a)$ geldt dan het volgende:
    \[ \phi(f(x)) = \frac{g(f(x)) - g(f(a))}{f(x)-f(a)} \]
    Omgevormd:
    \[ g(f(x)) - g(f(a)) = \phi(f(x))(f(x)-f(a)) \]
    Merk nu op dat de bovenstaande vergelijking ook geldt voor de $x\in A$ met $f(x)=f(a)$.
    \clarify{waarom?? is dit niet net het nut van de $\phi$ functie?}
    Bijgevolg geldt ze voor alle $x\in A$.
    Voor alle $x\in A$, verschillend van $a$, geldt nu het volgende:
    \[ 
    \frac{h(x)-h(a)}{x-a} = \frac{g(f(x))-g(f(a))}{x-a}  = \phi(f(x))\left(\frac{f(x)-f(a)}{x-a}\right)
    \]
    We weten nu dat de limieten voor $x$ gaande naar $a$ voor beide factoren bestaan en eindig zijn.\waarom
    De rekenregels van limieten leren ons dat de limiet voor $x$ gaande naar $a$ ook bestaat en er als volgt uitziet:
    \[ \lim_{x\rightarrow a}\frac{h(x)-h(a)}{x-a} = \lim_{x\rightarrow a}\phi(f(x))\lim_{x\rightarrow a}\left(\frac{f(x)-f(a)}{x-a}\right) = g'(f(a))f'(a) \]
    \extra{bewijs extra uitleggen}
  \end{proof}
\end{st}

\begin{st}
  Zij $f: A \subseteq \mathbb{R} \rightarrow \mathbb{R}$ een bijectie.
  Noteer de inverse bijectie met $g$.
  Zij $a \in A$ een ophopingspunt van $A$.
  Veronderstel dat $f$ afleidbaar is in $a$ en dat de afgeleide er niet nul is.
  Stel bovendien dat $g$ continu is in $f(a)$, dan is $f(a)$ een ophopingspunt van $B$ en is $g$ afleidbaar in $f(a)$:
  \[ g'(f(a)) = \frac{1}{f'(a)} \]

  \begin{proof}
    Bewijs in delen
    \begin{itemize}
    \item $f(a)$ is een ophopingspunt is van $B$.\\
      Kies een willekeurige $\eta \in \mathbb{R}_{0}^{+}$.
      Omdat $f$ afleidbaar is in $a$, is $f$ zeker continu in $a$.\prref{pr:afleidbaar-dan-continu}
      We kunnen dus een $\xi\in \mathbb{R}_{0}^{+}$ vinden zodat $f(x) \in \interval[open]{f(a)-\eta}{f(a)+\eta}$ geldt voor alle $x\in A$ die dichter dan $\xi$ bij $a$ liggen.
      Omdat $a$ een ophopingspunt is van $A$, kunnen we een $x_{0}\in A$ vinden, verschillend van $a$, die dichter dan $\xi$ bij $a$ ligt.
      Het beeld van $x_{0}$ onder $f$ zal dan in $\interval[open]{f(a)-\eta}{f(a)+\eta}\cap B$ liggen.
      Omdat $f$ injectief is zal $f(x_{0})$ verschillend zijn van $f(a)$.
      De volgende verzameling is dus niet leeg, wat bewijst dat $f(a)$ een ophopingspunt is van $B$.
      \[ \interval[open]{f(a)-\eta}{f(a)+\eta}\cap (B \setminus \{f(a)\}) \]
    \item $g$ is afleidbaar in $f(a)$.\\
      Kies een $x\in A\setminus \{a\}$ en noteer $y=f(x)$ en $b=f(a)$.
      \[ 
      \begin{array}{rl}
        \left| \frac{g(y)-g(b)}{y-b} - \frac{1}{f'(a)} \right|
        &= \left| \frac{x-a}{f(x)-f(a)} - \frac{1}{f'(a)} \right|\\
        &= \left| \frac{f'(a)(x-a)-(f(x)-f(a))}{(f(x)-f(a))f'(a)} \right|\\
        &= \left| \frac{1}{f'(a)}\right|\left| \frac{f'(a)(x-a)-(f(x)-f(a))}{f(x)-f(a)} \right|\\
        &= \left| \frac{1}{f'(a)}\right|\left|\frac{f'(a)(x-a)}{f(x)-f(a)} -1 \right|\\
        &= \left| \frac{1}{f'(a)} \right| \left| \frac{x-a}{f(x)-f(a)} \right| \left| f'(a)-\frac{(f(x)-f(a))}{x-a} \right|\\
      \end{array}
      \]
      Omdat $f$ afleidbaar is, kunnen we een $eta_{1}\in \mathbb{R}$ nemen zodat uit $0<|x-a|<\eta_{1}$ en $x\in A$ het volgende volgt: ($f'(a)$ is de limiet van de functie $\frac{f(x)-f(a)}{x-a}$.)
      \[ \left| \frac{f(x)-f(a)}{x-a} -f'(a)\right| < \frac{1}{2}|f'(a)| \]
      En dus:
      \[ \left| \frac{f(x)-f(a)}{x-a} \right| > \frac{1}{2}|f'(a)| \]
      Keren we dit om, dan krijgen we de volgende ongelijkheid:
      \[ \left| \frac{x-a}{f(x)-f(a)} \right| < \frac{2}{|f'(a)|} \]
      Kies nu een $\epsilon \in \mathbb{R}_{0}^{+}$. We gaan nu op zoek naar een $\delta\in\mathbb{R}_{0}^{+}$ zodat uit $0 <|y-b|< \delta$ en $y\in B$ het volgende volgt:
      \[ \left|\frac{g(y)-g(b)}{y-b} -\frac{1}{f'(a)} \right| < \epsilon\]
      Hieruit volgt dan de stelling.\waarom
      Kies nu een $\eta_{2}$ zodat uit $0<|x-a|< \eta_{2}$ en $x\in A$ het volgende volgt:
      \[ \left| \frac{f(x)-f(a)}{x-a} -f'(a) \right| < \epsilon\frac{|f'(a)|^{2}}{2} \]
      Omdat $g$ continu is in $b$, kunnen we een $\delta \in \mathbb{R}_{0}^{+}$ vinden zodat $|y-b| < \delta$ en $y\in B$ impliceert dat $|x-a|$ kleiner is dan $\min\{\eta_{1},\eta_{2}\}$.
      Er volgt nu uit $0<|x-a| <\min\{\eta_{1},\eta_{2}\}$ en $x\in A$ het volgende:
      \[
      \begin{array}{rl}
        \left| \frac{g(y)-g(b)}{y-b} - \frac{1}{f'(a)} \right|
        &= \left| \frac{1}{f'(a)} \right| \left| \frac{x-a}{f(x)-f(a)} \right| \left| f'(a)-\frac{(f(x)-f(a))}{x-a} \right|\\
        &< \left| \frac{1}{f'(a)} \right|\frac{2}{|f'(a)|}\epsilon\frac{|f'(a)|^{2}}{2}\\
        &= \epsilon
      \end{array}
      \]
\extra{bewijs verbeteren!!}
    \item $g'(f(a)) = \frac{1}{f'(a)}$\\
    \end{itemize}
  \end{proof}
\extra{opnieuw proberen bewijzen met propositie \ref{pr:limiet-van-functie-asa-limiet-van-beeld-van-rij}}
\end{st}


\end{document}

%%% Local Variables:
%%% mode: latex
%%% TeX-master: t
%%% End:
