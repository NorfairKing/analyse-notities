\documentclass[main.tex]{subfiles}
\begin{document}



\section{Het continu\"iteitsbegrip}
\label{sec:het-cont}

\begin{de}
  Zij $f$ een functie en $a\in A$:
  \[ f:\ A \subseteq \mathbb{R} \rightarrow \mathbb{R}:\ x \mapsto f(x) \]
  We noemen $f$ \term{continu} in $a$ als en slechts als het volgende geldt:
  \[ \forall \epsilon \in \mathbb{R}_{0}^{+}:\ \exists \delta \in \mathbb{R}_{0}^{+}:\ \forall x\in A:\ |x-a| < \delta \Rightarrow |f(x) -f(a)| < \epsilon \]
  We noemen $f$ \term{continu} op $A$ als $f$ continu is in elke $a\in A$.
\end{de}

\begin{vb}
  De functie $f$ is continu:
  \[ f:\ \mathbb{R} \rightarrow \mathbb{R}:\ x \mapsto x^{2} \]

  \begin{proof}
    Kies een willekeurige $a \in \mathbb{R}$.
    Voor elke $x\in \mathbb{R}$ geldt het volgende:
    \[ |f(x)-f(a)| = |x^{2}-a^{2}| = |x-a||x+a| \le |x-a|(|x|+|a|) \]
    Voor $|x-a|$ kleiner dan $1$ zal $|x| \le |a|+1$ gelden.
    \[ |f(x)-f(a)| \le (2|a|+1)|x-a| \]
    Kies een willekeurige $\epsilon \in \mathbb{R}_{0}^{+}$ en kies $\delta = \min\left\{ 1, \frac{\epsilon}{2|a|+1} \right\}$.
    Uit $|x-a|< \delta$ volgt nu het volgende.
    \[ |f(x)-f(a)| < (2|a|+1)\delta \le (2|a|+1)\frac{\epsilon}{2|a|+1} = \epsilon \]
    $f$ is dus continu in $a$.
  \end{proof}
\end{vb}

\begin{vb}
  De functie $f$ is continu:
  \[ f:\ \mathbb{R}_{0} \rightarrow \mathbb{R}:\ x \mapsto \frac{1}{x} \]
  
  \begin{proof}
    Kies een willekeurige $a \in \mathbb{R}_{0}$.
    Voor elke $x \in \mathbb{R}_{0}$ geldt het volgende:
    \[ |f(x)-f(a)| = \left|\frac{1}{x}-\frac{1}{a}\right| = \frac{|x-a|}{|x||a|} \]
    Voor $|x-a|$ kleiner dan $\frac{1}{2}|a|$ zal $|x| > \frac{1}{2}|a|$ gelden.
    \[ |f(x)-f(a)| \le \frac{2}{|a|^{2}}|x-a| \]
    Kies nu een willekeurige $\epsilon \in \mathbb{R}_{0}^{+}$ en kies $\delta = \min\left\{ \frac{1}{2}|a|, \frac{\epsilon|a|^{2}}{2} \right\}$.
    Uit $|x-a|< \delta$ volgt nu het volgende.
    \[ |f(x)-f(a)| < \frac{2}{|a|^{2}}\delta \le \frac{2}{|a|^{2}}\frac{\epsilon|a|^{2}}{2} = \epsilon \]
    $f$ is dus continu in elke $a\in \mathbb{R}_{0}$.
  \end{proof}
\end{vb}

\begin{vb}
  De functie $f$ is continu:
  \[ f:\ \mathbb{R} \setminus \{-1\} \rightarrow \mathbb{R}:\ x \mapsto \frac{1}{1+x} \]

  \begin{proof}
    Kies een willekeurige $a\in \mathbb{R}\setminus \{-1\}$.
    Voor elke $x\in \mathbb{R}\setminus \{-1\}$ geldt het volgende:
    \[ |f(x)-f(a)| = \left| \frac{1}{1+x}-\frac{1}{1+a} \right| = \left| \frac{(1+a)-(1+x)}{(1+x)(1+a)} \right|  = \frac{|x-a|}{|1+x||1+a|} \]
    Voor $|x-a|$ kleiner dan $|1+a|$ geldt nu het volgende:
    \[ |1+x| = |1+a-a+x| = |1+a+x-a| \le |1+a|+|x-a| \le 2|1+a| \]
    \[ |f(x)-f(a)| \le  \frac{1}{3|1+a|^{2}}|x-a| \]
    Kies nu een willekeurige $\epsilon \in \mathbb{R}_{0}^{+}$ en kies $\delta = \min\left\{ |1+a|,3\epsilon|1+a|^{2}\right\}$
    Voor $|x-a| < \delta$ geldt dan het volgende:
    \[ \frac{1}{3|1+a|^{2}}|x-a| < \frac{1}{3|1+a|^{2}}\delta \le \frac{3\epsilon|1+a|^{2}}{3|1+a|^{2}} = \epsilon\]
    $f$ is dus continu in elke $a \in \mathbb{R}\setminus \{-1\}$
  \end{proof}
\end{vb}

\begin{vb}
  De functie $f$ is continu:
  \[ f:\ \mathbb{R}^{+} \rightarrow \mathbb{R}:\ x \mapsto \sqrt{x} \]

  \begin{proof}
    Kies een willekeurige $a \in \mathbb{R}$.
    Voor elke $x \in \mathbb{R}$ geldt het volgende:
    \[ |f(x)-f(a)| = |\sqrt{x}-\sqrt{a}| = \left|\sqrt{x}-\sqrt{a}\right|\frac{\sqrt{x}+\sqrt{a}}{\sqrt{x}+\sqrt{a}} = \frac{|x-a|}{\sqrt{x}+\sqrt{a}} \]
    \begin{itemize}
    \item Voor $a>0$ kunnen we als volgt verder gaan :
      \[ \frac{|x-a|}{\sqrt{x}+\sqrt{a}} \le \frac{|x-a|}{\sqrt{a}} \]
      Voor een willekeurige $\epsilon$ kunnen we nu voor $\delta$ $\epsilon\sqrt{a}$ kiezen.
      Uit $|x-a|< \delta$ volgt dan het volgende:
      \[ |f(x)-f(a)| = \frac{|x-a|}{\sqrt{x}+\sqrt{a}} \le \frac{|x-a|}{\sqrt{a}} < \frac{\delta}{\sqrt{a}} = \frac{\epsilon\sqrt{a}}{\sqrt{a}} = \epsilon \]
    \item Voor $a=0$ is het eenvoudiger:
      \[ |f(x)-f(a)| =  \frac{|x-a|}{\sqrt{x}+\sqrt{a}} = \frac{|x|}{\sqrt{x}} \]
      Voor een willekeurige $\epsilon$ kunnen we nu voor $\delta$ $\epsilon^{2}$ kiezen. 
      Uit $|x-a|=|x|< \delta$ volgt dan het volgende:
      \[ \frac{|x|}{\sqrt{x}} < \frac{\delta}{\sqrt{\delta}} = \frac{\epsilon^{2}}{\epsilon} = \epsilon \]
    \end{itemize}
    $f$ is dus continu in $a$.
  \end{proof}
\end{vb}

\begin{vb}
  De functie $f$ is continu:
  \[ f:\ \mathbb{R} \rightarrow \mathbb{R}:\ x \mapsto \frac{1}{(1+x^{2})} \]

  \begin{proof}
    Kies een willekeurige $a\in \mathbb{R}$.
    Er geldt dan voor alle $x \in \mathbb{R}$ het volgende:
    \[
    |f(x)-f(a)|
    = \left| \frac{1}{1+x^{2}} - \frac{1}{1+a^{2}}\right|
    = \left| \frac{(1+a^{2})-(1+x^{2})}{(1+x^{2})(1+a^{2})}\right|
    = \frac{|a^{2}-x^{2}|}{|1+x^{2}||1+a^{2}|}
    = \frac{|a+x||a-x|}{|1+x^{2}||1+a^{2}|}
    \]
    Voor $|x-a| \le \frac{1}{2}|a|$ geldt $x \le \frac{3}{2}|a|$, $x \ge \frac{1}{2}|a|$ en dus het volgende:
    \[ |x+a| \ge ||x|-|a|| \ge |\frac{1}{2}|a| - |a|| = |-\frac{1}{2}|a|| = \frac{1}{2}|a| \]
    ,
    \[ |x+a| \le |x|+|a| \le \frac{3}{2}|a| + |a| = \frac{5}{2}|a| \]
    ... en bovendien het volgende:
    \[ |1+x^{2}| = 1 + |x^{2}| \le 1 + \left(\frac{3}{2}|a|\right)^{2} = 1 + \frac{9}{4}|a|^{2} \]
    \[
    |f(x)-f(a)|
    = \frac{|a+x||a-x|}{|1+x^{2}||1+a^{2}|}
    \le \frac{|a||a-x|}{2\left(1 + \frac{9}{4}|a|^{2}\right)|1+a^{2}|}
    \]
    Kies nu een willekeurige $\epsilon \in \mathbb{R}_{0}^{+}$ en kies $\delta = \min\{\frac{1}{2}|a|,\epsilon\frac{|1+a^{2}|2\left(1 + \frac{9}{4}|a|^{2}\right)}{|a|} \}$.
    Uit $|x-a| < \delta$ volgt dan het volgende:
    \[ 
    |f(x)-f(a)|
    \le \frac{|a||a-x|}{2\left(1 + \frac{9}{4}|a|^{2}\right)|1+a^{2}|}
    < \frac{|a|}{2\left(1 + \frac{9}{4}|a|^{2}\right)|1+a^{2}|}\delta
    \le \frac{\epsilon|a||1+a^{2}|2\left(1 + \frac{9}{4}|a|^{2}\right)}{2\left(1 + \frac{9}{4}|a|^{2}\right)|1+a^{2}||a|} = \epsilon
    \]
  \end{proof}
\feed
\end{vb}

\begin{vb}
  De functie $f:\ \mathbb{R} \rightarrow \mathbb{R}:\ x \mapsto \frac{x}{1+x^{2}}$ is continu.

  \extra{bewijs p II.4 onderaan.}
\end{vb}

\begin{tvb}
  De functie $f$, als volgt, is \textbf{niet} continu in $0$:
  \[
  f:\ \mathbb{R} \rightarrow \mathbb{R}:\ x\mapsto 
  \left\{
    \begin{array}{cl}
      1 & \text{ als } x \ge 0\\
      0 & \text{ als } x < 0
    \end{array}
  \right.
  \]

  \begin{proof}
    Kies $\epsilon = 1$, dan bestaat er voor elke $\delta$ een $x\in \mathbb{R}$ zodat $|x-0|<\delta$ en $|f(x)-f(0)|\ge \epsilon$ beide gelden.
    Voor een willekeurige $\delta$ kiezen we $x = -\frac{\delta}{2}$:
    \[ |x-0| = \frac{\delta}{2} \text{ en } |f(x)-f(0)| = 1 \ge \epsilon \]
    $f$ is dus niet continu in $0$.
  \end{proof}
\end{tvb}

\begin{tvb}
  De functie $f$, als volgt, is \textbf{niet} continu in $0$.
  \[
  f:\  \mathbb{R} \rightarrow \mathbb{R}:\ x \mapsto 
  \begin{cases}
    \sin\left(\frac{1}{x}\right) &\text{ als } x \neq 0\\
    0 &\text{ als } x = 0
  \end{cases}
  \]

  \begin{proof}
  Kies $\epsilon = 1$ en kies een willekeurige $\delta \in \mathbb{R}_{0}^{+}$.
  Kies $k$ voldoende groot zodat $\left(2k+\frac{1}{2}\right)\pi > \frac{1}{\delta}$ geldt en stel $x = \frac{1}{\left(2k+\frac{1}{2}\right)\pi}$.
  Dan geldt:
  \[ |x-0| < \delta \text{ en } |f(x)-f(0)| = 1 \ge \epsilon \]
  \end{proof}
\end{tvb}

\begin{vb}
  De absolute waardefunctie is continu:
  \[ f:\ \mathbb{R} \rightarrow \mathbb{R}:\ x \mapsto |x| \]

  \begin{proof}
    Kies een willekeurige $a\in \mathbb{R}$:
    Beschouw eerst $|f(x)-f(a)|$:
    \[ |f(x)-f(a)| = \left||x|-|a|\right| \le |x-a| \]
    Kies nu een willekeurige $\epsilon$ en kies $\delta = \epsilon$.
    Uit $|x-a| <\delta$ volgt nu het volgende:
    \[ |f(x)-f(a)| \le |x-a| < \delta = \epsilon \]
    $f$ is dus continu in $a$.
  \end{proof}
\end{vb}

\begin{vb}
  De functie $(\in \mathbb{Q})$ is nergens continu:
  \[
  (\in \mathbb{Q}):\ \mathbb{R} \rightarrow \mathbb{R}:\ 
  \left\{
    \begin{array}{rl}
      1 &\text{ als } x\in \mathbb{Q}\\
      0 &\text{ als } x\in \mathbb{R}\setminus \mathbb{Q}
    \end{array}
  \right.
  \]
  
  \begin{proof}
    Kies een willekeurige $a \in \mathbb{R}$.
    Gevalsonderscheid:
    \begin{itemize}
    \item $a\in \mathbb{Q}$
      Kies $\epsilon = \frac{1}{2}$, dan bestaat er voor elke $\delta \in \mathbb{R}_{0}^{+}$ een $x\in \mathbb{R}$ zodat uit $|x-a| < \delta$ $|((x \in \mathbb{Q})-(a\in \mathbb{Q})| \ge \epsilon$ volgt:
      Kies immers een willekeurige $\delta \in \mathbb{R}_{0}^{+}$, dan bestaat er een $x\in \mathbb{R}\setminus \mathbb{Q}$ met $|x-a| < \delta$. \needed
      Hiervoor geldt dan $|((x \in \mathbb{Q})-(a\in \mathbb{Q})| = 1 \ge \frac{1}{2} = \epsilon$.
    \item $a\in \mathbb{R}\setminus \mathbb{Q}$
      \extra{bewijs analoog}
    \end{itemize}
  \end{proof}
\end{vb}

\begin{vb}
  De volgende functie is negrens continu, maar de absolute waarde ervan is overal continu:
  \[ 
  f:\ \mathbb{R} \rightarrow \mathbb{R}:\ x \mapsto
  \begin{cases}
    1  &\text{ als } x\in \mathbb{Q}\\
    -1 &\text{ als } x\in \mathbb{R}\setminus \mathbb{Q}
  \end{cases}
  \]
\extra{bewijs}
\end{vb}

\begin{vb}
  De volgende functie is enkel continu in $0$.
  \[ 
  f:\ \mathbb{R} \rightarrow \mathbb{R}:\ x \mapsto
  \begin{cases}
    x  &\text{ als } x\in \mathbb{Q}\\
    -x &\text{ als } x\in \mathbb{R}\setminus \mathbb{Q}
  \end{cases}
  \]
\extra{bewijs}
\end{vb}

\begin{vb}
  (Bewijs via karakterisatie van continu\"iteit in termen van rijen.)
  De functie $f$ is continu:
  \[ f:\ \mathbb{R} \rightarrow \mathbb{R}: x \mapsto x^{2} \]
  
  \begin{proof}
    We moeten bewijzen dat voor een willekeurige rij $(x_{n})_{n}$ die naar $x$ convergeert, $(f(x_{n}))_{n}$ naar $f(x)$ convergeert.
    Kies daartoe een willekeurige $(x_{n})_{n}$ die naar $x$ convergeert.
    Voor een willekeurige $x_{n}$ uit de rij geldt het volgende:
    \[ |f(x_{n})-f(x)| = |x_{n}^{2}-x^{2}| = |x_{n}+x||x_{n}-x| \]
    Kies een willekeurige $\epsilon \in \mathbb{R}_{0}^{+}$ en kies $\delta = \frac{\epsilon}{|x_{n}+x|}$
    Omdat $(x_{n})_{n}$ naar $x$ convergeert, bestaat er dan een $m\in \mathbb{N}$ zodat het volgende geldt:
    \[ \forall n\in \mathbb{N}: n \ge m:\ |x_{n}-x| < \frac{\epsilon}{|x_{n}+x|} \]
    Voor $n \ge m$ geldt dus het volgende:
    \[ |f(x_{n})-f(x)| = |x_{n}+x||x_{n}-x| < |x_{n}+x|\frac{\epsilon}{|x_{n}+x|} = \epsilon \]
    De rij $(f(x_{n}))_{n}$ convergeert dus naar $f(x)$.
  \end{proof}
\end{vb}

\begin{vb}
  Zij $f: A \subseteq \mathbb{R} \rightarrow \mathbb{R}$ een functie die continu is in een punt $a\in A$ met $f(a) > 0$.
  Er bestaat dan een $\delta \in \mathbb{R}_{0}^{+}$ zodat $f(x)>0$ geldt voor alle $x\in A$ met $|x-a| < \delta$.

  \begin{proof}
    Kies $\epsilon = f(a)$, dan bestaat er een $\delta \in \mathbb{R}_{0}^{+}$ zodat de stelling geldt omdat $f$ continu is in $A$.
  \end{proof}
\end{vb}

\begin{st}
  Equivalente definitie van \term{continu\"iteit}:\\
  $f$ is continu in $a$ als en slechts als het volgende geldt:
  \[ \forall \epsilon \in \mathbb{R}_{0}^{+}:\ \exists \delta \in \mathbb{R}_{0}^{+}:\ \forall x\in A:\ f(\interval[open]{a-\delta}{a+\delta} \cap A) \subseteq \interval[open]{f(a)-\epsilon}{f(a)+\epsilon} \]
\extra{bewijs}
\end{st}


\begin{bpr}
  De \term{karakterisering van continu\"iteit in termen van rijen}.\\
  \label{pr:continu-asa-behoudt-convergentie}
  Zij $f:\ A \subseteq \mathbb{R} \rightarrow \mathbb{R}$ een functie en $a\in A$.
  $f$ is continu in $a$ als en slechts als er voor elke rij $(x_{n})_{n}$ in $A$ die naar $a$ convergeert geldt dat $(f(x_{n}))_{n}$ naar $f(a)$ convergeert.

  \begin{proof}
    Bewijs van een equivalentie.
    \begin{itemize}
    \item $\Rightarrow$\\
      Kies een willekeurige rij $(x_{n})_{n}$ in $A$ die naar $a$ convergeert.
      We moeten bewijzen dat $(f(x_{n}))_{n}$ naar $f(a)$ convergeert.
      Kies daartoe een willekeurige $\epsilon \in \mathbb{R}_{0}^{+}$.
      Omdat $f$ continu is in $a$ kunnen we een $\delta\in \mathbb{R}_{0}^{+}$ vinden zodat $|f(x)-f(a)| < \epsilon$ geldt voor alle $x\in A$ met $|x-a| < \delta$.
      Omdat $(x_{n})_{n}$ naar $a$ convergeert, kunnen we een $n_{0}\in \mathbb{N}$ vinden zodat voor alle volgende $n\in\mathbb{N}$ de afstandvan $x_{n}$ tot $a$ kleiner is dan $\delta$.
      Voor elke $n\ge n_{0}$ zal dan $|f(x_{n})-f(a)| < \epsilon$ gelden.
      De rij $(f(x_{n}))_{n}$ convergeert dus naar $f(a)$.
    \item $\Leftarrow$\\
      Contrapositie: Als $f$ niet continu is in $a$ zal er een rij in $A$ bestaan die naar een $a$ convergeert waarvoor $(f(x_{n}))_{n}$ niet naar $f(a)$ convergeert.
      Omdat $f$ niet continu is bestaat er een $\epsilon \in \mathbb{R}_{0}^{+}$ zodat er voor elke $\delta\in \mathbb{R}_{0}^{+}$ een $x$ kan genomen worden zodat het volgende geldt:
      \[ |x-a| < \delta \quad\wedge\quad |f(x)-f(a)| \ge \epsilon \]
      We kunnen dan een rij $(x_{n})_{n}$ in $A$ construeren zodat voor alle $n\in\mathbb{N}_{0}$ $x_{n}$ als volgt gekozen is:
      \[ |x_{n}-a| < \frac{1}{n} \quad\wedge\quad |f(x_{n})-f(a)| \ge \epsilon \]
      Uit de eerste ongelijkheid volgt dat $x_{n}$ naar $a$ convergeert terwijl uit de tweede ongelijkheid volgt dat $(f(x_{n}))_{n}$ niet kan convergeren naar $f(a)$.
    \end{itemize}
  \end{proof}
\end{bpr}

\begin{bpr}
  De \term{karakterisering van continu\"iteit in termen van opens}\\
  Zij $f:\ A \subseteq \mathbb{R} \rightarrow \mathbb{R}$ een functie.
  $f$ is continu op $A$ als en slechts als er voor elk open deel $V$ van $\mathbb{R}$ geldt dat $f^{-1}(V)$ relatief open is in $A$.
  \begin{proof}
    Bewijs van een equivalentie.
    \begin{itemize}
    \item $\Rightarrow$\\
      Kies een willekeurig open deel $V$ van $\mathbb{R}$.
      Als $f^{-1}(V)$ leeg is, is $f^{-1}(V)$ triviaal open.
      Als $f^{-1}(V)$ niet leeg is, dan bestaat er een $a\in f^{-1}(V)$.
      $f(a)$ zit dan in $V$ en omdat $V$ open is kunnen we een $\epsilon$ vinden zodat $\interval[open]{f(a)-\epsilon}{f(a)+\epsilon}$ een deel is van $V$.
      Omdat $f$ continu is in $a$ kunnen we een $\delta\in \mathbb{R}_{0}^{+}$ vinden zodat uit $x\in A$ en $|x-a| < \delta$ volgt dat $f(x) \in \interval[open]{f(a)-\epsilon}{f(a)+\epsilon}$
      Dit betekent dat $\interval[open]{a-\delta}{a+\delta} \cap A \subseteq f^{-1}(V)$ geldt en dus dat $f^{-1}(V)$ relatief open is in $A$.
    \item $\Leftarrow$\\
      Stel dat er voor elk open deel $V$ van $\mathbb{R}$ geldt dat $f^{-1}(V)$ relatief open is in $A$.
      Kies nu een willekeurige $a\in A$.
      We bewijzen dat $f$ continu is in $a$.
      Beschouw daarvoor de verzameling $U$ voor een willekeurige $\epsilon \in \mathbb{R}_{0}^{+}$:
      \[ U = f^{-1}(\interval[open]{f(a)-\epsilon}{f(a)+\epsilon}) \]
      $U$ bevat zeker $a$ en is bovendien relatief open in $A$.
      Er bestaat dus een $\delta \in \mathbb{R}_{0}^{+}$ zodat uit $x\in A$ en $|x-a|< \delta$ volgt dat $x$ tot $U$ behoort.
      Kies dan een willekeurige $x\in A$ met $|x-a|< \delta$, opdat $x$ tot $U$ behoort.
      $f(x)$ behoort dan ook tot $f(U) = \interval[open]{f(a)-\epsilon}{f(a)+\epsilon}$.
      Er geldt met andere woorden $|f(x)-f(a)|<\epsilon$, dus $f$ is continu in $a$.
    \end{itemize}
  \end{proof}
\end{bpr}
\extra{zelfde stelling voor gesloten: tegenvoorbeeld!}
\extra{I gesloten dan F(I) gesloten?}

\begin{tvb}
  Zij $f:\ A \subseteq \mathbb{R} \rightarrow \mathbb{R}$ een continue functie en $V$ een open deel van $\mathbb{R}$, dan hoeft $f(V)$ niet open te zijn.
  
  \begin{proof}
    Zij $f$ de absolute-waardefunctie:
    \[ f:\ \mathbb{R} \rightarrow \mathbb{R}: x \mapsto |x| \]
    Kies $V = \interval[open]{-1}{1}$, dan is $f(V)$ het halfopen interval $\interval[open right]{0}{1}$ en dus niet open.
  \end{proof}
\end{tvb}

\begin{de}
  Zij $f:\ A \subseteq \mathbb{R} \rightarrow \mathbb{R}$ een functie en $a\in A$.
  We noemen $f$ ...
  \begin{itemize}
  \item ...\term{linkscontinu} in $a$ als en slechts als de beperking van $f$ tot $\interval[open left]{-\infty}{a} \cap A$ continu is in $a$.
  \item ...\term{rechtscontinu} in $a$ als en slechts als de beperking van $f$ tot $\interval[open right]{a}{+\infty} \cap A$ continu is in $a$.
  \end{itemize}
\end{de}

\begin{st}
  \examenvraag{TTT 1 2015}
  Zij $f,g:\ \mathbb{R} \rightarrow \mathbb{R}$ twee continue functies zodat $\forall q\in \mathbb{Q}: f(q) = g(q)$ geldt, dan zijn $f$ en $g$ gelijk.

  \begin{proof}
    Voor elk getal $c \in \mathbb{R}$ bestaat er een Cauchyrij $(x_{n})_{n}$ van rationale getallen met $c$ als limiet.\waarom
    \[ \lim_{n\rightarrow \infty}x_{n} = c\]
    Omdat $f$ en $g$ continu zijn moet ook het volgende gelden:
    \[ \lim_{n\rightarrow \infty}f(x_{n}) = f(c) \text{ en } \lim_{n\rightarrow \infty}g(x_{n}) = g(c) \]
    Omdat $x_{n}$ rationaal is, geldt voor elke $n\in \mathbb{N}$ $f(x_{n})=g(x_{n})$.
    De limieten moeten dan gelijk zijn\waarom, en dus moet $f(c)$ gelijk zijn aan $g(c)$ voor alle $c\in \mathbb{R}$.
  \end{proof}
\end{st}

\begin{st}
  Zij $f:\ \mathbb{R} \rightarrow \mathbb{R}$ functie.
  $f$ is continu als en slechts het volgende geldt:
  \[ \forall X \subseteq \mathbb{R}:\ f^{-1}(\mathring{X}) \subseteq \mathring{\left(f^{-1}(X)\right)} \]
\extra{oefening}
\end{st}

\begin{st}
  Zij $f:\ \mathbb{R} \rightarrow \mathbb{R}$ functie.
  $f$ is continu als en slechts het volgende geldt:
  \[ \forall X \subseteq \mathbb{R}:\ \overline{f^{-1}(X)} \subseteq f^{-1}(\overline{X}) \]
\extra{oefening}
\end{st}


\end{document}
