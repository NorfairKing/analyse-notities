\documentclass[main.tex]{subfiles}
\begin{document}


\section{Basisbegrippen}
\label{sec:basisbegrippen}

\begin{de}
  Beschouw een functie $f: A \subseteq \mathbb{R} \rightarrow \mathbb{R}$ en een punt $a\in A$ dat ook een ophopingspunt is van $A$.
  We zeggen dat $f$ \term{afleidbaar} is in $a$ als de volgende limiet bestaat en eindig is:
  \[ L = \lim_{h\rightarrow 0}\frac{f(a+h)-f(a)}{h} \]
  We noemen $L$ de \term{afgeleide} van $f$ in $a$ en noteren $L$ als $f'(a)$.
  Als alle punten van $A$ ophopingspunten zijn van $A$ noemen we $f$ \term{afleidbaar} over $A$ als $f$ afleidbaar is in elke $x\in A$.
  In dit geval definieren we de \term{afgeleide functie} $f$ als volgt:
  \[ f':\ A \subseteq \mathbb{R} \rightarrow \mathbb{R}:\ x \mapsto f'(x) \]
\end{de}

\begin{st}
  De afgeleide van een functie in een punt is uniek (omdat dat punt een ophopingspunt is).
\extra{bewijs}
\end{st}

\TODO{equivalente definite voor afgeleide zonder $h$}

\begin{de}
  We schrijven $f'$ soms als $Df$ en $f'(x)$ soms als $\frac{df(x)}{dx}$
\end{de}

\subsection{Meetkundige betekenis}
\label{sec:meetk-betek}

\begin{de}
  Een \term{koorde} is een rechte tussen twee punten $(a,f(a))$ en $(a+h,f(a+h))$ op de grafiek van een functie.
  \[ y = \frac{f(a+h) -f(a)}{h} (x-a) + f(a) \]
\end{de}

\begin{de}
  De \term{raaklijn} aan een grafiek in een punt is de limiet van de koorde in dat punt van $h$ gaande naar nul.
  \[ y = \lim_{h\rightarrow 0}\frac{f(a+h) -f(a)}{h} (x-a) + f(a) \]
  We noemen de raaklijn de \term{eerste orde benadering} of de \term{standaard eerstegraadsbenadering} van $f$ rond $a$.
\end{de}

\begin{de}
  Beschouw een functie $f:\ A \subseteq \mathbb{R} \rightarrow \mathbb{R}$ en een punt $a \in A$.
  \begin{itemize}
  \item Als $a$ een ophopingspunt is van $A \cap \interval[open left]{-\infty}{a}$ en de beperking van $f$ tot $A \cap \interval[open left]{-\infty}{a}$ afleidbaar is in $a$, dan noemen we $f$ \term{linksafleidbaar} in $a$.
    We noemen de afgeleide van de beperkte functie in $a$ de \term{linkerafgeleide} $f'(a^{-})$ van $f$ in $a$.
  \item Als $a$ een ophopingspunt is van $A \cap \interval[open right]{a}{+\infty}$en de beperking van $f$ tot $A \cap \interval[open left]{a}{+\infty}$ afleidbaar is in $a$, dan noemen we $f$ \term{rechtsafleidbaar} in $a$.
    We noemen de afgeleide van de beperkte functie in $a$ de \term{rechterafgeleide} $f'(a^{+})$ van $f$ in $a$.
  \end{itemize}
\end{de}

\begin{st}
  $f$ is links-, respectievelijk rechtsafleidbaar in $a$ als en slechts de volgende limiet bestaat en eindig is.
  \[ \lim_{h\overset{<}{\rightarrow} 0}\frac{f(a+h)-f(a)}{h} \quad\text{en}\quad \lim_{h\overset{>}{\rightarrow} 0}\frac{f(a+h)-f(a)}{h} \]
\extra{bewijs}
\end{st}


\end{document}

%%% Local Variables:
%%% mode: latex
%%% TeX-master: t
%%% End:
