\documentclass[main.tex]{subfiles}
\begin{document}

\chapter{Voorkennis}
\label{cha:voorkennis}

\section{Orderelaties}
\label{sec:orderelaties}


\begin{de}
  Een relatie $R$ op een verzameling $X$ is \term{anti-symmetrisch} als het volgende geldt:
  \[ \forall x,y \in X: ((x,y) \in R \wedge (y,x) \in R) \Rightarrow x = y \]
\end{de}

\begin{de}
  Een (parti\"ele) \term{orderelatie} op $X$ is reflexief, transitief en anti-symmetrisch.
\end{de}

\begin{de}
  Een \term{grootste element} $a$ van een verzameling $A$ waarop een orderelatie $\prec$ is gedefinieerd is, is het element waarvoor geldt dan alle andere elementen kleiner zijn of gelijk aan $a$.
  \[ \forall x \in A: x \preceq a \] 
  Analoog wordt ook een \term{kleinste element} gedefinieerd.
\end{de}

\begin{de}
  Een \term{minimaal element} $a$ van $A$ waarop een orderelatie $\prec$ is gedefinieerd is, is het element waarvoor geldt dat er geen kleiner bestaat.
  \[ \not\exists x \in A: a \prec x \]
  Analoog wordt ook een \term{maximaal element} gedefinieerd.
\end{de}

\begin{opm}
  Een maximaal/minimaal element is niet noodzakelijk een grootste/kleinste element.    
\end{opm}

\begin{st}
  Een grootste/kleinste element is ook een maximaal/minimaal element.
\extra{bewijs}
\end{st}

\begin{de}
  Zij $(X,\preceq)$ een geordende verzameling en $A \subsetneq X$.
  $b \in X$ is een \term{bovengrens} van $A$ als het volgende geldt.
  \[ \forall x \in A: x \preceq b \]
  Analoog wordt een \term{ondergrens} gedefinieerd.
\end{de}

\begin{de}
  Een verzameling noemen we \term{naar boven begrensd} als ze een bovengrens is, \term{naar onder begrensd} als ze een ondergrens heeft en \term{begrensd} als ze \'e\'en van beide heeft.
\end{de}

\begin{opm}
  Een grens van een ordeverzameling hoeft dus niet in die verzameling te zitten.
\end{opm}

\begin{de}
  Een \term{supremum}(\term{infimum}) van een deelverzameling van een geordende verzameling is een bovengrens(ondergrens) die kleiner(groter) is dan elke andere bovengrens(ondergrens).
\end{de}

\begin{opm}
  Een supremum/infimum is een grens van een ordeverzameling en hoeft dus niet in die verzameling te zitten.
\end{opm}

\begin{st}
  Zij $A$ een partieel geordende verzameling met orderelatie $\prec$.
  Het kleinste/grootste element element van $A$ is uniek als het bestaat.
\extra{bewijs}
\end{st}

\begin{st}
  Zij $A$ een partieel geordende verzameling met orderelatie $\prec$.
  Het supremum/infimum van $A$ is uniek als het bestaat.
\extra{bewijs}
\end{st}

\begin{st}
  \label{st:deelverzameling-kleiner-supremum}
  Zij $(X,\preceq)$ een totaal geordende verzameling en $A \subsetneq X$.
  \[ sup X \ge sup A \quad\text{ en }\quad inf X \le inf A \]
  \TODO{bewijs: oefening}
\end{st}

\begin{de}
  Een \term{totale orderelatie} $\preceq$ is een partiele orderelatie met bijkomend de volgende eigenschap:
  \[ \forall x,y \in X: x \preceq y \vee y \preceq x \]
  Voor elke twee elementen zijn er dus precies drie mogelijkheden:
  \begin{itemize}
  \item $x \prec y$
  \item $x = y$
  \item $y \prec x$
  \end{itemize}
\end{de}

\begin{de}
  Zij $A$ een verzameling die volledig geordend is door de relatie $\prec$, dan noemen we $succ$ de successorfunctie als die gedefinieerd kan worden.
  \[ succ(x) = y \Leftrightarrow x < y \wedge (\not\exists z \in A:\ x < z < y \]
\end{de}

\begin{opm}
  De successorfunctie kan niet altijd gedefinieerd worden.
  Denk bijvoorbeeld aan de volgende volledige orderelatie over $\mathbb{Z}$:
  \[ |\le|:\ \mathbb{Z} \times \mathbb{Z}:\ x\ |\le|\ y \Leftrightarrow |x| \le |y| \]
\end{opm}

\begin{de}
  Wanneer we het symbool $\le$ gebruiken voor een totale orderelatie gebruiken we vaak de volgende afkortingen:
  \begin{itemize}
  \item '$a \ge b$' = '$b \le a$'
  \item '$a < b$' = '$\neg(b \le a)$'
  \item '$a > b$' = '$\neg(a \le b)$'
  \end{itemize}
\end{de}

\section{Functies}
\label{sec:functies}

\TODO{meer over functies?}

\begin{de}
  Een uitbreiding $g:A' \rightarrow B$ van een functie $f:A \rightarrow B$ (met $A \subseteq A'$) is een functie zodat de beperking van $g$ tot $A$ gelijk is aan $f$.
\end{de}


\end{document}
