\documentclass[main.tex]{subfiles}
\begin{document}


\section{Hogere orde afgeleiden}
\label{sec:hogere-orde-afgel}

\begin{de}
  Beschouw een functie $f:\ A \subseteq \mathbb{R} \rightarrow \mathbb{R}$.
  Stel dat $A$ enkel ophopingspunten bevat.
  Noteer de verzamelingen van punten van $A$ waarin $f$ afleidbaar is als $A_{1}$.
  Noteer bovendien met $f':\ A_{1} \rightarrow \mathbb{R}:\ x \mapsto f'(x)$ de afgeleide functie van $f$.

  Zij $a$ een element van $A$, dan noemen we $f$ \term{twee maal afleidbaar} in $a$ als en slechts als $a$ tot $A_{1}$ behoort, een ophopingspunt is van $A_{1}$ en de afgeleide functie $f'$ afleidbaar is in $a$.
  We noteren de \term{tweede orde afgeleide} functie $(f')'$ van $f$ als $f''$.
  We noemen $f$ \term{twee maal afleidbaar} op $A$ als $f$ twee maal afleidbaar is in elke $a\in A$.
\end{de}

\begin{st}
  \label{st:berekening-tweede-afgeleide}
  Als de tweede afgeleide van een functie $f$ in een punt $a$ bestaat, dan kunnen we ze als volgt berekenen:
  \[ f''(x) = \lim_{h \rightarrow 0}\frac{f(x+h)+f(x-h)-2f(x)}{h^{2}} \]

  \begin{proof}
    \[ 
    \begin{array}{rl}
      f''(x)
      &= \lim_{h \rightarrow 0}\frac{f'(x+h)-f'(x)}{h}\\
      &= \lim_{h \rightarrow 0}\frac{\lim_{h \rightarrow 0}\frac{f(x+h)-f(x+h-h)}{h}-\lim_{h \rightarrow 0}\frac{f(x)-f(x-h)}{h}}{h}\\
      &= \lim_{h \rightarrow 0}\frac{\lim_{h \rightarrow 0}\frac{f(x+h)-f(x+h-h)-f(x)+f(x-h)}{h}}{h}\\
      &= \lim_{h \rightarrow 0}\frac{\lim_{h \rightarrow 0}\frac{f(x+h)-2f(x)+f(x-h)}{h}}{h}\\
      &= \lim_{h \rightarrow 0}\frac{f(x+h)+f(x-h)-2f(x)}{h^{2}} \\
    \end{array}
    \]
    \question{waarom mogen we zomaar dezelfde $h$ gebruiken?}
    \question{Mogen we zomaar de andere definitie van een afgeleide gebruiken?}
  \end{proof}
\end{st}

\begin{de}
  Zij $I$ een interval in $\mathbb{R}$ dan noemen we $C(I)$ de verzameling van continue functies van $I$ naar $\mathbb{R}$.
  Voor $n\in \mathbb{N}$ noemen we $C^{n}(I)$ de verzameling van continue functies die $n$ keer afleidbaar zijn op $I$ en waarvoor $f^{(n)}$ continu is op $I$.
  Met $C^{\infty}(I)$ noteren we de verzameling van functies $I$ die onbeperkt afleidbaar zijn op $I$.
\end{de}

\begin{st}
  Zij $I$ een interval in $\mathbb{R}$, voor elke $n\in \mathbb{N}$ is $C^{n}(I)$ een vectorruimte.
\extra{bewijs}
\end{st}

\begin{st}
  Zij $I$ een interval in $\mathbb{R}$, voor elke $n\in \mathbb{N}$ is $C^{n}(I)$ gesloten onder het product:
  \[ \forall f,g \in C^{n}(I):\ fg \in C^{n}(I) \]
\extra{bewijs}
\end{st}

\begin{de}
  \label{de:convexe-functie}
  Zij $f:\ I \subseteq \mathbb{R} \rightarrow \mathbb{R}$ een functie gedefinieerd op een interval $I$, dan noemen we $f$ \term{convex} over een deelinterval $I_{0}$ van $I$ als en slechts als het volgende geldt:
  \[ \forall x,y \in I_{0}, \forall \lambda \in \interval{0}{1}:\ f((1-\lambda)x+\lambda y) \le (1-\lambda)f(x) + \lambda f(y)\]
\end{de}
\begin{de}
  We noemen $f$ \term{concaaf} over $I_{0}$ als en slechts als $-f$ convex is.
\end{de}

\begin{opm}
  Een functie kan zowel convex als concaaf zijn.
  $f: x \mapsto 0$ bijvoorbeeld.
\end{opm}

\begin{bpr}
  Beschouw een functie $f:\ I \subseteq \mathbb{R} \rightarrow \mathbb{R}$, gedefinieerd op een open interval $I$ die twee keer afleidbaar is.
  $f$ is convex over $I$ als en slechts als $f''$ positief is over heel $I$.

  \begin{proof}
    Bewijs van een equivalentie
    \begin{itemize}
    \item $\Rightarrow$\\
      Zij $f$ een convexe functie.
      Kies een willekeurige $x\in I$.
      Omdat $I$ open is, bestaat er een $\delta \in \mathbb{R}_{0}^{+}$ zodat $\interval[open]{x-\delta}{x+\delta}$ een deel is van $I$.
      Omdat $f$ convex is, gelden de volgende twee ongelijkheden voor alle $h \in \interval[open]{-\delta}{\delta}$:
      (Kies $\lambda = \frac{1}{2}$ in de definitie van convexiteit\deref{de:convexe-functie})
      \[ f\left(\frac{1}{2}(x-h) + \frac{1}{2}(x+h)\right) \le \frac{1}{2}f(x-h) + \frac{1}{2}f(x+h) \]
      \[ f(x) \le \frac{1}{2}f(x-h) + \frac{1}{2}f(x+h) \]
      \[ 2f(x) \le f(x-h) + f(x+h) \]
      \[ f(x-h) + f(x+h) -2f(x) \ge 0 \]
      \[ \frac{f(x+h)+f(x-h)-2f(x)}{h^{2}} \ge 0 \]
      Nemen we nu de limiet van deze ongelijkheid voor $h$ gaande naar nul, dan krijgen we het volgende\prref{pr:limiet-behoudt-orde}:
      \[ \lim_{h \rightarrow 0}\frac{f(x+h)+f(x-h)-2f(x)}{h^{2}} \ge \lim_{h \rightarrow 0}0 \]
      \[ f''(x) \ge 0 \]
      $f''$ is positief in een willekeurige $x$\stref{st:berekening-tweede-afgeleide} en dus over heel $I$.
    \item $\Leftarrow$\\
      Zij $f$ een functie zodat $f''$ positief is over een open interval $I$
      Kies willekeurig $x,y \in I$   met $x < y$ en $\lambda \in \interval{0}{1}$.
      Voor $\lambda = 0$ of $\lambda = 1$ is dit deel triviaal.
      Stel daarom dat $\lambda \in \interval[open]{0}{1}$ geldt en noteer $z = (1-\lambda)x + \lambda y$.
      Volgens de middelwaardestelling van Lagrange bestaan er dan $c_{x}\in \interval[open]{x}{z}$ en $c_{y}\in \interval[open]{z}{y}$ als volgt\stref{st:middelwaardestelling-lagrange}:
      \[ f'(c_{x}) = \frac{f(z)-f(x)}{z-x} \quad\text{ en }\quad f'(c_{y}) = \frac{f(y)-f(z)}{y-z} \]
      We kunnen nu $f(z)$ uitwerken:
      \[
      \begin{array}{rl}
        f(z) 
        &= (1-\lambda)f(z) + \lambda f(z)\\
        &= (1-\lambda)\left((z-x)f'(c_{x}) + f(x)\right) - \lambda\left( f'(c_{y})(y-z)-f(y)\right)\\
        &= (1-\lambda)f(x) + \lambda f(y) + (1-\lambda)(z-x)f'(c_{x}) + \lambda(z-y)f'(c_{y})\\
        &= (1-\lambda)f(x) + \lambda f(y) + \lambda(1-\lambda)(y-x)(f'(c_{x})-f'(c_{y}))\\
      \end{array}
      \]
      \question{hoe gebeurt die laatste stap?}
      Gebruiken we nu de middelwaardestelling van Lagrange, dan vinden we een $c \in \interval{c_{x}}{c_{y}}$ als volgt:\stref{st:middelwaardestelling-lagrange}
      \[ f''(c) = \frac{f'(c_{y})-f'(c_{x})}{c_{y}-c_{x}} \]
      Gaan we nu verder met de vorige gelijkheid, dan vinden we de volgende ongelijkheid, waar uit volgt dat $f$ convex is over $I$.
      \[ 
      \begin{array}{rl}
        f(z) &= (1-\lambda)f(x) + \lambda f(y) + \lambda(1-\lambda)(y-x)(f'(c_{x})-f'(c_{y}))\\
             &= (1-\lambda)f(x) + \lambda f(y) + \lambda(1-\lambda)(y-x)f''(c)(c_{y}-c_{x})\\
             &\le (1-\lambda)f(x) + \lambda f(y) 
      \end{array}
      \]
      In de laatste ongelijkheid gebruiken we $f''(c) \ge 0$, $y > x$, $c_{y}> c_{x}$ en $\lambda(1-\lambda) \ge 0$.
    \end{itemize}
  \end{proof}
\end{bpr}

\section{Middelwaardestelling van Taylor}
\label{sec:midd-van-tayl}

\begin{bst}
  Beschouw een functie $f:\ I \subseteq \mathbb{R} \rightarrow \mathbb{R}$, gedefinieerd op een interval $I$, die $n$ maal afleidbaar is op $I$ en $(n+1)$ maal op het inwendige $\mathring{I}$ van $I$.
  Stel dat $f^{(n)}$ continu is op $I$ en zij $a,x\in I$ met $a \neq x$, dan bestaat er een $c$, strikt tussen $a$ en $x$, als volgt:
  \[ 
  f(x) = \frac{f^{(n+1)}(c)}{(n+1)!}(x-a)^{n+1} + \sum_{n = 0}^{n}\frac{f^{(n)}(a)}{n!}(x-a)^{n} 
  \]
\TODO{bewijs p 43}
\end{bst}

\begin{de}
  Beschouw een functie $f:\ I \subseteq \mathbb{R} \rightarrow \mathbb{R}$, gedefinieerd op een interval $I$, die $n$ maal afleidbaar is op $I$ en $(n+1)$ maal op het inwendige $\mathring{I}$ van $I$.
  Stel dat $f^{(n)}$ continu is op $I$ en zij $a,x\in I$ met $a \neq x$, dan noemen we $P_{n}$ als volgt de \term{$n$-de orde benadering} van $f$ rond $a$.
  \[ 
  P_{n}(x) = \sum_{n = 0}^{n}\frac{f^{(n)}(a)}{n!}(x-a)^{n} 
  \]
  We noemen deze veelterm soms de \term{Taylorveelterm} van graad $n$ van $f$ rond $a$.
  In bovenstaande stelling noemen we de eerste term de \term{restterm} van graad $n$ en we noteren deze als $R_{n}(x)$.
  \[ 
  R_{n}(x) = \frac{f^{(n+1)}(c)}{(n+1)!}(x-a)^{n+1} 
  \]
\end{de}

\begin{de}
  We noemen het rechterlid als volgt de \term{Taylor(reeks)ontwikkeling} van $f$ rond $a$ in $x$.
  \[
  f(x) = \lim_{n \rightarrow \infty}\sum_{n = 0}^{n}\frac{f^{(n)}(a)}{n!}(x-a)^{n} 
  \]
\end{de}


\begin{de}
  We noemen een functie $f:\ A \subseteq \mathbb{R} \rightarrow \mathbb{R}$, gedefinieerd op een open deel $A$ van $\mathbb{R}$, \term{analytisch} op $A$ als er voor elke $a\in A$ een $\delta \in \mathbb{R}_{0}^{+}$ bestaat als volgt:
  \begin{itemize}
  \item $\interval[open]{a-\delta}{a+\delta} \subseteq A$
  \item Er bestaat een rij $(c_{n})_{n}$ in $\mathbb{R}$ zodat voor alle $\interval[open]{a-\delta}{a+\delta}$ het volgende geldt:
    \[ f(x) = \lim_{n\rightarrow +\infty}\sum_{k=0}^{n}c_{k}(x-a)^{k} \]
  \end{itemize}
\end{de}

\begin{bpr}
  Zij $f:\ I \subseteq \mathbb{R} \rightarrow \mathbb{R}$ een twee maal afleidbare functie op een open interval $I$ zodat de tweede afgeleide $f''$ continu is.
  Stel dat er een nulpunt $a\in I$ van $f'$ bestaat, dan geldt het volgende:
  \begin{itemize}
  \item Als $f''$ strikt positief is in $a$, dan bereikt $f$ in $a$ een lokaal minimum.
  \item Als $f''$ strikt negatief is in $a$, dan bereikt $f$ in $a$ een lokaal maximum.
  \end{itemize}
\TODO{bewijs p 48}
\end{bpr}

\begin{opm}
  Bovenstaande propositie zegt ons niets in het geval dat $f''$ nul is in $a$.
\end{opm}

\end{document}

%%% Local Variables:
%%% mode: latex
%%% TeX-master: t
%%% End:
