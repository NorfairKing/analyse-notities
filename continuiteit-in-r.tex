\documentclass[main.tex]{subfiles}
\begin{document}

\chapter{Continu\"iteit voor functies van $\mathbb{R}$ naar $\mathbb{R}$}
\label{cha:cont-in-r}

\section{Het continu\"iteitsbegrip}
\label{sec:het-cont}


\begin{de}
  Zij $f$ een functie en $a\in A$:
  \[ f:\ A \subseteq \mathbb{R} \rightarrow \mathbb{R}:\ x \mapsto f(x) \]
  We noemen $f$ \term{continu} in $a$ als en slechts als het volgende geldt:
  \[ \forall \epsilon \in \mathbb{R}_{0}^{+}:\ \exists \delta \in \mathbb{R}_{0}^{+}:\ \forall x\in A:\ |x-a| < \delta \Rightarrow |f(x) -f(a)| < \epsilon \]
  We noemen $f$ \term{continu} op $A$ als $f$ continu is in elke $a\in A$.
\end{de}

\begin{st}
  Equivalente definitie van \term{continu\"iteit}:\\
  $f$ is continu in $a$ als en slechts als het volgende geldt:
  \[ \forall \epsilon \in \mathbb{R}_{0}^{+}:\ \exists \delta \in \mathbb{R}_{0}^{+}:\ \forall x\in A:\ f(\interval[open]{a-\delta}{a+\delta} \cap A) \subseteq \interval[open]{f(a)-\epsilon}{f(a)+\epsilon} \]
\extra{bewijs}
\end{st}


\begin{pr}
  \label{pr:continu-asa-behoudt-convergentie}
  Zij $f:\ A \subseteq \mathbb{R} \rightarrow \mathbb{R}$ een functie en $a\in A$.
  $f$ is continu in $a$ als en slechts als er voor elke rij $(x_{n})_{n}$ in $A$ die naar $a$ convergeert geldt dat $(f(x_{n}))_{n}$ naar $f(a)$ convergeert.

  \begin{proof}
    Bewijs van een equivalentie.
    \begin{itemize}
    \item $\Rightarrow$\\
      Kies een willekeurige rij $(x_{n})_{n}$ in $A$ die naar $a$ convergeert.
      We moeten bewijzen dat $(f(x_{n}))_{n}$ naar $f(a)$ convergeert.
      Kies daartoe een willekeurige $\epsilon \in \mathbb{R}_{0}^{+}$.
      Omdat $f$ continu is in $a$ kunnen we een $\delta\in \mathbb{R}_{0}^{+}$ vinden zodat $|f(x)-f(a)| < \epsilon$ geldt voor alle $x\in A$ met $|x-a| < \delta$.
      Omdat $(x_{n})_{n}$ naar $a$ convergeert, kunnen we een $n_{0}\in \mathbb{N}$ vinden zodat voor alle volgende $n\in\mathbb{N}$ de afstandvan $x_{n}$ tot $a$ kleiner is dan $\delta$.
      Voor elke $n\ge n_{0}$ zal dan $|f(x_{n})-f(a)| < \epsilon$ gelden.
      De rij $(f(x_{n}))_{n}$ convergeert dus naar $f(a)$.
    \item $\Leftarrow$\\
      Contrapositie: Als $f$ niet continu is in $a$ zal er een rij in $A$ bestaan die naar een $a$ convergeert waarvoor $(f(x_{n}))_{n}$ niet naar $f(a)$ convergeert.
      Omdat $f$ niet continu is bestaat er een $\epsilon \in \mathbb{R}_{0}^{+}$ zodat er voor elke $\delta\in \mathbb{R}_{0}^{+}$ een $x$ kan genomen worden zodat het volgende geldt:
      \[ |x-a| < \delta \quad\wedge\quad |f(x)-f(a)| \ge \epsilon \]
      We kunnen dan een rij $(x_{n})_{n}$ in $A$ construeren zodat voor alle $n\in\mathbb{N}_{0}$ $x_{n}$ als volgt gekozen is:
      \[ |x_{n}-a| < \frac{1}{n} \quad\wedge\quad |f(x_{n})-f(a)| \ge \epsilon \]
      Uit de eerste ongelijkheid volgt dat $x_{n}$ naar $a$ convergeert terwijl uit de tweede ongelijkheid volgt dat $(f(x_{n}))_{n}$ niet kan convergeren naar $f(a)$.
    \end{itemize}
  \end{proof}
\end{pr}

\begin{pr}
  Zij $f:\ A \subseteq \mathbb{R} \rightarrow \mathbb{R}$ een functie en $a\in A$.
  $f$ is continu op $A$ als en slechts als er voor elk open deel $V$ van $\mathbb{R}$ geldt dat $f^{-1}(V)$ relatief open is in $A$.
  \begin{proof}
    Bewijs van een equivalentie.
    \begin{itemize}
    \item $\Rightarrow$\\
      Kies een willekeurig open deel $V$ van $\mathbb{R}$.
      Als $f^{-1}(V)$ leeg is, is $f^{-1}(V)$ triviaal open.
      Als $f^{-1}(V)$ niet leeg is, dan bestaat er een $a\in f^{-1}(V)$.
      $f(a)$ zit dan in $V$ en omdat $V$ open is kunnen we een $\epsilon$ vinden zodat $\interval[open]{f(a)-\epsilon}{f(a)+\epsilon}$ een deel is van $V$.
      Omdat $f$ continu is in $a$ kunnen we een $\delta\in \mathbb{R}_{0}^{+}$ vinden zodat uit $x\in A$ en $|x-a| < \delta$ volgt dat $f(x) \in \interval[open]{f(a)-\epsilon}{f(a)+\epsilon}$
      Dit betekent dat $\interval[open]{a-\delta}{a+\delta} \cap A \subseteq f^{-1}(V)$ geldt en dus dat $f^{-1}(V)$ relatief open is in $A$.
    \item $\Leftarrow$\\
      Stel dat er voor elk open deel $V$ van $\mathbb{R}$ geldt dat $f^{-1}(V)$ relatief open is in $A$.
      Kies nu een willekeurige $a\in A$.
      We bewijzen dat $f$ continu is in $a$.
      Beschouw daarvoor de verzameling $U$ voor een willekeurige $\epsilon \in \mathbb{R}_{0}^{+}$:
      \[ U = f^{-1}(\interval[open]{f(a)-\epsilon}{f(a)+\epsilon}) \]
      $U$ bevat zeker $a$ en is bovendien relatief open in $A$.
      Er bestaat dus een $\delta \in \mathbb{R}_{0}^{+}$ zodat uit $x\in A$ en $|x-a|< \delta$ volgt dat $x$ tot $U$ behoort.
      Kies dan een willekeurige $x\in A$ met $|x-a|< \delta$, opdat $x$ tot $U$ behoort.
      $f(x)$ behoort dan ook tot $f(U) = \interval[open]{f(a)-\epsilon}{f(a)+\epsilon}$.
      Er geldt met andere woorden $|f(x)-f(a)|<\epsilon$, dus $f$ is continu in $a$.
    \end{itemize}
  \end{proof}
\end{pr}

\begin{de}
  Zij $f:\ A \subseteq \mathbb{R} \rightarrow \mathbb{R}$ een functie en $a\in A$.
  We noemen $f$ ...
  \begin{itemize}
  \item ...\term{linkscontinu} in $a$ als en slechts als de beperking van $f$ tot $\interval[open left]{-\infty}{a} \cap A$ continu is in $a$.
  \item ...\term{rechtscontinu} in $a$ als en slechts als de beperking van $f$ tot $\interval[open right]{a}{+\infty} \cap A$ continu is in $a$.
  \end{itemize}
\end{de}

\section{Operaties met continue functies}
\label{sec:oper-met-cont}

\begin{pr}
  Zij $f:\ A \subseteq \mathbb{R} \rightarrow \mathbb{R}$ een functie die continu is in $a\in A$.
  \[ \lambda f:\ A \rightarrow \mathbb{R}: x \mapsto \lambda f(x) \text{ is continu in } A \]

  \begin{proof}
    Kies een willekeurige $\epsilon \in \mathbb{R}_{0}^{+}$.
    Omdat $f$ continu is in $a$ bestaat er een $\delta \in \mathbb{R}_{0}^{+}$ zodat uit $|x-a|<\delta$  $|f(x)-f(a)|<\frac{\epsilon}{|\lambda|}$ volgt.
    Hieruit volgt het volgende:
    \[ |x-a| < \delta \Rightarrow |\lambda||f(x)-f(a)| = |\lambda f(x)-\lambda f(a)| < \epsilon \]
  \end{proof}
\end{pr}

\begin{pr}
  Zij $f,g:\ A \subseteq \mathbb{R} \rightarrow \mathbb{R}$ functies die continu zijn in $a\in A$.
  \[ f+g:\ A \rightarrow \mathbb{R}: x \mapsto f(x)+g(x) \text{ is continu in } A \]

  \begin{proof}
    Kies een willekeurige $\epsilon \in \mathbb{R}_{0}^{+}$.
    Omdat $f$ en $g$ elk continu zijn in $a$ bestaan er $\delta_{x}, \delta_{y} \in \mathbb{R}_{0}^{+}$ zodat de volgende implicaties gelden:
    \[ \forall x\in A:\ |x-a|<\delta_{x} \Rightarrow |f(x)-f(a)|<\frac{\epsilon}{2} \quad\wedge\quad \forall x\in A:\ |x-a|<\delta_{y}  \Rightarrow |g(x)-g(a)|<\frac{\epsilon}{2} \]
    Hieruit volgt dan het volgende:
    \[ \forall x\in A:\ |x-a|<\min\{\delta_{x},\delta_{y}\}  \Rightarrow |(f+g)(x) - (f+g)(a)| \le |f(x)-f(a)| + |g(x)-g(a)| < \frac{\epsilon}{2} + \frac{\epsilon}{2} = \epsilon \]
    Bijgevolg is $(f+g)$ continu in $a$.
  \end{proof}
\end{pr}

\begin{pr}
  \label{pr:product-continu}
  Zij $f,g:\ A \subseteq \mathbb{R} \rightarrow \mathbb{R}$ functies die continu zijn in $a\in A$.
  \[ fg:\ A \rightarrow \mathbb{R}: x \mapsto \lambda f(x)g(x) \text{ is continu in } A \]

  \begin{proof}
    Kies een willekeurige $\epsilon \in \mathbb{R}_{0}^{+}$.
    Merk eerst het volgende op voor elke $x\in A$:
    \[
    \begin{array}{rl}
    |(fg)(x) - (fg)(a)| &= |f(x)g(x) - f(a)g(a)|\\
                        &= |f(x)g(x) - f(x)g(a) + f(x)g(a) - f(a)g(a)|\\
                        &\le |f(x)g(x) - f(x)g(a)| + |f(x)g(a) - f(a)g(a)|\\
                        &\le |f(x)||g(x)-g(a)| + |g(a)||f(x)-f(a)|\\
    \end{array}
    \]
    Omdat $f$ continu is in $A$ bestaat er een $\delta_{1} \in \mathbb{R}_{0}^{+}$ zodat voor alle $x\in A$ met $|x-a|<\delta_{1}$ geldt dat $f(x)$ dichter dan $1$ bij $f(a)$ ligt.
    Er geldt dan het volgende:
    \[
    \begin{array}{c}
      |f(x)-f(a)|<1\\
      |f(x)|-|f(a)|<1\\
      |f(x)|<1+|f(a)|\\
    \end{array}
    \]
    Combineren we deze ongelijkheden, dan krijgen we de volgende:
    \[ |(fg)(x) - (fg)(a)| \le (1+|f(a)|)|g(x)-g(a)| + |g(a)||f(x)-f(a)| \]
    We zullen nu de juiste $\delta$'s kiezen zodat het rechterlid kleiner dan $\epsilon$ wordt.
    Kies daarom een $\delta_{2}$ en $\delta_{3}$ zodat de volgende ongelijkheden gelden voor elk $x\in A$ dicht genoeg bij $a$.
    Dit kan omdat zowel $f$ als $g$ continu is in $a$.
    \[ 
    |f(x)-f(a)| < \frac{\epsilon}{2(1+|g(a)|)} \quad\wedge\quad |g(x)-g(a)| < \frac{\epsilon}{2(1+|f(a)|)}
    \]
    Kies nu $\delta = \min\{\delta_{1},\delta_{2},\delta_{3}\}$ zodat het volgende geldt.
    \[ 
    \begin{array}{rl}
    \forall x\in A: |x-a|<\delta \Rightarrow \\
    |f(x)||g(x)-g(a)| + |g(a)||f(x)-f(a)| &\le (1+|f(a)|)|g(x)-g(a)| + |g(a)||f(x)-f(a)|\\
                                          &\le \frac{(1+|f(a)|)\epsilon}{2(1+|f(a)|)} + \frac{|g(a)|\epsilon}{2(1+|g(a)|)}\\
                                          &\le \frac{\epsilon}{2} + \frac{\epsilon}{2}\\
                                          &= \epsilon
    \end{array}
    \]
    Dit bewijs dat $fg$ continu is in $a$.
  \end{proof}
\end{pr}

\begin{pr}
  Zij $f,g:\ A \subseteq \mathbb{R} \rightarrow \mathbb{R}$ functies die continu zijn in $a\in A$ met $g(a)\neq 0$.
  Noteer bovendien $A_{0} = \{ x \in A \mid g(x) \neq 0 \}$
  \[ \frac{f}{g}:\ A_{0} \rightarrow \mathbb{R}: x \mapsto \lambda \frac{f(x)}{g(x)} \text{ is continu in } A \]

  \begin{proof}
    Het volstaat om aan te tonen dat $\frac{1}{g}$ continu is in $a$.\prref{pr:product-continu}
    \[ \frac{1}{g}:\ A_{0}\rightarrow \mathbb{R}:\ x \mapsto \frac{1}{g(x)} \]
    Merk eerst het volgende op voor alle $x\in A_{0}$.
    \[ \left| \frac{1}{g(x)} - \frac{1}{g(a)} \right| = \frac{|g(x)-g(a)|}{|g(x)||g(a)|} \]
    We proberen het rechterlid nu kleiner te krijgen dan een willekeurige $\epsilon \in \mathbb{R}_{0}^{+}$ door de juiste $\delta$ te kiezen.
    Omdat $g$ continu is in $a$ en $g(a)$ niet nul is, kunnen we een $\delta_{1} \in \mathbb{R}_{0}^{+}$ kiezen zodat voor alle $x\in A_{0}$ dicht genoeg bij $a$ het volgende geldt:
    \[ 
    \begin{array}{c}
      |g(x)-g(a)| < \frac{|g(a)|}{2}\\
      |g(x)| \in \interval[open]{|g(a)|-\frac{|g(a)|}{2}}{|g(a)|+\frac{|g(a)|}{2}}\\
    \end{array}
    \]
    We zetten deze ongelijkheid samen met de eerste gelijkheid om de volgende ongelijkheid te bekomen.
    \[ \left| \frac{1}{g(x)} - \frac{1}{g(a)} \right| \le \frac{2|g(x)-g(a)|}{|g(a)|^{2}} \]
    We zetten nu de tweede stap om het rechterlid kleiner dan $\epsilon$ te krijgen.
    Daartoe kiezen we een $\delta_{2}$ zodat het volgende geldt voor alle $x\in A_{0}$ dicht genoeg bij $a$.
    \[ |g(x)-g(a)| < \frac{|g(a)|^{2}\epsilon}{2} \]
    Voor $x\in A$, dichter dan $\min\{\delta_{1},\delta_{2}\}$ bij $a$ geldt dan de het volgende:
    \[ 
    \begin{array}{c}
      \left| \frac{1}{g(x)} - \frac{1}{g(a)} \right| = \frac{|g(x)-g(a)|}{|g(x)||g(a)|}\\
      \left| \frac{1}{g(x)} - \frac{1}{g(a)} \right| < \frac{2|g(x)-g(a)|}{|g(a)|^{2}}\\
      \left| \frac{1}{g(x)} - \frac{1}{g(a)} \right| < \frac{2|g(a)|^{2}\epsilon}{2|g(a)|^{2}}\\
      \left| \frac{1}{g(x)} - \frac{1}{g(a)} \right| < \epsilon \\
    \end{array}
    \]
    Dit bewijst dat $\frac{1}{g}$ continu is in $a$.
  \end{proof}
\end{pr}

\begin{pr}
  Zij $f:\ A \subseteq \mathbb{R} \rightarrow B \subseteq \mathbb{R}$ en $g:\ B \rightarrow \mathbb{R}$ functies en zij $a\in A$.
  Als $f$ continu is in $a$ en $g$ continu in $f(a)$, dan is $g\circ f$ continu in $a$.

  \begin{proof}
    Kies een willekeurige $\epsilon \in \mathbb{R}_{0}^{+}$.
    Omdat $g$ continu is in $f(a)$ kunnen we een $\eta\in\mathbb{R}_{0}^{+}$ vinden zodat $|g(y)-g(f(a))|<\epsilon$ geldt voor alle $y\in B$ met $|y-f(a)|<\eta$.
    Omdat $f$ continu is in $a$ kunnen we een $\delta\in \mathbb{R}_{0}^{+}$ vinden zodat $|f(x)-f(a)| < \eta$ geldt voor alle $x\in A$ met $|x-a|<\delta$ geldt.
    Voor elke $x\in A$ met $|x-a|<\delta$ zal $|f(x)-f(a)| < \eta$ gelden en bijgevolg ook $|g(f(x))-g(f(a))|<\epsilon$.
    $g\circ f$ is dus continu in $a$.
  \end{proof}
\end{pr}

\begin{pr}
  Zij $A$ een gesloten begrensd deel van $\mathbb{R}$.
  Zij $f: \ A \subseteq \mathbb{R} \rightarrow \mathbb{R}$ een continue injectieve functie, dan is $f^{-1}: f(A) \rightarrow A$ ook continu.

  \begin{proof}
    Merk op dat $f$ injectief moet zijn opdat $f^{-1}$ een functie zou zijn.
    We zullen bewijzen dat voor elke rij $(y_{n})_{n}$ in $f(A)$ die convergeert naar een $y\in f(x) \in f(A)$ geldt dat $(f^{-1}(y_{n}))_{n}$ convergeert naar $x\in A$.
    Daaruit volgt dan de stelling.\prref{pr:continu-asa-behoudt-convergentie}
    Noem het invers beeld van $y_{n}$ onder $f$ $x_{n}$.
    We moeten argumenteren dat $(x_{n})_{n}$ naar $x$ convergeert.
    Stel immers dat $(x_{n})_{n}$ niet convergeert naar $x$, dan bestaat er een $\epsilon \in \mathbb{R}_{0}^{+}$ zodat $|x_{n}-x|> \epsilon$ geldt (dit geldt dan ook voor elke deelrij van $(x_{n})_{n}$.
    Omdat $A$ gesloten en begrensd is bestaat er een deelrij $(x_{n_{k}})_{k}$ die convergeert naar een andere $x'\neq x$.
    Omdat $f$ continu is, zal $(y_{n})_{n}$ naar $f(x')$ convergeren, maar omdat $f$ injectief is, zal $f(x')$ verschillend zijn van $f(x)$.
    We vinden dus dat de convergente rij $(y_{n_{k}})_{k}$ een convergente deelrij zou hebben met een andere limiet, wat niet kan.
  \end{proof}
\end{pr}

\section{Continue functies op intervallen}
\label{sec:continue-functies-op}

\begin{st}
  Beschouw een continue functie $f: \interval{a}{b} \rightarrow \mathbb{R}$ gedefinieerd op een gesloten begrensd interval $\interval{a}{b}$.
  $f$ is begrensd en bereikt op $\interval{a}{b}$ haar minimale en maximale waarde.
  Er bestaan dus waarden $c,d \in \interval{a}{b}$ als volgt:
  \[ f(c) = \sup\left\{ f(x) \mid x \in \interval{a}{b} \right\} \quad\text{ en }\quad f(d) = \sup\left\{ f(x) \mid x \in \interval{a}{b} \right\} \]

  \begin{proof}
    \begin{itemize}
    \item $f$ is begrensd.\\
      Stel immers dat $f$ niet begrensd is, dan kunnen we een rij $(x_{n})_{n}$ construeren zodat $|f(x_{n})|$ telkens groter is dan $n$.
      We verkrijgen zo een rij in $\interval{a}{b}$.
      Omdat $\interval{a}{b}$ gesloten en begrensd is, kunnen we een deelrij nemen die convergeert naar een $x\in\interval{a}{b}$.
      Omdat $f$ continu is moet $f(x_{n_{k}})_{k}$ dan convergeren naar $f(x)$.\prref{pr:continu-asa-behoudt-convergentie}
      Sterker nog, $(f(x_{n_{k}}))_{k}$ moet begrensd zijn.\prref{pr:convergente-rij-begrensd}
      Dit is echter strijdig met het feit dat voor alle $k$, $|f(x_{n_{k}})|$ groter is dan $k$.
    \item $f$ bereikt een maximale waarde op $\interval{a}{b}$.\\
      We weten al dat $f$ begrensd is, dus we kunnen $M$ het supremum van $f(\interval{a}{b})$ noemen.
      We kunnen nu voor alle $\mathbb{N}_{0}$ een $x_{n} \in \interval{a}{b}$ kiezen zodat het volgende geldt:\waarom
      \[ M- \frac{1}{n} < f(x_{n}) \le M \]
      We verkrijgen zo een rij $(x_{n})_{n}$ in $\interval{a}{b}$.
      Omdat $\interval{a}{b}$ gesloten en begrensd is, kunnen we een deelrij $(x_{n_{k}})_{k}$ vinden die convergeert naar een $c\in \interval{a}{b}$.
      Omdat $f$ continu is zal $(f(x_{n_{k}}))_{k}$ naar $f(c)$ convergeren.
      Omdat $f_{n_{k}}$ willekeurig dicht bij $M$ komt vanaf een geschikte $n_{0}\in\mathbb{N}_{0}$ zal $c$ gelijk zijn aan $M$.
    \item $f$ bereikt een minimale waarde op $\interval{a}{b}$.\\
      \extra{bewijs}
    \end{itemize}
  \end{proof}
\question{waar wordt de supremumeigenschap gebruikt?}
\end{st}

\begin{st}
  \label{st:tussenwaardestelling}
  De \term{tussenwaardestelling}\\
  Zij $f: I \subseteq \mathbb{R} \rightarrow \mathbb{R}$ een continue functie op een interval $I$.
  Zij $x_{1},x_{2}\in I$ en noem $y_{i}=f(x_{i})$.
  \[ \forall y \in \interval{y_{1}}{y_{2}}:\ \exists x \in I:\ f(x) = y \]

  \begin{proof}
    We beschouwen het geval waarin $x_{1}<x_{2}$ en $y_{1}<y_{2}$ gelden, de andere gevallen gaan analoog.\extra{uitwerken?}
    Kies vervolgens een willekeurige $y\in \interval{y_{1}}{y_{2}}$.
    \begin{itemize}
    \item Als $y$ gelijk is aan $y_{1}$ of $y_{2}$ kiezen we voor $x$ gewoon $x_{1}$,respectievelijk $x_{2}$.
    \item Als $y$ verschillend is van zowel $y_{1}$ als $y_{2}$ zoeken we in het interval $\interval{y_{1}}{y_{2}}$ binair naar een waarde $x$ met als functiewaarde $y$.
      \begin{itemize}
      \item Als het zoeken eindigt met een interval waarin $y$ een randpunt is, dan hebben we $x$ gevonden als het overeenkomstig randpunt van het interval van de $x$-en.
      \item Als het zoeken nooit eindigd, krijgen we een rij intervallen $(I_{n})_{n}$.
        De rij van linkereindpunten $(l_{n})_{n}$ is een stijgende, naar boven begrensde rij in $\mathbb{R}$ en convergeert bijgevolg.\stref{st:stijgend-dan-limiet}
        De rij van rechtereindpunten $(r_{n})_{n}$ is een dalende, naar onder begrensde rij in $\mathbb{R}$ en convergeert bijgevolg.\stref{st:dalend-dan-limiet}
        Omdat de lengte van de intervallen $I_{n}$ naar nulconvergeert is de limiet $l$ van $(l_{n})_{n}$ gelijk aan de limiet $r$ van $(r_{n})_{n}$.
        Noem deze limiet $x$.
        Omdat $f$ continu is, is $f(x)$ ook de limiet van $(f(l_{n}))_{n}$ en $(r(l_{n}))_{n}$.
        Omdat $f(x)$, per constructie, zowel kleiner of gelijk aan, als groter of gelijk aan $y$ moet zijn,\waarom moeten $f(x)$ en $y$ gelijk zijn.
        We hebben dan de gezochte $x$ gevonden.
      \end{itemize}
    \end{itemize}
  \end{proof}
\end{st}

\section{Uniforme continuiteit}
\label{sec:unif-cont}

\begin{st}
  \label{st:continue-functie-behoudt-cauchy}
  Zij $f:\ \mathbb{R} \rightarrow \mathbb{R}$ een continue functie en $(x_{n})_{n}$ een Cauchyrij in $\mathbb{R}$, dan is $(f(x_{n}))_{n}$ ook een Cauchyrij.

  \begin{proof}
    $(x_{n})_{n}$ is een Cauchyrij en daarom convergent.\prref{pr:cauchyrij-in-R-convergeert}
    Noem $x$ de limiet van $(x_{n})_{n}$.\prref{pr:cauchyrij-in-R-convergeert}
    Omdat $f$ continu is, zal $f(x)$ de limiet zijn van $(f(x_{n}))_{n}$.
    $(f(x_{n}))_{n}$ convergeert dus ook en is bijgevolg een Cauchyrij.\prref{pr:convergent-dan-cauchy}
  \end{proof}
\end{st}

\begin{de}
  We noemen een functie $f: A \subseteq \mathbb{R} \rightarrow \mathbb{R}$ \term{uniform continu} of \term{gelijkmatig continu} op $A$ als het volgende geldt:
  \[ \forall \epsilon \in \mathbb{R}_{0}^{+}:\ \exists \delta \in \mathbb{R}_{0}^{+}:\ \forall x,y \in A:\ |x-y| < \delta \Rightarrow |f(x)-f(y)| < \epsilon \]
\end{de}

\begin{st}
  \label{st:uniform-continu-dan-ook-gewoon-continu}
  Als een functie $f: A \subseteq \mathbb{R} \rightarrow \mathbb{R}$ uniform continu is, is hij ook continu.
\end{st}

\begin{tvb}
  Een continue functie $f: A \subseteq \mathbb{R} \rightarrow \mathbb{R}$ is niet noodzakelijk uniform continu
\extra{tegenvoorbeeld}
\end{tvb}

\begin{pr}
  \label{pr:uniform-continue-functie-behoudt-cauchy}
  Zij $f: A \subseteq \mathbb{R} \rightarrow \mathbb{R}$ een uniform continue functie en $(x_{n})_{n}$ een Cauchyrij in $\mathbb{R}$, dan is $(f(x_{n}))_{n}$ ook een Cauchyrij.

  \begin{proof}
    $f$ is uniform continu en dus ook gewoon continu.\stref{st:uniform-continu-dan-ook-gewoon-continu}.
    $(f(x_{n}))_{n}$ is dan ook een Cauchyrij.\stref{st:continue-functie-behoudt-cauchy}
  \end{proof}
\end{pr}

\begin{pr}
  Zij $f: A \subseteq \mathbb{R} \rightarrow \mathbb{R}$ een uniform continue functie.
  Als $B$ een begrensd deel is van $A$, dan is $f(B)$ ook begrensd.

  \begin{proof}
    Stel dat $f(B)$ niet begrensd zou zijn, dan kunnen we voor elke $n\in \mathbb{N}_{0}$ een $x_{n}\in B$ vinden zodat $|f(x_{n})|$ groter is dan $n$.
    We vinden zo een rij $(x_{n})_{n}$ in $B$.
    Omdat $B$ begrensd is, bestaat er een convergente deelrij $(x_{n_{k}})_{k}$.\stref{st:bolzano-rijen}
    Deze deelrij is een Cauchyrij.\prref{pr:convergent-dan-cauchy}
    De rij $(f(x_{n_{k}}))_{k}$ is dan ook een Cauchyrij.\prref{pr:uniform-continue-functie-behoudt-cauchy}
    Cauchyrijen zijn echter begrends.\prref{pr:cauchyrij-begrensd} Contradictie.
  \end{proof}
\end{pr}

\begin{st}
  Zij $f: A \subseteq \mathbb{R} \rightarrow \mathbb{R}$ een continue functie, gedefinieerd op een gesloten en begrensd deel $A$ van $\mathbb{R}$, dan is $f$ uniform continu.

  \begin{proof}
    Stel immers dat $f$ niet uniform continu zou zijn, dan bestaat er een $\epsilon \in \mathbb{R}_{0}^{+}$ zodat we voor elke $\delta = \frac{1}{n}$ punten $x_{n}$, $y_{n}$ kunen vinden zodat het volgende geldt.
    \[ |x_{n}-y_{n}| < \frac{1}{n} \quad\wedge\quad |f(x_{n})-f(y_{n})| \ge \epsilon \]
    Omdat $(x_{n})_{n}$ een rij is in een gesloten, begrensd deel van $\mathbb{R}$.
    Er bestaat daarom een deelrij $(x_{n_{k}})_{k}$ die convergeert naar een $x\in A$.
    Omdat $f$ continu is, zal $(f(x_{n_{k}}))_{k}$ naar $x$ convergeren.
    Analoog vinden we een deelrij $(y_{n_{k}})_{k}$ van $(y_{n})_{n}$ die convergeert naar een $y\in A$.
    Omdat de afstand tussen $x_{n}$ en $y_{n}$ willekeurig klein wordt zal $y$ gelijk zijn aan $x$ en $f(x)$ dus ook aan $f(y)$.
    We vinden dus dat de afstand tussen $f(x_{n_{k}})$ en $f(y_{n_{k}})$ willekeurig klein wordt, wat strijdig is met $|f(x_{n})-f(y_{n})| \ge \epsilon$.
    Contradictie.
  \end{proof}
\end{st}

\section{Rijen van functies}
\label{sec:rijen-van-functies}

\begin{de}
  Beschouw een rij $(f_{n})_{n}$ van functies op een verzameling $A$ met waarden in $\mathbb{R}$ (of $\mathbb{C}$).
  We zeggen dat $(f_{n})_{n}$ \term{puntsgewijs convergeert} op $A$ naar een functie $f: A \rightarrow \mathbb{R}$ als het volgende geldt:
  \[ \forall x\in A:\ \epsilon \in \mathbb{R}_{0}^{+}:\ \exists n_{0}\in \mathbb{N}:\ \forall n\in \mathbb{N}:\ n \ge n \Rightarrow |f_{n}(x)-f(x)| < \epsilon \]
\end{de}

\begin{de}
  Beschouw een rij $(f_{n})_{n}$ van functies op een verzameling $A$ met waarden in $\mathbb{R}$ (of $\mathbb{C}$).
  We zeggen dat $(f_{n})_{n}$ \term{uniform convergeert} of \term{gelijkmatig convergeert} naar een functie $f: A \rightarrow \mathbb{R}$ als het volgende geldt:
  \[ \forall \epsilon \in \mathbb{R}_{0}^{+}:\ \exists n_{0} \in \mathbb{N}:\ \forall x \in A:\ \forall n\in \mathbb{N}:\ n \ge n_{0} \Rightarrow |f_{n}(x)-f(x)| < \epsilon \]
\end{de}

\begin{st}
  \label{str:uniform-dan-puntsgewijs}
  Een rij functies $(f_{n})_{n}$ die uniform convergeert, convergeert puntsgewijs.
\extra{bewijs}
\end{st}

\begin{tvb}
  Een rij functies $(f_{n})_{n}$ die puntsgewijs convergeert, convergeert niet noodzakelijk uniform.
\extra{tegenvoorbeeld}
\end{tvb}

\begin{st}
  Beschouw een rij $(f_{n})_{n}$ van functies $f_{n}: A \subseteq \mathbb{R} \rightarrow \mathbb{R}$ die op $A$ uniform convergeert naar een functie $f:\ A \rightarrow \mathbb{R}$.
  \begin{itemize}
  \item Als $f_{n}$ continu is in een $a\in A$ voor elke $n$, dan is ook $f$ continu in $a$.
  \item Als $f_{n}$ uniform continu is op $A$ voor elke $n$, dan is ook $f$ uniform continu op $A$. 
  \end{itemize}

  \begin{proof}
    \begin{itemize}
    \item 
      Kies een willekeurige $\epsilon \in \mathbb{R}_{0}^{+}$.
      Omdat $(f_{n})_{n}$ uniform convergeert naar $f$, kunnen we een $n_{0}\in \mathbb{N}$ vinden zodat voor alle $y\in A$ en voor alle volgende $n\in \mathbb{N}$ het volgende geldt:
      \[ |f_{n}(y)-f(y)|<\frac{\epsilon}{3} \]
      Omdat $f_{n}$ continu is in $a$ kunnen we een $\delta$ vinden
      zodat voor alle $x\in A$ uit $|x-a|<\delta$ het volgende geldt:
      \[ |f_{n}(x) - f_{n}(a) < \frac{\epsilon}{3} \]
      Voor elke $x\in A$ met $|x-a|<\delta$ geldt nu het volgende:
      \[
      \begin{array}{rl}
        |f(x)-f(a)| &= |f(x) - f_{n}(x) + f_{n}(x) - f_{n}(a) + f_{n}(a) - f(a)|\\
                    &\le |f(x) - f_{n}(x)| + |f_{n}(x) - f_{n}(a)| + |f_{n}(a) - f(a)|\\
                    &\le \frac{\epsilon}{3} + \frac{\epsilon}{3} + \frac{\epsilon}{3}\\
                    &= \epsilon
      \end{array}
      \]
      $f$ is dus continu in $a$.
    \item \extra{bewijs}
    \end{itemize}
  \end{proof}
\end{st}

\section{Limieten van functies}
\label{sec:limi-van-funct}

\subsection{Het limietbegrip voor functies}

\begin{de}
  Zij $f:\ A \subseteq \mathbb{R} \rightarrow \mathbb{R}$ een functie en $a\in \mathbb{R}$ een ophopingspunt van $A$.
  We noemen de \term{limiet} van $f$ in $a$ ...
  \begin{itemize}
  \item ... $L\in \mathbb{R}$ als het volgende geldt:
    \[
    \lim_{x\rightarrow a}f(x) = L \quad\Leftrightarrow\quad
    \forall \epsilon \in \mathbb{R}_{0}^{+}: \exists \delta \in \mathbb{R}_{0}^{+}: \forall x\in A:\ 0 < |x-a| < \delta \Rightarrow |f(x) - L| < \epsilon
    \]
  \item ... $+\infty$ als het volgende geldt:
    \[
    \lim_{x\rightarrow a}f(x) = +\infty\quad\Leftrightarrow\quad
    \forall M \in \mathbb{R}: \exists \delta \in \mathbb{R}_{0}^{+}: \forall x\in A:\ 0 < |x-a| < \delta \Rightarrow f(x) > M
    \]
  \item ... $-\infty$ als het volgende geldt:
    \[
    \lim_{x\rightarrow a}f(x) = -\infty\quad\Leftrightarrow\quad
    \forall M \in \mathbb{R}: \exists \delta \in \mathbb{R}_{0}^{+}: \forall x\in A:\ 0 < |x-a| < \delta \Rightarrow f(x) < M
    \]
  \end{itemize}
\end{de}

\begin{de}
  Zij $A$ een deelverzameling van $\mathbb{R}$.
  We noemen ...
  \begin{itemize}
  \item ... $+\infty$ een \term{ophopingspunt} van $A$ als het volgende geldt:
    \[ \forall N \in \mathbb{R}:\ A \cap \interval[open right]{N}{+\infty} \neq \emptyset \]
  \item ... $-\infty$ een \term{ophopingspunt} van $A$ als het volgende geldt:
    \[ \forall N \in \mathbb{R}:\ A \cap \interval[open left]{-\infty}{N} \neq \emptyset \]
  \end{itemize}
\end{de}

\begin{de}
  Zij $f:\ A \subseteq \mathbb{R} \rightarrow \mathbb{R}$ een functie en $A \subseteq \mathbb{R}$ zodat $+\infty$ een ophopinspunt is van $A$.
  We noemen de \term{limiet} van $f$ in $+\infty$ ...
  \begin{itemize}
  \item ... $L\in \mathbb{R}$ als het volgende geldt:
    \[
    \lim_{x\rightarrow +\infty}f(x) = L \quad\Leftrightarrow\quad
    \forall \epsilon \in \mathbb{R}_{0}^{+}: \exists N \in \mathbb{R}: \forall x\in A:\ x > N \Rightarrow |f(x) - L| < \epsilon
    \]
  \item ... $+\infty$ als het volgende geldt:
    \[
    \lim_{x\rightarrow +\infty}f(x) = +\infty\quad\Leftrightarrow\quad
    \forall M \in \mathbb{R}: \exists N \in \mathbb{R}: \forall x\in A:\ x > N \Rightarrow f(x) > M
    \]
  \item ... $-\infty$ als het volgende geldt:
    \[
    \lim_{x\rightarrow +\infty}f(x) = -\infty\quad\Leftrightarrow\quad
    \forall M \in \mathbb{R}: \exists N \in \mathbb{R}: \forall x\in A:\ x > N \Rightarrow f(x) < M
    \]
  \end{itemize}
\end{de}

\begin{de}
  Zij $f:\ A \subseteq \mathbb{R} \rightarrow \mathbb{R}$ een functie en $A \subseteq \mathbb{R}$ zodat $-\infty$ een ophopinspunt is van $A$.
  We noemen de \term{limiet} van $f$ in $-\infty$ ...
  \begin{itemize}
  \item ... $L\in \mathbb{R}$ als het volgende geldt:
    \[
    \lim_{x\rightarrow -\infty}f(x) = L \quad\Leftrightarrow\quad
    \forall \epsilon \in \mathbb{R}_{0}^{+}: \exists N \in \mathbb{R}: \forall x\in A:\ x > N \Rightarrow |f(x) - L| < \epsilon
    \]
  \item ... $+\infty$ als het volgende geldt:
    \[
    \lim_{x\rightarrow -\infty}f(x) = +\infty\quad\Leftrightarrow\quad
    \forall M \in \mathbb{R}: \exists N \in \mathbb{R}: \forall x\in A:\ x < N \Rightarrow f(x) > M
    \]
  \item ... $-\infty$ als het volgende geldt:
    \[
    \lim_{x\rightarrow -\infty}f(x) = -\infty\quad\Leftrightarrow\quad
    \forall M \in \mathbb{R}: \exists N \in \mathbb{R}: \forall x\in A:\ x < N \Rightarrow f(x) < M
    \]
  \end{itemize}
\end{de}

\begin{pr}
  \label{pr:limiet-van-functie-asa-limiet-van-beeld-van-rij}
  Beschouw een functie $f:\ A \subseteq \mathbb{R} \rightarrow \mathbb{R}$ en een $L \in \mathbb{R} \cup \{-\infty,+\infty\}$.
  Zij $a \in \mathbb{R} \cup \{-\infty,+\infty\}$ een ophopingspunt van $A$.
  De limiet van $f(x)$ in $a$ is $L$ als en slechts als $L$ ook de limiet is van het beeld van elke rij $(x_{n})_{n}$ in $A\setminus\{a\}$ die $a$ als limiet heeft.

  \begin{proof}
    Gevalsonderscheid.
    \begin{itemize}
    \item $a = -\infty$
      \begin{itemize}
      \item $L = -\infty$
\extra{bewijs}
      \item $L \in \mathbb{R}$
\extra{bewijs}
      \item $L = +\infty$
\extra{bewijs}
      \end{itemize}
    \item $a\in\mathbb{R}$
      \begin{itemize}
      \item $L = -\infty$
\extra{bewijs}
      \item $L \in \mathbb{R}$
        \begin{itemize}
        \item $\Rightarrow$
          Zij $(x_{n})_{n}$ een rij in $A\setminus \{a\}$ met $a$ als limiet.
          Kies een willekeurige $\epsilon \in \mathbb{R}_{0}^{+}$.
          Omdat de limiet van $f$ in $a$ $L$ is, bestaat er dan een $\delta \in \mathbb{R}_{0}^{+}$ zodat uit voor alle $x\in A$ uit $|x-a|<\delta$ volgt dat $|f(x)-L|<\epsilon$ geldt.
          Omdat de rij convergeert naar $a$, bestaat er een $n_{0}\in \mathbb{N}$ zodat voor alle volgende $n\in\mathbb{N}$ geldt dat $|x_{n}-a|$ kleiner is dan $\delta$.
          Vanaf die $n_{0}$ is $|f(x_{n})-L|$ dus kleiner dan $\epsilon$.
        \item $\Leftarrow$
        \end{itemize}
      \item $L = +\infty$
\extra{bewijs}
      \end{itemize}
    \item $a = +\infty$
      \begin{itemize}
      \item $L = -\infty$
\extra{bewijs}
      \item $L \in \mathbb{R}$
\extra{bewijs}
      \item $L = +\infty$
\extra{bewijs}
      \end{itemize}
    \end{itemize}
  \end{proof}
\end{pr}

\begin{de}
  Zij $f:\ A \subseteq \mathbb{R} \rightarrow \mathbb{R}$ een functie en $a\in \mathbb{R}$.
  \begin{itemize}
  \item Als $a$ een ophopingspunt is van $\interval[open left]{-\infty}{a} \cap A$ zeggen we dat $f$ een \term{linkerlimiet} heeft in $a$ als de beperking van $f$ tot $\interval[open left]{-\infty}{a} \cap A$ een limiet $L \in \mathbb{R} \cup \{ -\infty, +\infty \}$ heeft.
    \[ \lim_{x \overset{<}{\rightarrow} a}f(x) = L \quad\text{of}\quad \lim_{x \rightarrow a^{-}}f(x) = L \quad\text{of}\quad f(a^{-}) = L \]
  \item Als $a$ een ophopingspunt is van $\interval[open right]{a}{+\infty} \cap A$ zeggen we dat $f$ een \term{rechterlimiet} heeft in $a$ als de beperking van $f$ tot $\interval[open right]{a}{+\infty} \cap A$ een limiet $L \in \mathbb{R} \cup \{ -\infty, +\infty \}$ heeft.
    \[ \lim_{x \overset{>}{\rightarrow} a}f(x) = L \quad\text{of}\quad \lim_{x \rightarrow a^{+}}f(x) = L \quad\text{of}\quad f(a^{+}) = L \]
  \end{itemize}
\end{de}

\begin{st}
  Een equivalente definities voor een linker- en rechterlimiet.\\
  Zij $f:\ A \subseteq \mathbb{R} \rightarrow \mathbb{R}$ een functie en $a\in\mathbb{R}$ een ophopingspunt van $A$.
  \begin{itemize}
  \item De linker limiet van $f$ in $A$ noemen we $L$ als het volgende geldt:
    \[ \forall \epsilon\in\mathbb{R}_{0}^{+},\ \exists \delta \in \mathbb{R}_{0}^{+}:\ \forall x\in A:\ a-\delta<x\le a \Rightarrow |f(x)-L|<\epsilon \]
  \item De rechter limiet van $f$ in $A$ noemen we $L$ als het volgende geldt:
    \[ \forall \epsilon\in\mathbb{R}_{0}^{+},\ \exists \delta \in \mathbb{R}_{0}^{+}:\ \forall x\in A:\ a\le x < a+\delta \Rightarrow |f(x)-L|<\epsilon\]
  \end{itemize}
  \extra{bewijs}
\end{st}

\subsection{Classificatie van discontinuiteiten}

\begin{pr}
  \label{pr:functie-continu-asa-limiet-is-beeld}
  Zij $f:\ A \subseteq \mathbb{R} \rightarrow \mathbb{R}$.
  Stel dat $a$ zowel een element van $A$ als een ophopingspunt van $A$ is.
  $f$ is continu in $a$ als de limiet van $f$ in $a$ $f(a)$ is.

  \begin{proof}
    $f$ is continu in $a$, dus voor elke rij $(x_{n})_{n}$ in $A$ die naar $a$ convergeert geldt dat $(f(x_{n}))_{n}$ naar $f(a)$ convergeert.\prref{pr:continu-asa-behoudt-convergentie}
    Dit is equivalent met de stelling dat $f(a)$ de limiet is van $f$ in $a$.\prref{pr:limiet-van-functie-asa-limiet-van-beeld-van-rij}
  \end{proof}
\end{pr}

\begin{pr}
  Zij $f:\ A \subseteq \mathbb{R} \rightarrow \mathbb{R}$ en $a\in \mathbb{R}$.
  Stel dat $a$ een ophopingspunt is van zowel $\interval[open right]{a}{+\infty} \cap A$ als $\interval[open left]{-\infty}{a} \cap A$.
  $f$ heeft in $a$ een limiet $L\in \mathbb{R}\cup \{ -\infty,+\infty\}$ als en slechts als de linker- en rechterlimiet van $f$ in $a$ bestaan en gelijk zijn aan $L$.

  \begin{itemize}
  \item $\Rightarrow$\\
    \extra{bewijs}
  \item $\Leftarrow$\\
    Stel dat $f$ zowel een linker- als een rechterlimiet heeft in $a$ en ze beide gelijk zijn aan $L$.
    Kies dan een willekeurige $\epsilon\in\mathbb{R}_{0}^{+}$.
    Er bestaat dan een $\delta_{l}$ en een $\delta_{r}$ als volgt.
    \[ \forall x\in A:\ a-\delta_{l}<x\le a \Rightarrow |f(x)-L|<\epsilon \]
    \[ \forall x\in A:\ a\le x < a+\delta_{r} \Rightarrow |f(x)-L|<\epsilon \]
    Noem nu $\delta = \min\{\delta_{1},\delta_{2}\}$ zodat het volgende geldt.
    \[ \forall x\in A:\ a\le |x-a|<\delta \Rightarrow |f(x)-L|<\epsilon \]
    $f$ heeft dus $L$ als limiet in $a$.
  \end{itemize}
\end{pr}

\begin{de}
  Beschouw een functie $f:\ A \subseteq \mathbb{R} \rightarrow \mathbb{R}$ en $a\in A$.
  Stel dat $f$ niet continu is in $a$, dan karakteriseren we deze discontinu\"eteit als volgt:
  \begin{itemize}
  \item Als de limiet van $f$ naar $a$ bestaat en eindig is, noemen we $a$ een \term{ophefbare discontinu\"iteit}.
  \item Als $a$ een ophopingspunt is van zowel $\interval[open right]{a}{+\infty} \cap A$ als $\interval[open left]{-\infty}{a} \cap A$ en als zowel de linker als de rechterlimiet van $f$ in $a$ bestaan, maar ze zijn niet gelijk, dan noem men $a$ een \term{discontinu\"iteit van de eerste soort} of \term{sprongdiscontinu\"iteit}.
  \item Als $a$ een ophopingspunt is van $\interval[open left]{-\infty}{a} \cap A$, maar de limiet van $f$ in $a$ niet bestaat, of als $a$ een ophopingspunt is van $\interval[open right]{a}{+\infty}$, maar de limiet van $f$ in $a$ niet bestaat, dan noemt men $a$ een \term{discontinu\"iteit van de tweede soort} of \term{essentie\"ele discontinu\"iteit}.
  \end{itemize}
\end{de}

\begin{pr}
  Zij $f:\ I \subseteq \mathbb{R} \rightarrow \mathbb{R}$ een monotone functie op een (niet-leeg) open interval $I$, dan kan $f$ enkel discontinu\"iteiten van de eerste soort vertonen.
  Bovendien is het aantal discontinu\"iteiten van $f$ aftelbaar.

  \begin{proof}
    Gevalsonderscheid:
    \begin{itemize}
    \item $f$ is monotoon stijgend.
      \begin{itemize}
      \item De linkerlimiet van $f$ in elk punt $a\in I$ bestaat en is eindig
        Die linkerlimiet wordt in het bijzonder gegeven als volgt:
        \[ \lim_{x \overset{<}{\rightarrow} a}f(x) = \sup\{ f(x)\mid x\in I \wedge x < a \} \]
        De verzameling $\{ f(x)\mid x\in I \wedge x < a \}$ is immers een niet-lege, naar boven begrensde verzameling.
        Noem $s$ het supremum van die verzameling.
        Kies een willekeurige $\epsilon \in \mathbb{R}_{0}^{+}$, dan
        bestaat er een $y\in I$ groter dan $a$ zodat $f(y)$ tussen
        $s-\epsilon$ en $s$ ligt.  Omdat $f$ stijgt, zal voor alle
        $x\in F$ tussen $y$ en $a$ gelden dat $f(x)$ tussen $f(y)$ en
        $s$ ligt.  Noem nu $\delta = a-y > 0$.  Voor alle
        $x\in I\cap\interval[open left]{-\infty}{a}$ met
        $0<|x-a|<\delta$ geldt dan dat $|f(x)-s|<\epsilon$ geldt.
      \item De rechterlimiet van $f$ in elk punt $a\in I$ bestaat en is eindig.
        Die rechterlimiet wordt in het bijzonder gegeven als volgt:
        \[ \lim_{x \overset{>}{\rightarrow} a}f(x) = \inf\{ f(x)\mid x\in I \wedge x > a \} \]
      \item Vermits de linker- en rechterlimiet van $f$ in elk punt van $I$ bestaan, kunnen we al besluiten dat $f$ enkel discontinu\"iteiten van de eerste soort kan hebben.
      \item Merk op dat de linkerlimiet kleiner of gelijk aan de rechterlimiet is in elk punt $a\in I$.
        Als en slechts als de limieten verschillend zijn is $f$ niet continu in $a$.
        (Omdat $f$ monotoon stijgt kan er geen ophefbare discontinu\"iteit zijn.)
        De verzameling $D$ van de discontinu\"iteiten van $f$ ziet er dus als volgt uit:
        \[ D = \left\{ a \in I \mid \lim_{x \overset{<}{\rightarrow} a}f(x)- \lim_{x \overset{>}{\rightarrow} a}f(x) > 0 \right\} \]
        We tonen aan dat $D$ afterlbaar is.\\
        Merk eerst op dat voor $k$ elementen $a_{i}$ uit $D$ tussen $c$ en $d$ in $I$ het volgende geldt:
        \[ \sum_{i=1}^{k}\left(\lim_{x \overset{<}{\rightarrow} a}f(x)- \lim_{x \overset{>}{\rightarrow} a}f(x)\right) \le f(d) - f(c) \]
        Omdat $I$ open is, kunnen we een strikt dalende rij $(c_{n})_{n}$ en een strikt stijgende rij $(d_{n})_{n}$ in $I$ nemen zodat $I$ de unie is van alle intervallen $\interval[open]{c}{d}$.
        Beschouw nu voor elk $n$ de deelverzameling $D_{n}$ van $D$.
        \[ D_{n} = \left\{ a \in \interval[open]{c_{n}}{d_{n}} \mid \left(\lim_{x \overset{<}{\rightarrow} a}f(x)- \lim_{x \overset{>}{\rightarrow} a}f(x)\right) > \frac{1}{n} \right\} \]
        Elke afstand $\left(\lim_{x \overset{<}{\rightarrow} a}f(x)- \lim_{x \overset{>}{\rightarrow} a}f(x)\right)$ is groter dan $\frac{1}{n}$, de som is dus groter dan $\frac{\#D_{n}}{n}$.
        De som is echter ook kleiner dan $f(d_{n})-f(c_{n})$, om tot de volgende ongelijkheid te komen.
        \[ \#D_{n} \le n(f(d_{n})-f(c_{n})) \]
        $\#D_{n}$ is dus eindig.
        $D$ is een aftelbare unie van eindige verzamelingen $D_{n}$ en daarom aftelbaar.
      \end{itemize}
    \item $f$ is monotoon dalend.\\
      \extra{bewijs}
    \end{itemize}
  \end{proof}
\end{pr}


\subsection{Eigenschappen en rekenregels voor limieten}
\label{sec:eigensch-en-rekenr}

\begin{pr}
  Beschouw functies $f,g:\ A \subseteq \mathbb{R} \rightarrow \mathbb{R}$ en zij $a\in \mathbb{R} \cup \{-\infty,+\infty\}$ een ophopingspunt van $A$.
  Stel dat $\forall a\in A: f(x) \le g(x)$ geldt, en dat de limiet van zowel $f$ als $g$ in $a$ bestaat, dan is de limiet van $f$ in $a$ kleiner dan of gelijk aan de limiet van $g$ in $a$.\
\extra{bewijs}
\end{pr}

\begin{st}
  De \term{insluitstelling voor functies}\\
  Beschouw functies $f,g,h:\ A \subseteq \mathbb{R} \rightarrow \mathbb{R}$ en zij $a\in \mathbb{R} \cup \{-infty,+\infty\}$ een ophopingspunt van $A$.
  Stel dat $\forall a\in A: f(x) \le g(x) \le h(x)$ geldt, en dat de limiet van zowel $f$ als $h$ in $a$ bestaat en deze gelijk zijn, dan bestaat de limiet van $g$ in $x$ en zijn de drie limieten gelijk.
\extra{bewijs}
\end{st}

\begin{pr}
  Beschouw functies $f,g:\ A \subseteq \mathbb{R} \rightarrow \mathbb{R}$ en zij $a\in \mathbb{R} \cup \{-infty,+\infty\}$ een ophopingspunt van $A$.
  Zij $\lambda \in \mathbb{R}$ en stel dat de limieten van $f$ en $g$ in $a$ bestaan en eindig zijn.
  \begin{itemize}
  \item $\lim_{a}(\lambda f) = \lambda \lim_{a} f$
  \item $\lim_{a}(f+g) = \lim_{a}f + \lim_{a}g$
  \item $\lim_{a}(fg) = \lim_{a}f \lim_{a}g$
  \item $\lim_{a}\frac{f}{g} = \frac{\lim_{a}f}{\lim_{a}g}$ als $\lim_{a}g \neq 0$ met $\frac{f}{g}:\ \{x\in A\mid g(x) \neq 0\} \rightarrow \mathbb{R}:\ x \mapsto \frac{f(x)}{g(x)}$
  \end{itemize}
\extra{bewijs}
\end{pr}

\extra{zelfde extra rekenregels voor oneindigheden etc (veel werk)}

\begin{pr}
  Beschouw functies $f:\ A \subseteq \mathbb{R} \rightarrow B \subseteq \mathbb{R}$ en $g:\ B \rightarrow \mathbb{R}$.
  Zij $a\in \mathbb{R} \cup \{-infty,+\infty\}$ een ophopingspunt van $A$.
  Stel dat de limiet $b$ van $f$ in $a$ bestaat.
  \begin{itemize}
  \item Als $b$ in $B$ zit en $g$ continu is in $b$, dan bestaat de limiet van $g\circ f$ in $a$:
    \[ \lim_{a}g\circ f = g(\lim_{a}f) \]
  \item Als $b$ niet tot $B$ behoort, dan is $b$ een ophopingspunt van $B$ en de limiet van $g$ in $b$ bestaat, dan bestaat de limiet van $g\circ f$ in $a$:
    \[ \lim_{a}g\circ f = \lim_{a}g \]
  \end{itemize}

  \begin{proof}
    Gevalsonderscheid
    \begin{itemize}
    \item
      Kies een willekeurige rij $(x_{n})_{n}$ in $A\setminus\{a\}$ die naar $a$ convergeert.
      We bewijzen dat $((g\circ f)(x_{n}))_{n}$ naar $(g\circ f)(a)$ convergeert.
      Hieruit volgt dan de stelling.\prref{pr:limiet-van-functie-asa-limiet-van-beeld-van-rij}
      Omdat $b$ de limiet is van $f$ in $a$ weten we dat $(f(x_{n})_{n}$ een rij is die naar $b$ convergeert.
      Omdat $g$ continu is in $b$ geldt dan het volgende:
      \[ \lim_{n\rightarrow \infty}(g\circ f)(x_{n}) = \lim_{n\rightarrow \infty}g(f(x_{n})) = g(b) \]
      Hiermee is het volgende aangetoond:
      \[ \lim_{a}(g\circ f)=g(b) \]
    \item \extra{bewijs}
    \end{itemize}
  \end{proof}
\end{pr}

\begin{st}
  Zij $(f_{n})_{n}$ een rij van functies van $A \subseteq \mathbb{R}$ naar $\mathbb{R}$ die uniform op $A$ convergeert naar een functie $f: A \rightarrow \mathbb{R}$.
  Zij $a\in \mathbb{R} \cup \{-\infty,+\infty\}$ een ophopingspunt van $A$.
  Stel dat voor alle $n$ de limiet $L_{n}$ van $f_{n}$ in $a$ bestaat en eindig is.
  \[ \lim_{a}f = \lim_{n \rightarrow +\infty} L_{n} \]
\TODO{bewijs p 43}
\end{st}












\end{document}

%%% Local Variables:
%%% mode: latex
%%% TeX-master: t
%%% End:
