\documentclass[main.tex]{subfiles}
\begin{document}

\chapter{Continu\"iteit voor functies van $\mathbb{R}$ naar $\mathbb{R}$}
\label{cha:cont-in-r}

\section{Het continu\"iteitsbegrip}
\label{sec:het-cont}


\begin{de}
  Zij $f$ een functie en $a\in A$:
  \[ f:\ A \subseteq \mathbb{R} \rightarrow \mathbb{R}:\ x \mapsto f(x) \]
  We noemen $f$ \term{continu} in $a$ als en slechts als het volgende geldt:
  \[ \forall \epsilon \in \mathbb{R}_{0}^{+}:\ \epsilon \in \mathbb{R}_{0}^{+}:\ \forall x\in A:\ |x-a| < \delta \Rightarrow |f(x) -f(a)| < \epsilon \]
  We noemen $f$ \term{continu} op $A$ als $f$ continu is in elke $a\in A$.
\end{de}

\begin{st}
  Equivalente definitie van \term{continu\"iteit}:\\
  $f$ is continu in $a$ als en slechts als het volgende geldt:
  \[ \forall \epsilon \in \mathbb{R}_{0}^{+}:\ \epsilon \in \mathbb{R}_{0}^{+}:\ \forall x\in A:\ f(\interval[open]{a-\delta}{a+\delta} \cap A) \subseteq \interval[open]{f(a)-\epsilon}{f(a)+\epsilon} \]
\extra{bewijs}
\end{st}

\begin{pr}
  Zij $f:\ A \subseteq \mathbb{R} \rightarrow \mathbb{R}$ een functie en $a\in A$.
  $f$ is continu in $a$ als en slechts als er voor elke rij $(x_{n})_{n}$ in $A$ die naar $a$ convergeert dat $(f(x_{n}))_{n}$ naar $f(a)$ convergeert.
\TODO{bewijs p 6}
\end{pr}

\begin{pr}
  Zij $f:\ A \subseteq \mathbb{R} \rightarrow \mathbb{R}$ een functie en $a\in A$.
  $f$ is continu op $A$ als er voor elk open deel $V$ van $\mathbb{R}$ geldt dat $f^{-1}(V)$ relatief open is in $A$.
\TODO{bewijs p 7}
\end{pr}

\begin{de}
  Zij $f:\ A \subseteq \mathbb{R} \rightarrow \mathbb{R}$ een functie en $a\in A$.
  We noemen $f$ ...
  \begin{itemize}
  \item ...\term{linkscontinu} in $a$ als en slechts als de beperking van $f$ tot $\interval[open left]{-\infty}{a} \cap A$ continu is in $a$.
  \item ...\term{rechtscontinu} in $a$ als en slechts als de beperking van $f$ tot $\interval[open right]{a}{+\infty} \cap A$ continu is in $a$.
  \end{itemize}
\end{de}

\section{Operaties met continue functies}
\label{sec:oper-met-cont}

\begin{pr}
  Zij $f:\ A \subseteq \mathbb{R} \rightarrow \mathbb{R}$ een functie die continu is in $a\in A$.
  \[ \lambda f:\ A \rightarrow \mathbb{R}: x \mapsto \lambda f(x) \text{ is continu in } A \]
\TODO{bewijs: oefening}
\end{pr}

\begin{pr}
  Zij $f,g:\ A \subseteq \mathbb{R} \rightarrow \mathbb{R}$ functies die continu zijn in $a\in A$.
  \[ f+g:\ A \rightarrow \mathbb{R}: x \mapsto \lambda f(x)+g(x) \text{ is continu in } A \]
\TODO{bewijs: oefening}
\end{pr}

\begin{pr}
  Zij $f,g:\ A \subseteq \mathbb{R} \rightarrow \mathbb{R}$ functies die continu zijn in $a\in A$.
  \[ fg:\ A \rightarrow \mathbb{R}: x \mapsto \lambda f(x)g(x) \text{ is continu in } A \]
\TODO{bewijs p 10}
\end{pr}

\begin{pr}
  Zij $f,g:\ A \subseteq \mathbb{R} \rightarrow \mathbb{R}$ functies die continu zijn in $a\in A$.
  Noteer bovendien $A_{0} = \{ x \in A \mid g(x) \neq 0 \}$
  \[ \frac{f}{g}:\ A_{0} \rightarrow \mathbb{R}: x \mapsto \lambda \frac{f(x)}{g(x)} \text{ is continu in } A \]
\TODO{bewijs p 10}
\end{pr}

\begin{pr}
  Zij $f:\ A \subseteq \mathbb{R} \rightarrow B \subseteq \mathbb{R}$ en $g:\ B \rightarrow \mathbb{R}$ functies en zij $a\in A$.
  Als $f$ continu is in $a$ en $g$ continu in $f(a)$, dan is $g\circ f$ continu in $a$.
\TODO{bewijs p 11}
\end{pr}

\begin{pr}
  Zij $A$ een gesloten begrensd deel van $\mathbb{R}$.
  Zij $f: \ A \subseteq \mathbb{R} \rightarrow \mathbb{R}$ een continue injectieve functie, dan is $f^{-1}: f(A) \rightarrow A$ ook continu.
\TODO{bewijs p 12}
\end{pr}

\section{Continue functies op intervallen}
\label{sec:continue-functies-op}

\begin{st}
  Beschouw een continue functie $f: \interval[open]{a}{b} \rightarrow \mathbb{R}$ gedefinieerd op een gesloten begrensd interval $\interval[open]{a}{b}$.
  $f$ is begrensd en bereikt op $\interval[open]{a}{b}$ haar minimale en maximale waarde.
  Er bestaan dus waarden $c,d \in \interval[open]{a}{b}$ als volgt:
  \[ f(c) = \sup\left\{ f(x) \mid x \in \interval[open]{a}{b} \right\} \quad\text{ en }\quad f(d) = \sup\left\{ f(x) \mid x \in \interval[open]{a}{b} \right\} \]
\TODO{bewijs p 14}
\question{waar wordt de supremumeigenschap gebruikt?}
\end{st}

\begin{st}
  \label{st:tussenwaardestelling}
  De \term{tussenwaardestelling}\\
  Zij $f: I \subseteq \mathbb{R} \rightarrow \mathbb{R}$ een continue functie op een interval $I$.
  Zij $x_{1},x_{2}\in I$ en noem $y_{i}=f(x_{i})$.
  \[ \forall y \in \interval[open]{y_{1}}{y_{2}}:\ \exists x \in I:\ f(x) = y \]
\TODO{bewijs p 15}
\end{st}

\TODO{bissectiemethode?}

\section{Uniforme continuiteit}
\label{sec:unif-cont}

\begin{st}
  Zij $f:\ \mathbb{R} \rightarrow \mathbb{R}$ een continue functie en $(x_{n})_{n}$ een Cauchyrij in $\mathbb{R}$, dan is $(f(x_{n}))_{n}$ ook een Cauchyrij.
\extra{bewijs}
\end{st}

\begin{de}
  We noemen een functie $f: A \subseteq \mathbb{R} \rightarrow \mathbb{R}$ \term{uniform continu} of \term{gelijkmatig continu} op $A$ als het volgende geldt:
  \[ \forall \epsilon \in \mathbb{R}_{0}^{+}:\ \exists \delta in \mathbb{R}_{0}^{+}:\ \forall x,y \in A:\ |x-y| < \delta \Rightarrow |f(x)-f(y)| < \epsilon \]
\end{de}

\begin{st}
  Als een functie $f: A \subseteq \mathbb{R} \rightarrow \mathbb{R}$ uniform continu is, is hij ook continu.
\extra{bewijs}       
\end{st}

\begin{tvb}
  Een continue functie $f: A \subseteq \mathbb{R} \rightarrow \mathbb{R}$ is niet noodzakelijk uniform continu
\extra{tegenvoorbeeld}
\end{tvb}

\begin{pr}
  Zij $f: A \subseteq \mathbb{R} \rightarrow \mathbb{R}$ een uniform continue functie en $(x_{n})_{n}$ een Cauchyrij in $\mathbb{R}$, dan is $(f(x_{n}))_{n}$ ook een Cauchyrij.
\TODO{bewijs: oefening}
\end{pr}

\begin{pr}
  Zij $f: A \subseteq \mathbb{R} \rightarrow \mathbb{R}$ een uniform continue functie.
  Als $B$ een begrensd deel is van $A$, dan is $f(B)$ ook begrensd.
\TODO{bewijs p 23}
\end{pr}

\begin{st}
  Zij $f: A \subseteq \mathbb{R} \rightarrow \mathbb{R}$ een continue functie, gedefinieerd op een gesloten en begrensd deel $A$ van $\mathbb{R}$, dan is $f$ uniform continu.
\TODO{bewijs p 23}
\end{st}

\section{Rijen van functies}
\label{sec:rijen-van-functies}

\begin{de}
  Beschouw een rij $(f_{n})_{n}$ van functies op een verzameling $A$ met waarden in $\mathbb{R}$ (of $\mathbb{C}$).
  We zeggen dat $(f_{n})_{n}$ \term{puntsgewijs convergeert} op $A$ naar een functie $f: A \rightarrow \mathbb{R}$ als het volgende geldt:
  \[ \forall x\in A:\ \epsilon \in \mathbb{R}_{0}^{+}:\ \exists n_{0}\in \mathbb{N}:\ \forall n\in \mathbb{n}:\ n \ge n \Rightarrow |f_{n}(x)-f(x)| < \epsilon \]
\end{de}

\begin{de}
  Beschouw een rij $(f_{n})_{n}$ van functies op een verzameling $A$ met waarden in $\mathbb{R}$ (of $\mathbb{C}$).
  We zeggen dat $(f_{n})_{n}$ \term{uniform convergeert} of \term{gelijkmatig convergeert} naar een functie $f: A \rightarrow \mathbb{R}$ als het volgende geldt:
  \[ \forall \epsilon \in \mathbb{R}_{0}^{+}:\ \exists n_{0} \in \mathbb{N}:\ \forall x \in A:\ \forall n\in \mathbb{N}:\ n \ge n_{0} \Rightarrow |f_{n}(x)-f(x)| < \epsilon \]
\end{de}

\begin{st}
  Een rij functies $(f_{n})_{n}$ die uniform convergeert, convergeert puntsgewijs.
\extra{bewijs}
\end{st}

\begin{tvb}
  Een rij functies $(f_{n})_{n}$ die puntsgewijs convergeert, convergeert niet noodzakelijk uniform.
\extra{tegenvoorbeeld}
\end{tvb}

\begin{st}
  Beschouw een rij $(f_{n})_{n}$ van functies $f_{n}: A \subseteq \mathbb{R} \rightarrow \mathbb{R}$ die op $A$ uniform convergeert naar een functie $f:\ A \rightarrow \mathbb{R}$.
  \begin{itemize}
  \item Als $f_{n}$ continu is in een $a\in A$ voor elke $n$, dan is ook $f$ continu in $a$.
  \item Als $f_{n}$ uniform continu is op $A$ voor elke $n$, dan is ook $f$ uniform continu op $A$. 
  \end{itemize}

\TODO{bewijs p 33}  
\end{st}

\section{Limieten van functies}
\label{sec:limi-van-funct}





















\end{document}
