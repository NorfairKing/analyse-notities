\documentclass[main.tex]{subfiles}
\begin{document}



\section{Eigenschappen van de Riemannintegraal}
\label{sec:eigenschappen-van-de}

\subsection{Eigenschappen in verband met integreerbaarheid}
\label{sec:eigensch-verb-met}

\begin{bpr}
  \label{pr:integreerbaar-itv-verdelingen}
  Een begrensde functie $f:\ \interval{a}{b} \rightarrow \mathbb{R}$ is Riemannintegreerbaar als en slechts als er voor elke $\epsilon > 0$ een verdeling $P$ van $\interval{a}{b}$ bestaat als volgt:
  \[ \overline{S}(f,P) - \underline{S}(f,P) < \epsilon \]

  \begin{proof}
    Bewijs van een equivalentie.
    \begin{itemize}
    \item $\Rightarrow$\\
      Stel $I = \int_{a}^{b}f$.
      Kies willekeurig een $\epsilon\in\mathbb{R}_{0}^{+}$.
      Omdat $I$ gelijk is aan $\sup_{P}\underline{S}(f,P)$, bestaat er een verdeling $P_{1}$ van $\interval{a}{b}$ als volgt:
      \[ \underline{S}(f,P_{1}) > I-\frac{\epsilon}{2} \]
      Omdat $I$ gelijk is aan $\inf_{P}\underline{S}(f,P)$, bestaat er een verdeling $P_{2}$ van $\interval{a}{b}$ als volgt:
      \[ \overline{S}(f,P_{2}) < I+\frac{\epsilon}{2} \]
      Kies nu een verdeling $P$, fijner dan $P_{1}$ en fijner dan $P_{2}$.\stref{st:verdeling-kan-altijd-nog-fijner}, dan geldt het volgende:\prref{pr:fijnere-verdeling-betere-afschatting}
      \[ \overline{S}(f,P) - \underline{S}(f,P) \le \overline{S}(f,P_{2}) - \underline{S}(f,P_{1}) < \left(I + \frac{\epsilon}{2}\right) - \left( I - \frac{\epsilon}{2} \right) = \epsilon \]
    \item $\Leftarrow$\\
      Er geldt onmiddeling het volgende voor alle $\epsilon\in\mathbb{R}_{0}^{+}$
      \[ 0 \le \overline{S}(f) - \underline{S}(f) < \epsilon \]
      Dit betekent dat $\overline{S}(f)$ en $\underline{S}(f)$ gelijk zijn.
    \end{itemize}
  \end{proof}
\end{bpr}



\begin{bpr}
  Zij $f:\ \interval{a}{b} \rightarrow \mathbb{R}$ een continue functie, dan is $f$ Riemannintegreerbaar.

  \begin{proof}
    Omdat $f$ continu is op $\interval{a}{b}$, zal $f$ zeker begrensd zijn.\needed
    Bovendien weten we dat $f$ uniform continu is.\needed
    Kies nu een willekeurige $\epsilon\in\mathbb{R}_{0}^{+}$.
    We kunnen dan een $\delta\in\mathbb{R}_{0}^{+}$ vinden als volgt:
    \[ \forall s,t\in\interval{a}{b}:\ |s-t| < \delta \Rightarrow \left|f(s)-f(t)\right| < \frac{\epsilon}{b-a} \]
    Kies nu een $n\in\mathbb{N}_{0}$ groot genoeg zodat $\frac{b-a}{n}$ strikt kleiner is dan $\delta$ en beschouw de verdeling $P$ die het interval $\interval{a}{b}$ in $n$ deelintervallen van gelijke lengte verdeelt.
    Vermits de lengte van elk deelinterval kleiner is dan $\delta$, zal voor elk deelinterval gelden dat het verschil tussen het supremum en het infimum van de waarden erin kleiner is dan, of gelijk aan $\frac{\epsilon}{b-a}$.
    Hieruit volgt dan meteen dit:
    \[ \overline{S}(f,P) - \underline{S}(f,P) \le \epsilon \]
  \end{proof}
\end{bpr}

\begin{st}
  Een begrensde functie met een eindig aantal discontinu\"iteiten is Riemannintegreerbaar.
\TODO{zie oefening 1 van reeks 2.7}
\end{st}

\begin{bpr}
  Zij $f:\ \interval{a}{b} \rightarrow \mathbb{R}$ een begrensde functie en zij $c\in \interval{a}{b}$.
  Noteer met $f_{1}$ de beperking van $f$ tot $\interval{a}{c}$ en met $f_{2}$ de beperking van $f$ tot $\interval{c}{b}$.
  Volgende uitspraken zijn dan equivalent:
  \begin{enumerate}
  \item $f$ is Riemannintegreerbaar over $\interval{a}{b}$
  \item $f_{1}$ en $f_{2}$ zijn Riemannintegreerbaar over $\interval{a}{c}$, respectievelijk $\interval{c}{b}$.
  \end{enumerate}
  Bovendien geldt het volgende:
  \[ \int_{a}^{b}f = \int_{a}^{c}f_{1} + \int_{c}^{b}f_{2} \]

  \begin{proof}
    Bewijs van een equivalentie.
    \begin{itemize}
    \item $(1)\Rightarrow (2)$\\
      Merk op dat $f_{1}$ en $f_{2}$ begrensd zijn als beperking van een begrensde functie.
      Kies nu een willekeurige $\epsilon\in\mathbb{R}_{0}^{+}$. Omdat $f$ Riemannintegreerbaar is,k unnen we een verdeling $P$ van $\interval{a}{b}$ nemen zodat $\overline{S}(f,P) - \underline{S}(f,P) < \epsilon$ geldt.
      Vermits het verschil tussen bovensom en ondersom niet vergroot als we de verdeling verfijnen, kunnen we steeds veronderstellen dat $c$ tot $P$ behoort.
      Stel nu $P_{1} = P \cap \interval{a}{c}$ en $P_{2} = P \cap \interval{c}{b}$.
      $P_{1}$ is dan een verdeling van $\interval{a}{c}$ en $P_{2}$ een verdeling van $\interval{c}{b}$.
      Bovendien geldt het volgende:
      \[ \overline{S}(f,P) = \overline{S}(f,P_{1}) + \overline{S}(f,P_{2}) \quad\text{en}\quad \underline{S}(f,P) = \underline{S}(f,P_{1}) + \underline{S}(f,P_{2})\]
      Voor beide delen geldt dus het volgende:
      \[ \overline{S}(f_{i},P_{i}) - \underline{S}(f_{i},P_{i}) \le \overline{S}(f,P) - \underline{S}(f,P) < \epsilon \]
      $(2)$ is dus voldaan.
    \item $(2)\Rightarrow (1)$\\
    \end{itemize}
    We tonen tenslotte de laatste gelijkheid aan.
    \begin{align*}
      \left| \int_{a}^{b}f - \int_{a}^{c}f_{1} - \int_{c}^{b}f_{2} \right|
      &\le \left| \int_{a}^{b}f - \underline{S}(f,P) \right| + \left|\underline{S}(f_{1},P_{1})- \int_{a}^{c}f_{1}\right| + \left|\underline{S}(f_{2},P_{2}) - \int_{c}^{b}f_{2} \right|
      &\le 3 \epsilon
    \end{align*}
    Vermits hierin $\epsilon \in \mathbb{R}_{0}^{+}$ willekeurig gekozen kan worden imliceert het de laatste gelijkheid.
  \end{proof}
\end{bpr}


\subsection{Lineariteit van de integraal}
\label{sec:lineariteit-van-de}

\begin{bpr}
  \label{pr:optelling-behoudt-integreerbaarheid}
  Zij $f,g:\ \interval{a}{b} \rightarrow \mathbb{R}$ Riemannintegreerbare functies, dan is $f+g$ is Riemannintegreerbaar en geldt het volgende:
  \[ \int_{a}^{b}(f+g) = \int_{a}^{b}f + \int_{a}^{b}g \]

  \begin{proof}
    Merk eerst op dat de som van begrensde functies opnieuw begrensd is.
    Het is bovendien niet moeilijk om in te zien dat voor elke verdeling $P$ van $\interval{a}{b}$ het volgende geldt:
    \[ \underline{S}(f,P) + \underline{S}(g,P) \le \underline{S}(f+g,P) \le \overline{S}(f+g,P) \le \overline{S}(f,P) + \overline{S}(g,P) \]
\TODO{details uitwerken}
  \end{proof}
\end{bpr}

\begin{bpr}
  Zij $f:\ \interval{a}{b} \rightarrow \mathbb{R}$ een Riemannintegreerbare functie, en $\lambda \in \mathbb{R}$, dan is $\lambda f$ is Riemannintegreerbaar en geldt $\int_{a}^{b}(\lambda f) = \lambda \int_{a}^{b}f$.

  \begin{proof}
    Het volstaat om dit aan te tonen in de gevallen $\lambda \in\mathbb{R}_{0}^{+}$ en $\lambda = -1$.\TODO{bewijs dat dit klopt}
    Als $\lambda$ strikt positief is, geldt het volgende:
    \[ \underline{S}(\lambda f,P) = \lambda \underline{S}(f,P) \quad\text{en}\quad \overline{S}(\lambda f,P) = \lambda \overline{S}(f,P) \]
    \TODO{werk de details uit}
    Anderzijds geldt ook het volgende.
    \TODO{hieruit volgt meteen hetzelfde voor $\lambda = -1$}
  \end{proof}
\end{bpr}

\begin{de}
  Noteer met $\mathcal{R}(\interval{a}{b})$ de verzameling van de Riemannintegreerbare functies $f:\ \interval{a}{b} \rightarrow \mathbb{R}$.
\end{de}

\begin{bgev}
  $\mathcal{R}(\interval{a}{b})$ in een re\"ele vectorruimte en de volgende afbeelding is lineair:
  \[ \int_{a}^{b}:\ \mathcal{R}(\interval{a}{b}) \rightarrow \mathbb{R}:\ f \mapsto \int_{a}^{b}f \]
\extra{bewijs}
\end{bgev}

\begin{st}
  Zij $f:\ \interval{a}{b} \rightarrow \mathbb{R}$ een Riemannintegreerbare functie.
  Zij $f'$ de functie $f$ met precies \'e\'en punt $y\in\interval{a}{b}$ verschillend:
  \[ f(y) \neq f'(y) \wedge \forall x\in \interval{a}{b}:\ f(x) = f'(x) \]
  $f'$ is dan ook Riemannintegreerbaar met dezelfde integraal.

  \begin{proof}
    Het volstaat om te bewijzen dat de volgende functie $g$ integreerbaar is met integraal $0$:
    \[
    g:\ \interval{a}{b} \rightarrow \mathbb{R}:\ x \mapsto
    \begin{cases}
      0 &\text{ als } x \neq y\\
      f'(y)-f(y) &\text{ als } x = y
    \end{cases}
    \]
    Kies een willekeurige $\epsilon\in\mathbb{R}_{0}^{+}$.\prref{pr:integreerbaar-itv-verdelingen}
    Voor elke verdeling zal de ondersom $0$ zijn.
    We zoeken dus een verdeling zodat de bovensom kleiner is dan $\epsilon$.
    Noem $M = f'(y) - f(y)$, en kies een verdeling als volgt:
    \[ P = \left\{ a, y-\frac{\epsilon}{2M}, y+\frac{\epsilon}{2M}, b\right\} \]
    Dit is een verdeling van het interval in $3$ stukken.
    Het linker- en het rechterdeel krijgen een bovensom van $0$.
    We bekijken daarom het middelste deel.
    Het supremum van het middenste deel is $M$ en de lengte ervan $\frac{\epsilon}{2}$ dus de bovensom is hoogstens $\epsilon$.
    We kunnen $\overline{S}(f,P) - \underline{S}(f,P)$ arbitrair klein krijgen door de juiste verdeling te kiezen dus $f'$ is Riemannintegreerbaar.\prref{pr:optelling-behoudt-integreerbaarheid}
    De integraal is bovendien dezelfde als die van $f$.
  \end{proof}
\end{st}


\subsection{Integraal en orde}
\label{sec:integraal-en-orde}

\begin{bpr}
  \label{pr:integraal-behoudt-orde-1}
  Zij $f:\ \interval{a}{b} \rightarrow \mathbb{R}$ een Riemannintegreerbare functie.
  \[ \left( \forall x\in \interval{a}{b}:\ f(x) \ge 0 \right) \Rightarrow \int_{a}^{b} f \ge 0 \]

  \begin{proof}
    Dit volgt meteen uit het feit dat $\underline{S}(f,P)$ positief is voor elke verdeling $P$ als $f$ positief is.
  \end{proof}
\end{bpr}

\begin{bpr}
  \label{pr:integraal-behoudt-orde-2}
  Zij $f,g:\ \interval{a}{b} \rightarrow \mathbb{R}$ twee Riemannintegreerbare functies.
  \[ \left( \forall x\in \interval{a}{b}:\ f(x) \ge g(x) \right) \Rightarrow \int_{a}^{b} f \ge \int_{a}^{b}g \]

  \begin{proof}
    Dit volgt meteen uit de vorige stelling, toegepast op de functie $f-g$.\prref{pr:integraal-behoudt-orde-1}
  \end{proof}
\end{bpr}

\begin{bpr}
  Zij $f:\ \interval{a}{b} \rightarrow \mathbb{R}$ een Riemannintegreerbare functie en $m,M \in \mathbb{R}$
  \[ \left( \forall x\in \interval{a}{b}:\ m \le f(x) \le M \right) \Rightarrow m(b-a) \le \int_{a}^{b}f \le M(b-a) \]

  \begin{proof}
    Dit is een onmiddelijk gevolg van de definitie van de integraal.
  \end{proof}
\end{bpr}

\begin{bpr}
  \label{pr:absolute-waarde-bijna-door-integraal}
  Zij $f:\ \interval{a}{b} \rightarrow \mathbb{R}$ een Riemannintegeerbare functie, dan is $|f|$ ook Riemannintegreerbaar en geldt het volgende:
  \[ \left| \int_{a}^{b}f \right| \le \int_{a}^{b}|f| \]

  \begin{proof}
    Merk eerst op dat $|f|$ begrensd is omdat $f$ begrensd is.
    Beschouw een willekeurige verdelinvg $P = \{x_{0},x_{1},\dotsc,x_{n}\}$ van $\interval{a}{b}$ en benoem de volgende symbolen:
    \[ m_{k} = \inf\{ f(x) \mid x\in \interval{x_{k-1}}{x_{k}} \} \]
    \[ M_{k} = \sup\{ f(x) \mid x\in \interval{x_{k-1}}{x_{k}} \} \]
    \[ a_{k} = \inf\{ |f(x)| \mid x\in \interval{x_{k-1}}{x_{k}} \} \]
    \[ A_{k} = \sup\{ |f(x)| \mid x\in \interval{x_{k-1}}{x_{k}} \} \]
    Er geldt het volgende volgens de teweede driehoeksongelijkheid.
    \[ \forall k\in\{1,2,\dotsc,n\}, \forall x,y\in\interval{x_{k-1}}{x_{k}}:\ A_{k}-a_{k} \le M_{k}-m_{k} \]
    Er geldt daarom het volgende:
    \[ \overline{S}(|f|,P) - \underline{S}(|f|,P) \le \overline{S}(f,P) - \underline{S}(f,P) \]
    Vermits deze ongelijkheid geldt voor elke verdeling $P$ impliceert de Riemannintegreerbaarheid van $f$ die van $|f|$.
    Om nu de ongelijkheid aan te tonen volstaat het op te merken dat $-|f| \le f \le |f|$ geldt.\prref{pr:integraal-behoudt-orde-1}\prref{pr:integraal-behoudt-orde-2}
  \end{proof}
\end{bpr}

\begin{de}
  Zij $f:\ \interval{a}{b} \rightarrow \mathbb{R}$ een Riemannintegreerbare functie.
  Het \term{gemiddelde} van $f$ over $\interval{a}{b}$ wordt gedefinieerd als volgt:
  \[ \frac{1}{b-a} \int_{a}^{b}f \]
\end{de}

\begin{bst}
  De \term{middelwaardestelling voor de integraal}\\
  Zij $f:\ \interval{a}{b} \rightarrow \mathbb{R}$ een continue functie, dan bestaat er minstens \'e\'en $c\in \interval{a}{b}$ waarvoor het volgende geldt:
  \[ f(c) = \frac{1}{b-a} \int_{a}^{b}f \]

  \begin{proof}
    Omdat $f$ continu is, bestaan er getallen $x_{1},x_{2}\in\interval{a}{b}$ als volgt:\stref{st:continue-functie-op-gesloten-begrensd-interval-bereikt-extrema}
    \[ f(x_{1}) = \inf\{f(x) \mid x\in\interval{a}{b}\} \quad\text{en}\quad f(x_{2}) = \sup\{f(x) \mid x\in\interval{a}{b}\} \]
    Er geldt dan het volgende:\prref{pr:integraal-behoudt-orde-2}
    \[ f(x_{1}) \le \frac{1}{b-a} \int_{a}^{b}f \le f(x_{2}) \]
    Uit de tussenwaardestelling volgt nu deze stelling.\stref{st:tussenwaardestelling}\waarom
  \end{proof}
\end{bst}


\subsection{De hoofdstelling en enkele gevolgen}
\label{sec:de-hoofdstelling-en}

\begin{bst}
  Zij $f:\ \interval{a}{b} \rightarrow \mathbb{R}$ een Riemannintegreerbare functie.
  Beschouw de functie $g$ als volgt.
  \[ g:\ \interval{a}{b} \rightarrow \mathbb{R}:\ x \mapsto \int_{a}^{b}f \]
  Er geldt dan het volgende:
  \begin{enumerate}
  \item $g$ is continu.
  \item Als $f$ continu is in een punt $x_{0}\in \interval{a}{b}$, dan is $g$ afleidbaar in $x_{0}$ en geldt $g'(x_{0}) = f(x_{0})$.
  \end{enumerate}

  \begin{proof}
    \noindent
    \begin{enumerate}
    \item 
      Kies willekeurig een $\epsilon\in\mathbb{R}_{0}^{+}$.
      Omdat $f$ Riemannintegreerbaar is, is $f$ zeker begrensd\needed en kunnen we dus een $M\in\mathbb{R}_{0}^{+}$ vinden als volgt:
      \[ \forall t\in\interval{a}{b}:\ |f(t)| \le M \]
      Kies nu willekeurig twee getallen $x$ en $y$ uit $\interval{a}{b}$ met $0 \le x-y \le \frac{\epsilon}{M}$.
      We vinden dan het volgende:
      \begin{align*}
        |g(x)-g(y)|
        &= \left|\int_{a}^{x}f-\int_{a}^{y}f\right|\\
        &= \left| \int_{y}^{x}f\right|\\
        &\le \int_{y}^{x}|f| \le M(x-y) < \epsilon
      \end{align*}
      Dit toont de (uniforme) continu\"iteit van $g$ aan.
      
    \item 
      Stel dat $f$ continu is in een punt $x_{0}\in\interval{a}{b}$.
      Kies nogmaals een $\epsilon\in\mathbb{R}_{0}^{+}$.
      Volgens de continu\"iteit van $f$ in $x_{0}$ kunnen we een $\delta\in\mathbb{R}_{0}^{+}$ vinden als volgt:
      \[ \forall t\in\interval{a}{b}:\ |t-x_{0}| < \delta \Rightarrow |f(t)-f(x_{0})| < \epsilon \]
      Kies nu een willekeurige $x\in\interval{a}{b}$ met $0 < |x-x_{0}| < \delta$.
      We vinden het volgende:
      \begin{align*}
        \left| \frac{g(x)-g(x_{0})}{x-x_{0}} - f(x_{0}) \right|
        &= \left| \frac{1}{x-x_{0}} \int_{x_{0}}^{x}f(t)\ dt - f(x_{0}) \right|\\
        &= \left| \frac{1}{x-x_{0}} \int_{x_{0}}^{x}(f(t)-f(x_{0}))\ dt \right|\\
        &\le \frac{1}{x-x_{0}} \int_{\min\{x_{0},x\}}^{\max\{x_{0},m\}}|f(t)-f(x_{0})|\ dt\\
        &< \epsilon
      \end{align*}
    \end{enumerate} 
  \end{proof}
\end{bst}

\begin{bst}
  \label{st:hoofdstelling}
  De \term{hoofdstelling}\\
  \question{...van wat?}
  Zij $f:\ \interval{a}{b} \rightarrow \mathbb{R}$ een continue functie, dan is de functie $g$ als volgt afleidbaar en de afgeleide functie ervan gelijk aan $f$.
  \[ g:\ \interval{a}{b} \rightarrow \mathbb{R}:\ x \mapsto \int_{a}^{x}f \]
  \extra{bewijs p 15}
\end{bst}

\begin{de}
  Zij $f:\ I \subseteq \mathbb{R} \rightarrow \mathbb{R}$ een functie op een interval $I$.
  We noemen een functie $F:\ I \rightarrow \mathbb{R}$ een \term{primitieve functie} van $f$ als $F$ afleidbaar is en $F'$ gelijk is aan $f$.
\end{de}

\begin{de}
  Een functie waarvoor een primitieve functie bestaat noemen we \term{primitieveerbaar}.
\end{de}

\begin{bpr}
  \label{pr:primitieve-functies-op-constante-na-gelijk}
  Zij $f:\ I \subseteq \mathbb{R}\rightarrow \mathbb{R}$ een functie op een interval $I$.
  Elke twee primitieve functies van $f$ zijn op een constante $C\in \mathbb{R}$ na gelijk.
\extra{bewijs: oefening}
\end{bpr}

\begin{bpr}
  Zij $f:\ \interval{a}{b} \rightarrow \mathbb{R}$ een continue functie en zij $F$ een primitieve functie van $f$, dan geldt volgende gelijkheid:
  \[ \int_{a}^{b}f = F(b) - F(a) \]

  \begin{proof}
    De functie $g$ als volgt is een primitieve functie van $f$.
    \[ g:\ \interval{a}{b} \rightarrow \mathbb{R}:\ x \mapsto g(x) = \int_{a}^{x}f \]
    Er bestaat dan een $C\in\mathbb{R}$ als volgt:\prref{pr:primitieve-functies-op-constante-na-gelijk}
    \[ \forall x\in\interval{a}{b}:\ \int_{a}^{x}f = F(x) + C \]
    Voor $x=a$ vinden we dan het volgende:
    \[ 0 = \int_{a}^{a}f = F(a) + C \]
    $C$ moet dus gelijk zijn aan $-F(a)$.
    We vinden dan het volgende:
    \[ \forall x\in\interval{a}{b}:\ \int_{a}^{x}f = F(x)-F(a) \]
    In het bijzonder geldt dit ook voor $b$.
  \end{proof}
\end{bpr}

\begin{bst}
  Zij $F:\ \interval{a}{b} \rightarrow \mathbb{R}$ een afleidbare functie.
  Als $F'$ Riemannintegreerbaar is over $\interval{a}{b}$, dan geldt volgende gelijkheid:
  \[ \int_{a}^{b}F' = F(b) - F(a) \]

  \begin{proof}
    Beschouw een willekeurige verdeling $P = \{x_{0},x_{1},\dotsc x_{n}\}$ van $\interval{a}{b}$.
    Pas op elk deelinterval de middelwaardestelling van Lagrange toe.\stref{st:middelwaardestelling-lagrange}
    Dit levert punten $c_{k}\in\interval[open]{x_{k-1}}{x_{k}}$ op als volgt:
    \[ F(x_{k})-F(x_{k-1}) = F'(c_{k})(x_{k}-x_{k-1}) \]
    Omdat $F'(c_{k})$ telkens tussen het infimum en het supremum ligt van de waarden van $F'$, zal het volgende gelden:
    \[ \underline{S}(F',P) \le \sum_{k=1}^{n}(F(x_{k})-F(x_{k-1})) \le \overline{S}(F',P) \]
    Merk nu op dat de middelste term gelijk is aan $F(b)-F(a)$.
    Vermits vorige ongelijkheid geldt voor elke verdeling $Y$ en vermits $F'$ Riemannintegreerbaar is, volgt het gevraagde resultaat.
  \end{proof}
\end{bst}

\begin{bpr}
  De \term{parti\"ele integratieregel}\\
  Zij $f,g:\ \interval{a}{b} \rightarrow \mathbb{R}$ afleidbare fnucties met continue afgeleiden, dan geldt volgende gelijkheid:
  \[ \int_{a}^{b}fg' = f(b)g(b)-f(a)g(a) - \int_{a}^{b}f'g \]

  \begin{proof}
    Beschouw de functie $F = fg$, dan is $F$ afleidbaar en $F' = f'g + fg'$.\prref{pr:productregel-afgeleiden}
    Merk op dat $F'$ continu is.\waarom
    We vinden dan het volgende:\needed\needed
    \[ F(b) - F(a) = \int_{a}^{b}F' = \int_{a}^{b}f'g + \int_{a}^{b}fg' \]
    Dit laat zich meteen herschrijven tot de stelling.
  \end{proof}
\end{bpr}

\begin{bpr}
  De \term{substitutieregel}\\
  Zij $f:\ \interval{c}{d}\rightarrow\mathbb{R}$ een continue functie en $g:\ \interval{a}{b} \rightarrow \interval{c}{d}$ een afleidbare functie waarvoor $g'$ continu is, dan geldt volgende ongelijkheid:
  \[ \int_{g(a)}^{g(b)}f(t)\ dt = \int_{a}^{b}f(g(x))g'(x)\ dx \]

  \begin{proof}
    Beschouw de functie $F$ en $G$ als volgt:
    \[ F:\ \interval{c}{d} \rightarrow \mathbb{R}:\ y \mapsto F(y) = \int_{g(a)}^{y}f(t)\ dt \]
    \[ G:\ \interval{a}{b} \rightarrow \mathbb{R}:\ x \mapsto G(x) = \int_{a}^{x}f(g(y))g'(y)\ dy \]
    De functies $F \circ g$ en $G$ zijn afleidbaar op $\interval{a}{b}$ en bovendien geldt het volgende:
    \[ \forall x\in\interval{a}{b}:\ (F \circ g)'(x) = F'(g(x))g'(x) = f(g(x))g'(x) = G'(x) \]
    Het verschil tussen $F\circ g$ en $G$ is dus een constante functie.
    Omdat $(F \circ g)(a)$ en $G(a)$ echter beide nul zijn, moeten $F\circ g$ en $G$ gelijk zijn.
    In het bijzonder geldt $F(g(b)) = G(b)$.
  \end{proof}
\end{bpr}


\subsection{Integralen en limieten van rijen van functies}
\label{sec:integr-en-limi}

\begin{bst}
  De \term{begrensde convergentiestelling}\\
  Zij $(f_{n})_{n}$ een rij van Riemannintegreerbare functies op een interval $\interval{a}{b}$ is die puntsgewijs convergeert naar een Riemannintegreerbare functie $f$.
  Stel dat die rij begrensd is:
  \[ \exists M\in \mathbb{R}:\ \forall n\in \mathbb{N}_{0}, \forall x \in \interval{a}{b}:\ \left|f_{n}(M)\right| \le M \]
  Er geldt dan de volgende ongelijkheid:
  \[ \lim_{n\rightarrow \infty}\int_{a}^{b}f_{n} = \int_{a}^{b}f \]
\zb
\end{bst}

\TODO{bewijs voor uniforme convergentie}


\end{document}

%%% Local Variables:
%%% mode: latex
%%% TeX-master: t
%%% End:
