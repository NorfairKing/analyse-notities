\documentclass[main.tex]{subfiles}
\begin{document}



\section*{Voorwoord}
\label{sec:voorwoord}

Dit boek is een uit de hand gelopen verzameling van notities.
Het boek spant ondertussen \pageref{LastPage} bladzijden, maar het is nog steeds niet af.
De lezer wordt hierbij uitgenodigd om eraan mee te helpen, voor meer details hierover, zie appendix \ref{cha:oproep}.
De lezer wordt eveneens uitgenodigd om enige fout meteen te melden aan de auteur.
Hopelijk kunnen deze notities nog dienen voor volgende studenten.
% Ze hebben de auteur alvast erdoor geholpen.

\subsection*{Over de betekenis van de opmaak}
\label{sec:over-de-betekenis}

Een kader rond een stelling, bewijs, gevolg,... betekent dat het letterlijk in de cursus staat, (maar niet noodzakelijk ook het bewijs)
Elke stuk zonder een kader er rond is dus extra vanuit voorbereidingen, oefeningen, vragen,... 
Geen enkel stukje is naast de kwestie.
Het wordt allemaal verwacht gekend te zijn naar het examen toe.
In deze zin is de cursus een deelverzameling van deze notities.
\textbf{Minstens alles wat in de cursus staat, staat ook in deze notities}
(Afgezien van de vervollediging van $\mathbb{Q}$ en de Riemann-Stieltjesintegraal)


\end{document}

%%% Local Variables:
%%% mode: latex
%%% TeX-master: t
%%% End:
