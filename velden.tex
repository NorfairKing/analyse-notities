\documentclass[main.tex]{subfiles}
\begin{document}



\section{Velden}
\label{sec:velden}

\begin{de}
  Een \term{veld} $\mathbb{F},+,\cdot$ is een verzameling $\mathbb{F}$ met twee bewerkingen $+$ en $\cdot$ met de volgende eigenschappen.
  \begin{itemize}
  \item $+$ is associatief:
    \[ \forall f,g,h \in \mathbb{F}:\ (f+g)+h = f+(g+h) \]
  \item Er bestaat een (uniek) neutraal element $0$ voor $+$:
    \[ \forall f\in \mathbb{F}:\ f + 0 = f = 0 + f \]
  \item Elk element $x$ heeft een (uniek) invers element $-x$ voor $+$:
    \[ \forall f\in \mathbb{F}: \exists (-f) \in \mathbb{F}: \ f + (-f) = 0 = (-f) + f \]
  \item $+$ is commutatief:
    \[ \forall f,g \in \mathbb{F}:\ f+g = g+f \]
  \item $\cdot$ is associatief:
    \[ \forall f,g,h \in \mathbb{F}:\ (f\cdot g) \cdot h = f\cdot (g\cdot h) \]
  \item Er bestaat een (uniek) neutraal element $1$ voor $\cdot$:
    \[ \forall f\in \mathbb{F}:\ f \cdot 1 = f = 1 \cdot f \]
  \item Elk element $x$ heeft een (uniek) invers element $x^{-1}$ voor $\cdot$:
    \[ \forall f\in \mathbb{F}: \exists f^{-1} \in \mathbb{F}: \ f \cdot f^{-1} = 0 = f^{-1} \cdot f \]
  \item $\cdot$ is commutatief:
    \[ \forall f,g \in \mathbb{F}:\ f\cdot g = g \cdot f \]
  \item $\cdot$ is distributief ten opzichte van $+$:
    \[ \forall f,g,h \in \mathbb{F}:\ f \cdot (g+h) = (f \cdot g) + (f \cdot h) \]
  \end{itemize}
\end{de}

\begin{de}
  \label{de:totaal-geordend-veld}
  Zij $\mathbb{F},+,\cdot$ een veld met een totale orderelatie $\le$, dan noemen we $\mathbb{F},+,\cdot,\le$ een \term{totaal geordend veld} als aan de volgende eigenschappen voldaan is.
  \begin{itemize}
  \item $\forall x,y,z \in \mathbb{F}:\ x \le y \Rightarrow (x+z) \le (y+z)$
  \item $\forall x,y,z \in \mathbb{F}:\ x \le y \wedge 0 \le z \Rightarrow x\cdot z \le y\cdot z$
  \end{itemize}
\end{de}

\begin{bpr}
  \label{pr:geordend-veld-optelling-ongelijkheden}
  Zij $\mathbb{F},+,\cdot,\le$ een totaal geordend veld.
  \[ \forall a,b,c,d \in \mathbb{F}:\ a\le b \wedge c \le d \Rightarrow (a+c) \le (b+d) \]

  \begin{proof}
    We gebruiken enkel de eerste eigenschap in de definitie van een totaal geordend veld.
    Omdat $a\le b$ geldt geldt ook $a + c \le b+c$.
    Bovendien geldt $c+b \le d+b$ omdat $c\le d$ geldt.
    Omdat $+$ commutatief is in $\mathbb{F}$, geldt dan ook het volgende:
    \[ a+c \le b+c \le b+d\]
  \end{proof}
\end{bpr}

\begin{bpr}
  \label{pr:geordend-veld-ongelijkheid-maal-min-een}
  Zij $\mathbb{F},+,\cdot,\le$ een totaal geordend veld.
  \[ \forall a,b\in \mathbb{F}:\ a \le b \Rightarrow -b \le -a \]

  \begin{proof}
    Bewijs uit het ongerijmbde\\
    Stel $a \le b$ en $-b > -a$ (dus $-a \le b$).
    Omdat $\mathbb{F}$ een totaal geordend veld is, volgt uit $-a \le -b$ zowel $a \le -b+2a$ en $-a+2b \le -b$.
    Tel deze ongelijkheden op\prref{pr:geordend-veld-optelling-ongelijkheden} om $2b \le 2a$ te bekomen.
    Uit $a\le b$ volgt echter dat $2a \le 2b$ geldt (tel immers $a\le b$ bij zichzelf op).
    Contradictie.
  \end{proof}
\end{bpr}

\begin{bpr}
  \label{pr:geordend-veld-tegengestelde-wisselt-teken}
  Zij $\mathbb{F},+,\cdot,\le$ een totaal geordend veld.
  \[ \forall b \in \mathbb{F}:\ 0 \le b \Leftrightarrow -a \le 0 \]

  \begin{proof}
    Gebruik in propositie \ref{pr:geordend-veld-ongelijkheid-maal-min-een} $a=0$.
  \end{proof}
\end{bpr}

\begin{bpr}
  \label{pr:geordend-veld-ongelijkheid-vermenigvuldiging}
  Zij $\mathbb{F},+,\cdot,\le$ een totaal geordend veld.
  \[ \forall a,b \in \mathbb{F}:\ 0 \le a \wedge 0 \le b \Rightarrow 0 \le ab \]

  \begin{proof}
    Omdat $\mathbb{F}$ een geordend veld is, volgt uit $0\le a$ dat $0b \le ab$ geldt vanwege $0 \le b$.
    Omdat $0$ het nulelement is van $\mathbb{F}$ geldt $0b = 0$.
  \end{proof}
\end{bpr}

\begin{bpr}
  \label{pr:nul-kleiner-dan-een}
  Zij $\mathbb{F},+,\cdot,\le$ een totaal geordend veld.
  \[ 0 < 1 \]

  \begin{proof}
    Bewijs uit het ongerijmde\\
    Stel dat $1 \le 0$ geldt, dan moet $1<0$ gelden omdat in een veld $1$ verschilt van $0$.
    Tel hierbij $-1$ op, dan bekomen we $0 \le -1$.
    Vanwege de tweede definierende eigenschap van een totaal geordend veld moet dan uit $1 \le 0$ ook $-1 \le 0$ volgen, maar dat is in strijd met $0 \le -1$ omdat $0$ verschilt van $1$.
  \end{proof}
\end{bpr}

\begin{bpr}
  \label{pr:geordend-veld-inverse-zelfde-teken}
  Zij $\mathbb{F},+,\cdot,\le$ een totaal geordend veld.
  \[ \forall a \in \mathbb{F}_{0}:\ 0 \le a \Rightarrow 0 \le a^{-1}\]

  \begin{proof}
    Bewijs uit het ongerijmde:
    Stel $0 \le a$ maar ook $a^{-1} < 0$ geldt.
    We mogen die tweede ongelijkheid dan vermenigvuldigen met $a$ om $aa^{-1}< 0$ te bekomen.
    Dit zou $1<0$ betekenen en dat is in contradictie met propositie \ref{pr:nul-kleiner-dan-een}.
  \end{proof}
\end{bpr}

\begin{bpr}
  \label{pr:geordend-veld-inverse-ongelijkheid-rekenregel}
  Zij $\mathbb{F},+,\cdot,\le$ een totaal geordend veld.
  \[ \forall a,b \in \mathbb{F}_{0}^{+}:\  a \le b \Rightarrow b^{-1} \le a^{-1}\]

  \begin{proof}
    Omdat zowel $0 \le a$ en $0 \le b$ geld, mogen we $a\le b$ vermenigvuldigen met $a^{-1}$ en $b^{-1}$\prref{pr:geordend-veld-inverse-zelfde-teken} om $b^{-1} \le a^{-1}$ te bekomen.
  \end{proof}
\end{bpr}


\end{document}

%%% Local Variables:
%%% mode: latex
%%% TeX-master: t
%%% End:
