\documentclass[main.tex]{subfiles}
\begin{document}

\section{Metrische ruimten}
\label{sec:metrische-ruimten}

\subsection{Het concept metrische ruimte}
\label{sec:het-conc-metr}

\begin{de}
  Een \term{Metrische ruimte} $V,d$ is een tupel van een verzameling $V$ en een functie $d$ met de volgende eigenschappen:
  \[ d:\ V \times V \rightarrow \mathbb{R}^{+}:\ (v,w) \mapsto d(v,w) \]
  \begin{itemize}
  \item $d$ is \textbf{symmetrisch}:
    \[ \forall v,w \in V:\ d(v,w) = d(w,v) \]
  \item $d$ is nul als en slechts als de argumenten gelijk zijn:
    \[ \forall v,w \in V:\ d(v,w) = 0 \Leftrightarrow v = w \]
  \item $d$ voldoet aan de \textbf{driehoeksongelijkheid}:
    \[ \forall u,v,w \in V:\ d(v,u) \le d(v,w) + d(w,u) \]
  \end{itemize}
  Men noemt $d$ de \term{metriek} of \term{afstandsfunctie} en $d(v,w)$ de \term{afstand} tussen $v$ en $w$.
\end{de}

\begin{vb}
  $\mathbb{R}$, uitgerust met een metriek gebaseerd op de absolute-waardefunctie, is een metrische ruimte:
  \[ d:\ \mathbb{R}\times\mathbb{R}\rightarrow (x,y) \mapsto d(x,y)=|x-y| \]
  Men noemt dit de \term{gewone metriek op $\mathbb{R}$}.
\end{vb}

\begin{vb}
  $\mathbb{R}$, uitgerust met de volgende functie als metriek, is een metrische ruimte:
  \[ d:\ \mathbb{R}\times\mathbb{R}\rightarrow (x,y) \mapsto d(x,y)=\frac{|x-y|}{1+|x-y|} \]
  \begin{proof}
    \begin{itemize}
    \item $d$ is symmestrisch.
      Dit volgt meteen uit de symmetrie van de gewone metriek op $\mathbb{R}$.
    \item $d$ is nu als en slechts als de argumentien nul zijn.
      Dit volgt uit dezelfde eigenschap van de gewone metriek op $\mathbb{R}$
    \item $d$ voldoet aan de driehoeksongelijkheid:\\
      Kies $x,y,z \in \mathbb{R}$ en houdt de driehoeksongelijkheid voor de gewone metriek op $\mathbb{R}$ in het achterhoofd.
      Merk op dat de functie $f$ als volgt stijgend is.
      \[ f:\ \mathbb{R}^{+} \rightarrow \mathbb{R}^{+}:\ t \mapsto \frac{t}{1+t} \]
      \begin{align*}
        d(x,y)
        &= f(|x-y|)\\
        &\le f(|x-z|+|z-y|)\\
        &= \frac{|x-z|+|z-y|}{1+|x-z|+|z-y|}\\
        &= \frac{|x-z|}{1+|x-z|+|z-y|}+\frac{|z-y|}{1+|x-z|+|z-y|}\\
        &\le \frac{|x-z|}{1+|x-z|}+\frac{|z-y|}{1+|z-y|}\\
        &= d(x,z) + d(z,y)
      \end{align*}
    \end{itemize}
  \end{proof}

\end{vb}

\begin{vb}
  $\mathbb{R}$, uitgerust met de volgende functie als metriek, is een metrische ruimte:
  \[ d:\ \mathbb{R}\times\mathbb{R}\rightarrow (x,y) \mapsto d(x,y)=\left| \frac{x}{1+|x|} - \frac{y}{1+|y|} \right| \]

\extra{bewijs algemener bij oef 1 van reeks VI.1.6}
\end{vb}

\begin{vb}
  $\mathbb{R}^{p}$, uitgerust met de volgende functie als metriek, is een metrische ruimte:
  \[ d:\ \mathbb{R}^{p}\times\mathbb{R}^{p}\rightarrow (x,y) \mapsto d(x,y)=\|x-y\| \]
  We noemen dit de \term{gewone metriek} of \term{euclidische metriek} op $\mathbb{R}^{p}$.
  \extra{bewijs}
\end{vb}

\begin{vb}
  $\mathbb{R}^{p}$, uitgerust met de volgende functie als metriek, is een metrische ruimte:
  \[ d:\ \mathbb{R}^{p}\times\mathbb{R}^{p}\rightarrow (x,y) \mapsto d(x,y)=\sum_{i=1}^{p}|x_{i}-y_{i}| \]
  We noemen dit de \term{city block metriek}.
\extra{bewijs}
\end{vb}

\begin{vb}
  $\mathbb{R}^{p}$, uitgerust met de volgende functie als metriek, is een metrische ruimte:
  \[ d:\ \mathbb{R}^{p}\times\mathbb{R}^{p}\rightarrow (x,y) \mapsto d(x,y)=\max_{i}|x_{i}-y_{i}| \]
  We noemen dit de \term{maximummetriek}.
\extra{bewijs}
\end{vb}

\begin{vb}
  $\mathbb{R}^{p}$, uitgerust met de volgende functie als metriek, is een metrische ruimte:
  \[
  d:\ \mathbb{R}^{p}\times\mathbb{R}^{p}\rightarrow (x,y) \mapsto d(x,y)=
  \begin{cases}
    \|x-y\| &\text{ als $x$ en $y$ lineair afhankelijk zijn}\\
    \|x\|+\|y\| &\text{ als $x$ en $y$ lineair onafhankelijk zijn}
  \end{cases}
  \]
  We noemen dit de \term{spoorwegmetriek}.
\extra{bewijs}
\end{vb}

\begin{de}
  We noteren met $\mathbb{R}^{\mathbb{N}}$ de verzamelingen van alle rijen in $\mathbb{R}$.
  Merk op dat $\mathbb{R}^{\mathbb{N}}$ een vectorruimte is over $\mathbb{R}$.
\end{de}

\begin{vb}
  $\mathbb{R}^{\mathbb{N}}$, uitgerust met de volgende functie als metriek, is een metrische ruimte:
  \[ d:\ \mathbb{R}^{\mathbb{N}} \times \mathbb{R}^{\mathbb{N}} \rightarrow \mathbb{R}^{+}:\ ((x_{n})_{n},(y_{n})_{n}) \mapsto \sum_{n=0}^{+\infty}\frac{|x_{n}-y_{n}|}{2^{n}(1+|x_{n}-y_{n}|)} \]
\extra{bewijs en bewijs waarom dit goed gedefinieerd is}
\end{vb}

\begin{vb}
  Noteer met $l^{\infty}(\mathbb{N})$ het volgende:
  \[ l^{\infty}(\mathbb{N}) = \{ (x_{n})_{n} \in \mathbb{R}^{\mathbb{N}} \mid (x_{n})_{n} \text{ is begrensd.}\} \]
  $l^{\infty}$, uitgerust met de volgende functie als metriek, is een metrische ruimte:
  \[ d:\ \mathbb{R}^{\mathbb{N}} \times \mathbb{R}^{\mathbb{N}} \rightarrow \mathbb{R}^{+}:\ ((x_{n})_{n},(y_{n})_{n}) \mapsto \sup\{|x_{n}-y_{n}| \mid n\in \mathbb{N}\} \]
\extra{bewijs}
\end{vb}

\begin{vb}
  Zij $X$ een gesloten en begrensde deelverzameling van $\mathbb{R}^{p}$ en noteer met $C(X)$ de vectorruimte van de continue functie van $X$ naar $\mathbb{R}$.
  $C(X)$, uitgerust met de volgende functie als metriek, is een metrische ruimte:
  \[ d:\ C(x)\times C(X)\rightarrow \mathbb{R}^{+}:\ (f,g) \mapsto d(f,g) = \sup \{ |f(x)-g(x)| \mid x \in X \} \]
\extra{bewijs}
\end{vb}

\begin{vb}
  Zij $\interval{a}{b}$ een gesloten begrensd interval en noteer met $C(\interval{a}{b})$ de vectorruimte van de continue functies van $\interval{a}{b}$ naar $\mathbb{R}$.
  $C(\interval{a}{b})$, uitgerust met de volgende functie als metriek, is een metrische ruimte:
  \[ d:\ C(\interval{a}{b})\times C(\interval{a}{b}) \rightarrow \mathbb{R}^{+}:\ (f,g) \mapsto \int_{a}^{b}|f(x)-g(x)|dx \]
\extra{bewijs}
\end{vb}

\begin{vb}
  Zij $\interval{a}{b}$ een gesloten begrensd interval en noteer met $C(\interval{a}{b})$ de vectorruimte van de continue functies van $\interval{a}{b}$ naar $\mathbb{R}$.
  $C(\interval{a}{b})$, uitgerust met de volgende functie als metriek, is een metrische ruimte:
  \[ d:\ C(\interval{a}{b})\times C(\interval{a}{b}) \rightarrow \mathbb{R}^{+}:\ (f,g) \mapsto \sqrt{\int_{a}^{b}|f(x)-g(x)|dx} \]
\extra{bewijs}
\end{vb}

\begin{vb}
  Zij $X$ een willekeurige, niet-lege verzameling.
  De \term{discrete metriek} of \term{triviale metriek} op $X$ wordt gegeven als volgt.
  \[
  d:\ X \times X \rightarrow \mathbb{R}^{+}:\ (x,y) \mapsto
  \begin{cases}
    0 &\text{ als } x = y\\
    1 &\text{ als } x \neq y
  \end{cases}
  \]
  $X,d$ vormt een metrische ruimte.
\end{vb}

\extra{product van metrische ruimten}

\begin{de}
  Zij $(\mathbb{R},V,+)$ een vectorruimte, dan is een \term{norm} op $V$ een afbeelding als volgt:
  \[ \|\cdot\|:\ V \rightarrow \mathbb{R}^{+}:\ v \mapsto \|v\| \]
  \begin{enumerate}
  \item $\forall v\in V:\ \|v\| = 0 \Leftrightarrow v=0$
  \item $\forall v\in V, \forall \lambda \in \mathbb{R}:\ \|\lambda v\| = |\lambda|\|v\|$
  \item $\forall v,w\in V:\ \|v+w\| < \|v\| + \|w\|$
  \end{enumerate}
\end{de}

\begin{de}
  Een vectorruimte $\mathbb{R},V,+$, uitgerust met een norm noemen we een \term{genormeerde vectorruimte}.
\end{de}

\begin{st}
  Een genormeerde vectorruimte $\mathbb{R},V,+,\|\cdot\|$, uitgerust met de volgende functie als metriek, vormt een metrische ruimte:
  \[ d:\ V \times V \rightarrow \mathbb{R}^{+}:\ (x,y) \mapsto \|x-y\| \]
\extra{bewijs}
\end{st}

\begin{de}
  Voor een niet-lege deelverzameling $A$ van $\mathbb{R}^{p}$ en een $r\in \mathbb{R}^{+}$ definie\"eren we de $r$-omhullende $[A]_{r}$ van $A$ als volgt:
  \[ [A]_{r} = \{r \in \mathbb{R}^{p} \mid \exists a \in A:\ d(x,a) \le r \} \]
  Hierin staat $d$ voor de gewone euclidische metriek.
\end{de}

\begin{de}
  Noteer met $\mathcal{F}$ de verzameling van niet-lege gesloten en begrensde deelverzamelingen van $\mathbb{R}^{p}$.
  \[ \mathcal{F} = \{ A \mid A \subseteq \mathbb{R}^{p} \} \]
\end{de}

\begin{de}
  We defini\"eren de \term{Hausdorffafstand} $h(F,G)$ tussen twee deelverzamelingen $F,G \in \mathcal{F}$ als volgt:
  \[ h(F,G) = \inf\{ r\in \mathbb{R}^{+} \mid F \subseteq [G]_{r} \text{ en } G \subseteq [F]_{r} \} \]
\end{de}

\begin{blem}
  Zij $A \subseteq \mathbb{R}^{p}$ en $r,s\in\mathbb{R}^{+}$ met $r\ge s \ge 0$.
  \[ [A]_{s} \subseteq [A]_{r} \]
\TODO{bewijs p 16}
\end{blem}

\begin{blem}
  Zij $A \subseteq \mathbb{R}^{p}$ en $r,s\in\mathbb{R}^{+}$:
  \[ \left[ [A]_{r}\right]_{s} \subseteq [A]_{r+s} \]
\TODO{bewijs p 16}
\end{blem}

\begin{blem}
  Zij $F,G \in \mathcal{F}$ en $r=h(F,G)$.
  \[ F \subseteq [G]_{r} \text{ en } G \subseteq [F]_{r} \]
\TODO{bewijs p 16}
\end{blem}

\begin{pr}
  De Hausdorffafstand $h$ is een metriek die $\mathcal{F},h$ een metrische ruimte maakt.
  \TODO{bewijs p 17}
\end{pr}

\begin{st}
  \[ \forall x,y \in \mathbb{R}^{p}:\ h(\{x\},\{y\}) = d(x,y) \]
\TODO{bewijs}
\end{st}

\TODO{voorbeelden van HDafstandberekeningen}

\subsection{Begrensde verzamelingen}
\label{sec:begr-verz}

\begin{de}
  Zij $V,d$ een metrische ruimte, dan noemen we een deelverzameling $W$ van $V$ \term{begrensd} als het volgende geldt:
  \[ \exists M\in \mathbb{R}^{+}, \forall v,w\in W:\ d(v,w) < M \]
\end{de}

\begin{de}
  Zij $V,d$ een metrische ruimte en $A$ een niet-lege begrensde deelverzameling van $V$, dan definieren we de \term{diameter} van $A$ als volgt:
  \[ diam(A) = \sup\{d(x,y) \mid x,y \in A \} \]
\end{de}

\begin{bpr}
  Zij $V,d$ een metrische ruimte en $A$ een niet-leeg deel van $X$, dan is $A$ begrensd als en slechts als er voor elke $x_{0}\in V$ een $R\in \mathbb{R}^{+}$ bestaat zodat $\forall a\in A:\ d(x_{0},a) \le R$ geldt.
\TODO{ bewpjs p 18}
\end{bpr}

\subsection{Rijen in metrische ruimten}
\label{sec:rijen-metr-ruimt}

\begin{de}
  Een rij $(x_{n})_{n}$ in een metrische ruimte $V,d$ noemen we \term{convergent} voor $d$ als er een $a\in V$ bestaat als volgt:
  \[ \forall \epsilon\in\mathbb{R}_{0}^{+}, \exists n_{0}\in \mathbb{N}, \forall n\in \mathbb{N}: n \ge n_{0} \Rightarrow d(x_{n},a) < \epsilon \]
  $a$ noemen we dan de \term{limiet} van de rij $(x_{n})_{n}$:
  \[ \lim_{n\rightarrow +\infty}x_{n} = a \]
\end{de}

\begin{bst}
  Zij $(x_{n})_{n}$ een convergente rij met limiet $x$ in een metrische ruimte $V,d$, dan is $x$ uniek.

  \begin{proof}
    Bewijs uit het ongerijmde: Stel dat er twee limieten $a$ en $b$ zijn van $(x_{n})_{n}$.\\
    Kies $\delta = \frac{d(a,b)}{2}$.
    Omdat $a$ de limiet is van $(x_{n})_{n}$, bestaat er een $n_{a}$ als volgt:
    \[ \forall n\in \mathbb{N}: n \ge n_{a} \Rightarrow d(x_{n},a) < \epsilon \]
    Omdat $b$ eveneens de limiet is van  $(x_{n})_{n}$, bestaat er ook een $n_{b}$ als volgt:
    \[ \forall n\in \mathbb{N}: n \ge n_{b} \Rightarrow d(x_{n},a) < \epsilon \]
    Merk nu voor alle $n\in \mathbb{N}$ de volgende ongelijkheid op die geldt vanuit de driehoeksongelijkheid:
    \[ d(a,b) \le d(a,x_{n}) + d(b,x_{n}) \]
    Kies $m = \max\{n_{a},n_{b}\}$, dan geldt voor alle $n\in \mathbb{N}$, groter dan $m$ de volgende contradictie.
    \[ 2\delta = d(a,b) \le d(a,x_{n}) + d(b,x_{n}) < \delta + \delta = 2\delta \]
  \end{proof}
\end{bst}
        
\begin{de}
  Zij $V,d$ een metrische ruimte, dan noemen we een rij $(x_{n})_{n}$ een \term{Cauchyrij} als en slechts als het volgende geldt:
  \[ \forall \epsilon, \exists n_{0} \in \mathbb{N}, \forall n,m\in \mathbb{N}:\ n,m\ge n_{0} \Rightarrow d(x_{n},x_{m}) < \epsilon \]
\end{de}

\begin{st}
  Beschouw de ruimte $C(X)$ van de re\"ele continue functies op een gesloten, begrensd deel $X$ van $\mathbb{R}^{p}$ en beschouw hierop de $d_{\infty}$-metriek.
  Een rij $(f_{n})_{n}$ in $C(X)$ zal volgens de $d_{\infty}$ metriek convergeren naar een $f\in C(X)$ als en slechts als $(f_{n})_{n}$ uniforum naar $f$ convergeert.
\TODO{bewijs p 20}
\TODO{dit is cool, moet dit niet ergens anders staan?}
\end{st}
\extra{voorbeeld p 20 onderaan, waar hoort dit?}

\begin{st}
  Een rij gesloten bollen $(B_{n})_{n}$ in $\mathbb{R}^{p}$ met middelpunten $x_{n}$ en stralen $r_{n}$ zal voor de Hausdorffmetriek convergeren naar een gesloten bol $B$ met middelpunt $x$ en straal $r$ als en slechts als $\lim_{n\rightarrow +\infty}x_{n}=x$ en $\lim_{n\rightarrow +\infty}r_{n} = r$ gelden.
\extra{bewijs}
\TODO{again, hoort dit wel hier?}
\end{st}
\extra{voorbeeld p 21}

\begin{bst}
  Zij $V,d$ een metrische ruimte, dan is elke onvergente rij hierin een Cauchyrij.

  \begin{proof}
    Zij $(x_{n})_{n}$ een convergente rij met limiet $x\in V$.
    Merk voor alle $n,m\in \mathbb{N}$ de volgende ongelijkheid op:
    \[ d\left(x_{n},x_{m}\right) \le d(x_{n},x) + d(x_{m},x) \]
    Kies willekeurig een $\delta \in \mathbb{R}_{0}^{+}$, dan bestaat er een $n_{0}\in \mathbb{N}$ als volgt:
    \[ \forall n\in \mathbb{N}:\ n \ge n_{0} \Rightarrow d(x_{n},x) < \frac{\delta}{2}\]
    Gebruiken we nu de vorige ongelijkheid, dan vinden we dat $(x_{n})_{n}$ een Cauchyrij is:
    \[ \forall n\in \mathbb{N}:\ n \ge n_{0} \Rightarrow d\left(x_{n},x_{m}\right) \le d(x_{n},x) + d(x_{m},x) < \frac{\delta}{2} + \frac{\delta}{2} = \delta\]
  \end{proof}
\end{bst}

\begin{de}
  Zij $V,d$ een metrische ruimte, dan noemen we deze \term{volledig} als elke Cauchyrij convergent is in $V$.
\end{de}

\begin{st}
  Een convergente rij is begrensd.
\extra{zeker?}
\end{st}

\subsection{Continue functies tussen metrische ruimten}
\label{sec:cont-funct-tuss}

\begin{de}
  Zij $V,d_{V}$ en $W,d_{W}$ twee metrische ruimtes en $f:\ A \subseteq V \rightarrow W$ een functie, dan noemen we $f$ \term{continu} in een punt $a\in A$ als het volgende geldt:
  \[ \forall \epsilon \in \mathbb{R}_{0}^{+}, \exists \delta \in \mathbb{R}_{0}^{+}, \forall b \in A:\ d_{V}(a,b)< \delta \Rightarrow d_{W}(f(a),f(b)) < \epsilon \]
\end{de}

\begin{de}
  Zij $V,d_{V}$ en $W,d_{W}$ twee metrische ruimtes en $f:\ A \subseteq V \rightarrow W$ een functie, dan noemen we $f$ \term{uniform continu} op $A$ als het volgende geldt:
  \[ \forall \epsilon \in \mathbb{R}_{0}^{+}, \exists \delta \in \mathbb{R}_{0}^{+}, \forall a,b \in A:\ d_{V}(a,b)< \delta \Rightarrow d_{W}(f(a),f(b)) < \epsilon \]
\end{de}

\begin{vb}
  Zij $d_{X}$ de triviale metriek voor een verzameling $X$, is elke functie $f:\ X \rightarrow Y$ met $Y,d_{Y}$ een metrische ruimte, continu.
\extra{bewijs}
\end{vb}

\begin{vb}
  Beshouw $I$ als volgt:
  \[ I:\ C(\interval{a}{b}) \Rightarrow \mathbb{R}:\ f \mapsto \int_{a}^{b}f(x)dx \]
  Beschouw op $C(\interval{a}{b}$ de $d_{\infty}$ metriek.
  $I$ is continu.
  \extra{bewijs}
\end{vb}

\begin{de}
  Zij $V,d_{V}$ en $W,d_{W}$ metrische ruimten, dan noemen we een funcie $f:\ X \rightarrow Y$ een \term{contractie} als het volgende geldt:
  \[ \forall x,y\in X:\ d_{Y}(f(x),f(y)) \le d_{X}(x,y) \]
\end{de}

\begin{de}
  Zij $V,d_{V}$ en $W,d_{W}$ metrische ruimten, dan noemen we een funcie $f:\ X \rightarrow Y$ een \term{strikte contractie} als er een $c\in \interval[open right]{0}{1}$ bestaat als volgt:
  \[ \forall x,y\in X:\ d_{Y}(f(x),f(y)) \le c d_{X}(x,y) \]
  De kleinste $c$ die hieraan voldoet noemen we de \term{contractiefactor}
\end{de}

\begin{st}
  Contracties zijn uniform continu.
\TODO{bewijs}
\end{st}

\begin{de}
  Zij $V,d_{V}$ en $W,d_{W}$ metrische ruimten, dan noemen we een funcie $f:\ X \rightarrow Y$ een \term{isometrie} als $f$ afstanden bewaart:
  \[ \forall x,y \in X: d_{W}(f(x),f(y)) = d_{V}(x,y) \]
\end{de}

\begin{st}
  Elke isometrie is injectief.
\TODO{bewijs}
\end{st}

\begin{de}
  Zij $V,d_{V}$ en $W,d_{W}$ metrische ruimten, dan zeggen we dat $V,d_{V}$ \term{isometrisch ingebed} kan worden in $W,d_{W}$ als er een isometrie $f:\ X \rightarrow Y$ bestaat.
\end{de}

\begin{de}
  Twee metrische ruimten $V,d_{V}$ en $W,d_{W}$ noemen we \term{isometrisch} als en slechts als er een bijectieve isometrie $f:\ X \rightarrow Y$ bestaat.
\end{de}

\TODO{zoek hiervan de voorbeelden hierboven}

\extra{voorbeeld p 23}





\end{document}

%%% Local Variables:
%%% mode: latex
%%% TeX-master: t
%%% End:
