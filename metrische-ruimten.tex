\documentclass[main.tex]{subfiles}
\begin{document}

\section{Metrische ruimten}
\label{sec:metrische-ruimten}

\subsection{Het concept metrische ruimte}
\label{sec:het-conc-metr}

\begin{de}
  Een \term{Metrische ruimte} $V,d$ is een tupel van een verzameling $V$ en een functie $d$ met de volgende eigenschappen:
  \[ d:\ V \times V \rightarrow \mathbb{R}^{+}:\ (v,w) \mapsto d(v,w) \]
  \begin{itemize}
  \item $d$ is \textbf{symmetrisch}:
    \[ \forall v,w \in V:\ d(v,w) = d(w,v) \]
  \item $d$ is nul als en slechts als de argumenten gelijk zijn:
    \[ \forall v,w \in V:\ d(v,w) = 0 \Leftrightarrow v = w \]
  \item $d$ voldoet aan de \textbf{driehoeksongelijkheid}:
    \[ \forall u,v,w \in V:\ d(v,u) \le d(v,w) + d(w,u) \]
  \end{itemize}
  Men noemt $d$ de \term{metriek} of \term{afstandsfunctie} en $d(v,w)$ de \term{afstand} tussen $v$ en $w$.
\end{de}


\begin{st}
  Zij $f:\ \mathbb{R}^{+} \rightarrow \mathbb{R}^{+}$ een stijgende concave functie zodat $f(t)$ nul is als en slechts als $t$ nul is.
  Zij $X,d$ een metrische ruimte, dan is $d_{f} = f\circ d$ een metriek op $X$.

  \begin{proof}
    We gaan alle eigenschappen na.
    \begin{itemize}
    \item $d_{f}$ is symmetrisch: evident.
    \item $d_{f}$ is nul als en slechts als de argumenten gelijk: aanname.
    \item $d_{f}$ voldoet aan de driehoeksongelijkheid:
      \[ f(d(x,z)) \le f(d(x,y) + d(y,z)) \]
\TODO{en nu?}
    \end{itemize}
  \end{proof}
\end{st}

\subsection{Begrensde verzamelingen}
\label{sec:begr-verz}

\begin{de}
  Zij $V,d$ een metrische ruimte, dan noemen we een deelverzameling $W$ van $V$ \term{begrensd} als het volgende geldt:
  \[ \exists M\in \mathbb{R}^{+}, \forall v,w\in W:\ d(v,w) < M \]
\end{de}

\begin{de}
  Zij $V,d$ een metrische ruimte en $A$ een niet-lege begrensde deelverzameling van $V$, dan definieren we de \term{diameter} van $A$ als volgt:
  \[ diam(A) = \sup\{d(x,y) \mid x,y \in A \} \]
\end{de}

\begin{bpr}
  Zij $V,d$ een metrische ruimte en $A$ een niet-leeg deel van $X$, dan is $A$ begrensd als en slechts als er voor elke $x_{0}\in V$ een $R\in \mathbb{R}^{+}$ bestaat zodat $\forall a\in A:\ d(x_{0},a) \le R$ geldt.

  \begin{proof}
    Bewijs van een equivalentie.
    \begin{itemize}
    \item $\Rightarrow$\\
      Stel dat $A$ begrensd is, dan bestaat er een $M$ als volgt:
      \[ \forall x,a \in A:\ d(x,a) \le M \]
      Kies willekeurig een $x_{0}\in X$ en een $x\in A$.
      Noem nu $R=M + d(x_{0},x)$.
      Kies vervolgens een willekeurige $a\in A$.
      We vinden nu het volgende volgens de driehoeksongelijkheid:
      \[ d(x_{0},a) \le d(x_{0},x) + d(x,a) \le d(x_{0},x) + M = R \]
    \item $\Leftarrow$\\
      Kies willekeurig een $x_{0}\in X$.
      Kies willekeurig twee elementen $x$ en $a$ uit $A$.
      \[ d(x,a) \le d(x,x_{0}) + d(x_{0},a) \le R + R = 2R\]
      Kies dan eenvoudigweg $M=2R$.
    \end{itemize}
  \end{proof}
\end{bpr}

\subsection{Convergente rijen in metrische ruimten}
\label{sec:rijen-metr-ruimt}

\begin{de}
  Een rij $(x_{n})_{n}$ in een metrische ruimte $V,d$ noemen we \term{convergent} voor $d$ als er een $a\in V$ bestaat als volgt:
  \[ \forall \epsilon\in\mathbb{R}_{0}^{+}, \exists n_{0}\in \mathbb{N}, \forall n\in \mathbb{N}: n \ge n_{0} \Rightarrow d(x_{n},a) < \epsilon \]
  $a$ noemen we dan de \term{limiet} van de rij $(x_{n})_{n}$:
  \[ \lim_{n\rightarrow +\infty}x_{n} = a \]
\end{de}

\begin{bst}
  Zij $(x_{n})_{n}$ een convergente rij met limiet $x$ in een metrische ruimte $V,d$, dan is $x$ uniek.

  \begin{proof}
    Bewijs uit het ongerijmde: Stel dat er twee limieten $a$ en $b$ zijn van $(x_{n})_{n}$.\\
    Kies $\delta = \frac{d(a,b)}{2}$.
    Omdat $a$ de limiet is van $(x_{n})_{n}$, bestaat er een $n_{a}$ als volgt:
    \[ \forall n\in \mathbb{N}: n \ge n_{a} \Rightarrow d(x_{n},a) < \epsilon \]
    Omdat $b$ eveneens de limiet is van  $(x_{n})_{n}$, bestaat er ook een $n_{b}$ als volgt:
    \[ \forall n\in \mathbb{N}: n \ge n_{b} \Rightarrow d(x_{n},a) < \epsilon \]
    Merk nu voor alle $n\in \mathbb{N}$ de volgende ongelijkheid op die geldt vanuit de driehoeksongelijkheid:
    \[ d(a,b) \le d(a,x_{n}) + d(b,x_{n}) \]
    Kies $m = \max\{n_{a},n_{b}\}$, dan geldt voor alle $n\in \mathbb{N}$, groter dan $m$ de volgende contradictie.
    \[ 2\delta = d(a,b) \le d(a,x_{n}) + d(b,x_{n}) < \delta + \delta = 2\delta \]
  \end{proof}
\end{bst}
        
\begin{de}
  Zij $V,d$ een metrische ruimte, dan noemen we een rij $(x_{n})_{n}$ een \term{Cauchyrij} als en slechts als het volgende geldt:
  \[ \forall \epsilon, \exists n_{0} \in \mathbb{N}, \forall n,m\in \mathbb{N}:\ n,m\ge n_{0} \Rightarrow d(x_{n},x_{m}) < \epsilon \]
\end{de}


\begin{bst}
  Zij $V,d$ een metrische ruimte, dan is elke onvergente rij hierin een Cauchyrij.

  \begin{proof}
    Zij $(x_{n})_{n}$ een convergente rij met limiet $x\in V$.
    Merk voor alle $n,m\in \mathbb{N}$ de volgende ongelijkheid op:
    \[ d\left(x_{n},x_{m}\right) \le d(x_{n},x) + d(x_{m},x) \]
    Kies willekeurig een $\delta \in \mathbb{R}_{0}^{+}$, dan bestaat er een $n_{0}\in \mathbb{N}$ als volgt:
    \[ \forall n\in \mathbb{N}:\ n \ge n_{0} \Rightarrow d(x_{n},x) < \frac{\delta}{2}\]
    Gebruiken we nu de vorige ongelijkheid, dan vinden we dat $(x_{n})_{n}$ een Cauchyrij is:
    \[ \forall n\in \mathbb{N}:\ n \ge n_{0} \Rightarrow d\left(x_{n},x_{m}\right) \le d(x_{n},x) + d(x_{m},x) < \frac{\delta}{2} + \frac{\delta}{2} = \delta\]
  \end{proof}
\end{bst}

\begin{de}
  \label{de:metrische-ruimte-volledig}
  Zij $V,d$ een metrische ruimte, dan noemen we deze \term{volledig} als elke Cauchyrij convergent is in $V$.
\end{de}

\begin{st}
  Een convergente rij is begrensd.
\extra{zeker?}
\end{st}

\subsection{Continue functies tussen metrische ruimten}
\label{sec:cont-funct-tuss}

\begin{de}
  Zij $V,d_{V}$ en $W,d_{W}$ twee metrische ruimtes en $f:\ A \subseteq V \rightarrow W$ een functie, dan noemen we $f$ \term{continu} in een punt $a\in A$ als het volgende geldt:
  \[ \forall \epsilon \in \mathbb{R}_{0}^{+}, \exists \delta \in \mathbb{R}_{0}^{+}, \forall b \in A:\ d_{V}(a,b)< \delta \Rightarrow d_{W}(f(a),f(b)) < \epsilon \]
\end{de}

\begin{de}
  Zij $V,d_{V}$ en $W,d_{W}$ twee metrische ruimtes en $f:\ A \subseteq V \rightarrow W$ een functie, dan noemen we $f$ \term{uniform continu} op $A$ als het volgende geldt:
  \[ \forall \epsilon \in \mathbb{R}_{0}^{+}, \exists \delta \in \mathbb{R}_{0}^{+}, \forall a,b \in A:\ d_{V}(a,b)< \delta \Rightarrow d_{W}(f(a),f(b)) < \epsilon \]
\end{de}

\begin{de}
  Zij $V,d_{V}$ en $W,d_{W}$ metrische ruimten, dan noemen we een funcie $f:\ X \rightarrow Y$ een \term{contractie} als het volgende geldt:
  \[ \forall x,y\in X:\ d_{Y}(f(x),f(y)) \le d_{X}(x,y) \]
\end{de}

\begin{de}
  Zij $V,d_{V}$ en $W,d_{W}$ metrische ruimten, dan noemen we een funcie $f:\ X \rightarrow Y$ een \term{strikte contractie} als er een $c\in \interval[open right]{0}{1}$ bestaat als volgt:
  \[ \forall x,y\in X:\ d_{Y}(f(x),f(y)) \le c d_{X}(x,y) \]
  De kleinste $c$ die hieraan voldoet noemen we de \term{contractiefactor}
\end{de}

\begin{st}
  Contracties zijn uniform continu.
\TODO{bewijs}
\end{st}

\begin{de}
  Zij $V,d_{V}$ en $W,d_{W}$ metrische ruimten, dan noemen we een funcie $f:\ X \rightarrow Y$ een \term{isometrie} als $f$ afstanden bewaart:
  \[ \forall x,y \in X: d_{W}(f(x),f(y)) = d_{V}(x,y) \]
\end{de}

\begin{st}
  Elke isometrie is injectief.

  \begin{proof}
    Kies twee elementen $x$ en $y$ uit $V$ met hetzelfde beeld $z\in W$ omdat $d(z,z)$ nul is, en omdat $f$ een isometrie is, moet $d(x,y)$ nul zijn en $x$ en $y$ dus gelijk zijn.
  \end{proof}
\end{st}

\begin{de}
  Zij $V,d_{V}$ en $W,d_{W}$ metrische ruimten, dan zeggen we dat $V,d_{V}$ \term{isometrisch ingebed} kan worden in $W,d_{W}$ als er een isometrie $f:\ X \rightarrow Y$ bestaat.
\end{de}

\begin{de}
  Twee metrische ruimten $V,d_{V}$ en $W,d_{W}$ noemen we \term{isometrisch} als en slechts als er een bijectieve isometrie $f:\ X \rightarrow Y$ bestaat.
\end{de}

\TODO{zoek hiervan de voorbeelden hierboven}

\extra{voorbeeld p 23}





\end{document}

%%% Local Variables:
%%% mode: latex
%%% TeX-master: t
%%% End:
