\documentclass[main.tex]{subfiles}
\begin{document}

\section{Topologie in metrische ruimten}
\label{sec:topol-metr-ruimt}

\subsection{Open en gesloten verzamelingen}
\label{sec:open-en-gesloten}

\begin{de}
  Zij $V,d$ een metrische ruimte, dan noemen we een deelverzameling $W$ van $V$ \term{open} als het volgende geldt:
  \[ \forall v\in W, \exists \delta \in \mathbb{R}_{0}^{+}, \forall w\in W:\ d(v,w) < \delta \Rightarrow w \in W \]
\end{de}

\begin{de}
  Zij $V,d$ een metrische ruimte, dan noemen we een deelverzameling $W$ van $V$ \term{gesloten} als het complement $V\setminus W$ van $W$ in $V$ open is.
\end{de}

\begin{de}
  Zij $V,d$ een metrische ruimte, $x\in V$ en $\delta \in \mathbb{R}_{0}^{+}$, dan noemen we de verzameling $B(x,\delta)$ als volgt, de \term{open bol} met middelpunt $x$ en straal $\delta$.
  \[ B(x,\delta) = \{ y\in V \mid d(x,y) < \delta \} \]
\end{de}

\begin{de}
  Zij $V,d$ een metrische ruimte, $x\in V$ en $\delta \in \mathbb{R}_{0}^{+}$, dan noemen we de verzameling $B\interval{x}{\delta}$ als volgt, de \term{gesloten bol} met middelpunt $x$ en straal $\delta$.
  \[ B(x,\delta) = \{ y\in V \mid d(x,y) \le \delta \} \]
\end{de}

\begin{st}
  Een open bol $B(x,\delta)$ in een metrische ruimte $V,d$ is een open deelverzameling van $V$.

  \begin{proof}
    Kies een willekeurig punt $y\in V$ met $d(x,y) < \delta$. 
    \begin{lem}
      Er bestaat een open bol $B(y,\epsilon)$ rond elk punt $y\in B(x,\delta)$ die een deel is van $B(x,\delta)$.
      \begin{proof}
        Noem
      \end{proof}
    \end{lem}
  \end{proof}
\end{st}

\begin{st}
  Een gesloten bol $B\interval{x}{\delta}$ in een metrische ruimte $V,d$ is een gesloten deelverzameling van $V$.
\extra{bewijs}
\end{st}



\begin{pr}
  Zij $V,d$ een metrische ruimte, dan is de unie van open verzamelingen in $V$ ook open in $V$.

  \begin{proof}
    Zij $\mathcal{O}$ een niet-lege verzameling van open deelverzamelingen van $V$.
    Noem $U = \bigcup_{A\in \mathcal{O}}A$.
    \begin{itemize}
    \item Als $U$ leeg is, is $U$ trivialerwijs open.
    \item Als $U$ niet leeg is, dan bestaat er een $a$ in een $A \subseteq \mathcal{O}$ als volgt:
      \[ \forall v\in A, \exists \delta \in \mathbb{R}_{0}^{+}, \forall w\in V:\ d(v,w) < \delta \Rightarrow w \in A \]
      Omdat $A$ een deel is van $U$, zal hetzelfde gelden voor $U$.
    \end{itemize}
  \end{proof}
\end{pr}

\begin{pr}
  Zij $V,d$ een metrische ruimte, dan is een \textbf{eindige} doorsnede van open verzamelingen in $V$ ook open in $V$.
  
  \begin{proof}
    Beschouw een eindig aantal ($n$) open deelverzamelingen $A_{i}$ van $V$.
    Noem $D = \bigcap_{i}A_{x}$.
    \begin{itemize}
    \item Als $D$ leeg is, is $D$ trivialerwijs open.
    \item Als $D$ niet leeg is, dan bestaat er een $a\in D$ als volgt:
      \[ \forall i, \exists \delta_{i} \in \mathbb{R}_{0}^{+}, \forall w\in V:\ d(a,w) < \delta \Rightarrow w \in A_{i} \]
      Kies nu $\delta = \min_{i}\delta_{i}$, dan geldt het volgende:
      \[ \forall w\in V:\ d(a,w) < \delta \Rightarrow w\in D \]
    \end{itemize}
  \end{proof}
\end{pr}

\begin{opm}
  Deze stelling geldt niet voor een oneindige doorsnede van verzamelingen omdat $\min_{i}\delta_{i}$ dan niet noodzakelijk bestaat.
\end{opm}
\extra{tegenvoorbeeld}

\begin{pr}
  Een doorsnede van gesloten verzamelingen is gesloten.
\extra{bewijs}
\end{pr}

\begin{pr}
  Een \textbf{eindige} unie van gesloten verzamelingen is gesloten
\extra{bewijs}
\end{pr}

\begin{pr}
  Zij $V,d$ een metrische ruimte en $A$ een niet-leeg deel van $V$, dan is $A$ gesloten als en slechts als de limiet van elke convergente rij in $A$ ook tot $A$ behoort.

  \begin{proof}
    Bewijs van een equivalentie.
    \begin{itemize}
    \item $\Rightarrow$\\
      Kies een willekeurige convergente rij $(x_{n})_{n}$ in $A$ en noem de limiet $x$.
      Veronderstel dat $x$ niet tot $A$ zou behoren, dan behoort $x$ tot het open deel $V \setminus A$ van $V$.
      We kunnen dus een $\delta \in \mathbb{R}_{0}^{+}$ vinden als volgt:
      \[ \forall w \in A:\ d(x,w) < \delta \Rightarrow w \in V \setminus A \]
      Omdat $x$ de limiet is van $(x_{n})_{n}$, kunnne we eveneens een $n_{0}\in \mathbb{N}$ vinden als volgt:
      \[ \forall n\in \mathbb{N}:\ n \ge n_{0} \Rightarrow d(x_{n},x) < \delta \]
      Nemen we deze beweringen samen, dan bestaat er minstens \'e\'en $x_{n}$ met $n\ge n_{0}$ in $V \setminus A$.
      Contridictie.
    \item $\Leftarrow$\\
      Bewijs uit het ongerijmde: Stel dat $A$ niet gesloten is.
      $A^{c}$ is dan niet open en er bestaat dus een $a\in V \setminus A$ als volgt:
      \[ \forall \delta \in \mathbb{R}_{0}^{+}, \exists b\in V:\ d(a,b) < \delta \wedge b \in A \]
      We kunnen een rij construeren door voor $\delta_{n}$ telkens $\frac{1}{n}$ te kiezen en zo een $x_{n}$ te bekomen.
      Per constructie geldt voor elke $n\in \mathbb{N}$ dan $d(x_{n},a) < \frac{1}{n}$ en zal de rij dus naar $a$ convergeren.
      We hebben nu een convergente rij $(x_{n})_{n}$ in $A$ geconstrueerd waarvoor de limiet niet tot $A$ behoort.
      Contradictie.
    \end{itemize}
  \end{proof}
\end{pr}
 
\begin{de}
  Zij $V,d$ een metrische ruimte, dan noemen we de unie $\mathring{W}$ van alle open deelverzamelingen van een deelverzameling $W$ van $V$ het \term{inwendige} van $W$.
\end{de}

\begin{de}
  Zij $V,d$ een metrische ruimte, dan noemen we de doorsnede $\overline{W}$ van alle gesloten oververzamelingen van een deelverzameling $W$ van $V$ de \term{sluiting} van $W$.
\end{de}



\begin{st}
  Zij $X,d$ een metrische ruimte en $A$ een niet-leeg deel van $X$.
  Als elke rij een convergente deelrij heeft met limiet in $A$ is $A$ gesloten en begrensd is.
\TODO{bewijs}
\end{st}

\begin{tvb}
  De omgekeerde implicatie geldt niet.
\TODO{tegenvoorbeeld}
\end{tvb}


\end{document}

%%% Local Variables:
%%% mode: latex
%%% TeX-master: t
%%% End:
