\documentclass[main.tex]{subfiles}
\begin{document}



\section{$\exp$, $\sin$ en $\cos$}
\label{sec:exp-sin-en}

\subsection{$\exp$}
\label{sec:exp}

\begin{de}
  De \term{exponentie\"ele functie} op $\mathbb{C}$ is de functie $\exp$.
  \[ \exp:\ \mathbb{C} \rightarrow \mathbb{C}:\ z \mapsto \exp(z) = \sum_{n=0}^{\infty}\frac{z^{n}}{n!} \]
\end{de}

\begin{st}
  De convergentiestraal van $\exp$ is $+\infty$.

  \begin{proof}
    \begin{align*}
      \lim_{n \rightarrow +\infty}\frac{|c_{n}|}{|c_{n+1}|} = \lim_{n
        \rightarrow +\infty}\frac{\frac{1}{n!}}{\frac{1}{(n+1)!}}  =
      \lim_{n \rightarrow +\infty}\frac{(n+1)!}{n!}  = \lim_{n
        \rightarrow +\infty}n+1 = +\infty
    \end{align*}
  \end{proof}
\end{st}

\begin{gev}
  De machtreeks $\exp$, convergeert dus overal absoluut.
\end{gev}

\begin{bst}
  De exponentiele functie is afleidbaar op $\mathbb{C}$ en is zijn eigen afgeleide.

  \begin{proof}
    \begin{align*}
      \lim_{x \rightarrow z}\frac{exp(x)-exp(z)}{x-z}
      &= \lim_{x \rightarrow z}\frac{\sum_{n=0}^{\infty}\frac{x^{n}}{n!}-\sum_{n=0}^{\infty}\frac{z^{n}}{n!}}{x-z}\\
      &= \lim_{x \rightarrow z}\frac{\sum_{n=0}^{\infty}\frac{x^{n}-z^{n}}{n!}}{x-z}\\
      &= \lim_{x \rightarrow z}\frac{\sum_{n=0}^{\infty}x^{n}-z^{n}}{(x-z)n!}\\
      &= \sum_{n=0}^{\infty}\frac{z^{n}}{n!}
    \end{align*}
  \end{proof}
\end{bst}

\begin{bpr}
  \label{pr:epx-0-is-1}
    $\exp(0) = 1$
  \begin{proof}
    \begin{align*}
      exp(0)
      = \sum_{n=0}^{\infty}\frac{0^{n}}{n!}
      = \frac{0^{0}}{0!} + \sum_{n=1}^{\infty}\frac{0}{n!}
      = 1 + \sum_{n=1}^{\infty}0 = 1
    \end{align*}
  \end{proof}
\end{bpr}

\begin{bpr}
  \label{pr:exponent-toegevoegde}
  $\forall z \in \mathbb{C}:\ \overline{\exp(z)} = \exp(\overline{z})$

  \begin{proof}
    \begin{align*}
      \forall z \in \mathbb{C}:\ \overline{\exp(z)}
      = \overline{\sum_{n=0}^{\infty}\frac{z^{n}}{n!}}
      = \sum_{n=0}^{\infty}\overline{\frac{z^{n}}{n!}}
      = \sum_{n=0}^{\infty}\frac{\overline{z^{n}}}{n!}
      = \sum_{n=0}^{\infty}\frac{\overline{z}^{n}}{n!}
      = \exp(\overline{z})
    \end{align*}
  \end{proof}
\end{bpr}

\begin{pr}
  \label{pr:optelling-exp}
  $\forall x,y \in \mathbb{C}:\ \exp(x+y) = \exp(x)\exp(y)$

  \begin{proof}
    Kies willekeurig $x$ en $y$ uit $\mathbb{c}$.
    noem $x_{n} = \frac{x^{n}}{n!}$ en $y_{n} = \frac{y^{n}}{n!}$:
    \[ \exp(x) = \sum_{n=0}^{+\infty}x_{n} \quad\text{ en }\quad \exp(y) = \sum_{n=0}^{+\infty}y_{n} \]
    we weten dat beide reeksen absoluut convergeren en dat de productreeks en dus als volgt uitziet:\stref{st:productreeks-van-absoluut-en-gewoon-convergent}
    
    \begin{align*}
      \exp(x)\exp(y)
      &= \left(\sum_{n=0}^{\infty}\frac{z^{n}}{n!}\right)\left(\sum_{n=0}^{\infty}\frac{z^{n}}{n!}\right)\\
      &= \sum_{n=0}^{+\infty}\sum_{k=0}^{n}x_{n}y_{k-n}\\
      &= \sum_{n=0}^{+\infty}\sum_{k=0}^{n}\frac{x^{k}}{k!}\frac{y^{n-k}}{(n-k)!}\\
      &= \sum_{n=0}^{+\infty}\frac{1}{n!}\sum_{k=0}^{n}\frac{n!}{k!(n-k)!}x^{k}y^{n-k}\\
      &= \sum_{n=0}^{+\infty}\frac{(x+y)^{n}}{n!}\\
    \end{align*}
  \end{proof}
\end{pr}


\begin{bpr}
  \label{pr:exp-nooit-nul}
  $\forall z \in \mathbb{c}:\ \exp(z) \neq 0$

  \begin{proof}
    bewijs uit het ongerijmde: stel dat er een $a\in \mathbb{c}$ bestaat zodat $\exp(a)$ nul is.
    voor elke $b\in \mathbb{c}$ moet $\exp(a+b)$ dan ook $0$ zijn:\prref{pr:optelling-exp}
    \[ \exp(a+b) = \exp(a)\exp(b) = 0\exp(b) = 0 \]
    omdat $b$ hierin willekeurig is moet dan $\exp$ overal nul zijn, maar we weten al dat $\exp(0) = 1 \neq 0$ geldt.\prref{pr:epx-0-is-1}
    contradictie.
  \end{proof}
\end{bpr}
\feed 

\begin{bpr}
  \label{pr:inverse-exponent}
  $\forall z \in \mathbb{c}:\ \exp(-z) = \frac{1}{\exp(z)}$

  \begin{proof}
    \begin{align*}
      1 = \exp(0)
      = \exp(z+(-z))
      = \exp(z)\exp(-z)
      \rightarrow
      \exp(-z)
      = \frac{1}{\exp(z)}
    \end{align*}
    merk op dat dit enkel steek houdt omdat $\exp(z)$ nooit $0$ kan zijn.\prref{pr:exp-nooit-nul}
  \end{proof}
\end{bpr}

\begin{bpr}
  \label{pr:lengte-imaginaire-exponentiatie-1}
  $\forall t \in \mathbb{r}:\ |\exp(it)| = 1$

  \begin{proof}
    \begin{align*}
      1
      &= \exp(0)\\
      &= \exp(it-it)\\
      &= \exp(it)\exp(-it)\\
      &= \exp(it)\exp(\overline{it})\\
      &= \exp(it)\overline{\exp(it)}\\
      &= |\exp(it)|^{2}
    \end{align*}
  \end{proof}
\end{bpr}

\begin{bpr}
  \label{pr:reele-exponentiatie-positief}
  $\forall t \in \mathbb{R}:\ \exp(t) > 0$

  \begin{proof}
    Bewijs uit het ongerijmde: stel dat er een $t\in \mathbb{R}$ bestaat zodat $\exp(t)$ negatief is.
    $\exp(t)$ is zeker geen nul\prref{pr:exp-nooit-nul} en gelijk aan $\exp\left(\frac{t}{2}\right)\exp\left(\frac{t}{2}\right)$.\prref{pr:optelling-exp}
    Omdat $\exp(t)$ negatief is moeten deze factoren een verschillend teken hebben.
    Contradictie.
  \end{proof}
\end{bpr}
  \feed

\begin{bpr}
  $\forall z \in \mathbb{C}:\ |\exp(z)| = \exp(Re(z))$

  \begin{proof}
    \begin{align*}
      |\exp(z)|
      &= |\exp(Re(z) + Im(z)i)|\\
      &= |\exp(Re(z))\exp(Im(z)i)|\\
      &= |\exp(Re(z))||\exp(Im(z)i)|\\
      &\overset{\prref{pr:lengte-imaginaire-exponentiatie-1}}{=} |\exp(Re(z))|\\
      &\overset{\prref{pr:reele-exponentiatie-positief}}{=} \exp(Re(z))\\
    \end{align*}
  \end{proof}
\end{bpr}

\begin{pr}
  $\exp(\mathbb{R}) \subseteq \mathbb{R}$

  \begin{proof}
    We zullen bewijzen dat voor een willeurige $r\in \mathbb{R}$, $\exp(r)$ gelijk is aan $\overline{\exp(r)}$.
    \begin{align*}
      \overline{\exp(r)}
      =
      \exp\left(\overline{r}\right) = \exp(r)
    \end{align*}
  \end{proof}
\end{pr}

\begin{bpr}
  \label{pr:exp-van-positief-getal-groter-dan-1}
  $\forall t \in \mathbb{R}_{0}^{+}:\ \exp(t) > 1$

  \begin{proof}
    Kies een willekeurige $t\in \mathbb{R}_{0}^{+}$.
    \begin{align*}
      \exp(t)
      = \sum_{n=0}^{\infty}\frac{t^{n}}{n!}
      = \frac{t^{0}}{n!} + \sum_{n=0}^{\infty}\frac{t^{n}}{n!}
      = 1 + \sum_{n=0}^{\infty}\frac{t^{n}}{n!}
      > 1
    \end{align*}
  \end{proof}
\end{bpr}

\begin{bpr}
  \label{pr:exponent-naar-oneindig-oneindig}
  De functie $\exp: \mathbb{R} \rightarrow \mathbb{R}_{0}^{+}:\ t \rightarrow \exp(t)$ is strikt stijgend.

  \begin{proof}
    Kies twee elemnten $x$ en $y$ uit $\mathbb{R}$ met $x < y$.
    We weten dan dat $y-x$ strikt positief is en daarmee het volgende:\prref{pr:exp-van-positief-getal-groter-dan-1}
    \[ \exp(y-x) > 1 \]
    \begin{align*}
      \exp(y-x)
      = \frac{\exp(y)}{\exp(x)}
      > 1 \Rightarrow \exp(y)
      > \exp(x)
    \end{align*}
  \end{proof}
\end{bpr}

\begin{bpr}
  $\lim_{t \rightarrow +\infty}\exp(t) = +\infty$

  \begin{proof}
    Voor een willekeurige $t\in \mathbb{R}_{0}^{+}$ vinden we $\exp(t) > 1 + t$ uit de reeksontwikkeling.
    Dit, samen met het strikt stijgen van $\exp$, betekent dat $\exp$ naar $+\infty$ divergeert.
  \end{proof}
\end{bpr}

\begin{bpr}
  $\lim_{t \rightarrow -\infty}\exp(t) = 0$

  \begin{proof}
    Dit volgt meteen uit de rekenregel voor de inverse exponent\prref{pr:inverse-exponent} en de limiet naar oneindig van $\exp$.\prref{pr:exponent-naar-oneindig-oneindig}
  \end{proof}
\end{bpr}

\begin{bpr}
  De functie $\exp: \mathbb{R} \rightarrow \mathbb{R}_{0}^{+}:\ t \rightarrow \exp(t)$ is een bijectie.

  \begin{proof}
    Bewijs van bijectiviteit
    \begin{itemize}
    \item $\exp$ is injectief\\
      $\exp$ is overal strikt stijgend en dus injectief.\prref{pr:exponent-naar-oneindig-oneindig}
    \item $\exp$ is surjectief.\\
      Omdat $\exp$ afleidbaar is, is $\exp$ continu.
      Nu is $\exp$ surjectief vanwege de tussenwaardestelling voor continue functies.\stref{st:tussenwaardestelling}
    \end{itemize}
  \end{proof}
\end{bpr}

\begin{de}
  De inverse van de exponenti\"ele functie noemen we de \term{logaritmische functie}
  \[ \ln:\ \mathbb{R}_{0}^{+}\rightarrow \mathbb{R}:\ x \mapsto \exp^{-1}(x) \]
\end{de}

\extra{eigenschappen onderaan p 29}

\begin{de}
  We definieren $e = \exp(1)$.
\end{de}

\begin{bpr}
  $\forall q \in \mathbb{Q}:\ \exp(q) = e^{q}$

  \begin{proof}
    Noem $q = \frac{n}{m}$ en neem aan dat $q$ strikt positief is (in het andere geval kunnen we dit herleiden tot een positieve $q$).
    Merk eerst op dat $\exp(n)$ gelijk is aan $\exp(1)^{n} = e^{n}$.
    Merk nu het volgende op:
    \[ \exp(q)^{m} = \exp(mq) = \exp(n) = e^{n} \]
    Omdat $q$ strikt positief is mogen we het volgende besluiten: 
    \[ \exp(q) = \sqrt[m]{e^{n}} = e^{q} \]
  \end{proof}
\end{bpr}

\begin{bpr}
  Zij $x\in \mathbb{R}$ en $(q_{n})_{n}$ een rij in $\mathbb{Q}$ die naar $x$ convergeert.
  \[ \exp(x) = \lim_{n\rightarrow +\infty}e^{q_{n}} \]

  \begin{proof}
    Dit volgt meteen uit de continu\"iteit van $\exp$.\needed
  \end{proof}
\end{bpr}


\subsection{$\sin$ en $\cos$}
\label{sec:sin-en-cos}

\begin{de}
  We definie\"eren de \term{cosinusfunctie} $\cos$ als volgt:
  \[ \cos:\ \mathbb{C} \rightarrow \mathbb{C}:\ z \mapsto \cos(z) = \frac{e^{iz} + e^{-iz}}{2} \]
\end{de}

\begin{de}
  We definie\"eren de \term{sinusfunctie} $\cos$ als volgt:
  \[ \sin:\ \mathbb{C} \rightarrow \mathbb{C}:\ z \mapsto \sin(z) = \frac{e^{iz} - e^{-iz}}{2i} \]
\end{de}

\begin{bpr}
  $\forall z \in \mathbb{C}:\ cos(z) = \sum_{n=0}^{+\infty}(-1)^{n}\frac{z^{2n}}{(2n)!}$

  \begin{proof}
    \begin{align*}
      \cos(z)
      &= \frac{e^{iz} + e^{-iz}}{2}\\
      &= \frac{\sum_{n=0}^{\infty}\frac{(iz)^{n}}{n!} + \sum_{n=0}^{\infty}\frac{(-iz)^{n}}{n!}}{2}\\
      &= \sum_{n=0}^{\infty}\frac{(iz)^{n} + (-iz)^{n}}{2(n!)}\\
      &= \sum_{n=0}^{\infty}\frac{(iz)^{n} + (-iz)^{n}}{2(n!)}\\
      &= \sum_{n=0}^{+\infty}(-1)^{n}\frac{z^{2n}}{(2n)!}
    \end{align*}
  \end{proof}
\extra{dit kan beter uitgeschreven}
\end{bpr}

\begin{bpr}
  $\forall z \in \mathbb{C}:\ sin(z) = \sum_{n=0}^{+\infty}(-1)^{n}\frac{z^{2n+1}}{(2n+1)!}$

  \begin{proof}
    \begin{align*}
      \sin(z)
      &= \frac{e^{iz} - e^{-iz}}{2i}\\
      &= \frac{\sum_{n=0}^{\infty}\frac{(iz)^{n}}{n!} - \sum_{n=0}^{\infty}\frac{(-iz)^{n}}{n!}}{2i}\\
      &= \sum_{n=0}^{\infty}\frac{(iz)^{n} - (-iz)^{n}}{2i(n!)}\\
      &= \sum_{n=0}^{+\infty}(-1)^{n}\frac{z^{2n+1}}{(2n+1)!}
    \end{align*}
  \end{proof}
\extra{dit kan beter uitgeschreven}
\end{bpr}

\begin{bpr}
  $\forall z \in \mathbb{C}: \cos(-z) = \cos(z)$

  \begin{proof}
    \begin{align*}
      \cos(-z)
      = \frac{e^{-zi} + e^{iz}}{2}
      = \frac{e^{zi} + e^{-iz}}{2}
      = \cos(z)
    \end{align*}
  \end{proof}
\end{bpr}

\begin{bpr}
  $\forall z \in \mathbb{C}: \sin(-z) = -\sin(z)$

  \begin{proof}
    \begin{align*}
      \sin(-z)
      = \frac{e^{-zi} - e^{iz}}{2i}
      = -\frac{e^{zi} - e^{-iz}}{2i}
      = -\sin(z)
    \end{align*}
  \end{proof}
\end{bpr}

\begin{bpr}
  $\cos(0) = 1$

  \begin{proof}
    \begin{align*}
      \cos(0)
      = \frac{e^{i0} + e^{-i0}}{2}
      = \frac{1+1}{2}
      = 1
    \end{align*}
  \end{proof}
\end{bpr}

\begin{bpr}
  $\sin(0) = 0$
  \begin{proof}
    \begin{align*}
      \sin(0)
      = \frac{e^{i0} - e^{-i0}}{2i}
      = \frac{1-1}{2i}
      = 0
    \end{align*}
  \end{proof}
\end{bpr}

\begin{bpr}
  $\cos$ is afleidbaar op $\mathbb{C}$ met $\cos'$ als afgeleide:
  \[ \cos' = -\sin \]
  
  \begin{proof}
    Kies een willekeurige $z\in \mathbb{C}$
    \begin{align*}
      \cos'(z)
      = \left(\frac{e^{iz} + e^{-iz}}{2}\right)'
      = \frac{\left(e^{iz}\right)' + \left(e^{-iz}\right)'}{2}
      = i\frac{e^{iz} - e^{-iz}}{2}
      = -\frac{e^{iz} - e^{-iz}}{2i}
      = -\sin(z)
    \end{align*}
  \end{proof}
\end{bpr}

\begin{bpr}
  $\sin$ is afleidbaar op $\mathbb{C}$ met $\sin'$ als afgeleide:
  \[ \sin' = cos \]

  \begin{proof}
    Kies een willekeurige $z\in \mathbb{C}$
    \begin{align*}
      \sin'(z)
      = \left(\frac{e^{iz} - e^{-iz}}{2i}\right)'
      = \frac{\left(e^{iz}\right)' - \left(e^{-iz}\right)'}{2i}
      = i\frac{e^{iz}+e^{iz}}{2i}
      = \frac{e^{iz}+e^{iz}}{2}
      = \cos(z)
    \end{align*}
  \end{proof}
\end{bpr}

\begin{bpr}
  \label{pr:cos-van-reel-reel}
  $\forall x\in \mathbb{R} \cos(x) = Re(e^{ix}) \in \mathbb{R}$

  \begin{proof}
    Kies willekeurig een $x\in \mathbb{R}$
    \begin{align*}
      \cos(x)
      = \frac{e^{ix} + e^{-ix}}{2}
      = \frac{e^{ix} + \overline{e^{ix}}}{2}
      = \frac{2Re(e^{ix})}{2}
      = Re(e^{ix})
      \in \mathbb{R}
    \end{align*}
  \end{proof}
\end{bpr}

\begin{bpr}
  \label{pr:sin-van-reel-reel}
  $\forall x\in \mathbb{R} \sin(x) = Im(e^{ix}) \in \mathbb{R}$
  
  \begin{proof}
    Kies willekeurig een $x\in \mathbb{R}$
    \begin{align*}
      \sin(x)
      = \frac{e^{ix} - e^{-ix}}{2i}
      = \frac{e^{ix} - \overline{e^{ix}}}{2i}
      = \frac{2Im(e^{ix})}{2i}
      = Im(e^{ix})
      \in \mathbb{R}
    \end{align*}
  \end{proof}
\end{bpr}

\begin{bpr}
  $\forall t\in \mathbb{R}:\ e^{it} = \cos(t) + i\sin(t)$

  \begin{proof}
    Dit volgt meteen uit de vorige twee stellingen.\prref{pr:cos-van-reel-reel}\prref{pr:sin-van-reel-reel}
  \end{proof}
\end{bpr}

\begin{bpr}
  $\forall t\in \mathbb{R}:\ \cos^{2}(t) + \sin^{2}(t) = 1$

  \begin{proof}
    Kies willekeurig een $x\in \mathbb{R}$
    \begin{align*}
      \cos^{2}(t) + \sin^{2}(t)
      &=  \left(\frac{e^{it} + e^{-it}}{2}\right)^{2} + \left(\frac{e^{it} - e^{-it}}{2i}\right)^{2}\\
      &=  \frac{(e^{it} + e^{-it})^{2}-(e^{it} - e^{-it})^{2}}{4}\\
      &=  \frac{\cancel{e^{2it}} +2e^{it}e^{-it} +\cancel{e^{-2it}}-\cancel{e^{2it}} +2e^{it}e^{-it}- \cancel{e^{-2it}}}{4}\\
      &=  \frac{4e^{it}e^{-it}}{4}\\
      &=  \frac{4\frac{e^{it}}{e^{it}}}{4}\\
      &=  \frac{4}{4}
      = 1
    \end{align*}
  \end{proof}
\end{bpr}

\begin{bpr}
  $\forall x,y\in \mathbb{R}:\ \cos(x+y) = \cos(x)\cos(y) - \sin(x)\sin(y)$

  \begin{proof}
    \begin{align*}
      &\cos(x)\cos(y) - \sin(x)\sin(y)\\
      &= \frac{e^{ix} + e^{-ix}}{2}\frac{e^{iy} + e^{-iy}}{2} - \frac{e^{ix} - e^{-ix}}{2i}\frac{e^{iy} - e^{-iy}}{2i}\\
      &= \frac{\left(e^{ix} + e^{-ix}\right)\left(e^{iy} + e^{-iy}\right)}{4} + \frac{\left(e^{ix} - e^{-ix}\right)\left(e^{iy} - e^{-iy}\right)}{4}\\
      &= \frac{e^{i(x+y)} + \cancel{e^{ix - iy}} + \cancel{e^{-ix +iy}} + e^{-i(x+y)} + e^{i(x+y)} - \cancel{e^{ix-iy}} -\cancel{e^{-ix+iy}} + e^{-i(x+y)}}{4}\\
      &= \frac{e^{i(x+y)} + e^{-i(x+y)}}{2}\\
      &= \cos(x+y)
    \end{align*}
  \end{proof}
\end{bpr}

\begin{bpr}
  $\forall x,y\in \mathbb{R}:\ \sin(x+y) = \sin(x)\cos(y) + \cos(x)\sin(y)$

  \begin{proof}
    \begin{align*}
      &\sin(x)\cos(y) + \cos(x)\sin(y)\\
      &=\frac{e^{ix} - e^{-ix}}{2i}\frac{e^{iy} + e^{-iy}}{2} + \frac{e^{ix} + e^{-ix}}{2}\frac{e^{iy} - e^{-iy}}{2i}\\
      &= \frac{\left(e^{ix} - e^{-ix}\right)\left(e^{iy} + e^{-iy}\right)}{4i} + \frac{\left(e^{ix} + e^{-ix}\right)\left(e^{iy} - e^{-iy}\right)}{4i}\\
      &= \frac{e^{i(x+y)}+\cancel{e^{ix-iy}}-\cancel{e^{-ix+iy}}-e^{-i(x+y)} + e^{i(x+y)}-\cancel{e^{ix-iy}}+\cancel{e^{-ix+iy}}-e^{-i(x+y)}}{4i}\\
      &= \frac{e^{i(x+y)} - e^{-i(x+y)}}{2i}\\
      &= \sin(x+y)
    \end{align*}
  \end{proof}
\end{bpr}

\begin{bpr}
  \label{pr:inleiding-pi}
  Er bestaat een unieke $t_{0} \in \interval[open]{0}{2}$ zodat het volgende geldt:
  \begin{itemize}
  \item $\cos(t_{0}) = 0$
  \item $\forall t\in \interval[open right]{0}{t_{0}}:\ \cos(t) > 0$
  \item $\sin(t_{0})=1$
  \item $\forall t\in \interval[open left]{0}{t_{0}}:\ \sin(t)> 0$ 
  \end{itemize}

  \begin{proof}
    In delen:
    \begin{itemize}
    \item 
    We weten dat $\cos(0) = 1$ geldt en beweren nu dat $\cos(2)$ strikt negatief is:
    \[ \cos(2) = 1 - \frac{2^{2}}{2!} + \frac{2^{4}}{4!} - \sum_{n=2}^{+\infty}\left( \frac{2^{4n-2}}{(4n-2)!} - \frac{2^{4n}}{(4n!)}\right) \]
    De eerste drie termen hierin zijn al strikt negatief...
    \[ 1 - \frac{2^{2}}{2!} + \frac{2^{4}}{4!} = 1-2+\frac{16}{24} < 0 \]
    ... en de volgende termen samen zijn ook negatief:
    \[
    \forall n\in \mathbb{N}:\ n \ge 2
    \Rightarrow -\frac{2^{4n-2}}{(4n-2)!} - \frac{2^{4n}}{(4n!)}
    = -\frac{2^{4n-2}}{(4n-2)!}\left( 1 - \frac{4}{4n(4n-1)} \right)
    > 0
    \]
    Omdat $\cos$ continu is volgt uit de tussenwaardestelling\stref{st:tussenwaardestelling} dat $\cos$ minstens \'e\'en nulpunt moet heben in $\interval[open]{0}{2}$.
    De verzameling $N = \{ t \in \interval[open]{0}{2} \mid \cos(t) = 0\}$ is dus niet leeg en naar onder begrensd.
    Noem $t_{0}$ het infimum van deze verzameling.
    Omdat $\cos$ continu is, moet $\cos(t_{0})$ nul zijn. \waarom
  \item 
    Omdat bovendien $cos(0) =1$ geldt, moet $t_{0}$ strikt positief zijn.
    Per constructie heeft $\cos$ geen nulpunten tussen $0$ en $t_{0}$.
    Omdat $\cos(0)$ positief is, volgt nu uit de continu\"iteit van $\cos$ dat $\forall t\in \interval[open right]{0}{t_{0}}:\ \cos(t)>0$ geldt.
  \item
    Omdat $\sin'=\cos$ geldt, stijgt $\sin$ strikt op $\interval{0}{t_{0}}$.
    Omdat $\sin(0) = 0$ geldt, betekent dit dat $\forall t\in \interval[open left]{0}{t_{0}}:\ \sin(t)>0$ geldt.
  \item Vanuit $\sin^{2}(t_{0}) + \cos^{2}(t_{0}) = 1$ vinden we nu dat $\sin(t_{0})$ gelijk is aan $1$.
    \end{itemize}
  \end{proof}
\end{bpr}

\begin{de}
  We definieren $\pi$ als $2t_{0}$ uit de vorige propositie:
  \[ \cos\left(\frac{\pi}{2}\right) = 0 \quad\text{ en }\quad \sin\left(\frac{\pi}{2}\right) = 1 \]
\end{de}

\begin{bpr}
  \label{pr:cos-en-sin-periodiek}
  De sinusfuncie en de cosinusfunctie zijn periodiek op $\mathbb{R}$ met periode $2\pi$.

  \begin{proof}
    Beschouw eerst voor alle $t\in \mathbb{R}$:
    \begin{align*}
      \cos\left(t+\frac{\pi}{2}\right)
      = \cos(t)\cos\left(\frac{\pi}{2}\right) - \sin(t)\sin\left(\frac{\pi}{2}\right)
      = \cos(t)0 - \sin(t)1
      = - \sin(t)
    \end{align*}
    ... en ...
    \begin{align*}
      \sin\left(t+\frac{\pi}{2}\right)
      = \sin(t)\cos\left(\frac{\pi}{2}\right) + \cos(t)\sin\left(\frac{\pi}{2}\right)
      = \sin(t)0 + \cos(t)1
      = \cos(t)
    \end{align*}
    Het volstaat dus om de periodiciteit van $\cos$ aan te tonen.
    Beschouw nu $\cos(t+2\pi)$:
    \begin{align*}
      \cos(t+2\pi)
      = -\sin\left(t+\frac{3}{2}\pi\right)
      = -\cos(t+\pi)
      = \sin\left(t + \frac{\pi}{2}\right)
      = \cos(t)
    \end{align*}
    Om periodiciteit aan te tonen moeten we nog aantonen dat er geen kleinere positieve $T$ bestaat zodat $\forall \cos(t+T)=\cos(t)$ geldt.
    We tonen nog iets sterker aan: $\forall T\in \interval[open]{0}{2\pi}: \cos(T) < \cos(0) = 1$.
    Kies daartoe een willekeurige $T\in \interval[open]{0}{2\pi}$ en kies $t=\frac{T}{4}$, dan geldt $t\in\interval[open]{0}{\frac{\pi}{2}}$ en bijgevolg ook $\cos(t) > 0$.
    Anderzijds geldt ook $\cos(t) < 1$ want er bestaat een $c\in \interval[open]{0}{1}$ als volgt:\stref{st:middelwaardestelling-lagrange}\prref{pr:inleiding-pi}
    \[ \frac{\cos(t)-\cos(0)}{t-0} = \cos'(c) = -sin(c) < 0 \]
    We bekijken nu $\cos(T)$:
    \begin{align*}
      \cos(T)
      &= Re(\exp(iT))\\
      &= Re(e^{4it})\\
      &= Re\left(\left(\cos(t) +i\sin(t)\right)^{4}\right)\\
      &= \cos^{4}(t) - 6\cos^{2}(t)\sin^{2}(t) + \sin^{4}(t)\\
      &= 8\cos^{4}(t) - 8\cos^{2}(t) + 1\\
      &= 1-8\left(t-\cos^{2}(t)\right)\cos^{2}(t)\\
      &<1
    \end{align*}
  \end{proof}
\end{bpr}

\begin{bpr}
  Neem $t\in \interval[open]{0}{2\pi}$ en beschouw het punt $(\cos(t),\sin(t))$ op de cirkel in $\mathbb{R}^{2}$ met middelpunt $(0,0)$ en straal $1$.
  $t$ is dan de booglengte van de boog op de cirkel die in $(1,0)$ begint en eindigt in $(\cos(t),\sin(t))$.
  De cirkel wordt hierbij doorlopen in tegenzin.

  \begin{proof}
    Kies $n\in \mathbb{N}_{0}$ willekeurig en beschouw $z_{k}$ als volgt:
    \[ z_{k} = \left(\cos\frac{kt}{n},\sin\frac{kt}{n}\right) \]
    We bewijzen dat de totale afstand tussen opeenvolgende punten $z_{k}$ naar $t$ gaat in de limiet voor $k$ gaande naar $+\infty$.
    Beschouw eerst het kwadraat van de afstand tussen twee opeenvolgende punten $z_{k}$ en $z_{k+1}$.
    \begin{align*}
      \left\|z_{k+1}-z_{k}\right\|^{2}
      &= \left( \cos\left(\frac{(k+1)t}{n}\right) - \cos\left(\frac{kt}{n}\right) \right)^{2}
       + \left( \sin\left(\frac{(k+1)t}{n}\right) - \sin\left(\frac{kt}{n}\right) \right)^{2}\\
      &= \cos^{2}\left(\frac{(k+1)t}{n}\right) - 2\cos\left(\frac{(k+1)t}{n}\right)\cos\left(\frac{kt}{n}\right) + \cos^{2}\left(\frac{kt}{n}\right)\\
      &+ \sin^{2}\left(\frac{(k+1)t}{n}\right) - 2\sin\left(\frac{(k+1)t}{n}\right)\cos\left(\frac{kt}{n}\right) + \sin^{2}\left(\frac{kt}{n}\right)\\
      &= 2 - 2\cos\left(\frac{(k+1)t}{n}\right)\cos\left(\frac{kt}{n}\right) - 2\sin\left(\frac{(k+1)t}{n}\right)\cos\left(\frac{kt}{n}\right)\\
      &= 2 - \left(2\cos\left(\frac{(k+1)t}{n}\right)\cos\left(-\frac{kt}{n}\right) - 2\sin\left(\frac{(k+1)t}{n}\right)\cos\left(-\frac{kt}{n}\right)\right)\\
      &= 2-2\cos\left( \frac{(k+1)t}{n} - \frac{kt}{n}\right)\\
      &= 2-2\cos\left(\frac{t}{n}\right)\\
      &= 2-2\cos\left(\frac{t}{2n}+ \frac{t}{2n}\right)\\
      &= 2-2\left(\cos^{2}\left(\frac{t}{2n}\right)-\sin^{2}\left(\frac{t}{2n}\right)\right)\\
      &= 2\left(1-\cos^{2}\left(\frac{t}{2n}\right)\right) + 2\sin^{2}\left(\frac{t}{2n}\right)\\
      &= 4\sin^{2}\left(\frac{t}{2n}\right)
    \end{align*}
    Voor $n$ groot genoeg is $\sin\left(\frac{t}{2n}\right)$ strikt positief en ziet de afstand tussen twee opeenvolgende punten er als volgt uit:
    \[ \left\|z_{k+1}-z_{k}\right\| = 2\sin\left(\frac{t}{2n}\right) \]
    Merk op dat dit onafhankelijk is van $k$.
    De totale lengte van al deze koorden is dan $2n\sin\left(\frac{x}{2n}\right)$.
    \begin{align*}
      \lim_{n\rightarrow +\infty}2n\sin\left(\frac{t}{2n}\right)
      = \lim_{n\rightarrow +\infty}\frac{2nt}{t}\sin\left(\frac{t}{2n}\right)
      = \lim_{n\rightarrow +\infty}t\frac{\sin\left(\frac{t}{2n}\right)}{\frac{t}{2n}}
      = t
    \end{align*}
  \end{proof}
\end{bpr}

\begin{bpr}
  \[ \forall z\in \mathbb{C}:\ \exp(z+2\pi i) = \exp(z) \quad\text{ en }\quad \exp(z) = 1 \Leftrightarrow \exists k\in \mathbb{Z}:\ z = 2k\pi i \]

  \begin{proof}
    \[ \forall z \in \mathbb{C}:\ \exp(z + 2\pi i) = \exp(z)\exp(2\pi i) = \exp(z) \]
    \question{waarom is $\exp(2\pi i)$ precies $1$?}
    Zij $x+yi \in \mathbb{C}$ met $x,y \in \mathbb{R}$ met $\exp(z)=1$.
    Beschouw nu $|\exp(z)| = 1$:
    \begin{align*}
      \left|\exp(z)\right|
      = \left|\exp(x+yi)\right|
      = \left|\exp(x)\right|\left|\exp(yi)\right|
      = \left|\exp(x)\right|1
      = \exp(x)
    \end{align*}
    Hieruit volgt dat $x$ nul moet zijn.\waarom
    We houden over dat $e^{iy}=1$ geldt.
    $\cos(y)$ moet dus $1$ zijn.
    We weten dat $\cos(t) < 1$ geldt voor alle $t\in \interval[open]{0}{2\pi}$.
    Omdat $\cos$ ook periode $2\pi$ heeft\prref{pr:cos-en-sin-periodiek} moet $\forall t\in (\mathbb{R} \setminus 2\pi\mathbb{Z}):\ \cos(t) < 1$ gelden.
    Bijgevolg moet $y$ een element zijn van $2\pi \mathbb{Z}$ en er dus een $k$ bestaat zodat $y$ gelijk is aan $2\pi k$.
  \end{proof}
\end{bpr}

\begin{bpr}
  Zij een $z\in \mathbb{C}_{0}$, dan bestaat er een unieke $\theta \in \interval[open right]{0}{2\pi}$ zodat $z=e^{i\theta}|z|$ geldt.

  \begin{proof}
    Kies een willekeurige $z\in \mathbb{C}_{0}$ met $|z| = 1$.
    Het geval waarin $|z|$ niet $1$ is, valt te herleiden naar een geval waarin $|z|=1$ wel geldt, met dezelfde $\theta$. \waarom
    \begin{itemize}
    \item Bestaan\\
      Noem $z=x+yi$ met $x,y\in \mathbb{R}$ en merk op dat $x^{2}+y^{2}=1$ geldt.
      We maken een gevalsonderscheid voor de tekens van $x$ en $y$.
      \begin{itemize}
      \item $x\ge 0$ en $y\ge 0$\\
        Omdat $x$ in $\interval{0}{1}$ zit, omdat $\cos(0) = 1$ en $\cos\left(\frac{\pi}{2}\right) = 0$ gelden en omdat $\cos$ continu is, bestaat er een $\theta \in \interval{0}{\frac{\pi}{2}}$ zodat $\cos(\theta) = x$ geldt.
        Dan geldt, omdat $\sin(\theta)$ positief is\waarom, het volgende:
        \[ y = \sqrt{1-x^{2}} = \sqrt{1-\cos^{2}(\theta)} = \sqrt{\sin^{2}(\theta)} = \sin(\theta) \]
        Bijgevolg geldt $x+yi = \cos(\theta) + i \sin(\theta) = e^{i\theta}$.\needed
      \item $x<0$ en $y \ge 0$\\
        Omdat $x$ in $\interval[open right]{-1}{0}$ zit, $\cos(\pi) = -1$ en $\cos\left(\frac{\pi}{2}\right) = 0$ gelden en omdat $\cos$ continu is,  $\theta \in \interval[open left]{\frac{\pi}{2}}{\pi}$ zodat $\cos(\theta) = x$ geldt.
        Analoog aan het vorige geval volgt uit $\sin(\theta)>0$ dat $y$ gelijk is aan $\sin(\theta)$ en dus dat $x+yi = \cos(\theta) + i \sin(\theta) = e^{i\theta}$ geldt.
      \item $x<0$ en $y < 0$\\
        Omdat $x$ in $\interval[open right]{-1}{0}$ zit, omdat $\cos(\pi) = -1$ en $\cos\left(\frac{3\pi}{2}\right) = 0$\waarom gelden en omdat $\cos$ continu is, bestaat er een $\theta \in \interval[open]{\pi}{\frac{3\pi}{2}}$ zodat $x = \cos(\theta)$ geldt.
        Nu is $\sin(\theta)$ strikt negatief en hieruit volgt dan het volgende:
        \[ y = -\sqrt{1-x^{2}} = -\sqrt{1-\cos^{2}(\theta)} = -\sqrt{\sin^{2}(\theta)} = \sin(\theta) \]
        Bijgevolg geldt $x+yi= e^{i\theta}$.
      \item $x\ge 0$ en $y< 0$\\
        Omdat $x$ in $\interval{0}{1}$ zit, omdat $\cos()=$ en $\cos\left(\right)=$ gelden en omdat $\cos$ continu is, bestaat er een $\theta \in \interval[open right]{\frac{3\pi}{2}}{2\pi}$ zodat $\cos(\theta) = x$x geldt.
        Analoog aan het vorige geval volgt uit $\sin(\theta)<0$ dat $y$ gelijk is aan $\sin(\theta)$ en dus dat $x+yi = \cos(\theta) + i \sin(\theta) = e^{i\theta}$ geldt.
      \end{itemize}
    \item Uniciteit\\
      Dit volgt meteen uit de periodiciteit van $\exp$.
      Stel immers dat er twee getallen $\theta_{1}$ en $\theta_{2}$ zouden bestaan als volgt:
      \[ e^{i\theta_{1}}|z| = e^{i\theta_{2}}|z| \Rightarrow e^{i(\theta_{1}-\theta_{2})} = 1 \]
      \[ \Leftrightarrow \exists k\in \mathbb{Z}:\ \theta_{1}-\theta_{2} = k2\pi \]
      Omdat $\theta_{1}$ beide in $\interval[open right]{0}{2\pi}$ moeten zitten leidt dit tot een contradictie.
    \end{itemize}
  \end{proof}
\end{bpr}

\begin{bst}
  Zij $P:\ \mathbb{C} \rightarrow \mathbb{C}:\ z \mapsto P(z) = \sum_{k=0}^{n}a_{k}z^{k}$ een veeltermfunctie van graad $n\in \mathbb{N}_{0}$ met co\"efficienten $a_{k}\in \mathbb{C}$, dan bestaat er een $z_{0}\in \mathbb{C}$ waarvoor $P(z_{0})=0$ geldt.

  \begin{proof}
    Zonder beperking van de algemeenheid veronderstellen we dat $a_{n}=1$ geldt.
    In het andere geval delen we de $a_{k}$ gewoon door $a_{n}$.
    Dit kan omdat we het over nulpunten hebben.
    \begin{itemize}
    \item Noem $m = \inf\{ |P(x)| \mid z\in \mathbb{C} \}$.
      Kies nu willekeurig een $r\in \mathbb{R}_{0}^{+}$.
      \[
      \forall z\in \mathbb{C}:\ |z| = r \Rightarrow
      \begin{array}{rl}
        |z^{n}|
        &= \left|P(z) - \sum_{k=0}^{n}a_{k}z^{k}\right|\\
        &= \left|P(z) + \sum_{k=0}^{n}-a_{k}z^{k}\right|\\
        &\le \left|P(z)\right| + \sum_{k=0}^{n}|-a_{k}||z|^{k}\\
        r^{n}&\le \left|P(z)\right| + r^{n}\sum_{i=0}^{n-1}\frac{|a_{k}|}{r^{k}}\\
        |P(z)| &\ge r^{n}\left(1-\sum_{i=0}^{n-1}\frac{|a_{k}|}{r^{k}}\right)
      \end{array}     
      \]
      Duidelijk is dat het rechterlid hierin naar $+\infty$ gaat voor $r$ gaande naar $+\infty$.
      Er bestaat dus een $r_{0}\in \mathbb{R}_{0}^{+}$ zodat $|P(z)| > $ geldt voor alle $z\in \mathbb{C}$ met $|z| > r_{0}$.
      Bijgevolg ziet $m$ er als volgt uit:
      \[ m =  \inf\{ |P(x)| \mid z\in \mathbb{C}, |z| \le r_{0} \} \]
      Omdat $z \mapsto |P(z)|$ een continue functie is en $\{ |P(x)| \mid z\in \mathbb{C}, |z| \le r_{0} \}$ een gesloten, begrensde verzameling, bestaat er een $z_{0}\in \mathbb{C}$ met $|z_{0}| \le r_{0}$ zodat $m = |P(z_{0})|$ geldt.\waarom
      We beweren nu dat $m$ nul is en dat $z_{0}$ dus een nulpunt is van $P$.
      Veronderstel immers dat $m$ strikt positief is, beschouw dan de veeltermfunctie $Q$ als volgt:
      \[ Q:\ \mathbb{C} \rightarrow \mathbb{C}:\ z \mapsto \frac{P(z+z_{0})}{P(z_{0})} \]
      Er geldt nu $Q(0)=1$ en $\forall z \in \mathbb{C}:\ |Q(z)| \ge 1$.
      Merk op dat $Q(z)$ een veelterm is van graad $n$ en noteer met $k$ de kleinste strikt positieve macht van $z$ waarvan de co\"efficient niet nul is.
      \[ Q(z) = 1+ \sum_{i=k}^{n}b_{i}z^{i} \]
      Er bestaat nu een $\theta$ als volgt:\needed
      \[ e^{ik\theta}b_{k} = -|b_{k}| \]
      Als $r\in \mathbb{R}_{0}^{+}$ geldt en $r$ klein genoeg is, opdat $r^{k}|b_{k}|<1$ geldt, zal het volgende gelden:
      \[ |1+r^{k}e^{ik\theta}b_{k} = |1-r^{k}|b_{k}|| = 1-r^{k}|b_{k}| \]
      Hieruit volgt dan dit:
      \[ |Q(re^{i\theta})| \le 1-r^{k}\left(|b_{k}| - \sum_{i=1}^{n-k}r^{i}|b_{k+i}|\right) \]
      Voor $r$ klein genoeg is het rechterlid strikt pasitief en $|Q(re^{i\theta})|$ dus strikt kleiner dan $1$.
      Contradictie.
    \end{itemize}
  \end{proof}
\extra{hermaken met meer uitleg en agenda erbij}
\end{bst}


\end{document}

%%% Local Variables:
%%% mode: latex
%%% TeX-master: t
%%% End:
