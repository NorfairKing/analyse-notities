\theoremstyle{plain}
\newtheorem{thm}{Theorem}[chapter] %Reset counter elk hoofdstuk
\theoremstyle{definition}
\newmdtheoremenv{de}[thm]{Definitie} % Definitie met frame
\newmdtheoremenv{bei}[thm]{Eigenschap}
\newtheorem{ei}[thm]{Eigenschap}
\newmdtheoremenv{bst}[thm]{Stelling}
\newtheorem{st}[thm]{Stelling}
\newtheorem{gev}[thm]{Gevolg}
\newmdtheoremenv{bgev}[thm]{Gevolg}
\newtheorem{pr}[thm]{Propositie}
\newmdtheoremenv{bpr}[thm]{Propositie}
\newtheorem{opm}[thm]{Opmerking}
\newtheorem{vb}[thm]{Voorbeeld}
\newtheorem{tvb}[thm]{Tegenvoorbeeld}
\newtheorem{lem}[thm]{Lemma}
\newmdtheoremenv{blem}[thm]{Lemma}
\newtheorem{al}[thm]{Algoritme}
\newmdtheoremenv{bal}[thm]{Algoritme}
\newtheorem{gst}[thm]{Zeker geen stelling}

\newcommand{\deref}[1]{\footnote{Zie definitie \ref{#1} op pagina \pageref{#1}.}}
\newcommand{\stref}[1]{\footnote{Zie stelling \ref{#1} op pagina \pageref{#1}.}}
\newcommand{\eiref}[1]{\footnote{Zie eigenschap \ref{#1} op pagina \pageref{#1}.}}
\newcommand{\gevref}[1]{\footnote{Zie gevolg \ref{#1} op pagina \pageref{#1}.}}
\newcommand{\prref}[1]{\footnote{Zie propositie \ref{#1} op pagina \pageref{#1}.}}
\newcommand{\opmref}[1]{\footnote{Zie opmerking \ref{#1} op pagina \pageref{#1}.}}
\newcommand{\vbref}[1]{\footnote{Zie voorbeeld \ref{#1} op pagina \pageref{#1}.}}
\newcommand{\tvbref}[1]{\footnote{Zie tegenvoorbeeld \ref{#1} op pagina \pageref{#1}.}}
\newcommand{\lemref}[1]{\footnote{Zie lemma \ref{#1} op pagina \pageref{#1}.}}
\newcommand{\alref}[1]{\footnote{Zie algoritme \ref{#1} op pagina \pageref{#1}.}}
\newcommand{\secref}[1]{\footnote{Zie sectie \ref{#1} op pagina \pageref{#1}.}}
\newcommand{\figref}[1]{\footnote{Zie figuur \ref{#1} op pagina \pageref{#1}.}}


\newcommand{\hint}[1]{Hint: #1}

\newenvironment{idee}{Idee:}{}


\newcounter{todocounter}
\newcounter{importantcounter}
\newcommand{\todonum}[2][]{\stepcounter{todocounter}\todo[#1]{#2}}
\newcommand{\todoi}[2][]{\stepcounter{importantcounter}\todonum[#1]{#2}}


% Mooiere TODO's
\newcommand{\TODO}[1]{\todoi[color=red,inline,size=\small]{TODO: #1}}
\newcommand{\extra}[1]{\todonum[color=orange,inline,size=\small]{EXTRA: #1}}
\newcommand{\clarify}[1]{\todonum[color=yellow,inline,size=\small]{CLARIFY: #1}}
\newcommand{\question}[1]{\todonum[color=green,inline,size=\small]{QUESTION: #1}}
\newcommand{\mst}[1]{\extra{mogelijke stelling: #1}}

\newcommand{\waarom}[0]{\clarify{waarom?}}
\newcommand{\needed}[0]{\clarify{referentie?}}

\newcommand{\term}[1]{\index{#1}\textbf{#1}}
\newcommand{\zb}[0]{{\footnotesize {\it Zonder bewijs\/}}}

\newcommand{\feed}[0]{\todonum[color=pink,inline,size=\small]{GET FEEDBACK}}

% badges
\newcommand{\bx}[1]{ \framebox[1.1\width]{#1} }
\newcommand{\examen}[0]{\bx{{\bf Examenvraag!}}\\}
\newcommand{\examenvraag}[1]{\bx{{\bf Examenvraag: #1 !}}\\}

% Voor meer plaats tussen wiskunde
\renewcommand{\arraystretch}{1.25}

\renewcommand{\qedsymbol}{$\square$}

\definecolor{solarback}{HTML}{FDF6E3}
\definecolor{solarfront}{HTML}{657A81}
\mdfdefinestyle{klad}{linewidth=0pt,backgroundcolor=solarback,fontcolor=solarfront}
\newenvironment{klad}{\begin{mdframed}[style=klad]}{\end{mdframed}}

