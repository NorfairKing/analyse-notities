\documentclass[main.tex]{subfiles}
\begin{document}

\section{Reeksen van functies - Machtreeksen}
\label{sec:reeksen-van-functies}

\begin{de}
  Zij $(f_{n})_{n}$ een rij van functies $f_{n}:\ W  \rightarrow \mathbb{C}$.
  We construeren hiermee een nieuwe rij $(s_{n})_{n}$ van functies $s_{n}:\ W \rightarrow \mathbb{C}$ als volgt:
  \[ s_{n}:\ A \rightarrow \mathbb{C}:\ x \mapsto s_{n}(x) = \sum_{k=0}^{n}f_{k}(x) \]
  We noemen het tupel $((f_{n})_{n},(s_{n})_{n})$ de \term{reeks} van de rij $(f_{n})_{n}$.
  We noemen $s_{n}$ de $n$-de \term{parteelsomfunctie} van de reeks.
\end{de}

\begin{de}
  We korten de reeks $((f_{n})_{n},(s_{n})_{n})$ vaak af als $\sum_{n}f_{n}$.
\end{de}

\begin{de}
  We zeggen dat een reeks $((f_{n})_{n},(s_{n})_{n})$ van functies $f_{n}:\ W \rightarrow \mathbb{C}$ \term{puntsgewijs convergeert} als en slechts als de rij $(s_{n}(x))_{n}$ voor elke $x\in W$ een limiet heeft in $\mathbb{C}$.
  Als de reeks puntsgewijs convergeert, dan definieert men de \term{puntsgewijze limietfunctie} $\sum_{k=0}^{\infty}f_{k}$ (notatie) als volgt:
  \[ \sum_{k=0}^{\infty}f_{k}:\ W \rightarrow \mathbb{C}:\ x \mapsto \lim_{n \rightarrow \infty}\sum_{k=0}^{n}f_{k}(x) \]
\end{de}

\begin{de}
  We zeggen dat een reeks $((f_{n})_{n},(s_{n})_{n})$ van functies $f_{n}:\ W \rightarrow \mathbb{C}$ \term{uniform convergeert} als de rij $(s_{n})_{n}$ uniform convergeert op $W$.
\end{de}

\extra{uniforme convergentie impliceert puntsgewijze convergentie, maar niet omgekeerd}.

\begin{de}
  We zeggen dat een reeks $((f_{n})_{n},(s_{n})_{n})$ van functies $f_{n}:\ W \rightarrow \mathbb{C}$ \term{absoluut convergeert} als de rij $(s_{n})_{n}$ absoluut convergeert voor elke $x \in W$
\end{de}

\begin{bst}
  De \term{M-test van Weierstra\ss}
  Zij $\sum_{n}f_{n}$ een reeks van functies $f_{n}:\ W \rightarrow \mathbb{C}$.
  Stel dat er een rij $(M_{k})_{k}$ in $\mathbb{R}^{+}$ bestaat zodat $\sum_{k}M_{k}$ convergeert als volgt:
  \[ \forall k\in \mathbb{N}, \forall x \in W:\ |f_{k}(x)| \le M_{k} \]
  Dan convergeert $\sum_{n}f_{n}$ absoluut en uniform op $W$.

  \begin{proof}
    Volgens de vergelijkingstest convergeert $\sum_{k}f_{k}(x)$ absoluut voor elke $x\in A$.\stref{st:vergelijkingstest}
    Noteer de limietfunctie met $f$.\\
    Kies een willekeurige $\epsilon \in \mathbb{R}_{0}^{+}$.
    Beschouw de rij $(s_{n})_{n}$ partieelsomfuncties:
    \[ s_{n} = \sum_{k=0}^{n}f_{k} \]
    Noteer met $(t_{n})_{n}$ de rij van de partieesommen van de reeks $\sum_{k}M_{k}$.
    \[ t_{n} = \sum_{k=0}^{n}M_{k} \]
    Omdat $\sum_{k}M_{k}$ convergeert, kunnen we een $n_{0}\in \mathbb{N}$ vinden als volgt:\prref{pr:convergent-dan-cauchy}
    \[ \forall n \in \mathbb{N}, \forall p\in \mathbb{N}_{0}:\ n \ge n_{0} \Rightarrow |t_{n+p}-t_{n}| < \epsilon \]
    We vinden dan het volgende:
    \[
    \begin{array}{rl}
      \forall n\in \mathbb{N}, \forall p\in \mathbb{N}_{0}, \forall x\in A:\ n \ge n_{0} \Rightarrow\\
      |s_{n+p}(x) - s_{n}(x)|
      &= \left|\sum_{k=n+1}^{n+p}f_{k}(x)\right|\\
      &\le \sum_{k=n+1}^{n+p}|f_{k}(x)|\\
      &\le \sum_{k=n+1}^{n+p}M_{K}\\
      &=|t_{n+p}-t_{n}| < \epsilon\\
    \end{array}
    \]
    Wanneer we hiervan de limiet voor $p$ gaande naar $+\infty$ nemen, vinden we dit:
    \[ \forall n \in \mathbb{N}, \forall x\in A:\ n \ge n_{0} \Rightarrow |f(x)-s_{n}(x)| < \epsilon \]
  \end{proof}
\end{bst}

\begin{bpr}
  \label{pr:reeks-uniform-convergent-dan-limietfunctie-continu}
  Zij $\sum_{n}f_{n}$ een reeks van continue functies $f_{n}:\ A \subseteq \mathbb{C} \rightarrow \mathbb{C}$, die uniform convergeert op $A$, dan is de limietfunctie $\sum_{k=0}^{\infty}f_{k}$ ook continu.

  \begin{proof}
    Noteer $s_{n} = \sum_{k=0}^{n}f_{k}$ en kies een willekeurige $\epsilon \in \mathbb{R}_{0}^{+}$.
    \begin{itemize}
    \item Beschouw eerst volgende afschatting:
      \[
      \begin{array}{rl}
        |f(x)-f(a)| &= |f(x) - s_{n}(x) + s_{n}(x) - s_{n}(a) + s_{n}(a) - f(a)|\\
        &\le |f(x) - s_{n}(x)| + |s_{n}(x) - s_{n}(a)| + |s_{n}(a) - f(a)|\\
      \end{array}
      \]
    \item Omdat $\sum_{n}f_{n}$ uniform convergeert naar $f$, kunnen we een $n_{0}\in \mathbb{N}$ vinden als volgt:
      \[ \forall y \in A, \forall n\in \mathbb{N}:\ n\ge n \Rightarrow |s_{n}(y)-f(y)|<\frac{\epsilon}{3} \]
    \item Omdat $f_{n}$ continu is in $a$, en elke eindige som van continue functies opnieuw continu\waarom, is $s_{n}$ continu in $a$ en kunnen we een $\delta$ vinden als volgt:
      \[ \forall x\in A:\ |x-a| <\delta \Rightarrow |s_{n}(x) - s_{n}(a)| < \frac{\epsilon}{3} \]
    \item 
      Voor elke $x\in A$ met $|x-a|<\delta$ en $n\ge n_{0}$ geldt nu het volgende:
      \[
      \begin{array}{rl}
        |f(x)-f(a)| &\le |f(x) - s_{n}(x)| + |s_{n}(x) - s_{n}(a)| + |s_{n}(a) - f(a)|\\
        &< \frac{\epsilon}{3} + \frac{\epsilon}{3} + \frac{\epsilon}{3}\\
        &= \epsilon
      \end{array}
      \]
    \end{itemize}
    $f$ is dus continu in $a$.
\feed
  \end{proof}
\end{bpr}

\begin{bpr}
  \label{pr:reeks-uniform-convergent-etc-dan-limietfunctie-afleidbaar}
  Zij $\sum_{n}f_{n}$ een reeks van afleidbare functies, gedefineerd op een open, convex, deel $A$ van $\mathbb{C}$ met waarden in $\mathbb{C}$.
  Stel dat $\sum_{n}f_{n}$ puntsgewijs convergeert naar een functie $f:\ A\subseteq \mathbb{C} \rightarrow \mathbb{C}$ en dat de reeks $\sum_{n}f'_{n}$ uniform convergeert op $A$, dan geldt:
  \begin{itemize}
  \item $f$ is afleidbaar
  \item $\forall x\in A:\ f'(x) = \sum_{k=0}^{\infty}f'_{k}(x)$
  \end{itemize}
\TODO{bewijs p 15 onderaan}
\end{bpr}

\begin{de}
  Zij $a \in \mathbb{C}$.
  Een \term{machtreeks (in $z$) rond $a$} is een reeks van de volgende vorm waarbij $z\in\mathbb{C}$ en waarbij $c_{n}$ gegeven elementen in $\mathbb{C}$ zijn.
  \[ \sum_{n}c_{n}(z-a)^{n} \]
  Men noemt de $c_{n}$ de \term{co\"effici\"entien} van de machtreeks.
\end{de}

\begin{opm}
  We zien $z \in \mathbb{C}$ hier als een variabele.
  Een machtreeks is dus eigenlijk een functiereeks $\sum_{n}f_{n}$ met $f_{n}$ als volgt:
  \[ f_{n}:\ \mathbb{C} \rightarrow \mathbb{C}:\ z \mapsto f_{n}(z) = c_{n}(z-a)^{n} \]
\end{opm}

\begin{bst}
  Beschouw een machtreeks $\sum_{n}c_{n}(z-a)^{n}$ rond een $a\in \mathbb{C}$.
  Benoem als volgt:
  \[ \gamma = \limsup_{n \rightarrow \infty}\sqrt[n]{|c_{n}|} \quad\text{ en }\quad R = \frac{1}{\gamma} \]
  We noteren ook $\frac{1}{+\infty}=0$ en $\frac{1}{0} = +\infty$.
  \begin{itemize}
  \item Als $|z-a| < R$ geldt, dan convergeert de machtreeks absoluut.
  \item Als $|z-a| > R$ geldt, dan convergeert de machtreeks niet.
  \end{itemize}

  \begin{proof}
    Kies een $z\in \mathbb{C}$ en noem $x_{n}=c_{n}(z-a)^{n}$.
    \[ \limsup_{n\rightarrow +\infty}\sqrt[n]{|x_{n}|} = |z-a|\limsup_{n\rightarrow +\infty}\sqrt[n]{|c_{n}|} = |z-a|\gamma \]
    Uit de worteltest van Cauchy volgt nu meteen de stelling.\stref{st:worteltest-cauchy}
  \end{proof}
\end{bst}

\extra{wat gebeurt er bij $|z-a| = R$?}

\begin{de}
  We noemen de $R$ in de stelling hierboven de \term{convergentiestraal} van de machtreeks.
\end{de}

\begin{vb}
  \stiekem{Juni 2014}
  De volgende machtreeks convergeert absoluut en uniform op $W$:
  \[ f:\ \interval[open]{-1}{1} \rightarrow \mathbb{R}:\ x \mapsto \sum_{k}x^{k}\cos(kx) \]
  
  \begin{proof}
    Omdat voor alle $y\in \mathbb{R}$, $\cos(y)$ tot $\interval{-1}{1}$ behoort, kunnen we de volgende afschatting maken:
    \[ \cos(kx) \le 1  \text{ en bijgevolg } x^{k}\cos(kx) \le x^{k} \]
    Nu weten we dus al dat het linkerlid convergeert als het rechterlid convergeert.\stref{st:vergelijkingstest}
    Beschouw nu de reeks $\sum_{k}x^{k}$.
    Dit is een machtreeks (rond $0$), met convergentiestraal $1$:
    \[ \lim_{k\rightarrow \infty}\sqrt[k]{1} = 1 \]
    In een straal van $1$ rond $0$ zal de reeks $\sum_{k}x^{k}$ dus absoluut convergeren en bijgevolg ook $\sum_{k}x^{k}\cos(kx)$.
  \end{proof}
\end{vb}

\extra{voorbeelden onderaan p 17}

\begin{bpr}
  Zij $\sum_{n}c_{n}(z-a)^{n}$ een machtreeks rond een $a\in \mathbb{C}$.
  Stel dat $c_{n}$ vanaf een bepaalde $n_{0}$ nooit $0$ is en dat de limiet $\lim_{n \rightarrow +\infty}\frac{|c_{n}|}{|c_{n+1}|}$ bestaat, dan is die limiet gelijk aan de convergentiestraal van de machtreeks.

  \begin{proof}
    Noem $\rho = \lim_{n\rightarrow +\infty}\frac{|c_{n}|}{|c_{n+1}|}$ en noem de convergentiestraal van de machtreeks $R$.
    Kies een $z\in \mathbb{C}$, verschillend van $a$ en stel $x_{n} = c_{n}(z-a)^{n}$, dan geldt het volgende:
    \[
    \lim_{n\rightarrow +\infty}\frac{|x_{n}+1|}{|x_{n}|} = 
    \left\{
      \begin{array}{cl}
        +\infty & \text{ als } \rho = 0\\
        \frac{|z-a|}{\rho} & \text{ als } \rho \in \interval[open]{0}{+\infty}\\
        0 & \text{ als } \rho = +\infty\\
      \end{array}
    \right.
    \]
    Met behulp van de verhoudingstest kunnen we volgende besluiten trekken:
    \begin{itemize}
    \item Als $\rho=0$ geldt, zal de machtreeks nooit convergeren voor $z \neq a$ dus geldt $R=0=\rho$.
    \item Als $\rho \in \interval[open]{0}{+\infty}$ geldt, zal de machtreeks absoluut convergeren als voor $\frac{|z-a|}{\rho}<1$ en niet convergeren voor $\frac{|z-a|}{\rho} >1$ dus geldt $R = \rho$.
    \item Als $\rho=+\infty$ geldt, zal de machtreeks voor alle $z\in \mathbb{C}$ convergeren en geldt dus $R = +\infty = \rho$.
    \end{itemize}
  \end{proof}
\end{bpr}

\begin{vb}
  De machtreeks $\sum_{n=1}^{+\infty}\frac{1}{n}z^{n}$ rond $0$ heeft convergentiestraal $1$.

  \begin{proof}
    \[
    \lim_{n\rightarrow +\infty}\frac{\left|\frac{1}{n}\right|}{\left|\frac{1}{n+1}\right|} 
    = \lim_{n\rightarrow +\infty}\frac{\frac{1}{n}}{\frac{1}{n+1}} 
    = \lim_{n\rightarrow +\infty} \frac{n+1}{n} = 1
    \]
    Merk hier op dat de reeks niet convergeert voor $z=1$ maar wel voor $z=-1$.
\extra{uitwerken.}
  \end{proof}
\end{vb}

\begin{vb}
  De machtreeks $\sum_{n}n!z^{n}$ rond $0$ heeft convergentiestraal $0$
  
  \begin{proof}
    \[
    \lim_{n\rightarrow +\infty}\frac{\left|n!\right|}{\left| (n+1)!\right|}
    = \lim_{n\rightarrow +\infty}\frac{n!}{(n+1)!}
    = \lim_{n\rightarrow +\infty}\frac{1}{n+1}
    = 0
    \]
  \end{proof}
\end{vb}

\begin{vb}
  De machtreeks $\sum_{n}\frac{1}{n^{n}}z^{n}$ heeft convergentiestraal $+\infty$.
  
  \begin{proof}
    \[
    \lim_{n \rightarrow +\infty} \frac{\frac{1}{n^{n}}}{\frac{1}{(n+1)^{n+1}}}
    = \lim_{n \rightarrow +\infty}\frac{(n+1)^{n+1}}{n^{n}}
    = +\infty
    \]
    ... maar het kan nog simpeler:
    \[ \limsup_{n\rightarrow +\infty}\sqrt[n]{\frac{1}{n^{n}}} = \limsup_{n\rightarrow +\infty}\frac{1}{n} = 0 \]
  \end{proof}
\end{vb}

\begin{bst}
  \label{st:machtreeks-convergeert-uniform-op-open-convergentieschijf}
  Zij $\sum_{n}c_{n}(z-a)^{n}$ een machtreeks rond een $a\in \mathbb{C}$ met convergentiestraal $R$.
  \[ \forall \rho \in \interval[open]{0}{R} \subseteq \mathbb{R}:\ \text{ de reeks convergeert uniform op } \left\{ z \in \mathbb{C} \mid |z-a| \le \rho \right\} \]

  \begin{proof}
    Kies een willekeurige $\rho \in \interval[open]{0}{R} \subseteq \mathbb{R}$.
    De reeks $\sum_{n}|c_{n}|\rho^{n}$ zal zeker convergeren.
    Anderzijds geldt ook het volgende:
    \[ \forall z\in \mathbb{C}, \forall n\in \mathbb{N}:\ |z-a|\le \rho \Rightarrow |c_{n}(z-a)^{n}| \le |c_{n}|\rho^{n} \]
    De stelling volgt nu meteen uit de M-test van Weierstra\ss.
  \end{proof}
\end{bst}

\begin{gev}
  Een machtreeks convergeert uniform op begrensde gesloten delen van haar open convergentieschijf $B(a,R)$.
  \begin{proof}
    Zij $D$ een begrensd, gesloten deel $D$ van $B(a,R)$, dan bestaat er een $\rho \in \interval[open right]{0}{R}$ als volgt:
    \[ D \subseteq \{ z \in \mathbb{C} \mid |z-a| \le \rho \} \]
    Omdat $\rho \in \interval[open right]{0}{R}$ geldt, convergeert de machtreeks uniform op $\{ z \in \mathbb{C} \mid |z-a| \le \rho \}$\stref{st:machtreeks-convergeert-uniform-op-open-convergentieschijf} en dus op $D$.
  \end{proof}
\end{gev}

\begin{bgev}
  Zij $\sum_{n}c_{n}(z-a)^{n}$ een machtreeks rond een $a\in \mathbb{C}$ met convergentiestraal $R$.
  De volgende functie is continu:
  \[ f:\ \left\{ z \in \mathbb{C} \mid |z-a| \le R \right\} \rightarrow \mathbb{C}:\ z \mapsto \sum_{n=0}^{\infty}c_{n}(z-a)^{n}  \]

  \begin{proof}
    De machtreeks convergeert uniform op $\left\{ z \in \mathbb{C} \mid |z-a| \le R \right\}$\stref{st:machtreeks-convergeert-uniform-op-open-convergentieschijf}.
    Bijgevolg is de limietfunctie continu.\prref{pr:reeks-uniform-convergent-dan-limietfunctie-continu}
  \end{proof}
\end{bgev}

\begin{blem}
  \label{lem:afgeleide-reeks-zelfde-limsup}
  Voor elke rij $(c_{n})_{n}$ in $\mathbb{C}$ geldt $\limsup_{n\rightarrow +\infty}\sqrt[n]{n|c_{n}|} = \limsup_{n\rightarrow +\infty}\sqrt[n]{|c_{n}|}$.

  \begin{proof}
    Noteer $a=\limsup_{n\rightarrow +\infty}\sqrt[n]{n|c_{n}|}$ en $b= \limsup_{n\rightarrow +\infty}\sqrt[n]{|c_{n}|}$.
    Omdat $n|c_{n}| \ge |c_{n}|$ geldt voor alle $n \ge 1$ moet $a\ge b$ gelden.
    Als $b$ $+\infty$ is, is het bewijs hierbij geleverd.
    Beschouw daarom het geval waarin $b < +\infty$ geldt.
    Kies een willekeurige $\epsilon \in \mathbb{R}_{0}^{+}$.
    Omdat $\lim_{n\rightarrow +\infty}\sqrt[n]{n} = 1$ geldt\waarom, kunnen we een $n_{0}\in \mathbb{N}$ nemen als volgt:
    \[ \forall n\in \mathbb{N}:\ n \ge n_{0} \Rightarrow \sqrt[n]{n} < 1+\epsilon \]
    Voor $n\ge n_{0}$ geldt dan ook $\sqrt[n]{n|c_{n}|} = \sqrt[n]{n}\sqrt[n]{|c_{n}|} < (1+\epsilon)\sqrt[n]{|c_{n}|}$.
    Hieruit volgt $a\le (1+\epsilon)b$.\prref{pr:limiet-behoudt-orde}
    $\epsilon$ kan hierin willekeurig klein gekozen worden, dus $a\le b$ moet gelden.
    Uit $a\ge b$ en $a \le b$ volgt dat $a$ en $b$ gelijk zijn.
  \end{proof}
\end{blem}

\begin{gev}
  \label{gev:afgeleide-reeks-zelfde-straal}
  De ``afgeleide machtreeks'' $\sum_{n=1}nc_{n}(z-a)^{n-1}$ van een machtreeks $\sum_{n}c_{n}(z-a)^{n}$ rond $a$ heeft dezelfde convergentiestraal.

  \begin{proof}
    De convergentiestraal is de inverse van $\limsup_{n\rightarrow +\infty}\sqrt[n]{|c_{n}|}$, en die is voor beide reeksen gelijk.\lemref{lem:afgeleide-reeks-zelfde-limsup}
  \end{proof}
\end{gev}

\begin{bst}
  Zij $\sum_{n}c_{n}(z-a)^{n}$ een machtreeks rond een $a\in \mathbb{C}$ met convergentiestraal $R$.
  De volgende functie is afleidbaar.
  \[ f:\ \left\{ z \in \mathbb{C} \mid |z-a| \le R \right\} \rightarrow \mathbb{C}:\ z \mapsto \sum_{n=0}^{\infty}c_{n}(z-a)^{n}  \]
  Bovendien vinden we de afgeleide functie als volgt:
  \[ f':\ \left\{ z \in \mathbb{C} \mid |z-a| \le R \right\} \rightarrow \mathbb{C}:\ z \mapsto \sum_{n=1}^{\infty}nc_{n}(z-a)^{n-1}  \]

  \begin{proof}
    We weten al dat $\sum_{n}f_{n}$ uniform, en dus puntsgewijs\needed, convergeert naar $f$ op $\left\{ z \in \mathbb{C} \mid |z-a| \le R \right\}$.\stref{st:machtreeks-convergeert-uniform-op-open-convergentieschijf}
    We weten eveneens dat $(f'_{n})_{n}$ uniforum convergeert op $\left\{ z \in \mathbb{C} \mid |z-a| \le R \right\}$.\gevref{gev:afgeleide-reeks-zelfde-straal}\stref{st:machtreeks-convergeert-uniform-op-open-convergentieschijf}
    $\sum_{n}f'_{n}$ convergeert dan naar $f'$.\prref{pr:reeks-uniform-convergent-etc-dan-limietfunctie-afleidbaar}
  \end{proof}
\end{bst}

\begin{bgev}
  Zij $\sum_{n}c_{n}(z-a)^{n}$ een machtreeks rond een $a\in \mathbb{C}$ met convergentiestraal $R$.
  De volgende functie is oneindig keer afleidbaar.
  \[ f:\ \left\{ z \in \mathbb{C} \mid |z-a| \le R \right\} \rightarrow \mathbb{C}:\ z \mapsto \sum_{n=0}^{\infty}c_{n}(z-a)^{n}  \]
  Bovendien vinden we de $k$-de afgeleide functie als volgt voor elke $k\in \mathbb{N}$:
  \[ f^{(k)}:\ \left\{ z \in \mathbb{C} \mid |z-a| \le R \right\} \rightarrow \mathbb{C}:\ z \mapsto \sum_{n=k}^{\infty}\binom{n}{k}c_{n}(z-a)^{n-k}  \]
  In het bijzonder geldt $f^{(k)}(a) = k!c_{k}$

  \begin{proof}
    We itereren simpelweg bovenstaande stelling.\extra{meer uitleg? inductie misschien?}
  \end{proof}
\end{bgev}

\begin{de}
  Gegeven twee reeksen $\sum_{n}x_{n}$, $\sum_{m}y_{m}$ in $\mathbb{C}$, dan defini\"eren we de \term{productreeks} $\sum_{k}z_{k}$ als volgt:
  \[ z_{k} = \sum_{n=0}^{k}x_{n}y_{k-n} \]
\end{de}

\begin{gst}
  Zij $\sum_{n}x_{n}$ en $\sum_{m}y_{m}$ twee convergente machtreeksen in $\mathbb{C}$, dan zal de productreeks $\sum_{k}z_{k}$ \textbf{niet} noodzakelijk convergeren.
\extra{vindt een tegenvoorbeeld}
\end{gst}

\begin{bst}
  Beschouw twee reeksen $\sum_{n}x_{n}$, $\sum_{m}c_{n}y_{m}$ in $\mathbb{C}$.
  Als $\sum_{n}x_{n}$ absoluut convergeert en $\sum_{m}y_{m}$ (gewoon) convergeert, dan convergeert de productreeks $\sum_{k}z_{k}$ en ziet de limiet er als volgt uit:
  \[ \sum_{k=0}^{\infty}z_{k} = \left(\sum_{n=0}^{\infty}x_{n}\right) \cdot \left(\sum_{m=0}^{\infty}y_{m}\right) \]

  \begin{proof}
    Noteer:
    \[
    s_{n} = \sum_{k}^{n}x_{k},\quad
    s = \lim_{n\rightarrow +\infty}s_{n}
    \]
    \[
    t_{n} = \sum_{k}^{n}y_{k},\quad
    t = \lim_{n\rightarrow +\infty}t_{n},\quad
    \tau_{n} = t_{n}-t
    \]
    \[
    u_{n} = \sum_{k=0}^{n}z_{k}
    \]
    \[ \rho_{n} = \sum_{k=0}^{n}x_{k}\tau_{n-k} \]
    Merk het volgende op:
    \[
    \begin{array}{rll}
      u_{n}
      &= \sum_{k=0}^{n}\sum_{i=0}^{k}x_{i}y_{k-i}
      &= x_{0}y_{0} + (x_{0}y_{1}+x_{1}y_{0}) + \dotsb + (x_{0}y_{n} + x_{1}y_{n-1} + \dotsb + x_{n}y_{0})\\
      &= \sum_{k=0}^{n}x_{k}t_{n-k}
      &= x_{0}t_{n} + x_{1}t_{n-1} + \dotsb + x_{n}t_{0}\\
      &= \sum_{k=0}^{n}x_{k}(t+\tau_{n-k})
      &= x_{0}(t+\tau_{n}) + x_{1}(t+\tau_{n-1}) + \dotsb + x_{n}(t+\tau_{0})\\
      &= s_{n}t + \sum_{k=0}^{n}x_{k}\tau_{n-k}
      &= s_{n}t + x_{0}\tau_{n} + x_{1}\tau_{n-1} + \dotsb + x_{n}\tau_{0}\\
      &= s_{n}t + \rho_{n}\\
    \end{array}
    \]
    Om aan te tonen dat $\lim_{n\rightarrow +\infty}u_{n}$ gelijk is aan $st$ is het voldoende aan te tonen dat $\rho_{n}$ naar $0$ gaat voor $n$ gaande naar $+\infty$ omdat $s_{n}t$ al naar $st$ gaat.
    
    Noteer $\alpha = \sum_{n}|x_{n}|$.
    Kies een willekeurige $\epsilon \in \mathbb{R}_{0}^{+}$.
    Vermits $\tau_{n}$ naar $0$ gaat voor $n$ gaande naar $+\infty$, kunnen we een $n_{0}\in \mathbb{N}$ nemem als volgt:
    \[ \forall n\in \mathbb{N}:\ n \ge n_{0} \Rightarrow |\tau_{n}| < \epsilon \]
    Er geldt dan het volgende:
    \[
    \begin{array}{rl}
      |\rho_{n}|
      &= \left|\sum_{i=0}^{n}x_{k}\tau_{n-k}\right|\\
      &\le \left|\sum_{k=0}^{n_{0}}x_{k}\tau_{n-k}\right|+\left|\sum_{k=n_{0}}^{n}x_{k}\tau_{n-k}\right|\\
      &\le \left|\sum_{k=0}^{n_{0}}x_{k}\tau_{n-k}\right|+\epsilon\alpha\\
    \end{array}
    \]
    Nemen we hiervan de $\limsup$ voor $n$ gaande naar $+\infty$, dan vinden we het volgende:
    \[
      \limsup_{n\rightarrow +\infty}|\rho_{n}|
      \le \epsilon\alpha\\
    \]
    Omdat $\epsilon$ hier willekeurig klein gekozen kan worden mogen we besluiten dat de limiet van $\rho_{n}$ voor $n$ gaande naar $+\infty$ nul is.\waarom
  \end{proof}
\end{bst}

\begin{bpr}
  Gegeven twee reeksen $\sum_{n}x_{n}$, $\sum_{m}c_{n}y_{m}$ in $\mathbb{C}$ met respectievelijke convergentiestraal $R_{x}$ en $R_{y}$.
  Beschouw de machtreeks $\sum_{n}c_{n}(z-a)^{n}$ waarbij $c_{n} = \sum_{k=0}^{n}a_{k}b_{n-k}$, dan heeft deze machtreeks een convergentiestraal $R \ge \min\{R_{1},R_{2}\}$ en geldt het volgende:
  \[ \sum_{n}c_{n}(z-a)^{n} = \left( \sum_{n}a_{n}(z-a)^{n}\right)\cdot \left( \sum_{n}b_{n}(z-a)^{n}\right) \]
\TODO{bewijs oefening}
\end{bpr}



\end{document}

%%% Local Variables:
%%% mode: latex
%%% TeX-master: t
%%% End:
