\documentclass[main.tex]{subfiles}
\begin{document}

\section{Reeksen van functies - Machtreeksen}
\label{sec:reeksen-van-functies}

\begin{de}
  Zij $(f_{n})_{n}$ een rij van functies $f_{n}:\ W  \rightarrow \mathbb{C}$.
  We construeren hiermee een nieuwe rij $(s_{n})_{n}$ van functies $s_{n}:\ W \rightarrow \mathbb{C}$ als volgt:
  \[ s_{n}:\ A \rightarrow \mathbb{C}:\ x \mapsto s_{n}(x) = \sum_{k=0}^{n}f_{k}(x) \]
  We noemen het tupel $((f_{n})_{n},(s_{n})_{n})$ de \term{reeks} van de rij $(f_{n})_{n}$.
  We noemen $s_{n}$ de $n$-de \term{parteelsomfunctie} van de reeks.
\end{de}

\begin{de}
  We korten de reeks $((f_{n})_{n},(s_{n})_{n})$ vaak af als $\sum_{n}f_{n}$.
\end{de}

\begin{de}
  We zeggen dat een reeks $((f_{n})_{n},(s_{n})_{n})$ van functies $f_{n}:\ W \rightarrow \mathbb{C}$ \term{puntsgewijs convergeert} als en slechts als de rij $(s_{n}(x))_{n}$ voor elke $x\in W$ een limiet heeft in $\mathbb{C}$.
  Als de reeks puntsgewijs convergeert, dan definieert men de \term{puntsgewijze limietfunctie} $\sum_{k=0}^{\infty}f_{k}$ (notatie) als volgt:
  \[ \sum_{k=0}^{\infty}f_{k}:\ W \rightarrow \mathbb{C}:\ x \mapsto \lim_{n \rightarrow \infty}\sum_{k=0}^{n}f_{k}(x) \]
\end{de}

\begin{de}
  We zeggen dat een reeks $((f_{n})_{n},(s_{n})_{n})$ van functies $f_{n}:\ W \rightarrow \mathbb{C}$ \term{uniforum convergeert} als de rij $(s_{n})_{n}$ uniforum convergeert op $W$.
\end{de}

\extra{uniforume convergentie impliceert puntsgewijze convergentie, maar niet omgekeerd}.

\begin{de}
  We zeggen dat een reeks $((f_{n})_{n},(s_{n})_{n})$ van functies $f_{n}:\ W \rightarrow \mathbb{C}$ \term{absoluut convergeert} als de rij $(s_{n})_{n}$ absoluut convergeert voor elke $x \in W$
\end{de}

\begin{bst}
  De \term{M-test van Weierstra\ss}
  Zij $\sum_{n}f_{n}$ een reeks van functies $f_{n}:\ W \rightarrow \mathbb{C}$.
  Stel dat er een rij $(M_{k})_{k}$ in $\mathbb{R}^{+}$ bestaat zodat $\sum_{k}M_{k}$ convergeert als volgt:
  \[ \forall k\in \mathbb{N}, \forall x \in W:\ |f_{k}(x)| < M_{k} \]
  Dan convergeert $\sum_{n}f_{n}$ absoluut en uniforum op $W$.

  \begin{proof}
    Volgens de vergelijkingstest convergeert $\sum_{k}f_{k}(x)$ absoluut voor elke $x\in A$.\stref{st:vergelijkingstest}
    Noteer de limietfunctie met $f$.\\
    Kies een willekeurige $\epsilon \in \mathbb{R}_{0}^{+}$.
    Beschouw de rij $(s_{n})_{n}$ partieelsomfuncties:
    \[ s_{n} = \sum_{k=0}^{n}f_{k} \]
    Noteer met $(t_{n})_{n}$ de rij van de partieesommen van de reeks $\sum_{k}M_{k}$.
    \[ t_{n} = \sum_{k=0}^{n}M_{k} \]
    Omdat $\sum_{k}M_{k}$ convergeert, kunnen we een $n_{0}\in \mathbb{N}$ vinden als volgt:\prref{pr:convergent-dan-cauchy}
    \[ \forall n \in \mathbb{N}, \forall p\in \mathbb{N}_{0}:\ n \ge n_{0} \Rightarrow |t_{n+p}-t_{n}| < \epsilon \]
    We vinden dan het volgende:
    \[
    \begin{array}{rl}
      \forall n\in \mathbb{N}, \forall p\in \mathbb{N}_{0}, \forall x\in A:\ n \ge n_{0} \Rightarrow\\
      |s_{n+p}(x) - s_{n}(x)|
      &= \left|\sum_{k=n+1}^{n+p}f_{k}(x)\right|\\
      &\le \sum_{k=n+1}^{n+p}|f_{k}(x)|\\
      &\le \sum_{k=n+1}^{n+p}M_{K}\\
      &=|t_{n+p}-t_{n}| < \epsilon\\
    \end{array}
    \]
    Wanneer we hiervan de limiet voor $p$ gaande naar $+\infty$ nemen, vinden we dit:
    \[ \forall n \in \mathbb{N}, \forall x\in A:\ n \ge n_{0} \Rightarrow |f(x)-s_{n}(x)| < \epsilon \]
    \extra{meer uitleg? lijkt te simpel.}
  \end{proof}
\end{bst}


\begin{bpr}
  Zij $\sum_{n}f_{n}$ een reeks van continue functies $f_{n}:\ A \subseteq \mathbb{C} \rightarrow \mathbb{C}$, die uniforum convergeert op $A$, dan is de limietfunctie $\sum_{k=0}^{\infty}f_{k}$ ook continu.

  \begin{proof}
    Noteer $s_{n} = \sum_{k=0}^{n}f_{k}$ en kies een willekeurige $\epsilon \in \mathbb{R}_{0}^{+}$.
    \begin{itemize}
    \item Beschouw eerst volgende afschatting:
      \[
      \begin{array}{rl}
        |f(x)-f(a)| &= |f(x) - s_{n}(x) + s_{n}(x) - s_{n}(a) + s_{n}(a) - f(a)|\\
        &\le |f(x) - s_{n}(x)| + |s_{n}(x) - s_{n}(a)| + |s_{n}(a) - f(a)|\\
      \end{array}
      \]
    \item Omdat $\sum_{n}f_{n}$ uniform convergeert naar $f$, kunnen we een $n_{0}\in \mathbb{N}$ vinden als volgt:
      \[ \forall y \in A, \forall n\in \mathbb{N}:\ n\ge n \Rightarrow |s_{n}(y)-f(y)|<\frac{\epsilon}{3} \]
    \item Omdat $f_{n}$ continu is in $a$, en elke eindige som van continue functies opnieuw continu\waarom, is $s_{n}$ continu in $a$ en kunnen we een $\delta$ vinden als volgt:
      \[ \forall x\in A:\ |x-a| <\delta \Rightarrow |s_{n}(x) - s_{n}(a)| < \frac{\epsilon}{3} \]
    \item 
      Voor elke $x\in A$ met $|x-a|<\delta$ en $n\ge n_{0}$ geldt nu het volgende:
      \[
      \begin{array}{rl}
        |f(x)-f(a)| &\le |f(x) - s_{n}(x)| + |s_{n}(x) - s_{n}(a)| + |s_{n}(a) - f(a)|\\
        &< \frac{\epsilon}{3} + \frac{\epsilon}{3} + \frac{\epsilon}{3}\\
        &= \epsilon
      \end{array}
      \]
    \end{itemize}
    $f$ is dus continu in $a$.
\feed
  \end{proof}
\end{bpr}

\begin{bpr}
  Zij $\sum_{n}f_{n}$ een reeks van afleidbare functies, gedefineerd op een open, convex, deel $A$ van $\mathbb{C}$ met waarden in $\mathbb{C}$.
  Stel dat $\sum_{n}f_{n}$ puntsgewijs convergeert naar een functie $f:\ A\subseteq \mathbb{C} \rightarrow \mathbb{C}$ en dat de reeks $\sum_{n}f'_{n}$ uniforum convergeert op $A$, dan geldt:
  \begin{itemize}
  \item $f$ is afleidbaar
  \item $\forall x\in A:\ f'(x) = \sum_{k=0}^{\infty}f'_{k}(x)$
  \end{itemize}
\TODO{bewijs p 15 onderaan}
\end{bpr}

\begin{de}
  Zij $a \in \mathbb{C}$.
  Een \term{machtreeks (in $z$) rond $a$} is een reeks van de volgende vorm waarbij $z\in\mathbb{C}$ en waarbij $c_{n}$ gegeven elementen in $\mathbb{C}$ zijn.
  \[ \sum_{n}c_{n}(z-a)^{n} \]
  Men noemt de $c_{n}$ de \term{co\"effici\"entien} van de machtreeks.
\end{de}

\begin{opm}
  We zien $z \in \mathbb{C}$ hier als een variabele.
  Een machtreeks is dus eigenlijk een functiereeks $\sum_{n}f_{n}$ met $f_{n}$ als volgt:
  \[ f_{n}:\ \mathbb{C} \rightarrow \mathbb{C}:\ z \mapsto f_{n}(z) = c_{n}(z-a)^{n} \]
\end{opm}

\begin{bst}
  Beschouw een machtreeks $\sum_{n}c_{n}(z-a)^{n}$ rond een $a\in \mathbb{C}$.
  Benoem als volgt:
  \[ \gamma = \limsup_{n \rightarrow \infty}\sqrt[n]{|c_{n}|} \quad\text{ en }\quad R = \frac{1}{\gamma} \]
  We noteren ook $\frac{1}{+\infty}=0$ en $\frac{1}{0} = +\infty$.
  \begin{itemize}
  \item Als $|z-a| < R$ geldt, dan convergeert de machtreeks absoluut.
  \item Als $|z-a| > R$ geldt, dan convergeert de machtreeks niet.
  \end{itemize}
\TODO{bewijs p 17}
\end{bst}

\extra{wat gebeurt er bij $|z-a| = R$?}

\begin{de}
  We noemen de $R$ in de stelling hierboven de \term{convergentiestraal} van de machtreeks.
\end{de}

\extra{voorbeelden onderaan p 17}

\begin{bpr}
  Zij $\sum_{n}c_{n}(z-a)^{n}$ een machtreeks rond een $a\in \mathbb{C}$.
  Stel dat $c_{n}$ vanaf een bepaalde $n_{0}$ nooit $0$ is en dat de limiet $\lim_{n \rightarrow +\infty}\frac{|c_{n}|}{|c_{n+1}|}$ bestaat, dan is diee limiet gelijk aan de convergentiestraal van de machtreeks.
\TODO{bewijs p 18}
\end{bpr}

\extra{voorbeelden p 19}

\begin{bst}
  Zij $\sum_{n}c_{n}(z-a)^{n}$ een machtreeks rond een $a\in \mathbb{C}$ met convergentiestraal $R$.
  \[ \forall \rho \in \interval[open]{0}{R} \subseteq \mathbb{R}:\ \text{ de reeks convergeert uniforum op } \left\{ z \in \mathbb{C} \mid |z-a| \le \rho \right\} \]
\TODO{bewijs p 19}
\end{bst}

\begin{bgev}
  Zij $\sum_{n}c_{n}(z-a)^{n}$ een machtreeks rond een $a\in \mathbb{C}$ met convergentiestraal $R$.
  De volgende functie is continu:
  \[ f:\ \left\{ z \in \mathbb{C} \mid |z-a| \le R \right\} \rightarrow \mathbb{C}:\ z \mapsto \sum_{n=0}^{\infty}c_{n}(z-a)^{n}  \]
\TODO{bewijs: oefening p 20}
\end{bgev}

\begin{blem}
  Voor elke rij $(c_{n})_{n}$ in $\mathbb{C}$ geldt $\limsup_{n\rightarrow +\infty}\sqrt[n]{n|c_{n}|} = \limsup_{n\rightarrow +\infty}\sqrt[n]{|c_{n}|}$.
\TODO{bewijs p 20}
\end{blem}

\begin{bst}
  Zij $\sum_{n}c_{n}(z-a)^{n}$ een machtreeks rond een $a\in \mathbb{C}$ met convergentiestraal $R$.
  De volgende functie is afleidbaar.
  \[ f:\ \left\{ z \in \mathbb{C} \mid |z-a| \le R \right\} \rightarrow \mathbb{C}:\ z \mapsto \sum_{n=0}^{\infty}c_{n}(z-a)^{n}  \]
  Bovendien vinden we de afgeleide functie als volgt:
  \[ f':\ \left\{ z \in \mathbb{C} \mid |z-a| \le R \right\} \rightarrow \mathbb{C}:\ z \mapsto \sum_{n=1}^{\infty}nc_{n}(z-a)^{n-1}  \]
\TODO{bewijs sp 21}
\end{bst}

\begin{bgev}
  Zij $\sum_{n}c_{n}(z-a)^{n}$ een machtreeks rond een $a\in \mathbb{C}$ met convergentiestraal $R$.
  De volgende functie is oneindig keer afleidbaar.
  \[ f:\ \left\{ z \in \mathbb{C} \mid |z-a| \le R \right\} \rightarrow \mathbb{C}:\ z \mapsto \sum_{n=0}^{\infty}c_{n}(z-a)^{n}  \]
  Bovendien vinden we de $k$-de afgeleide functie als volgt voor elke $k\in \mathbb{N}$:
  \[ f^{(k)}:\ \left\{ z \in \mathbb{C} \mid |z-a| \le R \right\} \rightarrow \mathbb{C}:\ z \mapsto \sum_{n=k}^{\infty}\binom{n}{k}c_{n}(z-a)^{n-k}  \]
  In het bijzonder geldt $f^{(k)}(a) = k!c_{k}$
\TODO{bewijs}
\end{bgev}

\begin{de}
  Gegeven twee reeksen $\sum_{n}x_{n}$, $\sum_{m}c_{n}y_{m}$ in $\mathbb{C}$, dan defini\"eren we de \term{productreeks} $\sum_{k}z_{k}$ als volgt:
  \[ z_{k} = \sum_{n=0}^{k}x_{n}y_{k-n} \]
\end{de}

\extra{geen stelling p 23 bovenaan}

\begin{bst}
  Beschouw twee reeksen $\sum_{n}x_{n}$, $\sum_{m}c_{n}y_{m}$ in $\mathbb{C}$.
  Als $\sum_{n}x_{n}$ absoluut convergeert en $\sum_{m}y_{m}$ (gewoon) convergeert, dan convergeert de productreeks $\sum_{k}z_{k}$ en ziet de limiet er als volgt uit:
  \[ \sum_{k=0}^{\infty}z_{k} = \left(\sum_{n=0}^{\infty}x_{n}\right) \cdot \left(\sum_{m=0}^{\infty}y_{m}\right) \]
\TODO{bewijs p 23}
\end{bst}

\begin{bpr}
  Gegeven twee reeksen $\sum_{n}x_{n}$, $\sum_{m}c_{n}y_{m}$ in $\mathbb{C}$ met respectievelijke convergentiestraal $R_{x}$ en $R_{y}$.
  Beschouw de machtreeks $\sum_{n}c_{n}(z-a)^{n}$ waarbij $c_{n} = \sum_{k=0}^{n}a_{k}b_{n-k}$, dan heeft deze machtreeks een convergentiestraal $R \ge \min\{R_{1},R_{2}\}$ en geldt het volgende:
  \[ \sum_{n}c_{n}(z-a)^{n} = \left( \sum_{n}a_{n}(z-a)^{n}\right)\cdot \left( \sum_{n}b_{n}(z-a)^{n}\right) \]
\TODO{bewijs p 24}
\end{bpr}



\end{document}

%%% Local Variables:
%%% mode: latex
%%% TeX-master: t
%%% End:
