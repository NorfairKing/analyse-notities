\documentclass[main.tex]{subfiles}
\begin{document}



\section{Natuurlijke getallen}
\label{sec:natuurlijke-getallen}

\begin{de}
  De \term{natuurlijke getallen} zijn inductief gedefinieerd met de volgende axioma's.
  \begin{itemize}
  \item $0$ is een natuurlijk getal: $0 \in \mathbb{N}$.
  \item $\forall a \in \mathbb{N}: S(a) \in \mathbb{N}$.
  \item $\forall a,b\in \mathbb{N}: S(a) = S(b) \Rightarrow a=b$
  \item Als een verzameling $V$ $0$ bevat alsook de successor van elk getal in $V$, dan geldt $V=\mathbb{N}$.
  \end{itemize}
\end{de}

\begin{de}
  De \term{optellling} is gedefinieerd als volgt:
  \[ a + S(b) = s(a+b) \]
\end{de}

\begin{de}
  De \term{vermenigvuldiging} is gedefinieerd als volgt:
  \[ a \cdot S(b) = a+ s(a\cdot b) \]
\end{de}

\begin{de}
  De \term{orde} in $\mathbb{N}$ is gedefinieerd als volgt:
  \[ n \le S(m) \Leftrightarrow n = S(m) \vee n \le m \]
\end{de}

\begin{de}
  De \term{natuurlijke getallen} kunnen ook concreet gedefineerd worden:
  \[
  \left\{
  \begin{array}{rl}
    0 &= \emptyset\\
    s(n) &= n \cup \{ n \}\\
  \end{array}
  \right.
  \]
\end{de}


\end{document}

%%% Local Variables:
%%% mode: latex
%%% TeX-master: t
%%% End:
