\documentclass[main.tex]{subfiles}
\begin{document}


\section{Rijen in $\mathbb{R}$}
\label{sec:rijen-mathbbr}

\begin{de}
  Een \term{rij} in een verzameling $V$ is een functie als volgt:
  \[ x: \mathbb{N} \rightarrow V:\ n\mapsto x_{n} \]
  De functiewaarden noemen we \term{termen} en de rij wordt genoteerd met $(x_{n})_{n}$.
  Hierin noemen we de $n$ de \term{index} van de term.
\end{de}

\begin{de}
  We zeggen dat een rij $(x_{n})_{n}$ in een totaal geordend veld $F,+,\cdot,\le$ \term{convergeert} naar $a\in F$ als en slechts als het volgende geldt:
  \[ \forall \epsilon \in F_{0}^{+},\ \exists n_{0}\in \mathbb{N},\ \forall n\in \mathbb{N}:\ n \ge n_{0} \Rightarrow |x_{n}-a| < \epsilon \]
  We noemen $a$ de \term{limiet} van de rij $(x_{n})_{n}$:
  \[ a = \lim_{n\rightarrow \infty}x_{n} \]
  Een rij die convergeert noemen we een \term{convergente rij}.
  Een rij die niet convergeert noemen we een \term{divergente rij}.
\end{de}

\begin{vb}
  De rij $\left(\frac{(-1)^{n}}{1+n^{2}}\right)$ convergeert naar $0$.
\extra{bewijs p 26}
\end{vb}

\begin{vb}
  De rij $\left( 3 + \left(\frac{-1}{2
}\right)^{2}\right)$ convergeert naar $3$.
\extra{bewijs p 27}
\end{vb}

\begin{vb}
  De rij $(x_{n})_{n}$, gedefinieerd als volgt met willekeurige $a,b\in \mathbb{R}$ $a\neq b$, convergeert niet:
  \[
  x_{n} = 
  \begin{cases}
    a \text{ als } n \text{ even is}\\
    b \text{ als } n \text{ oneven is}\\
  \end{cases}
  \]
\end{vb}

\begin{vb}
  Zij $x_{0}=1$ en $x_{n+1} = \frac{1}{1+x_{n}}$.
  \extra{bewijs dat $(x_{n})$ naar $\frac{2}{1+\sqrt{5}}$ convergeert}
\end{vb}


\begin{de}
  We zeggen dat een rij $(r_{n})_{n}$ in een totaal geordend veld $F,+,\cdot,\le$ \term{divergeert naar plus oneindig} als en slechts het volgende geldt:
  \[ \forall M\in F,\ \exists n_{0}\in \mathbb{N},\ \forall n\in \mathbb{N}:\ n \ge n_{0} \Rightarrow x_{n} > M \]
  \[ + \infty = \lim_{n\rightarrow \infty}x_{n}\]
\end{de}

\begin{de}
  We zeggen dat een rij $(r_{n})_{n}$ in een totaal geordend veld $F,+,\cdot,\le$ \term{divergeert naar min oneindig} als en slechts het volgende geldt:
  \[ \forall M\in F,\ \exists n_{0}\in \mathbb{N},\ \forall n\in \mathbb{N}:\ n \ge n_{0} \Rightarrow x_{n} < M \]
  \[ - \infty = \lim_{n\rightarrow \infty}x_{n}\]
\end{de}

\begin{vb}
  De rij $(n)_{n}$ divergeert naar $+\infty$.
\extra{bewijs}
\end{vb}

\begin{vb}
  De rij $\left(\frac{n^{2}}{n+1}\right)_{n}$ divergeert naar $+\infty$.
\extra{bewijs p 29}
\end{vb}

\begin{bpr}
  Zij $p\in \mathbb{Z}$, dan heeft de rij $(n^{p})_{n}$ een limiet:
  \begin{itemize}
  \item $p>0 \rightarrow \lim_{n\rightarrow \infty}n^{p} = + \infty$
  \item $p=0 \rightarrow \lim_{n\rightarrow \infty}n^{p} = 1$
  \item $p<0 \rightarrow \lim_{n\rightarrow \infty}n^{p} = 0$
  \end{itemize}
  \extra{bewijs}
\end{bpr}

\begin{bpr}
  Als een rij een limiet heeft, dan is deze uniek.

  \begin{proof}
    Zij $(x_{n})_{n}$ een rij met twee limieten: $a,b \in \mathbb{F}\cup \{-\infty,+\infty\}$, dan tonen we aan dat deze gelijk zijn.
    Veronderstel dat deze verschillend zijn: $a \neq b$.
    Hernoem ze dan zo dat $a > b$ geldt.
    Er zijn nu vier gevallen:
    \begin{itemize}
    \item $a = +\infty, b \in F$\\
      \extra{bewijs}
    \item $a,b\in F$\\
      Noem $\delta = (a-b)/2$.
      Omdat $a$ groter is dan $b$ is $\delta$ positief.
      Omdat $a$ de limiet is van $(x_{n})_{n}$ is is $|x_{n}-a|<\delta$ vanaf een bepaalde $n\in \mathbb{N}$.
      Voor al de volgende $n$ geldt dan ook het volgende:
      \[ x_{n}>a-\delta = \frac{a+b}{2} \]
      Analoog vinden we een $m$ waar vanaf het volgende geldt:
      \[ x_{m}<b+\delta=\frac{a+b}{2} \]
      Kies nu een $p\in \mathbb{N}$ groter dan zowel $m$ als $n$, dan moet $x_{p}$ zowel strikt groter als strikt kleiner zijn dan $\frac{a+b}{2}$.
      Contradictie.
    \item $a\in F, b = -\infty$\\
      \extra{bewijs}
    \item $a= +\infty, b=-\infty$\\
      \extra{bewijs}
    \end{itemize}
  \end{proof}
\end{bpr}

\begin{bpr}
  \label{pr:convergente-rij-begrensd}
  Een convergente rij $(x_{n})_{n}$ is begrensd.
  \[ \exists M \in F^{+}: \forall n\in \mathbb{N}: |x_{n}| \le M \]

  \begin{proof}
    Zij $(x_{n})_{n}$ een convergente rij met limiet $a$.
    Vanaf een bepaalde $n\in \mathbb{N}$ komt de rij dichter dan $1$ bij $a$:
    \[ \exists n\in \mathbb{N}:\ \forall n'\in \mathbb{N}: n'>n\Rightarrow |x_{n}-a|<1 \]
    Vanaf die $n$ zijn alle $x_{n}$ in absolute waarde kleiner dan $1+|a|$.
    \[ |x_{n}| = |x_{n}-a+a| \le |x_{n}-a|+|a| <1+|a| \]
    De gezochte $M$ is nu het maximum van de absolute waarden van de termen met index kleiner dan $n$ en $1+|a|$.
    Per constructie is $(x_{n})_{n}$ dan begrensd door $M$.
  \end{proof}
\end{bpr}

\begin{opm}
  Merk op dat het omgekeerde niet geldt.
  \extra{tegenvoorbeeld}
\end{opm}

\begin{bpr}
  Beschouw twee rijen $(x_{n})_{n}$ en $(y_{n})_{n}$.
  Stel dat de staart van de rijen overeen komt, dus dat er een $k\in \mathbb{N}$ bestaat zodat $x_{n}$ gelijk is aan $y_{n}$ voor alle $n$ groter dan $k$.
  De rijen vertonen dan hetzelfde asymptotisch gedrag.

  \begin{proof}
    Er zijn drie mogelijkheden voor het asymptotisch gedrag van $(x_{n})_{n}$, voor elk van die mogelijkheden gaan we na dat $(y_{n})_{n}$ hetzelfde gedrag vertoont.
    \begin{itemize}
    \item $\lim_{n \rightarrow \infty}x_{n} = a\in F$\\
      Voor een willekeurige $\delta$ komt $x_{n}$ dichter dan $\delta$ bij $a$ vanaf een bepaalde $n\in \mathbb{N}$.
      Vanaf $\max\{n,k\}$ komt $y_{n}$ dan ook dichter dan $\delta$ bij $a$.
    \item $\lim_{n \rightarrow \infty}x_{n} = \pm \infty$\\
      Voor een willekeurige $M$ wordt $x_{n}$ groter/kleiner dan $M$/$m$ vanaf een bepaalde $n\in \mathbb{N}$.
      Vanaf $\max\{n,k\}$ wordt $y_{n}$ dan ook groter/kleiner dan $M$/$m$.
    \item $(x_{n})_{n}$ heeft geen limiet\\
      Uit het ongerijmde:
      Stel dat $(y_{n})_{n}$ een limiet $a$ heeft, dan volgt uit bovenstaande redenering dat $(x_{n})_{n}$ dezelfde limiet heeft.
      Contradictie.
    \end{itemize}
  \end{proof}
\end{bpr}

\begin{bpr}
  Voor elk re\"eel getal $x\in \mathbb{R}$ bestaat er een rij $(q_{n})_{n}$ in $\mathbb{Q}$ die convergeert naar $x$.

  \begin{proof}
    We zullen een rij construeren met $x$ als limiet.
    Beschouw voor elke $n\in \mathbb{N}$ de getallen $x\pm \frac{1}{n+1}$.
    We kunnen tussen deze getallen een $q\in \mathbb{Q}$ vinden.\prref{pr:q-dicht-in-r}
    Op die manier bekomen we een rij $(q_{n})_{n}$.
    Deze rij convergeert naar $x$.
    Neem immers een willekeurige $\epsilon \in \mathbb{R}_{0}^{+}$, dan geldt voor alle $n'\in\mathbb{N}$ vanaf een $n \ge \frac{1}{\epsilon}$ het volgende:
    \[ |q_{n}-x| < \frac{1}{n'+1} \le \frac{1}{n+1} < \epsilon \]
  \end{proof}
\end{bpr}

\begin{st}
  Zij $(y_{n})_{n}$ een rij in $\mathbb{R}$ die convergeert naar een $y\in \mathbb{R}$.
  Stel dat $y_{n}$ verschillend is van $y$ voor alle $n\in \mathbb{N}$, dan bevat de verzameling $Y = \{y_{n}\mid n\in \mathbb{N}\}$ oneindig veel elementen.

  \begin{proof}
    Bewijs uit het ongerijmde: stel dat $Y$ een eindig aantal elementen bevat.\\
    Omdat $Y$ een eindig aantal elementen bevat, bestaat er een $m\in \mathbb{N}$ zodat voor alle volgende $n$, de $y_{n}$ gelijk blijven:
    \[ \forall n_{1},n_{2}\in \mathbb{N}:\ n_{1},n_{2} \ge m \Rightarrow y_{m} = y_{n_{1}} = y_{n_{2}} \]
    Kies nu $\delta_{y} = |y_{m}-y|$, dan geldt het volgende:
    \[ \forall n\in \mathbb{N}:\ n \ge m \Rightarrow |y_{n}-y| = |y_{m}-y| = \delta_{y} \]
    Opdat $(y_{n})_{n}$ convergeert naar $y$ moet er voor $\delta_{y}$ ook het volgende gelden:
    \[ \exists n_{0} \in \mathbb{N}: \forall n \in \mathbb{N}:\ n \ge n_{0} \Rightarrow |y_{n}-y| < \delta_{y} \]
    Contradictie.
\feed
  \end{proof}
\end{st}

\begin{st}
  Zij $(x_{n})_{n}$ een convergente rij, dan is $(|x_{n}|)_{n}$ ook convergent.

  \begin{proof}
    Noem de limiet van $(x_{n})_{n}$ $x$.
    Kies willekeurig een $\epsilon\in\mathbb{R}_{0}^{+}$, dan bestaat er een $n_{0}\in\mathbb{N}$ als volgt:
    \[ \forall n\in\mathbb{N}:\ n\ge n_{0} \Rightarrow |x_{n}-x| < \epsilon \]
    De stelling volgt nu uit de volgende ongelijkheid:\prref{pr:tweede-driehoeksongelijkheid}
    \[ \left| |x_{n}| - |x| \right| \le |x_{n}-x| \]
  \end{proof}
\end{st}

\begin{tvb}
  De omgekeerde implicatie geldt niet.
  
  \begin{proof}
    Beschouw de rij $((-1)^{n})_{n}$.
  \end{proof}
\end{tvb}


\subsection{Limieten en orde}
\label{sec:limieten-en-orde}

\begin{de}
  We noemen een rij $(x_{n})_{n}$ in $\mathbb{R}$ ...
  \begin{itemize}
  \item ... \term{stijgend} als en slechts als $x_{n+1} \ge x_{n}$ geldt voor alle $n\in \mathbb{N}$.  
  \item ... \term{strikt stijgend} als en slechts als $x_{n+1} > x_{n}$ geldt voor alle $n\in \mathbb{N}$.  
  \item ... \term{dalend} als en slechts als $x_{n+1} \le x_{n}$ geldt voor alle $n\in \mathbb{N}$.  
  \item ... \term{strikt dalend} als en slechts als $x_{n+1} < x_{n}$ geldt voor alle $n\in \mathbb{N}$.  
  \end{itemize}
\end{de}

\begin{bst}
  \label{st:stijgend-dan-limiet}
  Een stijgende rij in $\mathbb{R}$ heeft atijd een limiet.
  Deze limiet is bovendien eindig als en slechts als de rij naar boven begrensd is.
  Die limiet is dan het supremum van de rij.

  \begin{proof}
    Gevalsonderscheid voor een willekeurige rij $(x_{n})_{n}$
    \begin{itemize}
    \item $(x_{n})_{n}$ is niet naar boven begrensd\\
      We tonen aan dat $(x_{n})_{n})$ $+\infty$ als limiet heeft.
      Kies daartoe een willekeurige $M\in\mathbb{R}$.
      Omdat de rij niet naar boven begrensd is is er minstens \'e\'en $n\in \mathbb{N}$ zodat $x_{n}$ groter is dan $M$.
      Omdat de rij stijgend is zullen voor alle volgende $n'$ de $x_{n'}$ ook groter zijn dan $M$.
    \item $(x_{n})_{n}$ is naar boven begrensd\\
      De verzameling $V = \{x_{n}\mid n\in \mathbb{N}\}$ is dan een niet-lege naar boven begrensde deelverzameling van $\mathbb{R}$ en heeft dus een supremum $s$.\stref{st:supremumeigenschap-R}
      Kies nu een willekeurige $\epsilon > 0$, dan is $s-\epsilon$ geen bovengrens meer voor $V$.
      Er moet daarom een $n\in \mathbb{N}$ bestaan zodat $x_{n}$ groter is dan $s-\epsilon$.
      Omdat de rij stijgend is zal voor elke grotere $n'\in \mathbb{N}$ $x_{n'}$ zowel groter als $s-\epsilon$ als kleiner dan $s$ zijn.
      \[ \forall n' \ge n:\ |x_{n'}-s| = s-x_{n'} \le x_{n}-s < \epsilon \]
      $s$ is dus de (eindige) limiet van $(x_{n})_{n}$
    \item Omgekeerd is de rij naar boven begrensd als ze een eindige limiet heeft.\prref{pr:convergente-rij-begrensd}
    \end{itemize}
  \end{proof}
\end{bst}

\begin{st}
  \label{st:dalend-dan-limiet}
  Een dalende rij heeft atijd een limiet.
  Deze limiet is bovendien eindig als en slechts als de rij naar onder begrensd is.
  Die limiet is dan het infimum van de rij.
  \extra{bewijs: oefening}
\end{st}

\begin{bpr}
  Zij $r\in \mathbb{R}$, dan heeft de rij $(r^{n})_{n}$ ...
  \begin{itemize}
  \item ... limiet plus oneindig als $r>1$ geldt.
  \item ... limiet $1$ als $r=1$ geldt.
  \item ... limiet $0$ als $|r|<1$ geldt.
  \item ... geen limiet als $r\le-1$ geldt.
  \end{itemize}

  \begin{proof}
    \begin{itemize}
    \item $r>1$\\
      De rij $(r_{n})_{n}$ is dan stijgend.
      We argumenteren dat de rij niet naar boven begrensd is om aan te tonen dat ze limiet $+\infty$ heeft.\stref{st:stijgend-dan-limiet}
      Stel daarom dat de rij wel naar boven begrensd zou zijn, dan zou de verzameling $V=\{ r^{n}\mid n\in \mathbb{N}\}$ een supremum $a$ hebben:\stref{st:supremumeigenschap-R}
      \[
      \begin{array}{c}
        \forall n\in \mathbb{N}: r^{n} \le a\\
        \forall n\in \mathbb{N}: r^{n-1} \le \frac{a}{r}\\
        \forall n\in \mathbb{N}: r^{n} \le \frac{a}{r}\\
      \end{array}
      \]
      Uit de bovenstaande argumentatie volgt dat $\frac{a}{r}$ ook een bovengrens is voor $V$, maar omdat $r$ groter is dan $1$ is dit een kleinere bovengrens dan $a$, wat in tegenspraak is met het supremum-zijn van $a$.
    \item $r=1$: $\lim_{n \rightarrow \infty}r^{n}=\lim_{n \rightarrow \infty}1=1$.
    \item $|r|<1$:
      \begin{itemize}
      \item $0<r<1$:
        De rij is in dit geval dalend \needed en naar onder begrensd (want alle termen zijn positief), dus de limiet is gelijk aan het infimum.\stref{st:dalend-dan-limiet}
        We bewijzen nog dat dit infimum $0$ is.
        Stel immers dat het groter zou zijn dan $0$:
        \[
        \begin{array}{c}
          \forall n\in \mathbb{n}: r^{n} \ge a\\
          \forall n\in \mathbb{n}: r^{n-1} \ge \frac{a}{r}\\
          \forall n\in \mathbb{n}: r^{n} \ge \frac{a}{r}\\
        \end{array}
        \]
        We vinden dat $\frac{a}{r}$ ook een ondergrens zou zijn van de $r^{n}$, maar kleiner dan $a$ (want $r$ is kleiner dan $1$) en dat is in tegenspraak met het infimum-zijn van $a$.
      \item $r=0$: $\lim_{n \rightarrow \infty}0^{n}=\lim_{n \rightarrow \infty}0=0$.
      \item $-1<r<0$:
        \extra{bewijs}
      \end{itemize}
    \item $r \le -1$
      \begin{itemize}
      \item $r=-1$
        \question{in het boek staat 'triviaal', maar ik ben niet overtuigd.}
      \item $r<-1$
        Omdat $|r|>1$ geldt weten we al dat $|r|>1$ naar $+\infty$ gaat.
        De rij $(|r|^{n})_{n}$ is dus zeker niet begrensd \needed en kan bijgevolg zeker geen eindige limiet hebben. \stref{st:stijgend-dan-limiet}
        Anderzijds kan de rij ook zeker niet $+\infty$ als limiet hebben want dan zouden vanaf een bepaalde $n$ alle $r^{n'}$ groter moeten zijn dan $r^{n}$.
        Dit kan niet het geval zijn want $r^{n}$ is negatief voor alle oneven $n$.
        Analoog kan de rij ook niet $-\infty$ als limiet hebben.
      \end{itemize}
    \end{itemize}
  \end{proof}
\end{bpr}

\begin{bpr}
  Zij $A$ een niet-leeg, naar boven begrensd deel van $\mathbb{R}$, dan bestaat er een stijgende rij in $A$ die convergeert naar het supremum van $A$.
  
  \begin{proof}
    Noem $s$ het supremum van $A$ (dit bestaat immers altijd\stref{st:supremumeigenschap-R})
    Gevalsonderscheid:
    \begin{itemize}
    \item $s\in A$: de rij $(s)_{n}$ voldoet.
    \item $s\not \in A$:\\
      \begin{itemize}
      \item Noem $\epsilon_{1} = 1$.  Omdat $s$ de kleinste bovengrens
        is van $A$ is $s-\epsilon_{1}$ geen bovengrens meer.  We
        kunnen dus een $x_{1}\in A$ nemen zodat $s-\epsilon_{1} \le
        x_{1}$ geldt.  Omdat $s$ een bovengrens is van $A$, maar niet
        in $A$ zit, geldt ook $x_{1}< s$.
      \item Noem $\epsilon\_{i} = min\{ \nicefrac{1}{i}, s-x_{i-1} \}$,
        $\epsilon$ is dan zeker positief (want $s$ is groter dan $x_{i-1}$ en $\nicefrac{1}{i}$ is positief).
        Omdat $s$ de kleinste bovengres is van $A$ is $s-\epsilon_{i}$ geen bovengrens meer.
        We kunnen dus een $x_{i}$ nemen tussen $s-\epsilon_{i}$ en $s$.
        Die $x_{i}$ is dan groter dan $x_{i-1}$.\waarom

      \item We hebben nu een stijgende rij $(x_{n})_{n}$ geconstrueert zodat het volgende geldt:
        \[ \forall n\in \mathbb{N}_{0}:\ |x_{n}-s| \le \frac{1}{n} \]
        Hieruit volgt meteen dat $s$ de limiet is van $(x_{n})_{n}$.\waarom
      \end{itemize}
    \end{itemize}
  \end{proof}
\end{bpr}

\begin{de}
  De \term{uitgebreidde orde in een totaal geordend veld} $F,+,\cdot,\le$.\\
  We moeten soms de orde van een totaal geordend veld uitbreiden over $F \cup \{ -\infty,+\infty\}$.
  De notatie blijft dan hetzelfde maar we voegen het volgende toe.
  \[ \forall a\in F: -\infty \le a \quad\text{ en }\quad \forall a \in F:\ a \le +\infty \]
\end{de}
\TODO{oneindigheden zelf ook eens definieren?}

\begin{bpr}
  \label{pr:limiet-behoudt-orde}
  Zij $(x_{n})_{n}$ en $(y_{n})_{n}$ twee rijen zodat voor alle $n\in \mathbb{N}$ $x_{n}\le y_{n}$ geldt, dan geldt het volgende:
  \[ \lim_{n\rightarrow \infty}x_{n} \le \lim_{n\rightarrow \infty}y_{n} \]

  \begin{proof}
    Noem $a$ en $b$ respectievelijk de limieten van $(x_{n})_{n}$ en $(y_{n})_{n}$.
    We bewijzen nu uit het ongerijmde dat $a \le b$ geldt.
    Stel daarom dat $a$ groter is dan $b$.
    Er zijn nu vier gevallen:
    \begin{itemize}
    \item $a = +\infty, b \in \mathbb{R}$\\
      \extra{bewijs}
    \item $a,b\in \mathbb{R}$\\
      Noem $\delta = \frac{a-b}{2}$.
      $\delta$ is dan positief omdat $a$ groter is dan $b$.
      Omdat $(x_{n})_{n}$ naar $a$ gaat kunnen we een $n\in \mathbb{N}$ zodat voor alle grotere $n'$ het volgende geldt:
      \[ |x_{n'}-a|<\delta \]
      \[ x_{n'} > a - \delta = \frac{a+b}{2} \]
      Analoog vinden we het volgende voor de andere rij:
      \[ y_{m'} < b+\delta = \frac{a+b}{2} \]
      Neem nu een getal $p\in \mathbb{N}$ groter dan $n$ en $m$, dan geldt het volgende, maar dat is in strijd met $x_{p} \le y_{p}$(gegeven).
      \[ y_{p} < \frac{a+b}{2} < x_{p} \]
    \item $a\in \mathbb{R}, b = -\infty$\\
      \extra{bewijs}
    \item $a= +\infty, b=-\infty$\\
      \extra{bewijs}
    \end{itemize}
  \end{proof}
\end{bpr}

\begin{bst}
  \label{st:insluitstelling}
  De \term{insluitstelling}\\
  Beschouw drie rijen $(x_{n})_{n}$, $(y_{n})_{n}$ en $(z_{n})_{n}$ zodat het volgende geldt:
  \[ \forall n\in \mathbb{N}: x_{n}\le y_{n}\le z_{n} \]
  Als $(x_{n})_{n}$ en $(z_{n})_{n}$ dezelfde limiet hebben, dan geldt het volgende:
  \[ \lim_{n\rightarrow \infty}x_{n} = \lim_{n\rightarrow \infty}y_{n} = \lim_{n\rightarrow \infty}z_{n} \]

  \begin{proof}
    Noem $a$, $b$ en $c$ de respectievelijke limieten van $(x_{n})_{n}$, $(y_{n})_{n}$ en $(z_{n})_{n}$.
    We bewijzen nu uit het ongerijmde dat $b$ gelijk is aan $c$ en $a$ door te stellen dat $b$ groter is dan $a$ en $c$.
    Er zijn nu vier gevallen:
    \begin{itemize}
    \item $b = +\infty, a,c \in \mathbb{R}$\\
      \extra{bewijs}
    \item $a,b,c\in \mathbb{R}$\\
      Noem $\delta = \frac{b-c}{2}$.
      $\delta$ is dan positief omdat $b$ groter is dan $c$.
      Omdat $(y_{n})_{n}$ naar $b$ gaat kunnen we een $n\in \mathbb{N}$ zodat voor alle grotere $n'$ het volgende geldt:
      \[ |y_{n'}-b|<\delta \]
      \[ y_{n'} > b - \delta = \frac{b+c}{2} \]
      Analoog vinden we het volgende voor $(z_{n})_{n}$:
      \[ z_{m'} < c+\delta = \frac{b+c}{2} \]
      Neem nu een getal $p\in \mathbb{N}$ groter dan $n$ en $m$, dan geldt het volgende, maar dat is in strijd met $y_{p} \le z_{p}$(gegeven).
      \[ z_{p} < \frac{b+c}{2} < y_{p} \]
      Zeer analoog vinden we ook dat $b$ niet kleiner kan zijn dan $a$.
      $b$ zit dus tussen $a$ en $c$ en is er bijgevolg gelijk aan.
    \item $b\in \mathbb{R}, a=c = -\infty$\\
      \extra{bewijs}
    \item $b= +\infty, a=c=-\infty$\\
      \extra{bewijs}
    \end{itemize}
  \end{proof}
\end{bst}


\begin{st}
  Zij $(x_{n})_{n}$ een convergente rij in $\mathbb{R}^{+}$ met limiet $x$, dan convergeert ook de rij $\left(\sqrt[k]{x_{n}}\right)_{n}$.

\extra{bewijs}
\end{st}


\subsection{Limieten en bewerkingen}
\label{sec:limi-en-bewerk}

\begin{bst}
  Zij $(x_{n})_{n}$ een convergente rij en $\lambda\in \mathbb{R}$ een re\"eel getal.
  \[ \lim_{n \rightarrow \infty}(\lambda x_{n}) = \lambda \lim_{n\rightarrow \infty}x_{n} \]

  \begin{proof}
    Noem $a$ de limiet van $(x_{n})_{n}$.
    We bewijzen nu dat $\lambda a$ de limiet is van $(\lambda x_{n})_{n}$.
    Kies daartoe een willekeurige $\epsilon \in \mathbb{R}_{0}^{+}$.
    Omdat $a$ de limiet is van $(x_{n})_{n}$ bestaat er een $n'\in \mathbb{N}$ zodat voor alle volgende $n$ de afstand van $x_{n}$ tot kleiner is dan $\frac{\epsilon}{\lambda}.$
    \[ |x_{n}-a| < \frac{\epsilon}{\lambda} \Leftrightarrow |\lambda x_{n}-\lambda a| < \epsilon \]
  \end{proof}
\end{bst}

\begin{bst}
  \label{st:som-van-limieten-is-limiet-van-som}
  Zij $(x_{n})_{n}$ en $(y_{n})_{n}$ twee convergente rijen.
  \[ \lim_{n \rightarrow \infty}(x_{n}+y_{n}) = \lim_{n\rightarrow \infty}x_{n} + \lim_{n\rightarrow \infty}y_{n} \]

  \begin{proof}
    Noem $a$ en $b$ de respectievelijke limieten van $(x_{n})_{n}$ en $(y_{n})_{n}$.
    Voor een willekeurige $\epsilon \in \mathbb{R}_{0}^{+}$ bestaan er $n'_{x}$ en $n'_{y}$ zodat $x_{n}$ en $y_{n}$ dichter dan $\frac{\epsilon}{2}$ bij $a$ en $b$ komen voor alle volgende $n_{x}$, $n_{y}$.
    \[ \left( |x_{n}-a| < \frac{\epsilon}{2} \wedge |y_{n}-a| < \frac{\epsilon}{2} \right) \Rightarrow |x_{n}+y_{n} - a-b| < \epsilon \]
  \end{proof}
\end{bst}

\begin{bst}
  Zij $(x_{n})_{n}$ een convergente rij en $(y_{n})_{n}$ een rij zonder limiet.
  $(x_{n}+y_{n})_{n}$ heeft geen limiet.
  
  \begin{proof}
    Noem $x$ de limiet van $(x_{n})_{n}$.
    Stel dat $(x_{n}+y_{n})_{n}$ een limiet $z$ zou hebben, dan zou $(y_{n})_{n}$ ook een limiet hebben:
    Kies een $\epsilon \in \mathbb{R}_{0}^{+}$.
    Er bestaat dan een $n_{0}\ in \mathbb{N}$ zodat voor alle volgende $n\in \mathbb{N}$ het volgende geldt:
    \[ |x_{n}-x| < \frac{\epsilon}{2} \]
    Er bestaat ook een $n_{1}\in \mathbb{N}$ zodat voor alle volgende $n\in \mathbb{N}$ het volgende geldt:
    \[ |x_{n}+y_{n}-z| < \frac{\epsilon}{2} \]
    Noem nu $n_{2} = \max\{n_{0},n_{1}\}$, dan geldt voor alle volgende $n\in \mathbb{N}$ het volgende:
    \[ |x_{n}+y_{n}-z|+|x_{n}-x| < \epsilon \]
    \[ |x_{n}+y_{n}-z|+|-(x_{n}-x)| < \epsilon \]
    \[ |x_{n}+y_{n}-z-(x_{n}-x)| < \epsilon \]
    \[ |y_{n}-z-x| < \epsilon \]
    \[ |y_{n}-(z+x)| < \epsilon \]
    $y$ convergeert dus naar $z+x$.
\feed
  \end{proof}
\end{bst}


\begin{bst}
  \label{st:product-van-limieten-is-limiet-van-product}
  Zij $(x_{n})_{n}$ en $(y_{n})_{n}$ twee convergente rijen.
  \[ \lim_{n \rightarrow \infty}(x_{n}y_{n}) = \lim_{n\rightarrow \infty}x_{n} \cdot \lim_{n\rightarrow \infty}y_{n} \]

  \begin{proof}
    Noem $a$ en $b$ de respectievelijke limieten van $(x_{n})_{n}$ en $(y_{n})_{n}$.
    Kies een wilekeurige $\epsilon \in \mathbb{R}_{0}^{+}$.
    We kunnen $|x_{n}y_{n}-ab|$ noor boven afschatten als volgt:
    \[ |x_{n}y_{n}-ab| = |x_{n}y_{n} -x_{n}b + x_{n}b -ab| = |x_{n}(y_{n}-b) + (x_{n}-a)b| \le |x_{n}||y_{n}-b| + |x_{n}-a||b| \]
    We zullen zorgen dat we een $n$ kunnen vinden zodat beide termen kleiner zijn dan $\frac{\epsilon}{2}$. Hieruit volgt dan de stelling.
    \begin{itemize}
    \item We kunnen $|x_{n}-a|$ zo klein kiezen als we willen door $n$ groot genoeg te kiezen omdat $a$ de limiet is van $x_{n}$.
      Kies daarom een $n_{1}\in\mathbb{N}$ groot genoeg zodat het volgende geldt:
      \[ |x_{n}-a| < \frac{\epsilon}{2|b|+1} \]
      Op deze manier is het rechterlid kleiner dan $\frac{\epsilon}{2}$.
    \item We kunnen $|y_{n}-b|$ zo klein kiezen als we willen door $n$ groot genoeg te kiezen omdat $b$ de limiet is van $y_{n}$.
      Om er zeker van te zijn dat $|x_{n}|$ ons niet tegenwerkt gebruiken we dat $x_{n}$ begrensd is.\prref{pr:convergente-rij-begrensd}
      Noem die bovengrens $M\in\mathbb{R}_{0}^{+}$.
      Kies dan een $n_{2}\in\mathbb{N}$ groot genoeg zodat het volgende geldt:
      \[ |y_{n}-b| < \frac{\epsilon}{2M} \]
    \item Kies nu $n$ groter dan zowel $n_{1}$ als $n_{2}$.
    \end{itemize}
  \end{proof}
  \extra{mooier herschrijven zoals op p 41}
\end{bst}

\begin{tvb}
  Zij $(x_{n})_{n}$ een convergente rij en $(y_{n})_{n}$ een rij zonder limiet.
  $(x_{n}\cdot y_{n})_{n}$ heeft \textbf{niet} noodzakelijk \textbf{geen} limiet.

  \begin{proof}
    Zij $(x_{n})_{n}$ de constante rij $(0)_{n}$ en $(y_{n})_{n}$ de rij van alternerend $1$ en $-1$, dan is de rij $(x_{n}\cdot y_{n})_{n}$ convergent met limiet $0$.
  \end{proof}
\question{geldt dit wel als we nul niet meerekenen?}
\end{tvb}
  
\begin{bst}
  Zij $(x_{n})_{n}$ en $(y_{n})_{n}$ twee convergente rijen zodat $\forall n\in \mathbb{N}:\ y_{n}\neq 0$ en $\lim_{n\rightarrow \infty}y_{n} \neq 0$ gelden.
  \[ \lim_{n \rightarrow \infty}\left(\frac{x_{n}}{y_{n}}\right) = \frac{\lim_{n\rightarrow \infty}x_{n}}{\lim_{n\rightarrow \infty}y_{n}} \]

  \begin{proof}
    Noem $a$ en $b$ de respectievelijke limieten van $(x_{n})_{n}$ en $(y_{n})_{n}$.
    $b$ is dan verschillend van $0$.
    We tonen aan dat het volgende geldt, uit stelling \ref{st:som-van-limieten-is-limiet-van-som} volgt dan de stelling.
    \[ \lim_{n\rightarrow \infty}\frac{1}{y_{n}} = \frac{1}{b} \]
    Kies daartoe een willekeurige $\epsilon \in \mathbb{R}_{0}^{+}$.
    Merk op dat we $\left|\frac{1}{y_{n}}-\frac{1}{b}\right|$ kunnen herschrijven als volgt:
    \[ \left|\frac{1}{y_{n}}-\frac{1}{b}\right| = \frac{|b-y_{n}|}{|y_{n}||b|} \]
    Omdat $b$ de limiet is van $(y_{n})_{n}$ kunnen we een $n_{1}$ groot genoeg kiezen zodat het volgende geldt voor alle $n>n_{1}$:
    \[ |y_{n}-b| < \frac{|b|}{2} \]
    \[ |b| \le |b-y_{n}| +|y_{n}| < \frac{|b|}{2} + |y_{n}| \]
    $|y_{n}|$ is dan kleiner dan $\frac{|b|}{2}$.
    \clarify{waarom precies $\frac{|b|}{2}$?}
    We kunnnen ook een $n_{2}$ kiezen zodat voor alle grotere $n\in \mathbb{N}$ het volgende geldt:
    \[ |y_{n}-b| < \frac{\epsilon|b|^{2}}{2} \]
    Nemen we nu $n_{0}$ als het maximum van $n_{1}$ en $n_{2}$, dan geldt het volgende en bijgevolg de stelling.
    \[ \left|\frac{1}{y_{n}}-\frac{1}{b}\right| = \frac{|b-y_{n}|}{|y_{n}||b|} \le \frac{2|y_{n}-b|}{|b|^{2}} < \frac{2\epsilon|b|^{2}}{|b|^{2}2} = \epsilon \]
  \end{proof}
\end{bst}

\begin{bst}
  We kunnen bovenstaande stellingen uitbreiden om te gelden over $\mathbb{R}\cup\{ -\infty,+\infty\}$ als we de volgende rekenregels toevoegen.
  \[
  \begin{array}{crccccl}
    (1) &                        & (+\infty) &+    & (+\infty) &= +\infty\\
    (2) &                        & (-\infty) &+    & (-\infty) &= -\infty\\
    (3) &                        & (+\infty) &+    & (-\infty) & & \text{ is onbepaald.} \\
    (4) &                        & (-\infty) &+    & (+\infty) & & \text{ is onbepaald.} \\
  \end{array}
  \]
  \[
  \begin{array}{crccccl}
    (5) & \forall a \in \mathbb{R}:\        & a         &+    & (+\infty) &= + \infty \\
    (6) & \forall a \in \mathbb{R}:\        & (+\infty) &+    & a         &= + \infty \\
    (7) & \forall a \in \mathbb{R}:\        & a         &+    & (-\infty) &= - \infty \\
    (8) & \forall a \in \mathbb{R}:\        & (-\infty) &+    & a         &= - \infty \\
  \end{array}
  \]
  \[
  \begin{array}{crccccl}
    (9) &                          & (+\infty) &\cdot& (+\infty) &= +\infty\\
    (10) &                          & (-\infty) &\cdot& (-\infty) &= +\infty\\
    (11) &                          & (+\infty) &\cdot& (-\infty) &= -\infty\\
    (12) &                          & (-\infty) &\cdot& (+\infty) &= -\infty\\
  \end{array}
  \]
  \[
  \begin{array}{crccccl}
    (13) &                          & 0         &\cdot& (+\infty) & & \text{ is onbepaald.} \\
    (14) &                          & 0         &\cdot& (-\infty) & & \text{ is onbepaald.} \\
    (15) &                          & (+\infty) &\cdot& 0         & & \text{ is onbepaald.} \\
    (16) &                          & (-\infty) &\cdot& 0         & & \text{ is onbepaald.} \\
  \end{array}
  \]
  \[
  \begin{array}{crccccl}
    (17) & \forall a \in \mathbb{R}_{0}^{+}:\ & a         &\cdot& (+\infty) &= + \infty \\
    (18) & \forall a \in \mathbb{R}_{0}^{+}:\ & (+\infty) &\cdot& a         &= + \infty \\
    (19) & \forall a \in \mathbb{R}_{0}^{+}:\ & a         &\cdot& (-\infty) &= - \infty \\
    (20) & \forall a \in \mathbb{R}_{0}^{+}:\ & (-\infty) &\cdot& a         &= - \infty \\
  \end{array}
  \]
  \[
  \begin{array}{crccccl}
    (21) & \forall a \in \mathbb{R}_{0}^{-}:\ & a         &\cdot& (+\infty) &= + \infty \\
    (22) & \forall a \in \mathbb{R}_{0}^{-}:\ & (+\infty) &\cdot& a         &= + \infty \\
    (23) & \forall a \in \mathbb{R}_{0}^{-}:\ & a         &\cdot& (-\infty) &= - \infty \\
    (24) & \forall a \in \mathbb{R}_{0}^{-}:\ & (-\infty) &\cdot& a         &= - \infty \\
  \end{array}
  \]
  \[
  \begin{array}{crccccl}
    (25) & \forall a \in \mathbb{R}:\        & \frac{a}{+\infty}      &= 0 \\
    (26) & \forall a \in \mathbb{R}:\        & \frac{a}{-\infty}      &= 0 \\
    (27) & \forall a \in \mathbb{R}_{0}^{+}:\ & \frac{+\infty}{a}      &= +\infty \\
    (28) & \forall a \in \mathbb{R}_{0}^{+}:\ & \frac{-\infty}{a}      &= -\infty \\
    (29) & \forall a \in \mathbb{R}_{0}^{-}:\ & \frac{+\infty}{a}      &= -\infty \\
    (30) & \forall a \in \mathbb{R}_{0}^{-}:\ & \frac{-\infty}{a}      &= +\infty \\
  \end{array}
  \]
  \[
  \begin{array}{crccccl}
    (31) &                          & \frac{+\infty}{-\infty} &&&& \text{ is onbepaald.}\\
    (32) &                          & \frac{+\infty}{+\infty} &&&& \text{ is onbepaald.}\\
    (33) &                          & \frac{-\infty}{+\infty} &&&& \text{ is onbepaald.}\\
    (34) &                          & \frac{-\infty}{-\infty} &&&& \text{ is onbepaald.}\\
  \end{array}
  \]
  \[
  \begin{array}{crccccl}
    (35) &                          & \frac{0}{0}             &&&& \text{ is onbepaald.}\\
  \end{array}
  \]
  \begin{proof}
    Bewijs in (veel) delen.
    \begin{itemize}
    \item $(1)$\\
      Zij $(x_{n})_{n}$ en $(y_{n})_{n}$ twee rijen met limiet $+\infty$.
      Kies nu een willekeurige $M\in \mathbb{R}$.
      Kies dan $n_{1}\in \mathbb{N}$ en $n_{2}\in \mathbb{N}$ groot genoeg zodat het volgende geldt:
      \[ \forall n\in \mathbb{N}: n \ge n_{1} \Rightarrow x_{n}>\frac{M}{2} \]
      \[ \forall n\in \mathbb{N}: n \ge n_{2} \Rightarrow y_{n}>\frac{M}{2} \]
      Kies nu $n_{0} = \max\{n_{1},n_{2}\}$, dan geldt voor alle grotere $n\in\mathbb{N}$ het volgende:
      \[ x_{n}+y_{n} > \frac{M}{2}+\frac{M}{2} = M \]
    \item $(2)$ \extra{bewijs}
    \item $(3)$ en $(4)$\\
      We zoeken voor elke van de volgene mogelijke rekenregels (tegelijk) een tegenvoorbeeld:
      \[ 
      \begin{array}{cccc}
        (+\infty) &+& (-\infty) &= +\infty\\
        (+\infty) &+& (-\infty) &= -\infty\\
        (+\infty) &+& (-\infty) &= a \in \mathbb{R}\\
      \end{array}
      \]
      Kies daartoe de rijen $(n+(-1)^{n})_{n}$ en $(-n)_{n}$.
      De limiet van de som van deze rijen bestaat niet.
    \item $(5)$ en $(6)$\\
      Zij $(x_{n})_{n}$ en $(y_{n})_{n}$ twee rijen met respectievelijke limiet $a\in \mathbb{R}$ en $b = +\infty$.
      Kies nu een willekeurige $M \in \mathbb{R}$.
      Kies dan $n_{1}\in \mathbb{N}$ en $n_{2}\in \mathbb{N}$ groot genoeg zodat het volgende geldt:
      \[ \forall n\in \mathbb{N}: n \ge n_{1} \Rightarrow |x_{n}-a| < 1 \]
      \[ \forall n\in \mathbb{N}: n \ge n_{2} \Rightarrow y_{n} > M-a+1 \]
      Kies nu $n_{0} = \max\{n_{1},n_{2}\}$, dan geldt voor alle grotere $n\in\mathbb{N}$ het volgende:
      \[ x_{n}+y_{n} > (a-1) + (M-a+1) = M \]
    \item $(7)$ en $(8)$ \extra{bewijs}
    \item $(9)$ \extra{bewijs}
    \item $(10)$ \extra{bewijs}
    \item $(11)$ en $(12)$ \extra{bewijs}
    \item $(13)$ en $(15)$ \extra{bewijs}
    \item $(14)$ en $(16)$ \extra{bewijs}
    \item $(17)$ en $(18)$ \extra{bewijs}
    \item $(19)$ en $(20)$ \extra{bewijs}
    \item $(21)$ en $(22)$ \extra{bewijs}
    \item $(23)$ en $(24)$ \extra{bewijs}
    \item $(25)$ \extra{bewijs}
    \item $(26)$ \extra{bewijs}
    \item $(27)$ \extra{bewijs}
    \item $(28)$ \extra{bewijs}
    \item $(29)$ \extra{bewijs}
    \item $(30)$ \extra{bewijs}
    \item $(31)$ \extra{bewijs}
    \item $(32)$ \extra{bewijs}
    \item $(33)$ \extra{bewijs}
    \item $(34)$ \extra{bewijs}
    \item $(35)$ \extra{bewijs}
    \end{itemize}
  \end{proof}
  \TODO{bewijs p 42}
\end{bst}

\begin{de}
  We noemen een rij verzameling $(I_{n})_{n}$ een \term{geneste rij} als het volgende geldt:
  \[ \forall i \in \mathbb{N}_{0}: I_{i+1} \subseteq I_{i} \]
\end{de}

\begin{st}
  \label{st:geneste-intervallen}
  De \term{stelling van de geneste intervallen}\\
  Zij $(\interval{a_{n}}{b_{n}})_{n}$ een geneste rij van niet-lege, gesloten begrensde intervallen in $\mathbb{R}$ waarvan de lengte naar nul gaat.
  \[ \lim_{n \rightarrow \infty}(b_{n}-a_{n}) = 0 \]
  Er bestaat dan een uniek getal $x\in \mathbb{R}$ zodat $x$ in elk interval $\interval{a_{n}}{b_{n}}$ zit.
  \[ \bigcap_{n}\interval{a_{n}}{b_{n}} \text{ is een singleton } x \]

  \begin{proof}
    Merk eerst op dat, omdat de rij intervallen genest is, de rij $(a_{n})_{n}$ stijgend is en de rij $(b_{n})_{n}$ dalend.
    Omdat de rij intervallen begrensd is, moeten de rijen $(a_{n})_{n}$ en $(b_{n})_{n}$ ook begrensd zijn.
    $(a_{n})_{n}$ en $(b_{n})_{n}$ hebben dus elk een eindige limiet,\stref{st:stijgend-dan-limiet}\stref{st:dalend-dan-limiet} zeg $a$ en $b$ respectievelijk.
    (merk op dat het interval nooit leeg kan zijn omdat het gesloten is.)
    Zowel $a$ als $b$ zit in elk interval $\interval{a_{n}}{b_{n}}$.
    We bewijzen dat $a$ gelijk is aan $b$:
    We weten dat $a_{n}$ telkens klener is of gelijk aan dat $a$ en dat $b$ telkens kleiner is of gelijk aan $b_{n}$.
    We weten ook dat $a$ kleiner is of gelijk aan $b$:
    \[ a_{n} \le a \le b \le b_{n} \]
    Omdat $|a_{n}-b_{n}|$ willekeurig klein kan worden moet $a$ gelijk zijn aan $b$.
    Omdat elk punt dat in de doorsnede van de intervallen ligt tussen $a_{n}$ en $b_{n}$ moet liggen, is het punt $x=a=b$ het enige element in de doorsnede.
  \end{proof}
\end{st}

\extra{bewijs dat deze stelling niet geldt in $\mathbb{Q}$.}

\subsection{Deelrijen}
\label{sec:deelrijen}

\begin{de}
  Zij $(x_{n})_{n}$ een rij over een verzameling $V$ en $(n_{k})_{k}$ een strikt stijgende rij over $\mathbb{N}$, dan is de rij $(x_{n_{k}})_{k}$ een \term{deelrij} van $(x_{n})_{n}$.
\end{de}

\begin{vb}
  Beschouw de rij $\left(\frac{1}{n+1}\right)_{n}$ en neem $n_{k} = 2k$, dan ziet de deelrij met $n_{k}$ er uit als $\left( \frac{1}{2k+1} \right)_{k}$.
\end{vb}

\begin{vb}
  De deelrij met $n_{k}=2k+1$ van $\left((-1)^{n}\right)_{n}$ is de constante rij $(-1)_{k}$.
\end{vb}

\begin{vb}
  De deelrij met $n_{k}=m+k$ van een rij $(x_{n})_{n}$ met $m\in \mathbb{N}_{0}$ is de rij $(x_{k})_{k}$ waarin we $k$ laten beginnen van $m$ in plaats van van $1$.
\end{vb}

\begin{vb}
  Beschouw de rij $(x_{n})_{n}$ met $x_{n} = \frac{(-1)^{n}n}{n+1}$.
  Met $n_{k} = 2k$ verkrijgen we de deelrij $(x_{n_{k}})_{k}$ met $x_{n_{k}} = \frac{2k}{2k+1}$ en met $m_{k} = 2k+1$ verkrijgen we de deelrij $(x_{m_{k}})_{k}$ met $x_{m_{k}} = -\frac{2k+1}{2k+2}$.
\end{vb}

\begin{bpr}
  \label{pr:deelrij-zelfde-limiet-als-limiet-bestaat}
  Zij $(x_{n})_{n}$ een rij in $\mathbb{R}$ met een limiet in $\mathbb{R}\cup\{+\infty,-\infty\}$, dan heeft elke deelrij ervan dezelfde limiet.

  \begin{proof}
    Zij $(x_{n})_{n}$ een rij in $\mathbb{R}$ met een limiet $a$ in $\mathbb{R}\cup\{+\infty,-\infty\}$, en zij $(x_{n_{k}})_{k}$ een deelrij hiervan, dan tonen we het volgende aan:
    \[ \lim_{k\rightarrow \infty}x_{n_{k}} = a \]
    Er volgt een gevalsonderscheid:
    \begin{itemize}
    \item $a\in \mathbb{R}$\\
      Kies een willekeurige $\epsilon \in \mathbb{R}_{0}^{+}$.
      Neem $m$ zodat $|x_{n}-a| < \epsilon$ geldt voor alle grotere $n\in \mathbb{N}$.
      Omdat $(n_{k})_{k}$ strikit stijgend is in $\mathbb{N}$ geldt $n_{k} \ge k$ voor alle $k\in \mathbb{N}$.
      Voor alle $k \ge m$ geldt dus ook $n_{k} \ge m$ en daarom $|x_{n_{k}}-a| < \epsilon$.
    \item $a = +\infty$
      \extra{bewijs}
    \item $a = -\infty$
      \extra{bewijs}
    \end{itemize}
  \end{proof}
\end{bpr}

\begin{pr}
  Zij $(x_{n})_{n}$ een rij in $\mathbb{R}$ en $(x_{n_{k}})_{k}$ de deelrij gedefinieerd met $n_{k} = m+k$ (voor een bepaalde 'startwaarde' $m\in\mathbb{N}$) dan heeft deze deelrij $(x_{m+k})_{k}$ een limiet als en slechts als $(x_{n})_{n}$ een limiet heeft en zijn deze limieten bovendien gelijk.
  \extra{bewijs}
\end{pr}

\begin{bst}
  \label{st:bolzano-rijen}
  De \term{stelling van Bolzano-Weierstra\ss (rijen-versie)}\\
  Elke begrensde rij in $\mathbb{R}$ heeft een convergente deelrij.

  \begin{proof}
    Zij $(x_{n})_{n\in\mathbb{N}_{0}}$ een begrensde rij in $\mathbb{R}$.
    Definieer de volgende waarden:
    \begin{itemize}
    \item $t_{1} = sup\{ x_{n} \mid n \ge 1 \}$
    \item $n_{1}$ zodat $t_1-1 < x_{n_{1}} \le t_{1}$ geldt.
    \item $t_{i} = sup\{ x_{n} \mid n > n_{i-1} \}$
    \item $n_{i}$ zodat $t_{i}-\frac{1}{i} < x_{n_{k}} \le t_{k}$ geldt.
    \end{itemize}
    De rij $(t_{k})_{k\in \mathbb{N}_{0}}$ is nu een deelrij van $(x_{n})_{n\in\mathbb{N}_{0}}$ vanwege de supremumeigenschap van $\mathbb{R}$.\needed
    De rij $(t_{k})_{k\in \mathbb{N}_{0}}$ is dalend omdat voor grotere waarden van $k$ het $t_{k}$ het supremum is van een telkens kleinere verzameling.\stref{st:deelverzameling-kleiner-supremum}
    $(t_{k})_{k\in \mathbb{N}_{0}}$ is bovendien naar onder begrensd.\waarom
    $(t_{k})_{k\in \mathbb{N}_{0}}$ is daarom een convergente rij\stref{st:dalend-dan-limiet}, zeg met limiet $t$.
    Voor alle $k$ geldt nu $|x_{n_{k}} -t_{k}| < \frac{1}{k}$ en hieruit volgt de stelling.
    \clarify{waarom nemen we de $t_{i}$ precies zo?}
  \end{proof}
\end{bst}

\begin{opm}
  Deze stelling gaat niet op in $\mathbb{Q}$.
  \extra{bewijs}
\end{opm}

\begin{de}
  Zij $(x_{n})_{n}$ een begrensde rij.
  We defini\"eren de \term{limes superior} of \term{lim sup} en de \term{limes inferior} of \term{lim inf} van de rij $(x_{n})_{n}$ als volgt:
  \[ \limsup_{n\rightarrow \infty} x_{n} = \lim_{n\rightarrow \infty} sup\{x_{k}\mid k\ge n\} \quad\text{ en }\quad \liminf_{n\rightarrow \infty} x_{n} = \lim_{n\rightarrow \infty} inf\{ x_{k}\mid k\ge n\} \]
\end{de}

\extra{bewijzen dat $liminf$ en $limsup$ steeds bestaan.}

\begin{vb}
  Beschouw de rij $(x_{n})_{n}$ met $x_{n}=(-1)^{n}\frac{n}{n+1}$.
  \[ \forall n\in \mathbb{N}:\ \sup \left\{ (-1)^{k}\frac{k}{k+1} \mid k \ge n \right\} = 1 \]
  \[ \forall n\in \mathbb{N}:\ \inf \left\{ (-1)^{k}\frac{k}{k+1} \mid k \ge n \right\} = -1 \]
  We zien dat $\limsup_{n\rightarrow \infty}x_{n}$ $1$ zal zijn en dat $\liminf_{n\rightarrow \infty} x_{n}$ $-1$ zal zijn waardoor $(x_{n})_{n}$ zelf geen limiet heeft.
\end{vb}

\begin{vb}
  Beschouw de rij $(x_{n})_{n}$ met $x_{n} = (-1)^{n}\frac{n+2}{n+1}$.
  \[
  \forall n\in \mathbb{N}:\ \sup \left\{ (-1)^{k}\frac{k+2}{k+1} \mid k \ge n \right\}
  =
  \begin{cases}
    \frac{n+2}{n+1} & \text{ als $n$ even is}\\
    \frac{n+3}{n+2} & \text{ als $n$ oneven is}\\
  \end{cases}
  \]
  \[
  \forall n\in \mathbb{N}:\ \inf \left\{ (-1)^{k}\frac{k+2}{k+1} \mid k \ge n \right\}
  =
  \begin{cases}
    -\frac{n+2}{n+1} & \text{ als $n$ even is}\\
    -\frac{n+3}{n+2} & \text{ als $n$ oneven is}\\
  \end{cases}
  \]
  We zien dat $\limsup_{n\rightarrow \infty}x_{n}$ $1$ zal zijn en dat $\liminf_{n\rightarrow \infty} x_{n}$ $-1$ zal zijn waardoor $(x_{n})_{n}$ zelf geen limiet heeft.
\end{vb}

\begin{vb}
  Beschouw de rij $(x_{n})_{n}$ met $x_{n} = 3 + \left(-\frac{1}{2}\right)^{n}$.
  \[
  \forall n\in \mathbb{N}:\ \sup \left\{ 3 + \left(-\frac{1}{2}\right)^{k} \mid k \ge n \right\}
  =
  \begin{cases}
    3+\left(-\frac{1}{2}\right)^{n} & \text{ als $n$ even is}\\
    3+\left(-\frac{1}{2}\right)^{n+1} & \text{ als $n$ oneven is}\\
  \end{cases}
  \]
  \[
  \forall n\in \mathbb{N}:\ \inf \left\{ 3 + \left(-\frac{1}{2}\right)^{k} \mid k \ge n \right\}
  =
  \begin{cases}
    3+\left(-\frac{1}{2}\right)^{n+1} & \text{ als $n$ even is}\\
    3+\left(-\frac{1}{2}\right)^{n} & \text{ als $n$ oneven is}\\
  \end{cases}
  \]
  Zowel $\limsup_{n\rightarrow \infty}x_{n}$ als $\liminf_{n\rightarrow \infty} x_{n}$ is hier $3$ en $\lim_{n\rightarrow +\infty}x_{n}$ ook.
\end{vb}


\begin{bpr}
  Zij $(x_{n})_{n}$ een begrensde rij.
  \[ \liminf_{n\rightarrow \infty} x_{n} \le \limsup_{n\rightarrow \infty} x_{n} \]

  \begin{proof}
    Deze propositie geldt omdat een infimum altijd kleiner is dan, of gelijk aan, het supremum van dezelfde verzameling\stref{st:infimum-kleiner-dan-supremum}, en de limiet de orde bewaart.\prref{pr:limiet-behoudt-orde}
  \end{proof}
\end{bpr}


\begin{bpr}
  Zij $(x_{n})_{n}$ een begrensde rij.
  Als en slechts als $(x_{n})_{n}$ een limiet heeft geldt hetvolgende.
  \[ \limsup_{n\rightarrow \infty} x_{n} = \liminf_{n\rightarrow \infty} x_{n} \]
  De limiet van $(x_{n})_{n}$ is dan ook de gemeenschappelijke waarde van limsup en liminf.

  \begin{proof}
    Bewijs van een equivalentie.
    \begin{itemize}
    \item $\Rightarrow$\\
      Gevalsonderscheid, noem $\lim_{n\rightarrow \infty}x_{n} = a$.
      \begin{itemize}
      \item $a\in \mathbb{R}$\\
        Kies een willekeurige $\epsilon \in \mathbb{R}_{0}^{+}$.
        Kies dan $n_{0}\inf \mathbb{N}$ zodat voor alle volgende $n\in \mathbb{N}$ het volgende geldt:
        \[ |a - x_{n}| < \frac{\epsilon}{2} \Leftrightarrow a-frac{\epsilon}{2} < x_{n} < a + \frac{\epsilon}{2} \]
        Er geldt dan ook het volgende.
        \[ a-\frac{\epsilon}{2} \le \inf\{x_{k}\mid k>n\}\le \sup\{x_{k}\mid k>n\} \le a + \frac{\epsilon}{2} \]
        In het bijzonder geldt dan het volgende voor liminf en limsup:
        \[ a-\frac{\epsilon}{2} \le \liminf x_{n}\le \limsup x_{n} \le a + \frac{\epsilon}{2} \]
        \[ 0  \le \liminf x_{n} - \limsup x_{n} < \epsilon \]
        Hierut volgt meteen dat liminf en limsup gelijk zijn.
      \item $a=+\infty$
        \extra{bewijs}
      \item $a=-\infty$
        \extra{bewijs}
      \end{itemize}
    \item $\Leftarrow$\\
      Stel dat $\liminf_{n\rightarrow \infty} x_{n}$ gelijk is aan $\limsup_{n\rightarrow \infty} x_{n}$.
      Omdat $x_{n}$ steeds tussen het infimum en het supremum ligt, volgt uit de insluitstelling dat de limiet van $x_{n}$ gelijk is aan de gemeenschappelijke waarde van limsup en liminf.\stref{st:insluitstelling}
    \end{itemize}
  \end{proof}
\end{bpr}

\begin{bpr}
  Zij $(x_{n})_{n}$ een begrensde rij, dan bestaat er een deelrij van $(x_{n})_{n}$ die $\limsup_{n\rightarrow \infty} x_{n}$ als limiet heeft en een deelrij die $\liminf_{n\rightarrow \infty} x_{n}$ als limiet heeft.
  \extra{zie het bewijs van de stelling van Bolzano!}
\end{bpr}

\begin{bpr}
  Zij $(x_{k})_{k}$ deelrij met een limiet in $\bar{\mathbb{R}}$ van een begrensde rij $(x_{n})_{n}$.
  \[ \liminf_{n\rightarrow \infty} x_{n} \le \lim_{n\rightarrow \infty} x_{n} \le \limsup_{n\rightarrow \infty} x_{n} \]

  \begin{proof}
    Zij $(x_{n_{k}})_{k}$ een willekeurige convergente deelij van $(x_{n})_{n}$.
    Merk dan op dat voor alle $k\in \mathbb{N}$ het volgende geldt:
    \[ \inf\{x_{i}\mid i>n_{k}\} \le x_{n_{k}} \le \sup\{x_{i}\mid i>n_{k}\}\]
    Nu zijn $(\inf\{x_{i}\mid i>n_{k}\})_{k}$ en $(\sup\{x_{i}\mid i>n_{k}\})_{k}$ ook deelrijen van de convergente rijen $(\inf\{x_{k}\mid k>n\})_{n}$ en $(\sup\{x_{k}\mid k>n\})_{n}$.
    Ze hebben dus respectievelijk dezelfde limiet.
    Hieruit volgt dan samen met het behouden van de orde\prref{pr:limiet-behoudt-orde} de stelling.
  \end{proof}
\end{bpr}

\subsection{Cauchyrijen en de volledigheid van $\mathbb{R}$}
\label{sec:cauchyrijen-en-de}

\begin{de}
  We noemen een rij $(x_{n})_{n}$ over een totaal geordend veld $F,+,\cdot,\le$ een \term{Cauchyrij} als en slechts als het volgende geldt:
  \[ \forall \epsilon \in F_{0}^{+},\ \exists n_{0}\in \mathbb{N},\ \forall n,m \in \mathbb{N}_{0}:\ n,m \ge n_{0} \Rightarrow |x_{n}-x_{m}| < \epsilon \]
\end{de}

\begin{bpr}
  \label{pr:convergent-dan-cauchy}
  Elke convergente rij is een Cauchyrij.
  
  \begin{proof}
    Zij $(x_{n})_{n}$ een convergente rij met limiet $a$.
    Kies een willekeurige $\epsilon\in \mathbb{R}_{0}^{+}$.
    We kunnen dus een $n_{0}\in \mathbb{N}$ kiezen zodat voor alle volgende $n\in\mathbb{N}$ het de afstand tot $a$ kleiner is dan $\frac{\epsilon}{2}$.
    \[ |x_{n}-x_{m}| = |x_{n}-a+a-x_{m}| \le |x_{n}-a|+|a-x_{m}| \le \frac{\epsilon}{2}+\frac{\epsilon}{2}=\epsilon \]
  \end{proof}
\end{bpr}

\begin{bpr}
  In $\mathbb{Q}$ bestaan er Cauchyrijen die niet convergeren (in $\mathbb{Q}$).

  \begin{proof}
    Constructief bewijs van existentie.\\
    Benoem $A$ als volgt:
    \[ A = \{ q \in \mathbb{Q} \mid 1 \le q \le 2 \]
    Definieer dan recursief een rij $(p_{n})_{n}$ in $\mathbb{Q}$ als volgt waarbij $p_{0}\in\mathbb{Q}$ een willekeurig getal in $A$ genomen wordt.
    \[ p_{n+1} = \frac{1}{2}\left( p_{n} + \frac{2}{p_{n}} \right) \]
    Merk eerst al op dat $p_{n}$ steeds positief is.
    We bewijzen dat $(p_{n})_{n}$ een Cauchyrij is in $\mathbb{Q}$ met een niet-rationale limiet: $\sqrt{2}$.
    \begin{itemize}
    \item $(p_{n})_{n}$ is een Cauchyrij.\\
      Merk daartoe eerst op dat de afbeelding $f$ $A$ op een deel van zichzelf afstuurt.
      \[ f:\ \mathbb{Q}_{0} \rightarrow \mathbb{Q}:\ x \mapsto \frac{1}{2}\left(x+\frac{2}{x}\right) \]
      Daaruit volgt al dat alle punten $p_{n}$ tot $A$ behoren.
      Bovendien kunnen we de afstand tussen twee beelden herschrijven:
      \[
      \begin{array}{rl}
        |f(x) - f(y)| &=
        \left| \frac{1}{2}\left(x+\frac{2}{x}\right) - \frac{1}{2}\left(y+\frac{2}{y}\right) \right|\\
        &= \frac{1}{2} \left| x-y + \frac{2(y-x)}{xy} \right|\\
        &= \frac{1}{2} \left| 1 - \frac{2}{xy}\right| |x-y|
      \end{array}
      \]
      Omdat $x$ en $y$ beide in $A$ zitten, geldt het volgende:
      \[ \frac{1}{2} \left| 1 - \frac{2}{xy}\right| \le \frac{1}{2} \]
      Voor de rij $(p_{n})_{n}$ betekent dat het volgende:
      \[ |p_{n+1} - p_{m+1}| \le \frac{1}{2}|p_{n}-p_{m}| \]
      Wanneer we deze ongelijkheid een gepast aantal keer \clarify{hoeveel keer?} toepassen, krijgen we een afschatting voor $|p_{n}-p_{m}|$:
      \[ |p_{n}-p_{m}| \le \frac{1}{2^{m}}|p_{n-m}-p_{0}| \le \frac{1}{2^{m}} \]
      Door $m$ voldoende groot te kiezen kunnen we de afstand tussen $m$ en $n$ dus willekeurig klein krijgen.
      Dit betekent precies dat $(p_{n})_{n}$ een Cauchyrij is.
    \item $(p_{n})_{n}$ convergeert naar een element buiten $\mathbb{Q}$.
      Stel nu dat deze rij een limiet $p \in \mathbb{Q}$ zou hebben.
      Uit de definitie van $p_{n}$ volgt de volgende gelijkheid:
      \[ p_{n+1}p_{n} = 1 + \frac{1}{2}p_{n}^{2} \]
      In de limiet vinden we dan het volgende:
      \[ pp = 1 + \frac{1}{2}p^{2} \Leftrightarrow p^{2} = 2 \]
      We weten dat er geen $p\in\mathbb{Q}$ bestaat waarvan het kwadraat $2$ is.\needed
    \end{itemize}
  \end{proof}
  \extra{bewijs is niet evident, misschien nog meer uitleg over waarom we bepaalde dingen inzien}
\end{bpr}

\begin{bpr}
  \label{pr:cauchyrij-begrensd}
  Elke Cauchyrij over een totaal geordend veld $\mathbb{R},+,\cdot,\le$ is begrensd.

  \begin{proof}
    Zij $(x_{n})_{n}$ een Cauchyrij.
    Vanaf een bepaalde $n'\in \mathbb{N}$ geldt het volgende:
    \[ \forall m,n \in \mathbb{N}: |x_{n}-x_{m}| < 1 \]
    Vanaf die $n'$ zijn alle $x_{n}$ in absolute waarde kleiner dan $1+|x_{n'}|$.
    Per constructie is $(x_{n})_{n}$ nu begrensd door $\{ |x_{0}|, |x_{1}| \dotsc, |x_{n'-1}|, 1+|x_{n}| \}$.
  \end{proof}
\end{bpr} 

\begin{bpr}
  \label{pr-cauchyrij-met-convergente-deelrij-convergeert}
  Een Cauchyrij over een totaal geordend veld $\mathbb{R},+,\cdot,\le$ met een convergente deelrij convergeert naar dezelfde limiet als die deelrij.

  \begin{proof}
    Zij $(x_{n})_{n}$ een Cauchyrij in $\mathbb{R}$ met convergente deelrij $(x_{n_{k}})_{k}$ en noem $a$ de limiet van de convergente deelrij.\\
    Kies een willekeurige $\epsilon \in \mathbb{R}_{0}^{+}$.
    Kies een $n_{1} \in \mathbb{N}$ zodat voor alle volgende $n,m\in\mathbb{N}$ $|x_{n}-x_{m}| < \frac{\epsilon}{2}$ geldt.
    Kies ook een $n_{2} \in \mathbb{N}$ zodat voor alle volgende $k\in\mathbb{N}$ $|x_{n_{k}}| < \frac{\epsilon}{2} \frac{\epsilon}{2}$ geldt.
    Noem nu $n' = max\{n_{1},n_{2}\}$.
    Kies nu willekeurig een grotere $j\in \mathbb{N}$, dan geldt $n_{j} \ge j$ en ook het volgende:
    \[ |x_{j} - a| = |x_{j}-x_{n_{j}}+x_{n_{j}}-a| \le |x_{j}-x_{n_{j}}|+|x_{n_{j}}-a| < \frac{\epsilon}{2} + \frac{\epsilon}{2} = \epsilon \]
  \end{proof}
\end{bpr}

\begin{bpr}
  \label{pr:cauchyrij-in-R-convergeert}
  Elke Cauchyrij in $\mathbb{R}$ convergeert.

  \begin{proof}
    Zij $(x_{n})_{n}$ een Cauchyrij in $\mathbb{R}$.
    De rij is dan begrensd\prref{pr:cauchyrij-begrensd} en heeft een convergente deelrij\stref{st:bolzano-rijen}.
    Dit betekent dat de rij convergeert.\prref{pr-cauchyrij-met-convergente-deelrij-convergeert}
  \end{proof}
\end{bpr}

\begin{de}
  We noemen een totaal geordend veld $F,+,\cdot,\le$ \term{volledig} als elke Cauchyrij in $F$ een limiet heeft in $F$.
\end{de}

\begin{opm}
  $\mathbb{R}$ is dus volledig, maar $\mathbb{Q}$ niet.
\end{opm}

\extra{tekstje op blz 60 toch nog eens bekijken}



\end{document}

%%% Local Variables:
%%% mode: latex
%%% TeX-master: t
%%% End:
