\documentclass[main.tex]{subfiles}
\begin{document}



\section{Het product van metrische ruimten}
\label{sec:het-product-van}

\begin{de}
  Op een product van twee metrische ruimten $X_{1},d_{X_{1}}$ en $X_{2},d_{X_{2}}$, definieren we de $i$-de \term{selectie-metriek} $d_{i}$ als volgt:
  \[ d_{i}:\ X_{1}\times X_{2} \rightarrow \mathbb{R}^{+}:\ ((x_{1},x_{2}),(y_{1},y_{2})) \mapsto d_{X_{i}}(x_{i},y_{i}) \]
  De metriek 'selecteert' met andere woorden het resultaat van \'e\'en van de metrische ruimten.
\end{de}

\begin{opm}
  $X_{1}\times X_{2},d_{i}$ is dus volledig als en slechts als $X_{i}$ volledig is.
\end{opm}

\begin{vb}
  De volgende functie is een metriek voor een product van twee metrische ruimten $X_{1},d_{X_{1}}$ en $X_{2},d_{X_{2}}$:
  \[ d_{+}:\ X_{1}\times X_{2} \rightarrow \mathbb{R}^{+}:\ ((x_{1},x_{2}),(y_{1},y_{2})) \mapsto d_{X_{1}}(x_{1},y_{1}) + d_{X_{2}}(x_{2},y_{2}) \]
  \extra{nagaan dat dit inderdaad een metriek is.}
\end{vb}

\begin{vb}
  Het maximum van twee metrieken is, op een productruimte, weer een metriek:
  \[ d_{\max}:\ X_{1}\times X_{2} \rightarrow \mathbb{R}^{+}:\ ((x_{1},x_{2}),(y_{1},y_{2})) \mapsto \max\{ d(x_{1},y_{1}), d(x_{2},y_{2})\} \]
  \extra{nagaan dat dit echt een metriek is}
\end{vb}

\extra{city-block metriek op productruimten}

\end{document}

%%% Local Variables:
%%% mode: latex
%%% TeX-master: t
%%% End:
