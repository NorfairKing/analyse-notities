\documentclass[main.tex]{subfiles}
\begin{document}

\section{Reeksen in $\mathbb{C}$}
\label{sec:reeksen-mathbbc}

\begin{de}
  Zij $(x_{n})_{n}$ een rij in $\mathbb{C}$.
  Hiermee construeren we een nieuwe rij $(s_{n})_{n}$ als volgt:
  \[  s_{n} = \sum_{k=0}^{n}x_{k}  \]
  Men noemt het stel $((x_{n})_{n}, (s_{n})_{n})$ de \term{reeks} van de rij $(x_{n})_{n}$ .
  Men noemt $s_{n}$ de $n$-de \term{partieelsom} van de reeks en $x_{n}$ de $n$-de \term{term} van de reeks.
\end{de}

\begin{de}
  Men koort het tupel $((x_{n})_{n}, (s_{n})_{n})$ vaak af met $\sum_{k}x_{k}$.
\end{de}

\begin{opm}
  Het maakt natuurlijk niet uit met welk getal we de reeks/rij beginnen te nummeren.
  De volgorde maakt wel uit. \needed
\end{opm}

\begin{de}
  We zekken dat een reeks $((x_{n})_{n}, (s_{n})_{n})$ convergeert in $\mathbb{C}$ als de rij $(s_{n})_{n}$ een limiet heeft in $\mathbb{C}$.
  Men noteert die limiet dan als $\sum_{k=0}^{\infty}x_{k}$.
\end{de}

\begin{vb}
  De volgende reeks convergeert als en slechts als $p \in \mathbb{R}$ groter is dan $1$.
  \[ \sum_{n}\frac{1}{n^{p}} \]
\extra{bewijs}
\end{vb}

\begin{bpr}
  Stel dat een reeks $\sum_{k}x_{k}$ convergeert, dan geldt $\lim_{n \rightarrow \infty}x_{n} = 0$.
\TODO{bewijs: oefening p 3}
\end{bpr}

\begin{tvb}
  Het omgekeerde van bovenstaande stelling geldt niet.
  Zie hieronder voor een tegenvoorbeeld.
\end{tvb}

\begin{de}
  De \term{harmonische reeks} is de volgende reeks in $\mathbb{C}$.
  \[ \sum_{n}\frac{1}{n} \]
\end{de}

\begin{st}
  De termen van de harmonische reeks gaan naar nul.
\extra{bewijs}
\end{st}

\begin{st}
  De harmonische reeks convergeert niet.
\extra{bewijs}
\end{st}

\begin{st}
  Het \term{Criterium van Leibniz}\\
  Beschouw een dalende rij $(x_{n})_{n}$ van positieve getallen die naar nul gaat, dan convergeert $\sum_{n}(-1)^{n}x_{n}$.
\TODO{bewijs p 3}
\end{st}

\begin{de}
  De \term{alternerende harmonische reeks} is de volgende reeks in $\mathbb{C}$.
  \[ \sum_{n}(-1)^{n}\frac{1}{n} \]
\end{de}

\begin{st}
  De alternerende harmonische reeks convergeert.
\extra{bewijs}
\end{st}

\extra{voorbeeld van volgorde van termen maakt uit p 4}

\begin{de}
  Men noemt een reeks $\sum_{n}x_{n}$ in $\mathbb{C}$ \term{absoluut convergent} als de reeks $\sum_{n}|x_{n}|$ convergent is (in $\mathbb{R}$).
\end{de}

\begin{vb}
  De volgende reeks (met $z\in \mathbb{C}$) convergeert absoluut als $|z|$ kleiner is dan $1$ en anders helemaal niet
  \[ \sum_{n}z^{n} \]
\extra{bewijs}
\end{vb}

\begin{bpr}
  Een absoluut convergente reeks is convergent.
\TODO{bewijs p 6}
\end{bpr}

\begin{tvb}
  Het omgekeerde van bovenstaande stelling geldt niet.
\extra{tegenvoorbeeld}
\end{tvb}

\begin{bst}
  Zij $\sum_{n}x_{n}$ een absoluut convergente reeks in $\mathbb{C}$.
  Zij $\pi:\ \mathbb{N} \rightarrow \mathbb{N}$ een bijectie, dan convergeert de reeks $\sum_{n}x_{\pi(n)}$ en de limiet ervan is dezelfde als die van de reeks $\sum_{n}x_{n}$.
\TODO{bewijs p 7}
\end{bst}

\begin{st}
  Zij $(x_{n})_{n}$ een rij in $\mathbb{C}$ zodat voor alle bijecties $\pi:\ \mathbb{N} \rightarrow \mathbb{N}$, de reeks $\sum_{n}x_{\pi(n)}$ convergeert, dan moet de reeks $\sum_{n}x_{n}$ absoluut convergeren.
\extra{bewijs}
\end{st}

\begin{bst}
  De \term{vergelijkingstest}\\
  Zij $(x_{n})_{n}$ een rij in $\mathbb{C}$ en $(y_{n})_{n}$ een rij in $\mathbb{R}^{+}$.
  Stel dat er een $n\in \mathbb{N}$ bestaat zodat voor alle volgende $n\in \mathbb{N}$ $|x_{n}| < y_{n}$ geldt, dan convergeert $\sum_{n}x_{n}$ absoluut als $\sum_{n}y_{n}$ convergeert.
\TODO{bewijs p 7}
\end{bst}

\begin{st}
  Beschouw een reeks $\sum_{n}x_{n}$ in $\mathbb{C}$ met (alle?) termen verschillend van nul.
  \begin{itemize}
  \item Als $\limsup_{n\rightarrow \infty}\left| \frac{x_{n+1}}{x_{n}}\right|$ kleiner is dan $1$, dan convergeert de reeks asoluut.
  \item Als er een $n_{0}$ bestaat zodat $\left| \frac{x_{n+1}}{x_{n}}\right|$ groter wordt dan $1$ voor alle volgende $n\in \mathbb{N}$, dan convergeert de reeks niet.
  \end{itemize}
\TODO{bewijs p 9}
\end{st}

\begin{opm}
  Als $\lim_{n\rightarrow \infty}\left|\frac{x_{n+1}}{x_{n}}\right| = 1$ geldt, doet de verhoudingstest geen uitspraak.
\extra{waarom?} voorbeelden.
\end{opm}

\begin{vb}
  Beschouw de volgende reeks in $\mathbb{C}$ voor een willekeurige $z\in \mathbb{C}$.
  \[ \sum_{n}\frac{z^{n}}{n!} \]
  We toesten met de verhoudingstest of deze reeks convergeert.
\extra{doen p 9}
\end{vb}

\begin{st}
  De \term{Worteltest van Cauchy}\\
  Beschouw een reeks $\sum_{n}x_{n}$ in $\mathbb{C}$ en noem $L= \limsup_{n\rightarrow \infty}\sqrt[n]{|x_{n}|}$.
  \begin{itemize}
  \item Als $L<1$ geldt, dan convergeert de reeks absoluut.
  \item Als $L>1$ geldt, dan convergeert de reeks niet.
  \end{itemize}
\TODO{bewijs p 10}
\end{st}

\begin{opm}
  Als $\limsup_{n\rightarrow \infty}\sqrt[n]{|x_{n}|}= 1$ geldt, doet de wortel geen uitspraak.
\extra{waarom?} voorbeelden.
\end{opm}

\extra{voorbeelden p 10}

\begin{st}
  De worteltest is fijner dan de verhoudingstest.
\extra{formuleer p 11 en bewijs}
\end{st}





\end{document}

%%% Local Variables:
%%% mode: latex
%%% TeX-master: t
%%% End:
