\documentclass[main.tex]{subfiles}
\begin{document}

\section{Reeksen in $\mathbb{C}$}
\label{sec:reeksen-mathbbc}

\begin{de}
  Zij $(x_{n})_{n}$ een rij in $\mathbb{C}$.
  Hiermee construeren we een nieuwe rij $(s_{n})_{n}$ als volgt:
  \[  s_{n} = \sum_{k=0}^{n}x_{k}  \]
  Men noemt het stel $((x_{n})_{n}, (s_{n})_{n})$ de \term{reeks} van de rij $(x_{n})_{n}$ .
  Men noemt $s_{n}$ de $n$-de \term{partieelsom} van de reeks en $x_{n}$ de $n$-de \term{term} van de reeks.
\end{de}

\begin{de}
  Men koort het tupel $((x_{n})_{n}, (s_{n})_{n})$ vaak af met $\sum_{k}x_{k}$.
\end{de}

\begin{opm}
  Het maakt natuurlijk niet uit met welk getal we de reeks/rij beginnen te nummeren.
  De volgorde maakt wel uit. \needed
\end{opm}

\begin{de}
  We zekken dat een reeks $((x_{n})_{n}, (s_{n})_{n})$ convergeert in $\mathbb{C}$ als de rij $(s_{n})_{n}$ een limiet heeft in $\mathbb{C}$.
  Men noteert die limiet dan als $\sum_{k=0}^{\infty}x_{k}$.
\end{de}

\begin{vb}
  De volgende reeks convergeert als en slechts als $p \in \mathbb{R}$ groter is dan $1$.
  \[ \sum_{n}\frac{1}{n^{p}} \]
\extra{bewijs}
\end{vb}


\begin{vb}
  \[ \sum_{n}\frac{1}{n(n+1)} = 1 \]

  \begin{proof}
    Beschouw eerst de partieelsommen $S_{n}$:
    \[ S_{1} = \frac{1}{2},\ S_{2} = \frac{2}{3},\ S_{3} = \frac{3}{4},\quad S_{i} = \frac{i}{i+1} \]
    De limiet van $S_{n}$, voor $n$ gaande naar $+\infty$ is duidelijk $1$:
    \[ \sum_{n}\frac{1}{n(n+1)} = \lim_{n\rightarrow +\infty}\sum_{i=1}^{n}\frac{1}{i(i+1)} = \lim_{n\rightarrow +\infty} \frac{n}{n+1} = 1 \]
  \end{proof}
\end{vb}

\begin{vb}
  \[ \sum_{n}\frac{1}{(2n-1)(2n+1)} = \frac{1}{2} \]

  \begin{proof}
    Beschouw eerst de partieelsommen $S_{n}$:
    \[ S_{1} = \frac{1}{3},\ S_{2} = \frac{2}{5},\ S_{3} = \frac{3}{7},\quad S_{i} = \frac{i}{2*i+1} \]
    De limiet van $S_{n}$, voor $n$ gaande naar $+\infty$ is duidelijk $1$:
    \[ \sum_{n}\frac{1}{(2n-1)(2n+1)} = \lim_{n\rightarrow +\infty}\sum_{i=1}^{n}\frac{1}{(2i-1)(2i+1)} = \lim_{n\rightarrow +\infty} \frac{n}{2n+1} = \frac{1}{2} \]
  \end{proof}
\end{vb}

\begin{vb}
  \[ \sum_{n}\frac{\sqrt{n+1}-\sqrt{n}}{\sqrt{n}\sqrt{n+1}} = 1 \]

  \begin{proof}
    Merk eerst de volgende gelijkheid op voor alle $n\in\mathbb{N}_{0}$:
    \[ \frac{\sqrt{n+1}-\sqrt{n}}{\sqrt{n}\sqrt{n+1}} = \frac{\sqrt{n+1}}{\sqrt{n}\sqrt{n+1}} - \frac{\sqrt{n}}{\sqrt{n}\sqrt{n+1}} = \frac{1}{\sqrt{n}} - \frac{1}{\sqrt{n+1}} \]
    De partieelsommen zien er dan als volgt uit:
    \[
    S_{k}
    = \sum_{i=1}^{k}\left(\frac{1}{\sqrt{n}} - \frac{1}{\sqrt{n+1}}\right)
    = \frac{1}{\sqrt{1}} - \cancel{\frac{1}{\sqrt{2}}} + \cancel{\frac{1}{\sqrt{2}}} - \cancel{\frac{1}{\sqrt{3}}} + \dotsb + \cancel{\frac{1}{\sqrt{k}}} - \frac{1}{\sqrt{k+1}}
    = 1 - \frac{1}{\sqrt{k+1}}
    \]
    De limiet hiervan, voor $k$ gaande naar $+\infty$ is duidelijk $1$:
    \[ \lim_{k\rightarrow +\infty}S_{k} = \lim_{k \rightarrow +\infty}\left(1-\frac{1}{\sqrt{k+1}}\right) = 1 \]
  \end{proof}
\end{vb}

\begin{vb}
  \[ \sum_{n}\frac{2n+1}{n^{2}(n+1)^{2}} = 1 \]
  \extra{bewijs}
\end{vb}

\begin{bpr}
  \label{pr:termen-van-convergente-reeks-naar-0}
  Stel dat een reeks $\sum_{k}x_{k}$ convergeert, dan geldt $\lim_{n \rightarrow \infty}x_{n} = 0$.

  \begin{proof}
    Bewijs uit het ongerijmde:
    Stel dat $(x_{n})_{n}$ niet naar nul convergeert, dan geldt het volgende:
    \[ \exists \epsilon \in R_{0}^{+},\ \forall n_{0}\in \mathbb{N},\ \exists n\in \mathbb{N}:\ n \ge n_{0} \wedge |x_{n}| \ge \epsilon \]
    De reeks $\sum_{k}x_{k}$ kunnen we dan als volgt herschrijven:
    \[ \lim_{n\rightarrow +\infty}\sum_{k=1}^{n}x_{n} = \lim_{n \rightarrow \infty}n\epsilon = +\infty \]
    $\sum_{k}x_{k}$ kan dus niet convergeren.
    \feed
  \end{proof}
\end{bpr}

\begin{tvb}
  Het omgekeerde van bovenstaande stelling geldt niet.
  Zie hieronder voor een tegenvoorbeeld.
\end{tvb}

\begin{de}
  De \term{harmonische reeks} is de volgende reeks in $\mathbb{C}$.
  \[ \sum_{n}\frac{1}{n} \]
\end{de}

\begin{st}
  \label{st:termen-harmonische-reeks-naar-nul}
  De termen van de harmonische reeks gaan naar nul.
\extra{bewijs}
\end{st}

\begin{st}
  De harmonische reeks convergeert niet.

  \begin{proof}
    Bekijk de eerste paar partieelsommen:
    \[
    s_{1} = \frac{1}{1}\quad
    s_{2} = 1 + \frac{1}{2}\quad
    s_{4} = 1 + \frac{1}{2} + \frac{1}{3} + \frac{1}{4}\quad
    s_{8} = 1 + \frac{1}{2} + \frac{1}{3} + \frac{1}{4} + \frac{1}{5} + \frac{1}{6} + \frac{1}{7} + \frac{1}{8}
    \]
    Merk nu het volgende op:
    \[ \frac{1}{3} + \frac{1}{4} \ge \frac{1}{4} + \frac{1}{4} = \frac{1}{2} \quad\text{en in het bijzonder: }\quad \forall m\in \mathbb{N}:\ \sum_{i=2^{m}}^{2^{m+1}}\frac{1}{i} \le 2^{m}\frac{1}{2^{m+1}} = \frac{1}{2} \]
    We vinden het volgende:
    \[ \lim_{n\rightarrow +\infty}\sum_{i=1}^{n}\frac{1}{i} \le \lim_{n\rightarrow +\infty} 1+ \log_{2}(n)\frac{1}{2} = +\infty \]
\feed
  \end{proof}
\end{st}

\begin{st}
  Zij $(x_{n})_{n}$ een dalende rij in $\mathbb{R}_{0}$.
  Stel dat er oneindig veel getallen $n\in\mathbb{N}_{0}$ bestaan zodat $x_{n} > \frac{1}{n}$ geldt, dan convergeert $\sum_{n}x_{n}$ zeker niet.

  \begin{proof}
    Beschouw de rij $(n_{k})_{k}$ van de getallen waarvoor $\forall k\in\mathbb{N}:\ x_{n_{k}} > \frac{1}{n_{k}}$ geldt en beschouw dan de deelrij $(x_{n_{k}})_{k}$ van $(x_{n})_{n}$.
    De partieelsom $S'_{n_{k}}$ van deze deelrij is dan zeker groter dan de partieelsom van de harmonische reeks:
    \[ S'_{n_{k}} = \sum_{i=1}^{k}x_{n_{k}} > \sum_{i=1}^{k}\frac{1}{k} = F_{k}  \]
    Omdat de rij daalt, maar de reeks toch die specifieke eigenschap heeft, zijn de termen strikt positief.
    Omdat de harmonische reeks divergeert zal de reeks $\sum_{n_{k}}x_{n_{k}}$ ook divergeren en bijgevolg ook $\sum_{n}x_{n}$.
\feed
  \end{proof}
\end{st}

\begin{st}
  \label{st:criterium-leibniz}
  Het \term{Criterium van Leibniz}\\
  Beschouw een dalende rij $(x_{n})_{n}$ van positieve getallen die naar nul gaat, dan convergeert $\sum_{n}(-1)^{n}x_{n}$.

  \begin{proof}
    Noteer met $(s_{n})_{n}$ de rij van partieelsommen.
    Voor alle $n\in \mathbb{N}$ vinden we het volgende:
    \[ s_{2n} = s_{2(n-1)} - (x_{2n-1}-x_{2n}) \]
    \[ s_{2n+1} = s_{2(n-1)+1} + (x_{2n}-x_{2n+1}) \]
    $(s_{2n})_{n}$ is dus een dalende rij en $(s_{2n+1})_{n}$ een stijgende rij.
    Bovendien geldt $s_{2n+1} = s_{2n}-x_{2n+1}$ zodat $s_{2n+1}$ voor alle $n\in \mathbb{N}$ kleiner is dan, of gelijk aan $s_{2n}$.
    Samen levert dit dat $(s_{2n})_{n}$ naar onder begrensd is, $(s_{2n+1})_{n}$ naar boven begrensd is en dat $sup_{n}s_{2n+1} \le \inf_{n}s_{2n}$ geldt.\question{hoe werkt die notatie?}
    Omdat $s_{2n}-s_{2n+1} = x_{2n+1}$ geldt, en $(x_{2n+1})$ naar $0$ convergeert (omdat $(x_{n})_{n}$ naar nul convergeert), moet $sup_{n}s_{2n+1} = \inf_{n}s_{2n}$ gelden.
    Dit getal is dus de limiet van $(s_{n})_{n}$ voor $n$ gaande naar oneindig.
    \[ \lim_{n \rightarrow +\infty}s_{2n+1} = \lim_{n \rightarrow +\infty}s_{2n} = \lim_{n \rightarrow +\infty}s_{n} \]
    $\sum_{n}(-1)^{n}x_{n}$ is dus convergent.
  \end{proof}
\end{st}

\begin{de}
  De \term{alternerende harmonische reeks} is de volgende reeks in $\mathbb{C}$.
  \[ \sum_{n}(-1)^{n}\frac{1}{n} \]
\end{de}

\begin{st}
  De alternerende harmonische reeks convergeert.

  \begin{proof}
    De termen van de harmonische reeks convergeren naar $0$.\stref{st:termen-harmonische-reeks-naar-nul}
    Wanneer we er alternerend een min en een plus tussen zetten, moet de bekomen reeks convergeren.\stref{st:criterium-leibniz}
  \end{proof}
\end{st}

\extra{voorbeeld van volgorde van termen maakt uit p 4}
\extra{procedure om om het even welke limiet te bekomen.}

\begin{de}
  \label{de:absoluut-convergent}
  Men noemt een reeks $\sum_{n}x_{n}$ in $\mathbb{C}$ \term{absoluut convergent} als de reeks $\sum_{n}|x_{n}|$ convergent is (in $\mathbb{R}$).
\end{de}

\begin{vb}
  De volgende reeks (met $z\in \mathbb{C}$) convergeert absoluut als $|z|$ kleiner is dan $1$ en anders helemaal niet
  \[ \sum_{n}z^{n} \]
\extra{bewijs}
\end{vb}

\begin{bpr}
  Een absoluut convergente reeks is convergent.

  \begin{proof}
    Zij $\sum_{n}x_{n}$ een absoluut convergente reeks.
    Noteer de rij van partieelsommen van $\sum_{n}x_{n}$ als $(s_{n})_{n}$ en die van $\sum_{n}|x_{n}|$ als $(t_{n})_{n}$.
    We vinden nu voor alle $n\in \mathbb{N}$ en $p\in \mathbb{N}_{0}$ het volgende:
    \[
    |s_{n+p} - s_{n}|
    = \left| \sum_{k=n+1}^{n+p}x_{k} \right|
    \le \sum_{k=n+1}^{n+p}|x_{k}|
    = |t_{n+p}-t_{n}|
    \]
    Vermits de reeks absoluut convergeert, convergeert de rij $(t_{n})_{n}$\deref{de:absoluut-convergent} en is ze dus een cauchyrij.\prref{pr:convergent-dan-cauchy}
    Uit bovenstaande afschatting volgt dat $(s_{n})_{n}$ ook een Cauchyrij is en dus convergent.\stref{st:cauchyrij-in-C-asa-convergent}
  \end{proof}
\end{bpr}

\begin{tvb}
  Het omgekeerde van bovenstaande stelling geldt niet.
\extra{tegenvoorbeeld}
\end{tvb}

\begin{bst}
  Zij $\sum_{n}x_{n}$ een absoluut convergente reeks in $\mathbb{C}$.
  Zij $\pi:\ \mathbb{N} \rightarrow \mathbb{N}$ een bijectie, dan convergeert de reeks $\sum_{n}x_{\pi(n)}$ en de limiet ervan is dezelfde als die van de reeks $\sum_{n}x_{n}$.

  \begin{proof}
    Noteer:
    \[
    s_{n} = \sum_{k=1}^{n}x_{k},\quad
    s=\lim_{n\rightarrow +\infty}s_{n}\quad
    \text{ en }\quad
    t_{n} = \sum_{k=1}^{n}x_{\pi(k)}
    \]
    Kies een $\epsilon \in \mathbb{R}_{0}^{+}$.
    Omdat $\sum_{n}x_{n}$ absoluut convergeert, kunnen we een $n_{0}\in \mathbb{N}$ vinden als volgt:
    \[ \forall p\in \mathbb{N}:\ p \ge n_{0}\Rightarrow \sum_{k=n_{0}+1}^{p}|x_{k}| < \epsilon \]
    Kies nu $q_{0}$ groot genoeg, zodat het volgende geldt:
    \[ \{ 1,2,\dotsc,n_{0} \} \subseteq \{ \pi(1), \pi(2), \dotsc, \pi(q_{0}) \} \]
    Kies nu een willekeurige $q\in \mathbb{N}$, groter dan $q_{0}$.
    We vinden het volgende:
    \[ \forall m\in \mathbb{N}:\ m\ge \max\{\pi(j)\mid j\in \{1,2,\dotsc q\} \} \Rightarrow |s_{m}-t_{q}| = \left| \sum_{k=0}^{m}x_{k} - \sum_{j=0}^{q}x_{\pi(j)}\right| \le \sum_{k=n_{0}+1}^{m}|x_{k}| < \epsilon \]
    Neem hiervan de limiet voor $m$ gaande naar $+\infty$.
    Dit levert het volgende:
    \[ \forall q\in \mathbb{N}:\ q \ge q_{0} \Rightarrow |s-t_{q}| \le \epsilon \]
    Dit bewijst dat $(t_{q})_{q}$ convergeert naar $s$.
  \end{proof}
\end{bst}

\begin{st}
  Zij $(x_{n})_{n}$ een rij in $\mathbb{C}$ zodat voor alle bijecties $\pi:\ \mathbb{N} \rightarrow \mathbb{N}$, de reeks $\sum_{n}x_{\pi(n)}$ convergeert, dan moet de reeks $\sum_{n}x_{n}$ absoluut convergeren.
\extra{bewijs}
\end{st}

\begin{bst}
  \label{st:vergelijkingstest}
  De \term{vergelijkingstest}\\
  Zij $(x_{n})_{n}$ een rij in $\mathbb{C}$ en $(y_{n})_{n}$ een rij in $\mathbb{R}^{+}$.
  Stel dat er een $n\in \mathbb{N}$ bestaat zodat voor alle volgende $n\in \mathbb{N}$ $|x_{n}| < y_{n}$ geldt, dan convergeert $\sum_{n}x_{n}$ absoluut als $\sum_{n}y_{n}$ convergeert.

  \begin{proof}
    Noteer:
    \[ 
    s_{n} = \sum_{k=0}^{n}|x_{n}|,\quad
    \text{ en }\quad
    t_{n} = \sum_{k=0}^{n}y_{n}
    \]
    We vinden het volgende:
    \[ \forall n,m\in \mathbb{N}:\ m \ge n_{0} \wedge n > m \Rightarrow |s_{n}-s_{m}| = \sum_{k=m+1}^{n}|x_{k}| \le \sum_{k=m+1}^{n}y_{k} = |t_{n}-t_{m}| \]
    Omdat $\sum_{n}y_{n}$ convergent is, is $(t_{n})_{n}$ een Cauchyrij.\prref{pr:convergent-dan-cauchy}
    Uit bovenstaande afschatting volgt dan dat $(s_{n})_{n}$ ook een Cauchyrij is en dus convergent.\stref{st:cauchyrij-in-C-asa-convergent}
    Bijgevolg convergeert $\sum_{n}|x_{n}|$.
  \end{proof}
\end{bst}

\begin{vb}
  De volgende reeks is absoluut convergent:
  \[ \sum_{n}\frac{(-1)^{n}n}{n^{3}+1} \]
  
  \begin{proof}
    We weten dat het volgende geldt voor $n\ge 1$
    \[
    \left| \frac{(-1)^{n}n}{n^{3}+1} \right|
    = \frac{|n|}{|n^{3}+1|}
    \le \frac{1}{n^{2}+ \frac{1}{n}}
    \le \frac{1}{n^{2}} \]
    Vanuit de vergelijkingstest weten we dan dat de reeks absoluut convergeert omdat $\sum_{n}\frac{1}{n^{2}}$ convergeert.\stref{st:vergelijkingstest}
  \end{proof}
\end{vb}

\begin{vb}
  De volgende reeks convergeert niet.
  \[ \sum_{n}\frac{n}{\sqrt{n^{3}+1}} \]

  \begin{proof}
    We weten dat het volgende geldt voor $n\ge 1$
    \[
    \left| \frac{n}{\sqrt{n^{3}+1}}\right|
    = \frac{n}{\sqrt{n^{3}+1}}
    = \frac{1}{\sqrt{n + \frac{1}{n^{2}}}}
    \ge \frac{1}{2\sqrt{n}}
    \]
    Vanuit de vergelijkingstest weten we dan dat de reeks niet convergeert omdat $\sum\frac{1}{2\sqrt{n}}$ niet convergeert.\stref{st:vergelijkingstest}
    \question{kunnen we die laatste stap niet weglaten?}
  \end{proof}
\end{vb}

\begin{st}
  Zij $(x_{n})_{n}$ en $(y_{n})_{n}$ rijen in $\mathbb{R}_{0}^{+}$ zodat $\lim_{n\rightarrow +\infty}\frac{x_{n}}{y_{n}}$ bestaat, eindig en niet nul is.
  $\sum_{n}x_{n}$ convergeert als en slechts als $\sum_{n}y_{n}$ convergeert.

  \begin{proof}
    Kies willekeurig een $\epsilon \in \mathbb{R}_{0}^{+}$, dan bestaat er een $n_{0} \in \mathbb{N}$ als volgt:
    \[ \forall n\in \mathbb{N}:\ n \ge n_{0} \Rightarrow \left|\frac{x_{n}}{y_{n}} - z\right| < \epsilon \]
    Equivalent met de laatste ongelijkheid zijn deze:
    \[ -\epsilon < \frac{x_{n}}{y_{n}} - z < \epsilon \]
    \[ y_{n}(-\epsilon + z) < x_{n} < y_{n}(\epsilon + z) \]
    Vanaf die $n_{0}$ geldt dus voor alle grotere $n\in \mathbb{N}$ zowel $x_{n}< y_{n}(\epsilon + z)$ als $y_{n}< \frac{1}{-\epsilon+z}x_{n}$.
    Uit de vergelijkingstest volgt dat $\sum_{n}x_{n}$ convergeert als $\sum_{n}y_{n}$ convergeert vanuit de eerste ongelijkheid en dat $\sum_{n}y_{n}$ convergeert als $\sum_{n}x_{n}$ convergeert vanuit de tweede ongelijkheid.\stref{st:vergelijkingstest}
  \end{proof}
\end{st}

\begin{st}
  \label{st:verhoudingstest-dalembert}
  De \term{verhoudingstest van d'Alembert}\\
  Beschouw een reeks $\sum_{n}x_{n}$ in $\mathbb{C}$ met alle termen verschillend van nul.
  \begin{itemize}
  \item Als $\limsup_{n\rightarrow \infty}\left| \frac{x_{n+1}}{x_{n}}\right|$ kleiner is dan $1$, dan convergeert de reeks asoluut.
  \item Als er een $n_{0}$ bestaat zodat $\left| \frac{x_{n+1}}{x_{n}}\right|$ groter wordt dan $1$ voor alle volgende $n\in \mathbb{N}$, dan convergeert de reeks niet.
  \end{itemize}

  \begin{proof}
    Twee delen:
    \begin{itemize}
    \item Als $\limsup_{n\rightarrow \infty}\left| \frac{x_{n+1}}{x_{n}}\right| < 1$ geldt, kunnen we een $r\in \interval[open]{0}{1}$ vinden als volgt:
      \[ \limsup_{n\rightarrow \infty}\left| \frac{x_{n+1}}{x_{n}}\right| < r < 1 \]
      Er bestaat dan een $n_{0}\in \mathbb{N}$ als volgt:\waarom
      \[ k \ge n_{0} \Rightarrow \frac{|x_{k+1}|}{|x_{k}|} < r \]
      Dit impliceert het volgende:
      \[ \forall n\in \mathbb{N}:\ n \ge n_{0} \Rightarrow |x_{n}| < r|x_{n-1}| < r^{2}|x_{n-2}| < \dotsb < r^{n-n_{0}}|x_{n_{0}}| = (r^{-n_{0}}|x_{n_{0}}|)r^{n} \]
      Omdat $r$ kleiner is dan $1$, convergeert de reeks $\sum_{n}(r^{-n_{0}}|x_{n_{0}}|)r^{n}$.
      Uit de vergelijkingstes volgt dan dat $\sum_{n}x_{n}$ absoluut convergeert.\stref{st:vergelijkingstest}
    \item Als $\frac{|x_{n+1}|}{|x_{n}|} \ge 1$ geldt vanaf een zekere $n$, zal het volgende gelden:
      \[ \forall n\in \mathbb{N}:\ n \ge n_{0} \Rightarrow |x_{n}| \ge |x_{n_{0}}| \]
      Omdat $x_{n_{0}}$ niet nul is, zal $(x_{n})_{n}$ niet naar $0$ convergeren en kan de reeks dus ook niet convergeren.\prref{pr:termen-van-convergente-reeks-naar-0}
    \end{itemize}
  \end{proof}
\end{st}

\begin{opm}
  Als $\lim_{n\rightarrow \infty}\left|\frac{x_{n+1}}{x_{n}}\right| = 1$ geldt, doet de verhoudingstest geen uitspraak.
\extra{waarom?} voorbeelden.
\end{opm}

\begin{vb}
  Beschouw de volgende reeks in $\mathbb{C}$ voor een willekeurige $z\in \mathbb{C}$.
  \[ \sum_{n}\frac{z^{n}}{n!} \]
  We toetsen met de verhoudingstest of deze reeks convergeert.

  \begin{proof}
    Voor $z=0$ is de convergentie triviaal.
    Beschouw daarom het geval $z \neq 0$.
    \[ \lim_{n\rightarrow +\infty}\frac{\frac{|z^{n+1}|}{(n+1)!}}{\frac{|z^{n}|}{n!}} = \lim_{n \rightarrow +\infty }\frac{|z|}{n+1} = 0 \]
    Bijgevolg geldt hetzelfde voor $\limsup$ \question{... gok ik? wat hier staat is toch niet hetzelfde als in de stelling?} en convergeert de reeks absoluut.\stref{st:verhoudingstest-dalembert}
  \end{proof}
\end{vb}

\begin{vb}
  De volgende reeks convergeert:
  \[ \sum_{n}\frac{n!^{2}}{(2n)!} \]

  \begin{proof}
    We gebruiken de verhoudingstest van d' Alembert\stref{st:verhoudingstest-dalembert}.
    Beschouw eerst de volgende gelijkheid:
    \[
    \frac{\frac{(n+1)!^{2}}{(2n+2)!}}{\frac{(n!)^{2}}{(2n)!}}
    = \frac{(n+1)!^{2}}{(2n+2)!}\frac{(2n)!}{(n!)^{2}}
    = \frac{(n+1)^{2}}{(2n+2)(2n+1)}
    = \frac{n^{2}+2n+1}{4n^{2}+6n+2}
    \]
    De limiet hiervan, voor $n$ gaande naar $+\infty$ is $\frac{1}{4}$, wat kleiner is dan $1$, dus de reeks convergeert.
  \end{proof}
\end{vb}

\begin{vb}
  De volgende reeks convergeert:
  \[ \sum_{n}\frac{\ln(n)}{2n^{3}-1} \]

  \begin{proof}
    We gebruiken de verhoudingstest van d' Alembert\stref{st:verhoudingstest-dalembert}.
    Beschouw eerst de volgende gelijkheid:
    \begin{align*}
      \frac{\frac{\ln(n+1)}{2(n+1)^{3}-1}}{\frac{\ln(n)}{2n^{3}-1}}
      = \frac{\ln(n+1)}{2(n+1)^{3}-1}\frac{2n^{3}-1}{\ln(n)}
      = \frac{\ln(n+1)}{\ln(n)} \frac{\left(2n^{3}-1\right)}{2(n+1)^{3}-1}\\
      = \ln(n+1-n)  \frac{\left(2n^{3}-1\right)}{2(n+1)^{3}-1}
      = \ln(1)  \frac{\left(2n^{3}-1\right)}{2(n+1)^{3}-1}
      = 0
    \end{align*}
    De limiet hiervan, voor $n$ gaande naar $+\infty$ is $0$, wat kleiner is dan $1$, dus de reeks convergeert.
  \end{proof}
\end{vb}

\begin{vb}
  De volgende reeks convergeert:
  \[ \sum_{n}\left(\frac{1}{3}i\right)^{n-1} \]

  \begin{proof}
    We gebruiken de verhoudingstest van d' Alembert\stref{st:verhoudingstest-dalembert}.
    Beschouw eerst de volgende gelijkheden:
    \[ \left(\frac{1}{3}i\right)^{n-1} = \frac{i^{n-1}}{3^{n-1}} \]
    \begin{align*}
      \left| \frac{\frac{i^{n}}{3^{n}}}{\frac{i^{n-1}}{3^{n-1}}} \right|
      = \left| \frac{i}{3} \right|
      = \frac{1}{3}
    \end{align*}
    De limiet hiervan, voor $n$ gaande naar $+\infty$ is $\frac{1}{3}$, wat kleiner is dan $1$, dus de reeks convergeert.
  \end{proof}
\end{vb}

\begin{st}
  \label{st:worteltest-cauchy}
  De \term{Worteltest van Cauchy}\\
  Beschouw een reeks $\sum_{n}x_{n}$ in $\mathbb{C}$ en noem $L= \limsup_{n\rightarrow \infty}\sqrt[n]{|x_{n}|}$.
  \begin{itemize}
  \item Als $L<1$ geldt, dan convergeert de reeks absoluut.
  \item Als $L>1$ geldt, dan convergeert de reeks niet.
  \end{itemize}

  \begin{proof}
    Twee delen.
    \begin{itemize}
    \item Als $L$ kleiner is dan $1$ kunnen we een $r\in \interval[open]{L}{1} \cap \interval[open]{0}{1}$ nemen.
      Er bestaat dus een $n_{0}\in \mathbb{N}$ als volgt:
      \[ \forall k\in \mathbb{N}:\ k \ge n_{0} \Rightarrow \sqrt[k]{|x_{k}|} < r \]
      We herschrijven dit als volgt:
      \[ \forall k\in \mathbb{N}:\ k \ge n_{0} \Rightarrow |x_{k}| < r^{k} \]
      De absolute convergente volgt nu uit de vergelijkingstest.\stref{st:vergelijkingstest}
    \item Als $L$ groter is dan $1$ geldt per definite het volgende:
      \[ \forall n\in \mathbb{N}:\ \sup_{k\ge n}\sqrt[k]{|x_{k}|} > 1 \]
      Voor elke $n$ bestaat er dus een $k\ge n$ zodat $\sqrt[k]{|x_{k}|}$ groter is dan $1$ en $|x_{k}|$ dus ook.
      $(x_{n})_{n}$ kan dan niet naar $0$ convergeren en de reeks dus ook niet.\prref{pr:termen-van-convergente-reeks-naar-0}
    \end{itemize}
  \end{proof}
\end{st}

\begin{opm}
  Als $\limsup_{n\rightarrow \infty}\sqrt[n]{|x_{n}|}= 1$ geldt, doet de wortel geen uitspraak.
  Immers, voor $\sum_{n}\frac{1}{n}$ geldt $\limsup_{n}\sqrt[n]{|x_{n}|} = 1$ en de reeks niet, maar voor $\sum_{n}\frac{1}{n^{2}}$ geldt dit ook en die reeks convergeert wel.
\end{opm}

\begin{st}
  De worteltest is fijner dan de verhoudingstest.
\extra{formuleer p 11 en bewijs}
\end{st}

\begin{vb}
  Beschouw een reeks als volgt:
  \[ \frac{1}{2} + \frac{1}{3} + \frac{1}{2^{2}} + \frac{1}{3^{2}} + \frac{1}{2^{3}} + \frac{1}{3^{2}} + \dotsc \]
  We beginnen de termen te nummeren vanaf $1$.
  \begin{align*}
    \liminf_{n\rightarrow \infty}\frac{x_{n+1}}{x_{n}} = \lim_{n \rightarrow \infty}\left(\frac{2}{3}\right)^{n} = 0\\
    \liminf_{n\rightarrow \infty}\sqrt[n]{x_{n}} = \lim_{n \rightarrow \infty}\sqrt[2n]{\frac{1}{3^{n}}} = \frac{1}{\sqrt{3}}\\
    \limsup_{n\rightarrow \infty}\sqrt[n]{x_{n}} = \lim_{n\rightarrow \infty}\sqrt[2n]{\frac{1}{2^{n}}} = \frac{1}{\sqrt{2}}\\
    \limsup_{n\rightarrow \infty}\frac{x_{n+1}}{x_{n}} = \lim_{n\rightarrow \infty}\frac{1}{2}\left(\frac{3}{2}\right)^{n} = +\infty
  \end{align*}
  De worteltest geeft aan dat de reeks convergeert terwijl de verhoudingstest geen uitsluitsel geeft.
\end{vb}

\begin{vb}
  Beschouw de volgende reeks
  \[ \frac{1}{2} + 1 + \frac{1}{8} + \frac{1}{4} + \frac{1}{32} + \frac{1}{16} + \frac{1}{128} + \frac{1}{64} + \frac{1}{512} + \frac{1}{256} + \dotsc \]
  We beginnen de termen te nummeren vanaf $1$.
  \begin{align*}
    \liminf_{n\rightarrow\infty}\frac{x_{n+1}}{x_{n}}=\frac{1}{8}\\
    \limsup_{n\rightarrow\infty}\frac{x_{n+1}}{x_{n}}=2\\
    \lim_{n\rightarrow\infty}\sqrt[n]{x_{n}} = \frac{1}{2}
  \end{align*}
  De verhoudingstest geeft geen uitsluitsel terwijl de worteltest convergentie aantoont.
\end{vb}

\begin{vb}
  De volgende reeks convergeert niet:
  \[ \sum_{n}\frac{(i+1)^{n}}{n} \]

  \begin{proof}
    We gebruiken de worteltest.\stref{st:worteltest-cauchy}
    Beschouw eerst de volgende gelijkheid:
    \[ \sqrt[n]{\left| \frac{(i+1)^{n}}{n} \right|} = \left| \frac{i+1}{\sqrt[n]{n}} \right| = \frac{\sqrt{2}}{\sqrt[n]{n}} \]
    De limiet hiervan, voor $n$ gaande naar $+\infty$ is $\sqrt{2}$, dus de reeks convergeert niet.
  \end{proof}
\end{vb}


\subsection{Extra oefeningen}
\label{sec:extra-oefeningen}

Onderzoek de convergentie van de volgende reeksen:
\begin{align}
  \sum_{n}\frac{1}{3^{n}+1}\\
  \sum_{n}\frac{n^{2}}{2n^{3}-1}\\
  \sum_{n}\frac{3^{n}}{n^{3}}\\
  \sum_{n=2}\frac{n(-1)^{n+1}}{3n-1}\\
  \sum_{n} n^{2}\sin\left(\frac{\pi}{2^{n}}\right)\\
  \sum_{n} (-1)^{n-1}\frac{1}{n}\\
  \sum_{n} (-1)^{n-1}\frac{1}{\sqrt{n}}\\
  \sum_{n=2} (-1)^{n}\frac{\sqrt{n}}{n+1}\\
  \sum_{n} \ln\left(\frac{n}{n+1}\right)\\
  \sum_{n} \frac{1}{\sqrt{n(n+1)}}\\
  \sum_{n} \sin\left(\frac{1}{n}\right)\\
  \sum_{n} (-1)^{n} \left( 1 - n\sin\left(\frac{1}{n}\right)\right)
\end{align}

\end{document}

%%% Local Variables:
%%% mode: latex
%%% TeX-master: t
%%% End:
