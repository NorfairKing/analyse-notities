\documentclass[main.tex]{subfiles}
\begin{document}



\chapter{Relaties}
\label{cha:relaties}

\begin{de}
  Een (binaire) \term{relatie} $R$ is een verzameling koppels $(x,y)$, respectievelijk van een verzameling $X$ en $Y$. 
  Wanneer $(x,y)$ een koppel is in $R$ noteren we $x R y$.
  \[ R \subseteq X \times Y \]
  Vaak worden $X$ en $Y$ opgenomen in de identiteit van de relatie om over surjecties te kunnen spreken.
\end{de}

\begin{de}
  De \term{eenheidsrelatie} $I_{X}$ op een verzameling $X$ is de volgende verzameling:
  \[ \{(x,x) \in X \times X\ |\ x \in X \} \]
\end{de}

\begin{de}
  De \term{inverse} $R^{-1}$ van een relatie $R$ is de volgende relatie:
  \[ R^{-1} = \left\{ (x,y)\ |\ (y,x) \in R \right\} \]
\end{de}

\begin{st}
  De inverse van de inverse van een relatie is opnieuw de originele verzameling.
  \[ R^{-1^{-1}} = R \]

  \begin{proof}
    \[
    \begin{array}{rll}
      R^{-1^{-1}} &= \left\{ (x,y)\ |\ (y,x) \in \left\{ (u,v)\ |\ (v,u) \in R \right\} \right\} &\\
                &= \left\{ (x,y)\ |\ (x,y) \in R \right\} &= R
    \end{array}
    \]
  \end{proof}
\end{st}

\section{Samenstelling van relaties}
\label{sec:samenst-van-relat}

\begin{de}
  De \term{samenstelling $S \circ R$ van twee relaties} $R$ en $S$ (lees: ``$S$ na $R$'') is de volgende relatie.
  \[ \left\{ (x,y) \ |\ (\exists z) ((x,z) \in R \wedge (z,y) \in S) \right\} \]
\end{de}

\begin{st}
  \label{st:samenstelling-relaties-associatief}
  De \term{samenstelling van relaties is associatief}.
  \[ (T \circ S) \circ R = T \circ (S \circ R) \]
\extra{bewijs}
\end{st}

\begin{st}
  De \term{inverse} van een relatie nemen is distributief ten opzichte van de samenstelling van relaties.
  \[ (S \circ R)^{-1} = R^{-1} \circ S^{-1} \]
\extra{bewijs}
\end{st}

\begin{de}
  Zij $R$ een relatie. Het \term{domein} (domain) is als volgt gedefinieerd.
  \[ dom R = \left\{ x \ |\ (\exists y)(x,y) \in R \right\} \]
\end{de}

\begin{de}
  Zij $R$ een relatie. Het \term{beeld} (range) is als volgt gedefinieerd.
  \[ bld  R = ran R = \left\{ y \ |\ (\exists x)(x,y) \in R \right\} \]
\end{de}

\begin{st}
  Het domein van een relatie is het beeld van zijn inverse.
  \[ dom(R) = bld(R^{-1}) \]
\extra{bewijs}
\end{st}

\begin{st}
  Het beeld van een relatie is het domein van zijn inverse.
  \[ bld(R) = dom(R^{-1}) \]
\extra{bewijs}
\end{st}

\begin{st}
  Domein na samenstelling:
  \[ dom (R \circ S) \subseteq dom(S) \]
\extra{bewijs}
\end{st}
 
\begin{st}
  Beeld na samenstelling:
  \[ bld (R \circ S) \subseteq bld(R) \]
\extra{bewijs}
\end{st}

\begin{st}
  Domein na samenstelling (2):
  \[ bld S \subseteq dom R \Rightarrow dom(R \circ S) = dom S \]
\extra{bewijs}
\end{st}

\begin{de}
  Een $n$-aire relatie is, analoog aan een binaire relatie, een verzameling $n$-tallen.
\end{de}

\section{Equivalentierelaties}
\label{sec:equivalentierelaties}

\begin{de}
  \label{de:reflexief}
  Een relatie $R$ op $X \times X$ is \term{reflexief} wanneer voor alle $x\in X$ $xRx$ geldt.
  \[ \forall x \in X: (x,x) \in R \]
\end{de}

\begin{de}
  \label{de:symmetrisch}
  Een relatie $R$ op $X \times X$ is \term{symmetrisch} wanneer voor alle $x,y\in X$ $xRy \Leftrightarrow yRx$ geldt.
  \[ \forall x,y \in X: (x,y) \in R \Leftrightarrow (y,x) \in R\]
\end{de}

\begin{de}
  \label{de:transitief}
  Een relatie $R$ op $X \times X$ is \term{transitief} wanneer voor alle $x,y,z\in X$ $(xRy \wedge yRz) \Rightarrow xRz$ geldt.
  \[ \forall x,y,z \in X: ((x,y) \in R \wedge (y,z) \in R) \Rightarrow (x,z) \in R\]
\end{de}

\begin{de}
  \label{de:equivalentierelatie}
  Een \term{equivalentierelatie} $R$ is een relatie die reflexief, symmetrisch en transitief is.
\end{de}

\subsection{Equivalentieklassen}
\label{sec:equivalentieklassen}

\begin{de}
  Zij $\sim$ een equivalentierelatie op $X$ en zij $x \in X$.
  De \term{equivalentieklasse} van $x$ is de verzameling van elk element dat equivalent is met $x$.
  \[ [x] = \{ y \in X\ |\ x \sim y \} \]
\end{de}

\begin{de}
  De \term{quotientverzameling} van $X$ ten opzichte van een equivalentierelatie $\sim$ is de verzameling van alle equivalentieklassen.
  \[ X/\sim = \{[x] \ |\ x \in X \}\]
\end{de}

\begin{st}
  \label{st:element-in-equivalentieklasse}
  Zij $\sim$ een equivalentierelatie op $X$, dan is elk element van $X$ een element van diens equivalentieklasse.
  \[ \forall x \in X:\ x \in [x] \]
  
  \begin{proof}
    Zij $x$ een willekeurig element van $X$, dan geldt $x \sim x$ vanwege de reflexiviteit van een equivalentierelatie.\footnote{Zie definitie \ref{de:equivalentierelatie} en \ref{de:reflexief}.}
  \end{proof}
\end{st}

\begin{st}
  \label{st:element-gelijke-equivalentieklasse}
  Zij $\sim$ een equivalentierelatie op $X$.
  \[ \forall x,y \in X:\ y \in [x] \Leftrightarrow [y] = [x] \]

  \begin{proof}
    Bewijs van een equivalentie.
    \begin{itemize}
    \item $\Rightarrow$\\
      Kies een willekeurige $y$ in de equivalentieklasse van $x$.
      \[ y \sim x \]
      Kies een willekeurige $z\in [y]$. 
      $x \sim y$ geldt alsook $y \sim z$ bijgevolg geldt $x \sim z$.\footnote{Zie definitie \ref{de:equivalentierelatie} en \ref{de:transitief}.}
      $[y]$ is dus een deelverzameling van $[x]$.
      \[ [y] \subseteq [x] \]
      De omgekeerde richting is analoog.\footnote{Zie definitie \ref{de:equivalentierelatie} en \ref{de:symmetrisch}.}
      \[ [x] \subseteq [y] \]
    \item $\Leftarrow$\\
      Stel $[y] = [x]$, nu geldt $y \in [y]$\footnote{Zie stelling \ref{st:element-in-equivalentieklasse}.} en bijgevolg $y \in [x]$.
    \end{itemize}
  \end{proof}
\end{st}

\begin{st}
  De quotientverzameling $X/\sim$ van een equivalentierelatie $\sim$ op verzameling $X$ is een partitie van $X$.

  \begin{proof}
    We gaan de voorwaarden uit de definitie van een partitie na.\footnote{Zie definitie \ref{de:partitie}.}
    \begin{itemize}
    \item Een element $[x]$ van $A$ bevat steeds een element $x$ en is dus niet leeg.
    \item Stel dat er twee verschillend elementen $[x]$ en $[y]$ zijn van $X/\sim$ die niet onderling disjunct zijn, dan bestaat er een element $z$ dat in zowel $[x]$ als $[y]$ zit. Nu geldt zowel $[z] = [x]$ als  $[z] = [y]$.\footnote{Zie stelling \ref{st:element-gelijke-equivalentieklasse}.}
      Tenslotte geldt $[x] = [y]$. Contradictie.
    \item Voor elk element $x \in X$ zit de equivalentieklasse in $A$. $A$ overdekt dus minstens $X$.
    \end{itemize}
  \end{proof}
\end{st}

\begin{st}
  \label{st:partitie-equivalentierelatie}
  Zij $P$ een partitie van $X$.
  De volgende verzameling vormt dan een equivalentierelatie op $X$.
  \[ x \sim y \Leftrightarrow (\exists A \in P:\ x \in A \wedge y \in A )\]

  \begin{proof}
    We definieren een relatie $\sim$ als volgt:
    \[
    x \sim y \Leftrightarrow \text{ x en y zitten in dezelfde deelverzameling van } P
    \]
    Dat deze relatie een equivalentierelatie is volgt meteen uit het feit dat ``dezelfde ... als'' ook een equivalentierelatie is.
  \end{proof}
\end{st}

\section{Orderelaties}
\label{sec:orderelaties}

\begin{de}
  Een relatie $R$ op een verzameling $X$ is \term{anti-symmetrisch} als het volgende geldt:
  \[ \forall x,y \in X: ((x,y) \in R \wedge (y,x) \in R) \Rightarrow x = y \]
\end{de}

\begin{de}
  Een (parti\"ele) \term{orderelatie} op $X$ is reflexief, transitief en anti-symmetrisch.
\end{de}

\begin{de}
  Een \term{grootste element} $a$ van een verzameling $A$ waarop een orderelatie $\prec$ is gedefinieerd is, is het element waarvoor geldt dat alle andere elementen kleiner zijn of gelijk aan $a$.
  \[ \forall x \in A: x \preceq a \] 
  Analoog wordt ook een \term{kleinste element} gedefinieerd.
\end{de}

\begin{de}
  Een \term{minimaal element} $a$ van $A$ waarop een orderelatie $\prec$ is gedefinieerd is, is het element waarvoor geldt dat er geen kleiner bestaat.
  \[ \not\exists x \in A: a \prec x \]
  Analoog wordt ook een \term{maximaal element} gedefinieerd.
\end{de}

\begin{opm}
  Een maximaal/minimaal element is niet noodzakelijk een grootste/kleinste element.    
\end{opm}

\begin{st}
  Een grootste/kleinste element is ook een maximaal/minimaal element.
\extra{bewijs}
\end{st}

\begin{de}
  Zij $(X,\preceq)$ een geordende verzameling en $A \subsetneq X$.
  $b \in X$ is een \term{bovengrens} van $A$ als het volgende geldt.
  \[ \forall x \in A: x \preceq b \]
  Analoog wordt een \term{ondergrens} gedefinieerd.
\end{de}

\begin{de}
  Een verzameling noemen we \term{naar boven begrensd} als ze een bovengrens is, \term{naar onder begrensd} als ze een ondergrens heeft en \term{begrensd} als ze \'e\'en van beide heeft.
\end{de}

\begin{opm}
  Een grens van een ordeverzameling hoeft dus niet in die verzameling te zitten.
\end{opm}

\begin{de}
  Een \term{supremum}(\term{infimum}) van een deelverzameling van een geordende verzameling is een bovengrens(ondergrens) die kleiner(groter) is dan elke andere bovengrens(ondergrens).
\end{de}

\begin{opm}
  Een supremum/infimum is een grens van een ordeverzameling en hoeft dus niet in die verzameling te zitten.
\end{opm}

\begin{st}
  Zij $A$ een partieel geordende verzameling met orderelatie $\prec$.
  Het kleinste/grootste element element van $A$ is uniek als het bestaat.
\extra{bewijs}
\end{st}

\begin{st}
  Zij $A$ een partieel geordende verzameling met orderelatie $\prec$.
  Het supremum/infimum van $A$ is uniek als het bestaat.
\extra{bewijs}
\end{st}

\begin{st}
  \label{st:deelverzameling-kleiner-supremum}
  Zij $(X,\preceq)$ een totaal geordende verzameling en $A \subsetneq X$.
  \[ sup X \ge sup A \quad\text{ en }\quad inf X \le inf A \]
  \TODO{bewijs: oefening}
\end{st}

\begin{de}
  Een \term{totale orderelatie} $\preceq$ is een partiele orderelatie met bijkomend de volgende eigenschap:
  \[ \forall x,y \in X: x \preceq y \vee y \preceq x \]
  Voor elke twee elementen zijn er dus precies drie mogelijkheden:
  \begin{itemize}
  \item $x \prec y$
  \item $x = y$
  \item $y \prec x$
  \end{itemize}
\end{de}

\begin{de}
  Zij $A$ een verzameling die volledig geordend is door de relatie $\prec$, dan noemen we $succ$ de successorfunctie als die gedefinieerd kan worden.
  \[ succ(x) = y \Leftrightarrow x < y \wedge (\not\exists z \in A:\ x < z < y \]
\end{de}

\begin{opm}
  De successorfunctie kan niet altijd gedefinieerd worden.
  Denk bijvoorbeeld aan de volgende volledige orderelatie over $\mathbb{Z}$:
  \[ |\le|:\ \mathbb{Z} \times \mathbb{Z}:\ x\ |\le|\ y \Leftrightarrow |x| \le |y| \]
\end{opm}

\begin{de}
  Wanneer we het symbool $\le$ gebruiken voor een totale orderelatie gebruiken we vaak de volgende afkortingen:
  \begin{itemize}
  \item '$a \ge b$' = '$b \le a$'
  \item '$a < b$' = '$\neg(b \le a)$'
  \item '$a > b$' = '$\neg(a \le b)$'
  \end{itemize}
\end{de}

\begin{st}
  \label{st:infimum-kleiner-dan-supremum}
  Het infimum van een totaal geordende verzameling $F,\le$ is steeds kleiner dan, of gelijk aan, het supremum ervan.
\extra{bewijs}
\end{st}


\end{document}

%%% Local Variables:
%%% mode: latex
%%% TeX-master: t
%%% End:
