\documentclass[main.tex]{subfiles}
\begin{document}



\subsection{De discrete metriek}
\label{sec:de-discrete-metriek}

\begin{vb}
  Zij $X$ een willekeurige, niet-lege verzameling.
  De \term{discrete metriek} of \term{triviale metriek} op $X$ wordt gegeven als volgt.
  \[
  d_{X}:\ X \times X \rightarrow \mathbb{R}^{+}:\ (x,y) \mapsto
  \begin{cases}
    0 &\text{ als } x = y\\
    1 &\text{ als } x \neq y
  \end{cases}
  \]
  $X,d_{X}$ vormt een metrische ruimte.
\end{vb}

\begin{st}
  Elke verzameling is begrensd voor de discrete metriek.

  \begin{proof}
    Inderdaad, de afstand tussen twee elementen is nooit groter dan $1$.
  \end{proof}
\end{st}

\begin{vb}
  Een open bol rond $x\in \mathbb{R}$ met straal $\delta\in \mathbb{R}_{0}^{+}$ voor de gewone metriek ziet eruit als volgt:
  \[ B(x,\delta) = 
  \begin{cases}
    \{x\} &\text{ als } r\le 1\\
    X &\text{ als } r > 1
  \end{cases}
  \]
\extra{bewijs}
\end{vb}

\begin{st}
  Elke deelverzameling $A$ van een verzameling $X$ is zowel open als gesloten voor de triviale metriek.
\extra{bewijs}
\end{st}


\begin{vb}
  Voor de triviale metriek $d_{X}$ voor een verzameling $X$, is elke functie $f:\ X \rightarrow Y$ met $Y,d_{Y}$ een metrische ruimte, continu.
\extra{bewijs}
\end{vb}

\begin{st}
  De triviale metriek $d_{X}$ voor een verzameling $X$ is topologisch fijner dan elke andere metriek op $X$.

  \begin{proof}
    De triviale topologie is de machtsverzameling $\mathcal{P}(X)$ van $X$.
    Omdat elke topologie een verzameling van deelverzamelingen van $X$ is, moet ze dus een deel zijn van $\mathcal{T}_{d_{X}}$.
  \end{proof}
\end{st}

\begin{st}
  Een deelverzameling $A$ van een willekeurige verzameling $X$ is zijn eigen sluiting en inwendige voor de triviale metriek.
\extra{bewijs}
\end{st}

\begin{st}
  Alle punten van een deelverzameling $A$ van een willekeurige verzameling $X$ zijn ge\"isoleerde punten voor de triviale metriek.
\extra{bewijs}
\end{st}

\begin{st}
  De verzameling $\{ x_{n} \mid n\in \mathbb{N} \}$ van elementen van een convergente rij $(x_{n})_{n}$ voor de discrete metriek is steeds eindig.

  \begin{proof}
    Zij $(x_{n})_{n}$ een convergente rij in een verzameling $X$ en noem $x$ de limiet..
    Omdat $(x_{n})_{n}$ convergeert, bestaat er een $n_{0}\in \mathbb{N}$ zodat alle $x_{n}$ vanaf $n_{0}$ dichter dan $1$ bij $x$ liggen en dus gelijk zijn aan $x$.
    Er zijn dan nog hoogstens $n_{0}-1$ (en dus hoogstens een eindig aantal) andere elementen in de rij te vinden.
  \end{proof}
\end{st}

\begin{st}
  Voor de triviale metriek is elke verzameling volledig.

  \begin{proof}
    Zij $(x_{n})_{n}$ een Cauchyrij in een metrische ruimte $X,d$, dan bestaat er een $n_{0} \in \mathbb{N}$ zodat alle $x_{n}$ met $n\in \mathbb{N}$ groter dan $n_{0}$ hetzelfde element $x$ zijn.
    De rij $(x_{n})_{n}$ convergeert dus naar $x$.
  \end{proof}
\end{st}


\end{document}

%%% Local Variables:
%%% mode: latex
%%% TeX-master: t
%%% End:
