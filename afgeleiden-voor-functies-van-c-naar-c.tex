\documentclass[main.tex]{subfiles}
\begin{document}

\section{Afgeleiden voor functies van $\mathbb{C}$ naar $\mathbb{C}$}
\label{sec:afgel-voor-funct}

\TODO{definieer ophopingspunt in $\mathbb{C}$}
\begin{de}
  Zij $f:\ A \subseteq \mathbb{C} \rightarrow \mathbb{C}$ een complexe functie en $a \in \mathbb{C}$ een ophopingspunt van $A$.
  We noemen $L\in \mathbb{C}$ de \term{limiet} van $f$ in $a$, en noteren dit als volgt:
  \[ \lim_{z \rightarrow a}f(z) = L \]
  ... als het volgende geldt:
  \[ \forall \epsilon \in \mathbb{R}_{0}^{+}: \exists \delta \in \mathbb{R}_{0}^{+}: \forall z\in A:\ 0 < |z-a| < \delta \Rightarrow |f(z) - L| < \epsilon \]
\end{de}

\begin{pr}
  \label{pr:limiet-van-functie-asa-limiet-van-beeld-van-rij-in-c}
  Beschouw een functie $f:\ A \subseteq \mathbb{C} \rightarrow \mathbb{C}$ en een $L \in \mathbb{C}$.
  Zij $a \in \mathbb{C}$ een ophopingspunt van $A$.
  De limiet van $f(x)$ in $a$ is $L$ als en slechts als $L$ ook de limiet is van het beeld van elke rij $(z_{n})_{n}$ in $A\setminus\{a\}$ die $a$ als limiet heeft.

  \begin{proof}
    Bewijs van een equivalentie.
    \begin{itemize}
    \item $\Rightarrow$
      Zij $(z_{n})_{n}$ een rij in $A\setminus \{a\}$ met $a$ als limiet.
      Kies een willekeurige $\epsilon \in \mathbb{R}_{0}^{+}$.
      Omdat de limiet van $f$ in $a$ $L$ is, bestaat er dan een $\delta \in \mathbb{C}_{0}^{+}$ zodat uit voor alle $z\in A$ uit $|z-a|<\delta$ volgt dat $|f(z)-L|<\epsilon$ geldt.
      Omdat de rij convergeert naar $a$, bestaat er een $n_{0}\in \mathbb{N}$ zodat voor alle volgende $n\in\mathbb{N}$ geldt dat $|z_{n}-a|$ kleiner is dan $\delta$.
      Vanaf die $n_{0}$ is $|f(z_{n})-L|$ dus kleiner dan $\epsilon$.
    \item $\Leftarrow$
\extra{bewijs}
    \end{itemize}
  \end{proof}
\end{pr}

\TODO{rekenregels voor limieten in $\mathbb{C}$}

\begin{de}
  Zij $f$ een functie en $a\in A$:
  \[ f:\ A \subseteq \mathbb{C} \rightarrow \mathbb{C}:\ x \mapsto f(x) \]
  We noemen $f$ \term{continu} in $a$ als en slechts als het volgende geldt:
  \[ \forall \epsilon \in \mathbb{R}_{0}^{+}:\ \exists \delta \in \mathbb{R}_{0}^{+}:\ \forall z\in A:\ |z-a| < \delta \Rightarrow |f(z) -f(a)| < \epsilon \]
  We noemen $f$ \term{continu} op $A$ als $f$ continu is in elke $a\in A$.
\end{de}

\TODO{verband tussen limieten en continu\"iteit}

\begin{bpr}
  \label{pr:limiet-in-c-dan-beperking-limiet-in-r}
  Beschouw een functie $f:\ A \subseteq \mathbb{C} \rightarrow \mathbb{C}$ met $A \cap \mathbb{R} \neq \emptyset$.
  Zij $a \in \mathbb{R}$ een ophopingspunt van $A \cap \mathbb{R}$ en noteer $g=f|_{A \cap \mathbb{R}}$.
  Als $\lim_{z\rightarrow a}f(z)$ bestaat, dan bestaat ook $\lim_{x \rightarrow a}g(x)$ en dan geldt $\lim_{x \rightarrow a}g(x) = \lim_{z \rightarrow a}f(z)$.

  \begin{proof}
    Stel dat de limiet $L = lim_{z\rightarrow a}f(z)$ bestaat, dan geldt per definitie het volgende:
    \[ \forall \epsilon \in \mathbb{R}_{0}^{+}: \exists \delta \in \mathbb{R}_{0}^{+}: \forall z\in A:\ 0 < |z-a| < \delta \Rightarrow |f(z) - L| < \epsilon \]
    Het is nu evident dat ook het volgende geldt, wat de stelling bewijst:
    \[ \forall \epsilon \in \mathbb{R}_{0}^{+}: \exists \delta \in \mathbb{R}_{0}^{+}: \forall r\in A\cap \mathbb{R}:\ 0 < |r-a| < \delta \Rightarrow |f(r) - L| < \epsilon \]
  \end{proof}
\end{bpr}

\begin{tvb}
  \label{tvb:limiet-in-c-dan-beperking-limiet-in-r}
  Het omgekeerde van bovenstaande stelling geldt niet.

  \begin{proof}
    Beschouw de functie $f$ als volgt en kies $a=0$.
    \[ f:\ \mathbb{C}_{0} \rightarrow \mathbb{C}:\ z \mapsto \frac{\overline{z}}{z} \]
    $a$ is inderdaad een ophopingspunt van $A\cap \mathbb{R}$.
    Hier ziet $g$ er als volgt uit:
    \[ g:\ \mathbb{R}_{0} \rightarrow \mathbb{R}:\ x \mapsto 1 \]
    De limiet van $g$ in $0$ is nu natuurlijk $1$, maar de limiet van $f$ in $0$ bestaat niet.
    Kies ter illustratie de rij $\left(\frac{1}{n}\right)_{n}$ die convergeert naar nul.
    Het beeld van deze rij convergeert naar $1$.
    Kies ook die rij $\left(\frac{i}{n}\right)_{n}$ die eveneens convergeert naar nul.
    Het beeld van deze rij convergeert naar $-1$, dus de limiet van $f$ in $0$ kan niet bestaan.\prref{pr:limiet-van-functie-asa-limiet-van-beeld-van-rij-in-c}
  \end{proof}
\end{tvb}

\begin{de}
  Beschouw een functie $f:\ A \subseteq \mathbb{C} \rightarrow \mathbb{C}$ en een ophopingspunt $a\in A$ van $A$.
  We noemen $f$ \term{afleidbaar} in $a$ als en slechts als de volgende limiet bestaat in $\mathbb{C}$
  \[ \lim_{z \rightarrow a}\frac{f(z)-f(a)}{z-a} \]
  We noemen deze limiet de \term{afgeleide} van $f$ in $a$ en noteren hem met $f'(a)$.
  Als alle punten van $A$ ophopingspunten zijn van $A$, noemen we $f$ \term{afleidbaar op} $A$ als afleidbaar is in elke $z \in A$.
  We definieren dan de \term{afgeleide functie} $f'$ als volgt:
  \[ f':\ A \subseteq \mathbb{C} \rightarrow \mathbb{C}:\ z \mapsto f'(z) \]
\end{de}

\TODO{rekenregels voor afgeleiden in $\mathbb{C}$}

\begin{vb}
  Voor alle $n \in \mathbb{N}_{0}$ is de afgeleide functie $f'$ van $f:\ \mathbb{C} \rightarrow \mathbb{C}: z \mapsto z^{n}$ gegeven door $f':\ \mathbb{C} \rightarrow \mathbb{C}: z \mapsto nz^{n-1}$.
\extra{bewijs}
\end{vb}

\begin{vb}
  De afgeleide functie $f'$ van $f:\ \mathbb{C}_{0} \rightarrow \mathbb{C}: z \mapsto \frac{1}{z}$ is gegeven door $f':\ \mathbb{C}_{0} \rightarrow \mathbb{C}: z \mapsto -\frac{1}{z^{2}}$.
\extra{bewijs}
\end{vb}

\begin{tvb}
  De modulusfunctie is nergens complex afleidbaar:
  \[ f:\ \mathbb{C} \rightarrow \mathbb{C}:\ z \mapsto |z| \]

  \begin{proof}
    Kies een element $a\in \mathbb{C}$.
    De afgeleide van $f$ is $a$ ziet er uit als volgt als hij bestaat.
    \[ \lim_{z \rightarrow a}\frac{f(z)-f(a)}{z-a} \]
    We tonen aan dat deze limiet niet bestaat.
    Stel hiertoe $z=a+h$ met $h\in \mathbb{R}$:
    \[ \lim_{h\rightarrow 0}\frac{f(a+h)-f(a)}{h} = \frac{|a+h|-|a|}{h} \le \frac{|a| + |h| - |a|}{h} = 1\]
    Stel dan $z=a+k$ met $k = hi$:
    \[ \lim_{h\rightarrow 0}\frac{f(a+k)-f(a)}{k} = \frac{|a+hi|-|a|}{hi} \le \frac{|a| + |hi| - |a|}{hi} = \frac{|hi|}{hi} = -i \]
    De afgeleide van de modulusfunctie kan dus niet bestaan in $a$.
  \end{proof}
\end{tvb}

\begin{bpr}
  Beschouw een functie $f:\ A \subseteq \mathbb{C} \rightarrow \mathbb{C}$ met $A \cap \mathbb{R} \neq \emptyset$ en $f(A \cap \mathbb{R})\subseteq \mathbb{R}$.
  Zij $a \in \mathbb{R}$ een ophopingspunt van $A \cap \mathbb{R}$ en noteer $g=f|_{A \cap \mathbb{R}}$.
  Als $f$ afleidbaar is in $a$, dan is ook $g$ afleidbaar in $a$ en geldt $g'(a)=f'(a)$.

  \begin{proof}
    Dit volgt rechtstreeks uit propositie \ref{pr:limiet-in-c-dan-beperking-limiet-in-r} op pagina \pageref{pr:limiet-in-c-dan-beperking-limiet-in-r}.
  \end{proof}
\end{bpr}

\begin{tvb}
  Het omgekeerde van deze stelling geldt niet.

  \begin{proof}
    Kies de functie $f$ als volgt en kies $a=0$.
    \[ f:\ \mathbb{C} \rightarrow \mathbb{C}:\ z \mapsto \overline{z} \]
    $a$ is een ophopingspunt van $A \cap \mathbb{R}$.
    Hier ziet $g$ er als volgt uit:
    \[ g:\ \mathbb{R} \rightarrow \mathbb{R}:\ x \mapsto x \]
    $g$ is dus afleidbaar in $0$ met afgeleide $1$.
    Voor alle $z \neq 0$ in $\mathbb{C}$ is echter $\frac{f(z)-f(0)}{z}$ gelijk aan $\frac{\overline{z}}{z}$ en de limiet daarvan in $0$ bestaat niet.\tvbref{tvb:limiet-in-c-dan-beperking-limiet-in-r}
  \end{proof}
\end{tvb}

\mst{complexe afleidbaarheid is sterker dan re\"ele afleidbaarheid}

\begin{bst}
  Beschouw een functie $f:\ A \subseteq \mathbb{C} \rightarrow \mathbb{C}$, gedefinieerd op een open deel $A$ van $\mathbb{C}$.
  Stel dat $f$ \'e\'en maal afleidbaar is op $A$, dan geldt:
  \begin{itemize}
  \item $f$ is oneindig keer afleidbaar op $A$.
  \item Zij $a\in A$, $r\in \mathbb{R}_{0}^{+}$ zodat uit $|z-a|< r$ volgt dat $z\in A$ geldt, dana geldt:
    \[ \forall z\in \mathbb{C}:\ |z-a|< r \Rightarrow f(z) = \lim_{n \rightarrow \infty}\sum_{k=0}^{n}\frac{f^{(k)}(a)}{k!}(z-a)^{k} \]
  \end{itemize}
\zb
\end{bst}

\begin{st}
  Als een functie $f:\ \mathbb{C} \rightarrow \mathbb{C}$ afleidbaar is op $\mathbb{C}$ en begrensd, dan is $f$ constant.
\zb
\end{st}

\begin{st}
  Als $f: A \subseteq \mathbb{C} \rightarrow \mathbb{C}$ afleidbaar is op een open deel $A$ van $\mathbb{C}$ en als $\gamma$ een gesloten kromme is die samen met haar binnenste in $A$ ligt, dan is de waarde vnan $f$ in alle punten binnen de kromme $\gamma$ bepaald door de waarde die $f$ aanneemt in punten op de kromme $\gamma$.
\zb
\end{st}

\begin{st}
  Als $f: A \subseteq \mathbb{C} \rightarrow \mathbb{C}$ afleidbaar is op een open deel $A$ van $\mathbb{C}$, dan kan $|f|$ geen lokaal maximum bereiken in $A$.
\zb
\end{st}

\begin{gst}
  De stelling van Rolle geldt niet in $\mathbb{C}$.

  \begin{proof}
    Kies $f:\ \interval{0}{1} \rightarrow \mathbb{C}:\ z \mapsto ((i-1)z+1)^{4}$.
    $f$ is de beperking van een afleidbare functie, gedefinieerd op gans $\mathbb{C}$, met hetzelfde voorschrift, tot $\interval{0}{1}$.
    De voorwaarden voor de stelling van rolle zijn voldaan:\stref{st:rolle}
    \[ f(0) = 1 = f(1) \]
    Er bestaat echter geen $c\in \interval[open]{0}{1}$ waarin $f'(c)$ nul is.
    \extra{ga dit na! zie forum}
  \end{proof}
\end{gst}
\begin{gst}
  De stelling van Lagrange geldt dus ook niet meer in $\mathbb{C}$

  \begin{proof}
    Dit volg meteen uit de vorige stelling.
  \end{proof}
\end{gst}

\begin{de}
  We noemen in $\mathbb{C}$ het \term{lijnstuk} tussen twee punten $a,b\in \mathbb{C}$ de volgende verzameling:
  \[ \{(1-\lambda)a+\lambda b\mid \lambda \in \interval{a}{b} \} \]
\end{de}

\begin{de}
  Een verzameling $A \subseteq \mathbb{C}$ noemen we \term{convex} als ze de volgende eigenschap heeft.
  \[ \forall a,b \in A:\ \forall \lambda \in \interval{0}{1}:\ ((1-\lambda)a+\lambda b) \in A \]
\end{de}
\extra{voorbeelden van convexe deelverzamelingen van $\mathbb{C}$.}

\begin{bpr}
  \label{pr:lagrange-achtig-in-c}
  Beschouw een afleidbare functie $f:\ A \subseteq \mathbb{C} \rightarrow \mathbb{C}$, gedefinieerd op een open, convex deel $A$ van $\mathbb{C}$.
  \[ \forall a,b \in A:\ a \neq b \Rightarrow \left|\frac{f(b)-f(a)}{b-a}\right| \le \sup\left\{ \left|f'((1-\lambda)a+\lambda b)\right| \mid \lambda \in \interval{0}{1} \right\} \]

  \begin{proof}
    \extra{idee van dit bewijs?}
    Definieer $F$ als volgt:
    \[ F:\ \interval{0}{1} \rightarrow \mathbb{R}:\ \lambda \mapsto F(\lambda) = Re\left((\overline{f(b)}-\overline{f(a)})f((1-\lambda)a+\lambda b)\right) \]
    Deze functie is afleidbaar, en dus continu\needed, op $\interval{0}{1}$.\waarom
    De afgeleide functie ziet er bovendien als volgt uit.
    \[ F'(\lambda) = Re\left( (\overline{f(b)}-\overline{f(a)})f'((1-\lambda)a+\lambda b)(b-a)\right) \]
    We passen de middelwaardestelling toe op $F$ om een $\lambda \in \interval[open]{0}{1}$ te bekomen als volgt:\stref{st:middelwaardestelling-lagrange}
    \[ F(1)-F(0) = F'(\lambda) \]
    Nemen we van beide leden de modules, dan bekomen we het volgende: ... \prref{pr:modulus-door-optelling}\prref{pr:normaal-maal-toegevoegde-in-r}\prref{pr:modulus-in-termen-van-toegevoegde}
    \[
    \begin{array}{rl}
      \left| F(1)-F(0) \right|
      &= \left| Re\left((\overline{f(b)}-\overline{f(a)})(f(b)-f(a))\right) \right|\\
      &= \left| Re\left((\overline{f(b)-f(a)})(f(b)-f(a))\right) \right|\\
      &= \left| Re\left( \left(Re(f(b)-f(a))\right)^{2} + \left( Im(f(b)-f(a))\right)^{2} \right) \right|\\
      &= \left| \left(Re(f(b)-f(a))\right)^{2} + \left( Im(f(b)-f(a))\right)^{2} \right|\\
      &= \left| f(b)-f(a) \right|^{2}\\
    \end{array}
    \]
    ... en dit:\prref{pr:reel-deel-kleiner}
    \[ 
    \begin{array}{rl}
      \left|F'(\lambda)\right|
      &= \left| Re\left( (\overline{f(b)}-\overline{f(a)})f'((1-\lambda)a+\lambda b)(b-a)\right) \right|\\
      &\le \left| (\overline{f(b)}-\overline{f(a)})f'((1-\lambda)a+\lambda b)(b-a) \right|\\
      &= \left| (\overline{f(b)}-\overline{f(a)})\right|\left|f'((1-\lambda)a+\lambda b)\right|\left|(b-a) \right|\\
      &= \left| (\overline{f(b)-f(a)})\right|\left|(b-a) \right|\left|f'((1-\lambda)a+\lambda b)\right|\\
      &= \left| f(b)-f(a) \right|\left|(b-a) \right|\left|f'((1-\lambda)a+\lambda b)\right|\\
      &\le \left| f(b)-f(a) \right|\left|(b-a) \right|\sup\left\{ \left|f'((1-\lambda)a+\lambda b) \right| \mid \lambda \in \interval{0}{1} \right\}\\
    \end{array}
    \]
    We bekomen de gevraagde ongelijkheid:
    \[
    \begin{array}{rl}
      \left| f(b)-f(a) \right|^{2}
      &\le \left| f(b)-f(a) \right|\left|(b-a) \right|\sup\left\{ \left|f'((1-\lambda)a+\lambda b) \right| \mid \lambda \in \interval{0}{1} \right\}\\
      \left| f(b)-f(a) \right|
      &\le \left|(b-a) \right|\sup\left\{ \left|f'((1-\lambda)a+\lambda b) \right| \mid \lambda \in \interval{0}{1} \right\}\\
      \frac{\left| f(b)-f(a) \right|}{\left|(b-a) \right|}
      &\le \sup\left\{ \left|f'((1-\lambda)a+\lambda b) \right| \mid \lambda \in \interval{0}{1} \right\}\\
    \end{array}
    \]
  \end{proof}
\end{bpr}

\TODO{definieer convergentie van functies in $\mathbb{C}$?}

\begin{st}
  Beschouw een rij $(f_{n})_{n}$ van afleidbare functies, gedefineerd op een convex, open deel $A$ van $\mathbb{C}$ met waarden in $\mathbb{C}$, die puntsgewijs convergeert naar een functie $f:\ A \rightarrow \mathbb{C}$.
  Stel bovendien dat $(f'_{n})_{n}$ uniform convergeert op $A$, dan geldt:
  \begin{itemize}
  \item $f$ is afleidbaar.
  \item $\forall a\in A:\ f'(a) = \lim_{n \rightarrow \infty}f'_{n}(a)$
  \end{itemize}

  \begin{proof}
    Kies een willekeurige $a \in A$ en noteer $L= \lim_{n\rightarrow +\infty}f'_{n}(a)$.
    We tonen het volgende aan, daarmee is ineens heel de stelling bewezen.
    \begin{itemize}
    \item Merk eerst het volgende op voor alle $x\in I$, verchillend van $a$ en alle $m\in \mathbb{N}$:
      \[
      \begin{array}{rl}
        \left|\frac{f(x)-f(a)}{x-a}-L\right|
        &\le \left| \frac{f(x)-f(a)}{x-a} - \frac{f_{m}(x)-f_{m}(a)}{x-a} \right|\\
        & +  \left| \frac{f_{m}(x)-f_{m}(a)}{x-a} -f'_{m}(a) \right|\\
        & +  \left| f'_{m}(a)-L \right|\\
      \end{array}
      \]
    \item Kies een willekeurigie $\epsilon \in \mathbb{R}_{0}^{+}$.
      Omdat $(f'_{n})_{n}$ uniforum convergeert op $A$ kunnen we een $n_{0}\in \mathbb{N}$ vinden als volgt:
      \[ \forall n,m\in \mathbb{N}:\ \forall y\in A:\ \left| f'_{n}(y)-f'_{m}(y) \right| < \frac{\epsilon}{3} \]
      We vinden zo de volgende afschatting voor alle $x\in A$, verschillend van $a$ en voor alle $n,m\in \mathbb{N}$:\prref{pr:lagrange-achtig-in-c}\question{hoe precies volgt deze afschatting uit deze stelling?!}
      \[ \left|\frac{f_{n}(x)-f_{n}(a)}{x-a} - \frac{f_{m}(x)-f_{m}(a)}{x-a}\right|
      \le \sup\left\{ \left|f_{n}'((1-\lambda)a+\lambda x) - f_{m}'((1-\lambda)a+\lambda x)\right| \mid \lambda \in \interval{0}{1} \right\}\]
      We weten bovendien het volgende:\question{hoe?!}
      \[ \sup\left\{ \left|f_{n}'((1-\lambda)a+\lambda x) - f_{m}'((1-\lambda)a+\lambda x)\right| \mid \lambda \in \interval{0}{1} \right\} \le \left| f'_{n}(y)-f'_{m}(y) \right| \]
      ... en vinden daarom dit:
      \[ \forall x\in A, \forall n,m \in \mathbb{N}:\ x \neq a \wedge n,m\ge n_{0} \Rightarrow \left|\frac{f_{n}(x)-f_{n}(a)}{x-a} - \frac{f_{m}(x)-f_{m}(a)}{x-a}\right| < \frac{\epsilon}{3} \]
      Nemen we hiervan de limiet van, voor $n$ gaande naar $+\infty$, dan vinden we het volgende:
      \waarom
      \[ \forall x\in A, \forall m \in \mathbb{N}:\ x \neq a \wedge m\ge n_{0} \Rightarrow \left|\frac{f(x)-f(a)}{x-a} - \frac{f_{m}(x)-f_{m}(a)}{x-a}\right| < \frac{\epsilon}{3} \]
      
    \item Omdat $\left(f'_{n}(a)\right)_{n}$ naar $L$ convergeert voor $n$ gaande naar $+\infty$ kunnen we een $n_{1}\in \mathbb{N}$ vinden zodat $|f'_{m}(a)-L|<\frac{\epsilon}{3}$ geldt.
    \item Omdat elke $f_{m}$ afleidbaar is in $a$, kunnen we een $m$ kiezen en dan een $\delta \in \mathbb{R}_{0}^{+}$ vinden als volgt:
      \[ \forall x\in A:\ 0 < |x-a| < \delta \Rightarrow \left| \frac{f_{m}(x)-f_{m}(a)}{x-a}-f'_{m}(a) \right| < \frac{\epsilon}{3} \]
    \item Uit de eerste opmerking volgt nu het volgende:
      \[ \forall x\in A:\ 0 < |x-a| < \delta \Rightarrow \left|\frac{f(x)-f(a)}{x-a}-L\right| < \frac{\epsilon}{3} + \frac{\epsilon}{3}  + \frac{\epsilon}{3} = \epsilon \]
    \end{itemize}
  \end{proof}
\extra{bewijs opnieuw bekijken en vragen stellen!}
\end{st}


\end{document}

%%% Local Variables:
%%% mode: latex
%%% TeX-master: t
%%% End:
