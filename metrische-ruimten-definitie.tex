\documentclass[main.tex]{subfiles}
\begin{document}



\chapter{Metrische ruimten}
\label{cha:metrische-ruimten}

\section{Het concept metrische ruimte}
\label{sec:het-conc-metr}

\begin{de}
  \label{de:metrische-ruimte}
  Een \term{Metrische ruimte} $V,d$ is een tupel van een verzameling $V$ en een functie $d$ met de volgende eigenschappen:
  \[ d:\ V \times V \rightarrow \mathbb{R}^{+}:\ (v,w) \mapsto d(v,w) \]
  \begin{itemize}
  \item $d$ is \textbf{symmetrisch}:
    \[ \forall v,w \in V:\ d(v,w) = d(w,v) \]
  \item $d$ is nul als en slechts als de argumenten gelijk zijn:
    \[ \forall v,w \in V:\ d(v,w) = 0 \Leftrightarrow v = w \]
  \item $d$ voldoet aan de \textbf{driehoeksongelijkheid}:
    \[ \forall u,v,w \in V:\ d(v,u) \le d(v,w) + d(w,u) \]
  \end{itemize}
  Men noemt $d$ de \term{metriek} of \term{afstandsfunctie} en $d(v,w)$ de \term{afstand} tussen $v$ en $w$.
\end{de}

\begin{st}
  Zij $X,d$ een metrische ruimte en $f:\ X \rightarrow Y$ een bijectie, dan definieert $d'$ als volgt een metriek op $Y$:
  \[ d':\ Y^{2}\rightarrow \mathbb{R}:\ d'(f(x_{1}),f(x_{2})) = d(x_{1},x_{2}) \]

  \extra{bewijs: evident?}
\end{st}

\begin{st}
  Zij $X,d$ een metrische ruimte, dan definieert $d'$ als volgt een metriek op $X$:
  \[ d':\ X^{2}\rightarrow \mathbb{R}:\ d'(x,y) = \min\{1,d(x,y)\} \]

  \begin{proof}
    De eerste twee eigenschappen zijn evident vanuit de metriek $d$.

    \begin{align*}
      & d'(x,y)+d'(y,z)\\
      & = \min\{1,d(x,y)\} + \min\{1,d(y,z)\}\\
      & = \min\{1+1,1+d(x,y),d(y,z)+1,d(x,y)+d(y,z)\} \\
      & \ge \min\{1,d(x,y)+d(y,z)\} \\
      & \ge  \min\{1,d(x,z)\}\\
      & = d'(x,z)
    \end{align*}
  \end{proof}
\end{st}

\begin{st}
  Zij $f:\ \mathbb{R}^{+} \rightarrow \mathbb{R}^{+}$ een stijgende concave functie zodat $f(t)$ nul is als en slechts als $t$ nul is.
  Zij $X,d$ een metrische ruimte, dan is $d_{f} = f\circ d$ een metriek op $X$.

  \begin{proof}
    We gaan alle eigenschappen na.
    \begin{itemize}
    \item $d_{f}$ is symmetrisch: evident.
    \item $d_{f}$ is nul als en slechts als de argumenten gelijk: aanname.
    \item $d_{f}$ voldoet aan de driehoeksongelijkheid:
      \[ f(d(x,z)) \le f(d(x,y) + d(y,z)) \]
      \TODO{en nu?}
    \end{itemize}
  \end{proof}
\end{st}


\section{Begrensde verzamelingen}
\label{sec:begr-verz}

\begin{de}
  \label{de:metrische-ruimte-begrensde-verzameling}
  Zij $V,d$ een metrische ruimte, dan noemen we een deelverzameling $W$ van $V$ \term{begrensd} als het volgende geldt:
  \[ \exists M\in \mathbb{R}^{+}, \forall v,w\in W:\ d(v,w) < M \]
\end{de}

\begin{de}
  Zij $V,d$ een metrische ruimte en $A$ een niet-lege begrensde deelverzameling van $V$, dan definieren we de \term{diameter} van $A$ als volgt:
  \[ diam(A) = \sup\{d(x,y) \mid x,y \in A \} \]
\end{de}

\begin{bpr}
  Zij $V,d$ een metrische ruimte en $A$ een niet-leeg deel van $X$, dan is $A$ begrensd als en slechts als er voor elke $x_{0}\in V$ een $R\in \mathbb{R}^{+}$ bestaat zodat $\forall a\in A:\ d(x_{0},a) \le R$ geldt.

  \begin{proof}
    Bewijs van een equivalentie.
    \begin{itemize}
    \item $\Rightarrow$\\
      Stel dat $A$ begrensd is, dan bestaat er een $M$ als volgt:
      \[ \forall x,a \in A:\ d(x,a) \le M \]
      Kies willekeurig een $x_{0}\in X$ en een $x\in A$.
      Noem nu $R=M + d(x_{0},x)$.
      Kies vervolgens een willekeurige $a\in A$.
      We vinden nu het volgende volgens de driehoeksongelijkheid:
      \[ d(x_{0},a) \le d(x_{0},x) + d(x,a) \le d(x_{0},x) + M = R \]
    \item $\Leftarrow$\\
      Kies willekeurig een $x_{0}\in X$.
      Kies willekeurig twee elementen $x$ en $a$ uit $A$.
      \[ d(x,a) \le d(x,x_{0}) + d(x_{0},a) \le R + R = 2R\]
      Kies dan eenvoudigweg $M=2R$.
    \end{itemize}
  \end{proof}
\end{bpr}

\section{Convergente rijen in metrische ruimten}
\label{sec:rijen-metr-ruimt}

\begin{de}
  Een rij $(x_{n})_{n}$ in een metrische ruimte $V,d$ noemen we \term{convergent} voor $d$ als er een $a\in V$ bestaat als volgt:
  \[ \forall \epsilon\in\mathbb{R}_{0}^{+}, \exists n_{0}\in \mathbb{N}, \forall n\in \mathbb{N}: n \ge n_{0} \Rightarrow d(x_{n},a) < \epsilon \]
  $a$ noemen we dan de \term{limiet} van de rij $(x_{n})_{n}$:
  \[ \lim_{n\rightarrow +\infty}x_{n} = a \]
\end{de}

\begin{bst}
  Zij $(x_{n})_{n}$ een convergente rij met limiet $x$ in een metrische ruimte $V,d$, dan is $x$ uniek.

  \begin{proof}
    Bewijs uit het ongerijmde: Stel dat er twee limieten $a$ en $b$ zijn van $(x_{n})_{n}$.\\
    Kies $\delta = \frac{d(a,b)}{2}$.
    Omdat $a$ de limiet is van $(x_{n})_{n}$, bestaat er een $n_{a}$ als volgt:
    \[ \forall n\in \mathbb{N}: n \ge n_{a} \Rightarrow d(x_{n},a) < \epsilon \]
    Omdat $b$ eveneens de limiet is van  $(x_{n})_{n}$, bestaat er ook een $n_{b}$ als volgt:
    \[ \forall n\in \mathbb{N}: n \ge n_{b} \Rightarrow d(x_{n},a) < \epsilon \]
    Merk nu voor alle $n\in \mathbb{N}$ de volgende ongelijkheid op die geldt vanuit de driehoeksongelijkheid:
    \[ d(a,b) \le d(a,x_{n}) + d(b,x_{n}) \]
    Kies $m = \max\{n_{a},n_{b}\}$, dan geldt voor alle $n\in \mathbb{N}$, groter dan $m$ de volgende contradictie.
    \[ 2\delta = d(a,b) \le d(a,x_{n}) + d(b,x_{n}) < \delta + \delta = 2\delta \]
  \end{proof}
\end{bst}
        
\begin{de}
  Zij $V,d$ een metrische ruimte, dan noemen we een rij $(x_{n})_{n}$ een \term{Cauchyrij} als en slechts als het volgende geldt:
  \[ \forall \epsilon, \exists n_{0} \in \mathbb{N}, \forall n,m\in \mathbb{N}:\ n,m\ge n_{0} \Rightarrow d(x_{n},x_{m}) < \epsilon \]
\end{de}


\begin{bst}
  \label{st:convergent-dan-cauchy}
  Zij $V,d$ een metrische ruimte, dan is elke onvergente rij hierin een Cauchyrij.

  \begin{proof}
    Zij $(x_{n})_{n}$ een convergente rij met limiet $x\in V$.
    Merk voor alle $n,m\in \mathbb{N}$ de volgende ongelijkheid op:
    \[ d\left(x_{n},x_{m}\right) \le d(x_{n},x) + d(x_{m},x) \]
    Kies willekeurig een $\delta \in \mathbb{R}_{0}^{+}$, dan bestaat er een $n_{0}\in \mathbb{N}$ als volgt:
    \[ \forall n\in \mathbb{N}:\ n \ge n_{0} \Rightarrow d(x_{n},x) < \frac{\delta}{2}\]
    Gebruiken we nu de vorige ongelijkheid, dan vinden we dat $(x_{n})_{n}$ een Cauchyrij is:
    \[ \forall n\in \mathbb{N}:\ n \ge n_{0} \Rightarrow d\left(x_{n},x_{m}\right) \le d(x_{n},x) + d(x_{m},x) < \frac{\delta}{2} + \frac{\delta}{2} = \delta\]
  \end{proof}
\end{bst}

\begin{de}
  \label{de:metrische-ruimte-volledig}
  Zij $V,d$ een metrische ruimte, dan noemen we deze \term{volledig} als elke Cauchyrij convergent is in $V$.
\end{de}

\begin{st}
  \label{st:metrische-ruimte-cauchy-dan-begrensd}
  In een metrische ruimte is elke Cauchyrij begrensd.

  \begin{proof}
    Zij $(x_{n})_{n}$ een Cauchyrij in een metrische ruimte $X,d$.
    Omdat $(x_{n})_{n}$ een Cauchyrij is, bestaat er een $m\in \mathbb{N}$ zodat $d(x_{k},k_{l}) < 1$ geldt voor alle $k,l \in \mathbb{N}$, groter dan $m$.
    In het bijzonder geldt dan ook $d(x_{m},x_{n} < 1$ voor alle $n\in \mathbb{N}$ groter dan $m$.
    Kies nu $\alpha = \max\{ d(x_{m},x_{n}) \mid n\in \mathbb{N}, n < m \}$.
    Merk op dat voor alle $n\in \mathbb{N}$ dan $d(x_{n},x_{m}) < \alpha + 1$ geldt.
    Kies nu twee willekeurige elementen $x_{a},x_{b}$ uit de rij ($a,b\in \mathbb{N}$) en kies $M=2\alpha+2$.
    \[ d(x_{a},x_{b}) \le d(x_{a},x_{m}) + d(x_{m},x_{b}) < 2(\alpha+1) = M \]
    De rij $(x_{n})_{n}$ is dus begrensd door $M$.
  \end{proof}
\end{st}

\begin{gev}
  In elke metrische ruimte is dus in het bijzonder elke convergente rij begrensd.\stref{st:convergent-dan-begrensd}
\end{gev}

\section{Deelrijen}
\label{sec:deelrijen}

\begin{de}
  Zij $(x_{n})_{n}$ een rij over een verzameling $V\subseteq X$ in een metrische ruimte en $(n_{k})_{k}$ een strikt stijgende rij over $\mathbb{N}$, dan is de rij $(x_{n_{k}})_{k}$ een \term{deelrij} van $(x_{n})_{n}$.
\end{de}

\begin{st}
  \label{st:deelrij-zelfde-limiet-als-convergente-moederrij}
  Zij $(x_{n})_{n}$ een convergente rij in een metrische ruimte $X,d$ en $(x_{n_{k}})_{k}$ een deelrij van $(x_{n})_{n}$, dan convergent $(x_{n_{k}})_{k}$ naar dezelfde limiet.
\TODO{bewijs}
\end{st}

\begin{st}
  \label{st:cauchy-asa-convergente-deelrij}
  In een metrische ruimte $X,d$ is een Cauchyrij $(x_{n})_{n}$ convergent als en slechts als ze een convergente deelrij heeft. 

  \begin{proof}
    Bewijs van een equivalentie.
    \begin{itemize}
    \item $\Rightarrow$:
      De rij $(x_{n})_{n}$ is een convergente deelrij van zichzelf.
    \item $\Leftarrow$\\
      Zij $(x_{n_{k}})_{k}$ een convergente deelrij van de Cauchyrij $(x_{n})_{n}$ en noem de limiet $x$.
      Kies $\epsilon \in \mathbb{R}_{0}^{+}$ willekeurig.
      Er bestaat dan een $k_{0} \in \mathbb{N}$ als volgt:
      \[ \forall k\in \mathbb{N}:\ k \ge k_{0} \Rightarrow d(x_{n_{k}},x) < \frac{\epsilon}{2} \]
      Omdat $(x_{n})_{n}$ een Cauchyrij is, bestaat er een $n_{0} \in \mathbb{N}$ als volgt:
      \[ \forall m,n\in \mathbb{N}:\ m,n \ge n_{0}:\ d(x_{m},x_{n}) < \frac{\epsilon}{2} \]
      Kies nu $m = \max\{k_{0},n_{0}\}$.
      \[ \forall n\in \mathbb{N}:\ n \ge m \Rightarrow d(x_{n},x) \le d(x_{n},x_{n_{n}}) + d(x_{n_{n}},x) < \frac{\epsilon}{2} + \frac{\epsilon}{2} = \epsilon \]
    \end{itemize}
  \end{proof}
\end{st}

\begin{st}
  \label{st:rij-met-eindig-aantal-waarden-convergente-deelrij}
  Zij $(x_{n})_{n}$ een rij die een eindig aantal waarden aanneemt, dan bevat die zeker een convergente deelrij.

  \begin{proof}
    Dat een rij $(x_{n})_{n}$ een eindig aantal waarden aanneemt, houdt in dat de volgende verzameling eindig is.
    \[ A = \{ x_{n} \mid n \in \mathbb{N} \} \]
    Opdat $A$ eindig zou kunnen zijn, moet er een waarde $x_{n}$ in zitten die oneindig veel voorkomt in de rij $(x_{n})_{n}$.
    Nummer die waarden (met $k$) zodat $\{ x_{n_{k}} \mid k \in \mathbb{N} \}$ een constante rij is.
    Deze is convergent.\needed
  \end{proof}
\end{st}

\section{Continue functies tussen metrische ruimten}
\label{sec:cont-funct-tuss}

\begin{de}
  Zij $V,d_{V}$ en $W,d_{W}$ twee metrische ruimtes en $f:\ A \subseteq V \rightarrow W$ een functie, dan noemen we $f$ \term{continu} in een punt $a\in A$ als het volgende geldt:
  \[ \forall \epsilon \in \mathbb{R}_{0}^{+}, \exists \delta \in \mathbb{R}_{0}^{+}, \forall b \in A:\ d_{V}(a,b)< \delta \Rightarrow d_{W}(f(a),f(b)) < \epsilon \]
\end{de}

\begin{de}
  Zij $V,d_{V}$ en $W,d_{W}$ twee metrische ruimtes en $f:\ A \subseteq V \rightarrow W$ een functie, dan noemen we $f$ \term{uniform continu} op $A$ als het volgende geldt:
  \[ \forall \epsilon \in \mathbb{R}_{0}^{+}, \exists \delta \in \mathbb{R}_{0}^{+}, \forall a,b \in A:\ d_{V}(a,b)< \delta \Rightarrow d_{W}(f(a),f(b)) < \epsilon \]
\end{de}

\subsection{Contracties}
\label{sec:contracties}

\begin{de}
  Zij $V,d_{V}$ en $W,d_{W}$ metrische ruimten, dan noemen we een funcie $f:\ X \rightarrow Y$ een \term{contractie} als het volgende geldt:
  \[ \forall x,y\in X:\ d_{Y}(f(x),f(y)) \le d_{X}(x,y) \]
\end{de}

\begin{de}
  Zij $V,d_{V}$ en $W,d_{W}$ metrische ruimten, dan noemen we een funcie $f:\ X \rightarrow Y$ een \term{strikte contractie} als er een $c\in \interval[open right]{0}{1}$ bestaat als volgt:
  \[ \forall x,y\in X:\ d_{Y}(f(x),f(y)) \le c d_{X}(x,y) \]
  De kleinste $c$ die hieraan voldoet noemen we de \term{contractiefactor}
\end{de}

\begin{st}
  Contracties zijn uniform continu.

  \begin{proof}
    Zij $V,d_{V}$ en $W,d_{W}$ metrische ruimten, en $f:\ X \rightarrow Y$ een contractie.
    Kies willekeurig een $\epsilon\in\mathbb{R}_{0}^{+}$ en kies $\delta = \epsilon$.
    Kies willekeurig twee elementen $a,b\in V$:
    \[ d_{V}(a,b) < \epsilon \Rightarrow d_{W}(f(a),f(b)) \le d_{V}(a,b) < \epsilon \]
  \end{proof}
\end{st}

\subsection{Isometrie\"en}
\label{sec:isometrieen}

\begin{de}
  Zij $V,d_{V}$ en $W,d_{W}$ metrische ruimten, dan noemen we een funcie $f:\ X \rightarrow Y$ een \term{isometrie} als $f$ afstanden bewaart:
  \[ \forall x,y \in X: d_{W}(f(x),f(y)) = d_{V}(x,y) \]
\end{de}

\begin{st}
  \label{st:isometrie-dan-injectie}
  Elke isometrie is injectief.

  \begin{proof}
    Kies twee elementen $x$ en $y$ uit $V$ met hetzelfde beeld $z\in W$ omdat $d(z,z)$ nul is, en omdat $f$ een isometrie is, moet $d(x,y)$ nul zijn en $x$ en $y$ dus gelijk zijn.
  \end{proof}
\end{st}

\begin{de}
  Zij $V,d_{V}$ en $W,d_{W}$ metrische ruimten, dan zeggen we dat $V,d_{V}$ \term{isometrisch ingebed} kan worden in $W,d_{W}$ als er een isometrie $f:\ X \rightarrow Y$ bestaat.
\end{de}

\begin{de}
  Twee metrische ruimten $V,d_{V}$ en $W,d_{W}$ noemen we \term{isometrisch} als en slechts als er een bijectieve isometrie $f:\ X \rightarrow Y$ bestaat.
\end{de}

\TODO{zoek hiervan de voorbeelden hierboven}

\extra{voorbeeld p 23}


\section{Enkele resultaten}
\label{sec:enkele-resultaten}
\TODO{staan deze op hun plaats?}

\begin{st}
  Beschouw een metrische ruimte $X,d$ en $A$ een deelverzameling van $X$.
  Definieer de functie $d_{A}$ als volgt:
  \[ d_{A}:\ X \rightarrow \mathbb{R}^{+}:\ d_{A}(x) = \inf\{d(x,a) \mid a\in A\} \]
  Beschouw nu begrensde verzamelingen $A$ en $B$.
  \begin{enumerate}
  \item $d_{A\cup B}(x) = \min\{d_{A}(x),d_{B}(x)\}$
  \item $d_{A\cap B}(x) \ge \max\{d_{A}(x),d_{B}(x)\}$
  \end{enumerate}

  \begin{proof}
    \noindent
    Noem $d(x,a') = \{d(x,a) \mid a\in A\}$ en $d(x,b') = \{d(x,b) \mid b\in B\}$
    \begin{enumerate}
    \item Het infimum van $\inf\{d(x,a) \mid a\in A\cup B\}$ is ofwel $d(x,a')$ $d(x,b')$ en in het bijzonder het infimum van de twee, dus het minimum ervan.
      \question{wat valt er hier meer over te zeggen?}
    \item Het infimum van $\{d(x,a) \mid a\in A\cap B\}$ is $d(x,c)$ met $c\in A\cap B$ en $c$ \'e\'en van de $a'$ en $b'$ en dan is het inderdaad het maximum van $d_{A}(x)$ en $d_{B}(x)$.
      \question{dit klinkt niet juist, wat mist er?}
    \end{enumerate}
  \end{proof}
\end{st}

\begin{st}
  De functie $d_{A}$ zoals hierboven beschreven is continu.
  
  \begin{proof}
    Beschouw eerst volgende ongelijkheden voor willekeurige $x,y\in X$:
    \[ d_{A}(x) \le d(x,y) + d_{A}(x) \text{ en } d_{A}(y) \le d(x,y) + d_{A}(x) \]
    Hieruit volgt meteen deze:
    \[ d(x,y) \ge |d_{A}(x)-d_{A}(y)| \]
    Kies willekeurig een $x\in X$, $\epsilon \in \mathbb{R}_{0}^{+}$ en kies $\delta = \epsilon$, dan volgt uit bovenstaande ongelijkheid meteen continu\"iteit.
    \[ \forall y\in X:\ d(x,y) < \delta \Rightarrow |d_{A}(x)-d_{A}(y)| \le d(x,y) < \delta = \epsilon \]
    \feed
  \end{proof}
\end{st}

\section{Metrische Productruimten}
\label{sec:metrische-productruimten}

\begin{de}
  Zij $V,d_{V}$ en $W,d_{W}$ twee metrische ruimten. en $d_{V,W}$ een metriek op $V\times W$, dan noemen we $V\times W, d_{V,W}$ de productruimte van $V,d_{V}$ en $W,d_{W}$.
\end{de}

\begin{st}
  Metrieken zijn continu wanneer we als metriek op het het maximum van de deelmetrieken beschouwen.
  
  \begin{proof}
    Zij $X,d$ een metrische ruimte.
    Op de productruimte $X^{2}$ beschouwen we de volgende metriek:
    \[ d_{\max}:\ X\times X \rightarrow \mathbb{R}^{+}:\ ((x_{1},x_{2}),(y_{1},y_{2})) \mapsto \max\{ d(x_{1},y_{1}), d(x_{2},y_{2})\} \]
  \end{proof}
\end{st}

\begin{gev}
  \label{gev:metriek-door-limiet}
  We mogen metrieken door limieten halen.
\TODO{bewijs? is dit duidelijk genoeg? misschien nog een algemenere stelling formuleren.}
\end{gev}


\section{Genormeerde vectorruimten}
\label{sec:genorm-vect}

\begin{de}
  \label{de:norm}
  Zij $(\mathbb{R},V,+)$ een vectorruimte, dan is een \term{norm} op $V$ een afbeelding als volgt:
  \[ \|\cdot\|:\ V \rightarrow \mathbb{R}^{+}:\ v \mapsto \|v\| \]
  \begin{enumerate}
  \item $\forall v\in V:\ \|v\| = 0 \Leftrightarrow v=0$
  \item $\forall v\in V, \forall \lambda \in \mathbb{R}:\ \|\lambda v\| = |\lambda|\|v\|$
  \item $\forall v,w\in V:\ \|v+w\| \le \|v\| + \|w\|$
  \end{enumerate}
\end{de}

\begin{de}
  Een vectorruimte $\mathbb{R},V,+$, uitgerust met een norm noemen we een \term{genormeerde vectorruimte}.
\end{de}

\begin{st}
  Een genormeerde vectorruimte $\mathbb{R},V,+,\|\cdot\|$, uitgerust met de volgende functie als metriek, vormt een metrische ruimte:
  \[ d:\ V \times V \rightarrow \mathbb{R}^{+}:\ (x,y) \mapsto \|x-y\| \]

  \begin{proof}
    We gaan de eigenschappen van een metrische ruimte na.
    \begin{itemize}
    \item $d$ is symmetrisch: $\|x-y\| = \|(-1)(y-x)\| = |-1|\|y-x\| = \|y-x\|$
    \item $d$ is nul als en slechts als de argumenten gelijk zijn: $\|x-x\| = \|0\| = 0$
    \item $d$ voldoet aan de driehoeksongelijkheid: $\| x-z \| = \| (x-y)+(y-z) \| \le \|x-y\| + \| y-z \|$
    \end{itemize}
  \end{proof}
\end{st}

\begin{st}
  Equivalente definitie voor begrensde delen in een genormeerde vectorruimte:
  Een deel $A$ van een genormeerde vectorruimte $\mathbb{R},V,+$ is begrensd als en slechts als er een $M \in \mathbb{R}^{+}$ bestaat als volgt:
  \[ \forall x\in A:\ \|x\| \le M \]

  \begin{proof}
    Bewijs van een equivalentie.
    \begin{itemize}
    \item $\Rightarrow$\\
      Omdat $A$ begrensd is, bestaat er een $N\in \mathbb{R}^{+}$ als volgt:\deref{de:metrische-ruimte-begrensde-verzameling}
      \[ \forall v,w\in A:\ \|v-w\| < N \]
      Kies nu een willekeurige $a\in A$, dan geldt er voor elke $x\in A$ het volgende:
      \[ \|x\| = \|x+a-a\| \le \|a\| + \|x-a\| < \|a\| + N \]
      Kies nu $M = \|a\|+N$, dan is $A$ bij norm begrensd door $M\in \mathbb{R}^{+}$.
    \item $\Leftarrow$\\
      Kies twee willekeurige elementen $x,y\in A$, dan geldt het volgende:
      \[ \|x-y\| \le \|x\| + \|y\| < 2M \]
      $A$ is dus bij metriek begrensd door $N=2M \in \mathbb{R}^{+}$.
    \end{itemize}
  \end{proof}
\end{st}

\begin{st}
  Zij $B(x,\delta)$ een open bol in een genormeerde vectorruimte $\mathbb{R},V,+,\|\cdot\|$, dan is $B(x,\delta)$ een translatie van een open bol rond $0$:
  \[ B(x,\delta) = \left\{ t+x \mid t \in B(0,\delta) \right\} \]
\extra{bewijs}
\end{st}

\end{document}

%%% Local Variables:
%%% mode: latex
%%% TeX-master: t
%%% End:
