\documentclass[main.tex]{subfiles}
\begin{document}

\chapter{Voorbeelden}
\label{cha:voorbeelden}

\extra{voorbeelden van velden: $\mathbb{F}_{2}$.}

\section{Getallen}

\subsection{Intervallen}
\begin{vb}
  $\{x\in \mathbb{R} \mid i \le x \le 2\}$ is een interval.
\end{vb}

\begin{vb}
  $\{1,2\}$ is geen interval want $1 \le \nicefrac{3}{2} \le 2$ geldt maar $\nicefrac{3}{2}$ zit niet in $\{1,2\}$.  
\end{vb}

\begin{vb}
  $\{x\in \mathbb{R} \mid x > 5\}$ is een interval.
\end{vb}

\begin{vb}
  $\mathbb{R}_{0}$ is geen interval.
\end{vb}

\section{Continu\"iteit in $\mathbb{R}$}
\label{sec:continuiteit-mathbbr}

\begin{vb}
  De functie $f$ is continu:
  \[ f:\ \mathbb{R} \rightarrow \mathbb{R}:\ x \mapsto x^{2} \]

  \begin{proof}
    Kies een willekeurige $a \in \mathbb{R}$.
    Voor elke $x\in \mathbb{R}$ geldt het volgende:
    \[ |f(x)-f(a)| = |x^{2}-a^{2}| = |x-a||x+a| \le |x-a|(|x|+|a|) \]
    Voor $|x-a|$ kleiner dan $1$ zal $|x| \le |a|+1$ gelden.
    \[ |f(x)-f(a)| \le (2|a|+1)|x-a| \]
    Kies een willekeurige $\epsilon \in \mathbb{R}_{0}^{+}$ en kies $\delta = \min\left\{ 1, \frac{\epsilon}{2|a|+1} \right\}$.
    Uit $|x-a|< \delta$ volgt nu dat $f$ continu is in $a$.
    \[ |f(x)-f(a)| < (2|a|+1)\delta \le (2|a|+1)\frac{\epsilon}{2|a|+1} = \epsilon \]
  \end{proof}
\end{vb}

\begin{vb}
  De functie $f$ is continu:
  \[ f:\ \mathbb{R} \rightarrow \mathbb{R}:\ x \mapsto \frac{1}{x} \]
  
  \begin{proof}
    Kies een willekeurige $a \in \mathbb{R}$.
    Voor elke $x \in \mathbb{R}$ geldt het volgende:
    \[ |f(x)-f(a)| = \left|\frac{1}{x}-\frac{1}{a}\right| = \frac{|x-a|}{|x||a|} \]
    Voor $|x-a|$ kleiner dan $\frac{1}{2}|a|$ zal $|x| > \frac{1}{2}|a|$ gelden.
    \[ |f(x)-f(a)| \le \frac{2}{|a|^{2}}|x-a| \]
    Kies nu een willekeurige $\epsilon \in \mathbb{R}_{0}^{+}$ en kies $\delta = \min\left\{ \frac{1}{2}|a|, \frac{\epsilon|a|^{2}}{2} \right\}$.
    Uit $|x-a| < \delta$ volgt nu dat $f$ continu is in $a$.
    \[ |f(x)-f(a)| < \frac{2}{|a|^{2}}\delta \le \frac{2}{|a|^{2}}\frac{\epsilon|a|^{2}}{2} = \epsilon \]
  \end{proof}
\end{vb}


\end{document}
