\documentclass[main.tex]{subfiles}
\begin{document}

\chapter{Voorbeelden}
\label{cha:voorbeelden}

\extra{voorbeelden van velden: $\mathbb{F}_{2}$.}

\section{Getallen}

\subsection{Intervallen}
\begin{vb}
  $\{x\in \mathbb{R} \mid i \le x \le 2\}$ is een interval.
\end{vb}

\begin{vb}
  $\{1,2\}$ is geen interval want $1 \le \nicefrac{3}{2} \le 2$ geldt maar $\nicefrac{3}{2}$ zit niet in $\{1,2\}$.  
\end{vb}

\begin{vb}
  $\{x\in \mathbb{R} \mid x > 5\}$ is een interval.
\end{vb}

\begin{vb}
  $\mathbb{R}_{0}$ is geen interval.
\end{vb}

\section{Continu\"iteit in $\mathbb{R}$}
\label{sec:continuiteit-mathbbr}

\subsection{Het continu\"iteitsbegrip}
\label{sec:het-cont}

\begin{vb}
  De functie $f$ is continu:
  \[ f:\ \mathbb{R} \rightarrow \mathbb{R}:\ x \mapsto x^{2} \]

  \begin{proof}
    Kies een willekeurige $a \in \mathbb{R}$.
    Voor elke $x\in \mathbb{R}$ geldt het volgende:
    \[ |f(x)-f(a)| = |x^{2}-a^{2}| = |x-a||x+a| \le |x-a|(|x|+|a|) \]
    Voor $|x-a|$ kleiner dan $1$ zal $|x| \le |a|+1$ gelden.
    \[ |f(x)-f(a)| \le (2|a|+1)|x-a| \]
    Kies een willekeurige $\epsilon \in \mathbb{R}_{0}^{+}$ en kies $\delta = \min\left\{ 1, \frac{\epsilon}{2|a|+1} \right\}$.
    Uit $|x-a|< \delta$ volgt nu het volgende.
    \[ |f(x)-f(a)| < (2|a|+1)\delta \le (2|a|+1)\frac{\epsilon}{2|a|+1} = \epsilon \]
    $f$ is dus continu in $a$.
  \end{proof}
\end{vb}

\begin{vb}
  De functie $f$ is continu:
  \[ f:\ \mathbb{R} \rightarrow \mathbb{R}:\ x \mapsto \frac{1}{x} \]
  
  \begin{proof}
    Kies een willekeurige $a \in \mathbb{R}$.
    Voor elke $x \in \mathbb{R}$ geldt het volgende:
    \[ |f(x)-f(a)| = \left|\frac{1}{x}-\frac{1}{a}\right| = \frac{|x-a|}{|x||a|} \]
    Voor $|x-a|$ kleiner dan $\frac{1}{2}|a|$ zal $|x| > \frac{1}{2}|a|$ gelden.
    \[ |f(x)-f(a)| \le \frac{2}{|a|^{2}}|x-a| \]
    Kies nu een willekeurige $\epsilon \in \mathbb{R}_{0}^{+}$ en kies $\delta = \min\left\{ \frac{1}{2}|a|, \frac{\epsilon|a|^{2}}{2} \right\}$.
    Uit $|x-a|< \delta$ volgt nu het volgende.
    \[ |f(x)-f(a)| < \frac{2}{|a|^{2}}\delta \le \frac{2}{|a|^{2}}\frac{\epsilon|a|^{2}}{2} = \epsilon \]
    $f$ is dus continu in $a$.
  \end{proof}
\end{vb}

\begin{vb}
  De functie $f$ is continu:
  \[ f:\ \mathbb{R}^{+} \rightarrow \mathbb{R}:\ x \mapsto \sqrt{x} \]

  \begin{proof}
    Kies een willekeurige $a \in \mathbb{R}$.
    Voor elke $x \in \mathbb{R}$ geldt het volgende:
    \[ |f(x)-f(a)| = |\sqrt{x}-\sqrt{a}| = \left|\sqrt{x}-\sqrt{a}\right|\frac{\sqrt{x}+\sqrt{a}}{\sqrt{x}+\sqrt{a}} = \frac{|x-a|}{\sqrt{x}+\sqrt{a}} \]
    \begin{itemize}
    \item Voor $a>0$ kunnen we als volgt verder gaan :
      \[ \frac{|x-a|}{\sqrt{x}+\sqrt{a}} \le \frac{|x-a|}{\sqrt{a}} \]
      Voor een willekeurige $\epsilon$ kunnen we nu voor $\delta$ $\epsilon\sqrt{a}$ kiezen.
      Uit $|x-a|< \delta$ volgt dan het volgende:
      \[ |f(x)-f(a)| = \frac{|x-a|}{\sqrt{x}+\sqrt{a}} \le \frac{|x-a|}{\sqrt{a}} < \frac{\delta}{\sqrt{a}} = \frac{\epsilon\sqrt{a}}{\sqrt{a}} = \epsilon \]
    \item Voor $a=0$ is het eenvoudiger:
      \[ |f(x)-f(a)| =  \frac{|x-a|}{\sqrt{x}+\sqrt{a}} = \frac{|x|}{\sqrt{x}} \]
      Voor een willekeurige $\epsilon$ kunnen we nu voor $\delta$ $\epsilon^{2}$ kiezen. 
      Uit $|x-a|=|x|< \delta$ volgt dan het volgende:
      \[ \frac{|x|}{\sqrt{x}} < \frac{\delta}{\sqrt{\delta}} = \frac{\epsilon^{2}}{\epsilon} = \epsilon \]
    \end{itemize}
    $f$ is dus continu in $a$.
  \end{proof}
\end{vb}

\extra{vb p 4 onderaan}

\begin{tvb}
  De functie $f$ is niet continu in $0$:
  \[
  f:\ \mathbb{R} \rightarrow \mathbb{R}:\ x\mapsto 
  \left\{
    \begin{array}{cl}
      1 & \text{ als } x \ge 0\\
      0 & \text{ als } x < 0
    \end{array}
  \right.
  \]

  \begin{proof}
    Kies $\epsilon = 1$, dan bestaat er voor elke $\delta$ een $x\in \mathbb{R}$ zodat $|x-0|<\delta$ en $|f(x)-f(0)|\ge \epsilon$ beide gelden.
    Voor een willekeurige $\delta$ kiezen we $x = -\frac{\delta}{2}$:
    \[ |x-0| = \frac{\delta}{2} \text{ en } |f(x)-f(0)| = 1 \ge \epsilon \]
    $f$ is dus niet continu in $0$.
  \end{proof}
\end{tvb}

\extra{tvb p 5 onderaan}

\begin{vb}
  De functie $f$ is continu:
  \[ f:\ \mathbb{R} \rightarrow \mathbb{R}:\ x \mapsto |x| \]

  \begin{proof}
    Kies een willekeurige $a\in \mathbb{R}$:
    Beschouw eerst $|f(x)-f(a)|$:
    \[ |f(x)-f(a)| = \left||x|-|a|\right| \le |x-a| \]
    Kies nu een willekeurige $\epsilon$ en kies $\delta = \epsilon$.
    Uit $|x-a| <\delta$ volgt nu het volgende:
    \[ |f(x)-f(a)| \le |x-a| < \delta = \epsilon \]
    $f$ is dus continu in $a$.
  \end{proof}
\end{vb}

\subsection{Operaties met continue functies}
\label{sec:oper-met-cont}

\extra{voorbeelden}


\subsection{Continue functies op intervallen}
\label{sec:continue-functies-op}

\extra{voorbeelden}

\subsection{Uniforme continu\"iteit}
\label{sec:unif-cont}

\extra{voorbeeld 4.3.1}

\begin{vb}
  De functie $f$ is uniform continu:
  \[ f:\ \mathbb{R}^{+} \rightarrow \mathbb{R}:\ x \mapsto \sqrt{x} \]

  \begin{proof}
    Merk allereerst de volgende ongelijkheid op:
    \[ |\sqrt{x} - \sqrt{y}| \le \sqrt{|x-y|} \]
    Kies een willekeurige $\epsilon$ en definieer $\delta = \epsilon^{2}$.
    Kies nu twee willekeurige elementen $x$ en $y$ uit $\mathbb{R}^{+}$ zodat $|x-y|< \delta$ geldt.
    \[ |f(x) - f(y)| \le  \sqrt{|x-y|} < \sqrt{\delta} = \sqrt{\epsilon^{2}} = \epsilon \]
    $f$ is dus uniform continu over $\mathbb{R}^{+}$.
  \end{proof}
\end{vb}

\begin{tvb}
  De functie $f$ is niet uniform continu:
  \[ f:\ \mathbb{R}_{0}^{+} \rightarrow \mathbb{R}:\ x \mapsto \frac{1}{x} \]

  \begin{proof}
    Kies $\epsilon = 1$ en kies een willekeurige $\delta \in \mathbb{R}_{0}^{+}$.
    Stel bovendien $y = x + \frac{\delta}{2}$, dan is $|x-y|$ altijd kleiner dan $\delta$.
    Merk bovendien de volgende gelijkheid op:
    \[ |f(x)-f(y)| = \left| \frac{1}{x} - \frac{1}{y} \right| \le \left| \frac{1}{x} - \frac{1}{x + \frac{\delta}{2}} \right| = \left| \frac{x+\frac{\delta}{2}-x}{x(x+\frac{\delta}{2})}\right| = \left| \frac{\delta}{x(2x+\delta)} \right| = \frac{\delta}{x(2x+\delta)} \]
    Kies bijvoorbeeld $x = \frac{\delta}{4}\left(\sqrt{1+\frac{\delta}{4}}-1\right)$ (de oplossing van $\frac{1}{2}(2x^{2}+\delta x) = \delta$).
    $x$ is dan positief en $|f(x)-f(y)| = 2 > \epsilon$.
    $f$ is dus niet uniform continu op $\mathbb{R}_{0}^{+}$.
  \end{proof}
\end{tvb}

\extra{tvb 4.3.4}

\end{document}
