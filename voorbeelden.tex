\documentclass[main.tex]{subfiles}
\begin{document}

\chapter{Voorbeelden}
\label{cha:voorbeelden}

\extra{voorbeelden van velden: $\mathbb{F}_{2}$.}

\section{Getallen}

\subsection{Intervallen}
\begin{vb}
  $\{x\in \mathbb{R} \mid i \le x \le 2\}$ is een interval.
\end{vb}

\begin{vb}
  $\{1,2\}$ is geen interval want $1 \le \nicefrac{3}{2} \le 2$ geldt maar $\nicefrac{3}{2}$ zit niet in $\{1,2\}$.  
\end{vb}

\begin{vb}
  $\{x\in \mathbb{R} \mid x > 5\}$ is een interval.
\end{vb}

\begin{vb}
  $\mathbb{R}_{0}$ is geen interval.
\end{vb}

\section{Rijen}


\begin{vb}
  Zij $x_{0}=1$ en $x_{n+1} = \frac{1}{1+x_{n}}$.
  \extra{bewijs dat $(x_{n})$ naar $\frac{2}{1+\sqrt{5}}$ convergeert}
\end{vb}

\begin{vb}
  Zij $A = \interval[open left]{0}{1} \cup \{2\} $.
  \begin{itemize}
  \item De grootste open deelverzameling van $A$ is $\interval[open]{0}{1}$.
  \item De grootste gesloten deelverzameling van $A$ bestaat niet. 
  \item De kleinste open oververzameling van $A$ bestaat niet
  \item De kleinste gesloten oververzameling van $A$ is $\interval{0}{2}$.
  \end{itemize}
\feed
\end{vb}

\begin{vb}
  \begin{itemize}
  \item De grootste open deelverzameling van $\mathbb{Q}$ is $\emptyset$.
  \item De grootste gesloten deelverzameling van $\mathbb{Q}$ is $\emptyset$.
  \item De kleinste open oververzameling van $\mathbb{Q}$ is $\mathbb{R}$.
  \item De kleinste gesloten oververzameling van $\mathbb{Q}$ is $\mathbb{R}$.
  \end{itemize}
\feed
\end{vb}

\subsection{Topologie in $\mathbb{R}$}

\begin{vb}
  Zij $A= \interval[open left]{0}{1}$.
  \begin{itemize}
  \item $\mathring{A} = \interval[open]{0}{1}$.
  \item $\overline{A} = \interval{0}{1} \cup \{2\}$
  \item $\partial{A} = \{0\} \cup \{1\} \cup \{2\}$
  \item De ge\"isoleerde punten van $A$ zijn $\{2\}$.
  \item De ophopingspunten van $A$ zijn $\interval{0}{1}$.
  \end{itemize}
\end{vb}
\extra{zelfde voorbeeld voor $\{\frac{1}{n} \mid n \in \mathbb{N}_{0} \}$}
\extra{zelfde voorbeeld voor $\{\frac{1}{n} \mid n \in \mathbb{N}_{0} \} \cup 0$}
\extra{zelfde voorbeeld voor $\mathbb{N}$}
\extra{zelfde voorbeeld voor $\mathbb{Q}$}

\subsection{Relatieve topologie in $\mathbb{R}$}

\begin{vb}
  Beschouw $\interval[open right]{0}{1}$ als deelverzameling van $\interval[open right]{0}{2}$.
  \begin{itemize}
  \item $\interval[open right]{0}{1}$ is relatief open in $\interval[open right]{0}{2}$.
    $\interval[open right]{0}{1}$ is niet open in $\mathbb{R}$ omdat er geen punten links van $0$ in liggen, maar in $\interval[open right]{0}{2}$ liggen ook geen punten links van $0$, dus dat is geen probleem.
  \item $\interval[open right]{1}{2}$ is dan relatief gesloten.
  \end{itemize}
\end{vb}




\section{Continu\"iteit in $\mathbb{R}$}

\subsection{Het continu\"iteitsbegrip}

\begin{vb}
  De functie $f$ is continu:
  \[ f:\ \mathbb{R} \rightarrow \mathbb{R}:\ x \mapsto x^{2} \]

  \begin{proof}
    Kies een willekeurige $a \in \mathbb{R}$.
    Voor elke $x\in \mathbb{R}$ geldt het volgende:
    \[ |f(x)-f(a)| = |x^{2}-a^{2}| = |x-a||x+a| \le |x-a|(|x|+|a|) \]
    Voor $|x-a|$ kleiner dan $1$ zal $|x| \le |a|+1$ gelden.
    \[ |f(x)-f(a)| \le (2|a|+1)|x-a| \]
    Kies een willekeurige $\epsilon \in \mathbb{R}_{0}^{+}$ en kies $\delta = \min\left\{ 1, \frac{\epsilon}{2|a|+1} \right\}$.
    Uit $|x-a|< \delta$ volgt nu het volgende.
    \[ |f(x)-f(a)| < (2|a|+1)\delta \le (2|a|+1)\frac{\epsilon}{2|a|+1} = \epsilon \]
    $f$ is dus continu in $a$.
  \end{proof}
\end{vb}

\begin{vb}
  De functie $f$ is continu:
  \[ f:\ \mathbb{R}_{0} \rightarrow \mathbb{R}:\ x \mapsto \frac{1}{x} \]
  
  \begin{proof}
    Kies een willekeurige $a \in \mathbb{R}_{0}$.
    Voor elke $x \in \mathbb{R}_{0}$ geldt het volgende:
    \[ |f(x)-f(a)| = \left|\frac{1}{x}-\frac{1}{a}\right| = \frac{|x-a|}{|x||a|} \]
    Voor $|x-a|$ kleiner dan $\frac{1}{2}|a|$ zal $|x| > \frac{1}{2}|a|$ gelden.
    \[ |f(x)-f(a)| \le \frac{2}{|a|^{2}}|x-a| \]
    Kies nu een willekeurige $\epsilon \in \mathbb{R}_{0}^{+}$ en kies $\delta = \min\left\{ \frac{1}{2}|a|, \frac{\epsilon|a|^{2}}{2} \right\}$.
    Uit $|x-a|< \delta$ volgt nu het volgende.
    \[ |f(x)-f(a)| < \frac{2}{|a|^{2}}\delta \le \frac{2}{|a|^{2}}\frac{\epsilon|a|^{2}}{2} = \epsilon \]
    $f$ is dus continu in elke $a\in \mathbb{R}_{0}$.
  \end{proof}
\end{vb}

\begin{vb}
  De functie $f$ is continu:
  \[ f:\ \mathbb{R} \setminus \{-1\} \rightarrow \mathbb{R}:\ x \mapsto \frac{1}{1+x} \]

  \begin{proof}
    Kies een willekeurige $a\in \mathbb{R}\setminus \{-1\}$.
    Voor elke $x\in \mathbb{R}\setminus \{-1\}$ geldt het volgende:
    \[ |f(x)-f(a)| = \left| \frac{1}{1+x}-\frac{1}{1+a} \right| = \left| \frac{(1+a)-(1+x)}{(1+x)(1+a)} \right|  = \frac{|x-a|}{|1+x||1+a|} \]
    Voor $|x-a|$ kleiner dan $|1+a|$ geldt nu het volgende:
    \[ |1+x| = |1+a-a+x| = |1+a+x-a| \le |1+a|+|x-a| \le 2|1+a| \]
    \[ |f(x)-f(a)| \le  \frac{1}{3|1+a|^{2}}|x-a| \]
    Kies nu een willekeurige $\epsilon \in \mathbb{R}_{0}^{+}$ en kies $\delta = \min\left\{ |1+a|,3\epsilon|1+a|^{2}\right\}$
    Voor $|x-a| < \delta$ geldt dan het volgende:
    \[ \frac{1}{3|1+a|^{2}}|x-a| < \frac{1}{3|1+a|^{2}}\delta \le \frac{3\epsilon|1+a|^{2}}{3|1+a|^{2}} = \epsilon\]
    $f$ is dus continu in elke $a \in \mathbb{R}\setminus \{-1\}$
  \end{proof}
\end{vb}

\begin{vb}
  De functie $f$ is continu:
  \[ f:\ \mathbb{R}^{+} \rightarrow \mathbb{R}:\ x \mapsto \sqrt{x} \]

  \begin{proof}
    Kies een willekeurige $a \in \mathbb{R}$.
    Voor elke $x \in \mathbb{R}$ geldt het volgende:
    \[ |f(x)-f(a)| = |\sqrt{x}-\sqrt{a}| = \left|\sqrt{x}-\sqrt{a}\right|\frac{\sqrt{x}+\sqrt{a}}{\sqrt{x}+\sqrt{a}} = \frac{|x-a|}{\sqrt{x}+\sqrt{a}} \]
    \begin{itemize}
    \item Voor $a>0$ kunnen we als volgt verder gaan :
      \[ \frac{|x-a|}{\sqrt{x}+\sqrt{a}} \le \frac{|x-a|}{\sqrt{a}} \]
      Voor een willekeurige $\epsilon$ kunnen we nu voor $\delta$ $\epsilon\sqrt{a}$ kiezen.
      Uit $|x-a|< \delta$ volgt dan het volgende:
      \[ |f(x)-f(a)| = \frac{|x-a|}{\sqrt{x}+\sqrt{a}} \le \frac{|x-a|}{\sqrt{a}} < \frac{\delta}{\sqrt{a}} = \frac{\epsilon\sqrt{a}}{\sqrt{a}} = \epsilon \]
    \item Voor $a=0$ is het eenvoudiger:
      \[ |f(x)-f(a)| =  \frac{|x-a|}{\sqrt{x}+\sqrt{a}} = \frac{|x|}{\sqrt{x}} \]
      Voor een willekeurige $\epsilon$ kunnen we nu voor $\delta$ $\epsilon^{2}$ kiezen. 
      Uit $|x-a|=|x|< \delta$ volgt dan het volgende:
      \[ \frac{|x|}{\sqrt{x}} < \frac{\delta}{\sqrt{\delta}} = \frac{\epsilon^{2}}{\epsilon} = \epsilon \]
    \end{itemize}
    $f$ is dus continu in $a$.
  \end{proof}
\end{vb}

\begin{vb}
  De functie $f$ is continu:
  \[ f:\ \mathbb{R} \rightarrow \mathbb{R}:\ x \mapsto \frac{1}{(1+x^{2})} \]

  \begin{proof}
    Kies een willekeurige $a\in \mathbb{R}$.
    Er geldt dan voor alle $x \in \mathbb{R}$ het volgende:
    \[
    |f(x)-f(a)|
    = \left| \frac{1}{1+x^{2}} - \frac{1}{1+a^{2}}\right|
    = \left| \frac{(1+a^{2})-(1+x^{2})}{(1+x^{2})(1+a^{2})}\right|
    = \frac{|a^{2}-x^{2}|}{|1+x^{2}||1+a^{2}|}
    = \frac{|a+x||a-x|}{|1+x^{2}||1+a^{2}|}
    \]
    Voor $|x-a| \le \frac{1}{2}|a|$ geldt $x \le \frac{3}{2}|a|$, $x \ge \frac{1}{2}|a|$ en dus het volgende:
    \[ |x+a| \ge ||x|-|a|| \ge |\frac{1}{2}|a| - |a|| = |-\frac{1}{2}|a|| = \frac{1}{2}|a| \]
    ,
    \[ |x+a| \le |x|+|a| \le \frac{3}{2}|a| + |a| = \frac{5}{2}|a| \]
    ... en bovendien het volgende:
    \[ |1+x^{2}| = 1 + |x^{2}| \le 1 + \left(\frac{3}{2}|a|\right)^{2} = 1 + \frac{9}{4}|a|^{2} \]
    \[
    |f(x)-f(a)|
    = \frac{|a+x||a-x|}{|1+x^{2}||1+a^{2}|}
    \le \frac{|a||a-x|}{2\left(1 + \frac{9}{4}|a|^{2}\right)|1+a^{2}|}
    \]
    Kies nu een willekeurige $\epsilon \in \mathbb{R}_{0}^{+}$ en kies $\delta = \min\{\frac{1}{2}|a|,\epsilon\frac{|1+a^{2}|2\left(1 + \frac{9}{4}|a|^{2}\right)}{|a|} \}$.
    Uit $|x-a| < \delta$ volgt dan het volgende:
    \[ 
    |f(x)-f(a)|
    \le \frac{|a||a-x|}{2\left(1 + \frac{9}{4}|a|^{2}\right)|1+a^{2}|}
    < \frac{|a|}{2\left(1 + \frac{9}{4}|a|^{2}\right)|1+a^{2}|}\delta
    \le \frac{\epsilon|a||1+a^{2}|2\left(1 + \frac{9}{4}|a|^{2}\right)}{2\left(1 + \frac{9}{4}|a|^{2}\right)|1+a^{2}||a|} = \epsilon
    \]
  \end{proof}
\feed
\end{vb}

\extra{vb p 4 onderaan}

\begin{tvb}
  De functie $f$ is niet continu in $0$:
  \[
  f:\ \mathbb{R} \rightarrow \mathbb{R}:\ x\mapsto 
  \left\{
    \begin{array}{cl}
      1 & \text{ als } x \ge 0\\
      0 & \text{ als } x < 0
    \end{array}
  \right.
  \]

  \begin{proof}
    Kies $\epsilon = 1$, dan bestaat er voor elke $\delta$ een $x\in \mathbb{R}$ zodat $|x-0|<\delta$ en $|f(x)-f(0)|\ge \epsilon$ beide gelden.
    Voor een willekeurige $\delta$ kiezen we $x = -\frac{\delta}{2}$:
    \[ |x-0| = \frac{\delta}{2} \text{ en } |f(x)-f(0)| = 1 \ge \epsilon \]
    $f$ is dus niet continu in $0$.
  \end{proof}
\end{tvb}

\extra{tvb p 5 onderaan}

\begin{vb}
  De functie $f$ is continu:
  \[ f:\ \mathbb{R} \rightarrow \mathbb{R}:\ x \mapsto |x| \]

  \begin{proof}
    Kies een willekeurige $a\in \mathbb{R}$:
    Beschouw eerst $|f(x)-f(a)|$:
    \[ |f(x)-f(a)| = \left||x|-|a|\right| \le |x-a| \]
    Kies nu een willekeurige $\epsilon$ en kies $\delta = \epsilon$.
    Uit $|x-a| <\delta$ volgt nu het volgende:
    \[ |f(x)-f(a)| \le |x-a| < \delta = \epsilon \]
    $f$ is dus continu in $a$.
  \end{proof}
\end{vb}

\begin{vb}
  De functie $(\in \mathbb{Q})$ is nergens continu:
  \[
  (\in \mathbb{Q}):\ \mathbb{R} \rightarrow \mathbb{R}:\ 
  \left\{
    \begin{array}{rl}
      1 &\text{ als } x\in \mathbb{Q}\\
      0 &\text{ als } x\in \mathbb{R}\setminus \mathbb{Q}
    \end{array}
  \right.
  \]
  
  \begin{proof}
    Kies een willekeurige $a \in \mathbb{R}$.
    Gevalsonderscheid:
    \begin{itemize}
    \item $a\in \mathbb{Q}$
      Kies $\epsilon = \frac{1}{2}$, dan bestaat er voor elke $\delta \in \mathbb{R}_{0}^{+}$ een $x\in \mathbb{R}$ zodat uit $|x-a| < \delta$ $|((x \in \mathbb{Q})-(a\in \mathbb{Q})| \ge \epsilon$ volgt:
      Kies immers een willekeurige $\delta \in \mathbb{R}_{0}^{+}$, dan bestaat er een $x\in \mathbb{R}\setminus \mathbb{Q}$ met $|x-a| < \delta$. \needed
      Hiervoor geldt dan $|((x \in \mathbb{Q})-(a\in \mathbb{Q})| = 1 \ge \frac{1}{2} = \epsilon$.
    \item $a\in \mathbb{R}\setminus \mathbb{Q}$
      \extra{bewijs analoog}
    \end{itemize}
  \end{proof}
\end{vb}

\begin{vb}
  (Bewijs via karakterisatie van continu\"iteit in termen van rijen.)
  De functie $f$ is continu:
  \[ f:\ \mathbb{R} \rightarrow \mathbb{R}: x \mapsto x^{2} \]
  
  \begin{proof}
    We moeten bewijzen dat voor een willekeurige rij $(x_{n})_{n}$ die naar $x$ convergeert, $(f(x_{n}))_{n}$ naar $f(x)$ convergeert.
    Kies daartoe een willekeurige $(x_{n})_{n}$ die naar $x$ convergeert.
    Voor een willekeurige $x_{n}$ uit de rij geldt het volgende:
    \[ |f(x_{n})-f(x)| = |x_{n}^{2}-x^{2}| = |x_{n}+x||x_{n}-x| \]
    Kies een willekeurige $\epsilon \in \mathbb{R}_{0}^{+}$ en kies $\delta = \frac{\epsilon}{|x_{n}+x|}$
    Omdat $(x_{n})_{n}$ naar $x$ convergeert, bestaat er dan een $m\in \mathbb{N}$ zodat het volgende geldt:
    \[ \forall n\in \mathbb{N}: n \ge m:\ |x_{n}-x| < \frac{\epsilon}{|x_{n}+x|} \]
    Voor $n \ge m$ geldt dus het volgende:
    \[ |f(x_{n})-f(x)| = |x_{n}+x||x_{n}-x| < |x_{n}+x|\frac{\epsilon}{|x_{n}+x|} = \epsilon \]
    De rij $(f(x_{n}))_{n}$ convergeert dus naar $f(x)$.
  \end{proof}
\end{vb}

\begin{vb}
  Zij $f: A \subseteq \mathbb{R} \rightarrow \mathbb{R}$ een functie die continu is in een punt $a\in A$ met $f(a) > 0$.
  Er bestaat dan een $\delta \in \mathbb{R}_{0}^{+}$ zodat $f(x)>0$ geldt voor alle $x\in A$ met $|x-a| < \delta$.

  \begin{proof}
    Kies $\epsilon = \f(a)$, dan bestaat er een $\delta \in \mathbb{R}_{0}^{+}$ zodat de stelling geldt omdat $f$ continu is in $A$.
  \end{proof}
\end{vb}

\subsection{Operaties met continue functies}
\label{sec:oper-met-cont}

\extra{voorbeelden}


\subsection{Continue functies op intervallen}
\label{sec:continue-functies-op}

\extra{voorbeelden}

\subsection{Uniforme continu\"iteit}
\label{sec:unif-cont}

\extra{voorbeeld 4.3.1}

\begin{vb}
  De functie $f$ is uniform continu:
  \[ f:\ \mathbb{R}^{+} \rightarrow \mathbb{R}:\ x \mapsto \sqrt{x} \]

  \begin{proof}
    Merk allereerst de volgende ongelijkheid op:
    \[ |\sqrt{x} - \sqrt{y}| \le \sqrt{|x-y|} \]
    Kies een willekeurige $\epsilon$ en definieer $\delta = \epsilon^{2}$.
    Kies nu twee willekeurige elementen $x$ en $y$ uit $\mathbb{R}^{+}$ zodat $|x-y|< \delta$ geldt.
    \[ |f(x) - f(y)| \le  \sqrt{|x-y|} < \sqrt{\delta} = \sqrt{\epsilon^{2}} = \epsilon \]
    $f$ is dus uniform continu over $\mathbb{R}^{+}$.
  \end{proof}
\end{vb}

\begin{tvb}
  De functie $f$ is niet uniform continu:
  \[ f:\ \mathbb{R}_{0}^{+} \rightarrow \mathbb{R}:\ x \mapsto \frac{1}{x} \]

  \begin{proof}
    Kies $\epsilon = 1$ en kies een willekeurige $\delta \in \mathbb{R}_{0}^{+}$.
    Stel bovendien $y = x + \frac{\delta}{2}$, dan is $|x-y|$ altijd kleiner dan $\delta$.
    Merk bovendien de volgende gelijkheid op:
    \[ |f(x)-f(y)| = \left| \frac{1}{x} - \frac{1}{y} \right| \le \left| \frac{1}{x} - \frac{1}{x + \frac{\delta}{2}} \right| = \left| \frac{x+\frac{\delta}{2}-x}{x(x+\frac{\delta}{2})}\right| = \left| \frac{\delta}{x(2x+\delta)} \right| = \frac{\delta}{x(2x+\delta)} \]
    Kies bijvoorbeeld $x = \frac{\delta}{4}\left(\sqrt{1+\frac{\delta}{4}}-1\right)$ (de oplossing van $\frac{1}{2}(2x^{2}+\delta x) = \delta$).
    $x$ is dan positief en $|f(x)-f(y)| = 2 > \epsilon$.
    $f$ is dus niet uniform continu op $\mathbb{R}_{0}^{+}$.
  \end{proof}
\end{tvb}

\extra{tvb 4.3.4}

\end{document}
