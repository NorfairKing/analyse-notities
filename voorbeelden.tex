\documentclass[main.tex]{subfiles}
\begin{document}

\chapter{Voorbeelden}
\label{cha:voorbeelden}

\section{Rijen}
\subsection{Topologie in $\mathbb{R}$}

\extra{zelfde voorbeeld voor $\{\frac{1}{n} \mid n \in \mathbb{N}_{0} \}$}
\extra{zelfde voorbeeld voor $\{\frac{1}{n} \mid n \in \mathbb{N}_{0} \} \cup 0$}
\extra{zelfde voorbeeld voor $\mathbb{N}$}
\extra{zelfde voorbeeld voor $\mathbb{Q}$}

\subsection{Relatieve topologie in $\mathbb{R}$}

\begin{vb}
  Beschouw $\interval[open right]{0}{1}$ als deelverzameling van $\interval[open right]{0}{2}$.
  \begin{itemize}
  \item $\interval[open right]{0}{1}$ is relatief open in $\interval[open right]{0}{2}$.
    $\interval[open right]{0}{1}$ is niet open in $\mathbb{R}$ omdat er geen punten links van $0$ in liggen, maar in $\interval[open right]{0}{2}$ liggen ook geen punten links van $0$, dus dat is geen probleem.
  \item $\interval[open right]{1}{2}$ is dan relatief gesloten.
  \end{itemize}
\end{vb}



\end{document}
