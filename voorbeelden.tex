\documentclass[main.tex]{subfiles}
\begin{document}

\chapter{Voorbeelden}
\label{cha:voorbeelden}

\extra{voorbeelden van velden: $\mathbb{F}_{2}$.}

\section{Getallen}

\subsection{Intervallen}
\begin{vb}
  $\{x\in \mathbb{R} \mid i \le x \le 2\}$ is een interval.
\end{vb}

\begin{vb}
  $\{1,2\}$ is geen interval want $1 \le \nicefrac{3}{2} \le 2$ geldt maar $\nicefrac{3}{2}$ zit niet in $\{1,2\}$.  
\end{vb}

\begin{vb}
  $\{x\in \mathbb{R} \mid x > 5\}$ is een interval.
\end{vb}

\begin{vb}
  $\mathbb{R}_{0}$ is geen interval.
\end{vb}

\section{Rijen}


\begin{vb}
  Zij $x_{0}=1$ en $x_{n+1} = \frac{1}{1+x_{n}}$.
  \extra{bewijs dat $(x_{n})$ naar $\frac{2}{1+\sqrt{5}}$ convergeert}
\end{vb}

\begin{vb}
  Zij $A = \interval[open left]{0}{1} \cup \{2\} $.
  \begin{itemize}
  \item De grootste open deelverzameling van $A$ is $\interval[open]{0}{1}$.
  \item De grootste gesloten deelverzameling van $A$ bestaat niet. 
  \item De kleinste open oververzameling van $A$ bestaat niet
  \item De kleinste gesloten oververzameling van $A$ is $\interval{0}{2}$.
  \end{itemize}
\feed
\end{vb}

\begin{vb}
  \begin{itemize}
  \item De grootste open deelverzameling van $\mathbb{Q}$ is $\emptyset$.
  \item De grootste gesloten deelverzameling van $\mathbb{Q}$ is $\emptyset$.
  \item De kleinste open oververzameling van $\mathbb{Q}$ is $\mathbb{R}$.
  \item De kleinste gesloten oververzameling van $\mathbb{Q}$ is $\mathbb{R}$.
  \end{itemize}
\feed
\end{vb}

\subsection{Topologie in $\mathbb{R}$}

\begin{vb}
  Zij $A= \interval[open left]{0}{1}$.
  \begin{itemize}
  \item $\mathring{A} = \interval[open]{0}{1}$.
  \item $\overline{A} = \interval{0}{1} \cup \{2\}$
  \item $\partial{A} = \{0\} \cup \{1\} \cup \{2\}$
  \item De ge\"isoleerde punten van $A$ zijn $\{2\}$.
  \item De ophopingspunten van $A$ zijn $\interval{0}{1}$.
  \end{itemize}
\end{vb}
\extra{zelfde voorbeeld voor $\{\frac{1}{n} \mid n \in \mathbb{N}_{0} \}$}
\extra{zelfde voorbeeld voor $\{\frac{1}{n} \mid n \in \mathbb{N}_{0} \} \cup 0$}
\extra{zelfde voorbeeld voor $\mathbb{N}$}
\extra{zelfde voorbeeld voor $\mathbb{Q}$}

\subsection{Relatieve topologie in $\mathbb{R}$}

\begin{vb}
  Beschouw $\interval[open right]{0}{1}$ als deelverzameling van $\interval[open right]{0}{2}$.
  \begin{itemize}
  \item $\interval[open right]{0}{1}$ is relatief open in $\interval[open right]{0}{2}$.
    $\interval[open right]{0}{1}$ is niet open in $\mathbb{R}$ omdat er geen punten links van $0$ in liggen, maar in $\interval[open right]{0}{2}$ liggen ook geen punten links van $0$, dus dat is geen probleem.
  \item $\interval[open right]{1}{2}$ is dan relatief gesloten.
  \end{itemize}
\end{vb}



\end{document}
