\documentclass[main.tex]{subfiles}
\begin{document}


\section{Uitbreidingen van de Riemannintegraal}
\label{sec:uitbreidingen-van-de}

\subsection{Oneigenlijke Riemannintegraal}
\label{sec:oneig-riem}

\subsubsection{Integreren over een onbegrensd interval}
\label{sec:integreren-over-een}


\begin{de}
  We noemen een functie $f:\ \interval[open right]{a}{+\infty} \rightarrow \mathbb{R}$  \term{oneigenlijk Riemannintegreerbaar} over $\interval[open right]{a}{+\infty}$ als $f$ Riemannintegreerbaar is over elk interval $\interval{a}{b}$ en de volgende limiet bestaat:
  \[ \lim_{b\rightarrow +\infty}\int_{a}^{b}f \]
  We noemen deze limiet dan de \term{oneigenlijke Riemannintegraal} en noteren hem als volgt:
  \[ \int_{a}^{+\infty}f \]
\end{de}

\begin{de}
  We noemen een functie $f:\ \interval[open left]{-\infty}{b} \rightarrow \mathbb{R}$  \term{oneigenlijk Riemannintegreerbaar} over $\interval[open left]{-\infty}{b}$ als $f$ Riemannintegreerbaar is over elk interval $\interval{a}{b}$ en de volgende limiet bestaat:
  \[ \lim_{a\rightarrow -\infty}\int_{a}^{b}f \]
  We noemen deze limiet dan de \term{oneigenlijke Riemannintegraal} en noteren hem als volgt:
  \[ \int_{-\infty}^{b}f \]
\end{de}

\begin{vb}
  Beschouw de functie $f$ als volgt:
  \[ f:\ \interval[open right]{0}{+\infty} \rightarrow \mathbb{R}:\ x \mapsto e^{-x}\]

  \noindent
  \begin{minipage}{.45\textwidth}
    \begin{figure}[H]
      \centering
      \begin{tikzpicture}[scale=.75]
        \begin{axis}[ymin=0, ymax=1.5, xmin=0, xmax=3]
          \addplot[name path=A,smooth,color=blue,thick,domain=0:3]{e^(-x)};
          \addplot[name path=B,domain=0:3]{0};
          \addplot[blue!20] fill between[of=A and B];
        \end{axis}
      \end{tikzpicture}
    \end{figure}
  \end{minipage}
  \begin{minipage}{.45\textwidth}
    \[ f:\ \interval[open right]{0}{+\infty} \rightarrow \mathbb{R}:\ x \mapsto e^{-x}\]
  \end{minipage}

  $f$ is oneigenlijk Riemannintegreerbaar over $\interval[open right]{0}{+\infty}$ en de integraal ziet er als volgt uit:
  \[ \int_{0}^{+\infty}e^{-x}\ dx = 1 \]

  \begin{proof}
    Kies willekeurig een interval $\interval{0}{b}$, dan is $f$ zeker Riemannintegreerbaar over $\interval{0}{b}$ met volgende integraal:
    \[ \int_{0}^{b}e^{-x}\ dx = -e^{-x}\big|_{x=0}^{b} = 1-e^{-b}\]
    Nemen we hieran de limiet voor $b$ gaande naar $+\infty$, dan krijgen we het gevraagde.
    \[ \lim_{b\rightarrow +\infty}1-e^{-b} = 1 \]
  \end{proof}
\end{vb}

\subsubsection{Een onbegrensde functie integreren}
\label{sec:een-onbegr-funct}

\begin{de}
  We noemen een functie $f:\ \interval[open left]{a}{b} \rightarrow \mathbb{R}$  \term{oneigenlijk Riemannintegreerbaar} over $\interval[open left]{a}{b}$ als $f$ Riemannintegreerbaar is over elk interval $\interval{c}{b}$ en de volgende limiet bestaat:
  \[ \lim_{c\overset{>}{\rightarrow} a}\int_{a}^{b}f \]
  We noemen deze limiet dan de \term{oneigenlijke Riemannintegraal} en noteren hem als volgt:
  \[ \int_{a}^{b}f \]
\end{de}

\begin{de}
  We noemen een functie $f:\ \interval[open right]{a}{b} \rightarrow \mathbb{R}$  \term{oneigenlijk Riemannintegreerbaar} over $\interval[open right]{a}{b}$ als $f$ Riemannintegreerbaar is over elk interval $\interval{c}{b}$ en de volgende limiet bestaat:
  \[ \lim_{c\overset{<}{\rightarrow} b}\int_{a}^{b}f \]
  We noemen deze limiet dan de \term{oneigenlijke Riemannintegraal} en noteren hem als volgt:
  \[ \int_{a}^{b}f \]
\end{de}

\begin{vb}
  Beschouw de functie $f$ als volgt:
  \[ f:\ \interval[open right]{0}{+\infty} \rightarrow \mathbb{R}:\ x \mapsto e^{-x}\]

  \noindent
  \begin{minipage}{.45\textwidth}
    \begin{figure}[H]
      \centering
      \begin{tikzpicture}[scale=.75]
        \begin{axis}[ymin=0, ymax=5, xmin=0, xmax=2]
          \addplot[name path=A,smooth,color=blue,thick,domain=0:1]{1/(sqrt(x))};
          \addplot[name path=B,domain=0:1]{0};
          \addplot[blue!20] fill between[of=A and B];
        \end{axis}
      \end{tikzpicture}
    \end{figure}
  \end{minipage}
  \begin{minipage}{.45\textwidth}
    \[ f:\ \interval[open right]{0}{+\infty} \rightarrow \mathbb{R}:\ x \mapsto \frac{1}{\sqrt{x}} \]
  \end{minipage}

  $f$ is oneigenlijk Riemannintegreerbaar over $\interval[open left]{0}{+\infty}$ en de integraal ziet er als volgt uit:
  \[ \int_{0}^{+\infty}\frac{1}{\sqrt{x}}\ dx = 2 \]

  \begin{proof}
    $f$ is Riemannintegreerbaar over elk interval $\interval[open left]{c}{1}$ met $0 < c \le 1$:
    \[ \int_{c}^{1}\frac{1}{\sqrt{x}}\ dx = 2\sqrt{x}\big|_{x=c}^{1} = 2-2\sqrt{c} \]
    De rechterlimiet hiervan, voor $c$ gaande naar $0$, bestaat ook:
    \[ \lim_{c \overset{>}{\rightarrow} 0} 2-2\sqrt{c} = 2 \]
  \end{proof}
\end{vb}

\begin{de}
  Beschouw $a,b \in \mathbb{R} \cup \{ -\infty, +\infty \}$ met $a < b$ en beschouw een functie $g$ als volgt:
  \[ g:\ \interval[open]{a}{b} \times \interval[open]{a}{b} \rightarrow \mathbb{R} \]
  Zij verder $L \in \mathbb{R} \cup \{ -\infty, +\infty \}$.
  Als voor elke keuze van rijen $(c_{n})_{n}$ en $(d_{n})_{n}$ in $\interval[open]{a}{b}$ die respectievelijk naar $a$ en $b$ convergeren geldt dat $(g(c_{n},d_{n}))_{n}$ naar $L$ convergeert, dan zeggen we het volgende:
  \[ \lim_{\overset{d \overset{<}{\rightarrow} b}{c\overset{>}{\rightarrow} a}} g(c,d) = L \]
\end{de}

\begin{de}
  Beschouw $a,b\in \mathbb{R} \cup \{-\infty,+\infty\}$ met $a<b$.
  We noemen een functie $f:\ \interval[open left]{a}{b} \rightarrow \mathbb{R}$  \term{oneigenlijk Riemannintegreerbaar} over $\interval[open left]{a}{b}$ als $f$ Riemannintegreerbaar is over elk gesloten begrensd interval $\interval{c}{b} \subseteq \interval[open]{a}{b}$ en de volgende limiet bestaat:
  \[ \lim_{\overset{d \overset{<}{\rightarrow} b}{c\overset{>}{\rightarrow} a}}\int_{c}^{d}f \]
  We noemen deze limiet dan de \term{oneigenlijke Riemannintegraal} en noteren hem als volgt:
  \[ \int_{a}^{b}f \]
\end{de}

\begin{bpr}
  Beschouw $a,b\in \mathbb{R} \cup \{-\infty,+\infty\}$ met $a<b$.
  Zij $f:\ \interval[open]{a}{b} \rightarrow \mathbb{R}$ een functie.
  Kies een $c\in \interval[open]{a}{b}$.
  Volgende uitspraken zijn equivalent:
  \begin{enumerate}
  \item $f$ is oneigenlijk Riemannintegreerbaar over $\interval[open]{a}{b}$.
  \item $f_{\interval[open left]{a}{c}}$ is oneigenlijk Riemannintegreerbaar over $\interval[open]{a}{c}$ en $f_{\interval[open right]{c}{b}}$ is oneigenlijk Riemannintegreerbaar over $\interval[open]{c}{b}$.
  \end{enumerate}
  Er geldt dan volgende ongelijkheid:
  \[ \int_{a}^{b}f = \int_{a}^{c}f_{\interval[open left]{a}{c}} + \int_{c}^{b}f_{\interval[open right]{c}{b}} \]
  \TODO{bewijs: oefening}
\end{bpr}

\begin{bst}
  Zij $f,g:\ I \rightarrow \mathbb{R}$ functies op een niet ``eindig en gesloten'' gesloten interval $I$ met de volgende eigenschap:
  \[ \forall x\in I:\ |f(x)| \le g(x) \]
  Stel dat $f$ Riemannintegreerbaar is over elk gesloten begrensd deelinterval van $I$ en dat $g$ oneigenlijk Riemannintegreerbaar is over $I$, dan is $f$ oneigenlijk Riemannintegreerbaar over $I$.

  \begin{proof}
    Het is voldoende om het bewijs te geven in het geval $I = \interval[open]{a}{b}$ met $a,b\in\mathbb{R} \cup \{-\infty,+\infty\}$.\waarom
    Kies willekeurige rijen $(c_{n})_{n}$ en $(d_{n})_{n}$ in $\interval[open]{a}{b}$ met respectievelijke limieten $a$ en $b$.
    Stel $G_{n} = \int_{c_{n}}^{d_{n}}g$ en $F_{n} = \int_{c_{n}}^{d_{n}}f$.
    We moeten aantonen dat $(F_{n})_{n}$ een convergente rij is in $\mathbb{R}$.
    Het volstaat het geval te bekijken waarin $(c_{n})_{n}$ daalt, $(d_{n})_{n}$ stijgt en $\forall n\in\mathbb{N}:\ c_{n}<d_{n}$ geldt.\waarom
    Omdat $g$ oneigenlijk integreerbaar is, is $(G_{n})_{n}$ een convergente rij.
    We vinden het volgende:
    \[ \forall n,k\in\mathbb{N}_{0}:\ |F_{n+k}-F_{n}| \le |G_{n+k}-G_{n}| \]
    Uit deze afschatting volgt dat $(F_{n})_{n}$ een Cauchyrij is dus convergeert in $\mathbb{R}$.
    Hiermee is aangetoond dat $f$ oneigenlijk integreerbaar is over $I$.
  \end{proof}
\end{bst}

\begin{bst}
  Zij $\sum_{n}x_{n}$ een reeks met positieve termen.
  Zij $f:\ \mathbb{R}^{+}\rightarrow \mathbb{R}$ een dalende continue functie.
  Stel dat er een $N\in \mathbb{N}$ bestaat zodat $x_{n}$ gelijk is aan $f(n)$ voor alle volgende $n\in \mathbb{N}$, dan is $f$ oneigenlijk integreerbaar over $\interval[open right]{0}{+\infty}$ als en slechts als $\sum_{n}x_{n}$ convergeert.
  \TODO{bewijs: oefening}
\end{bst}

\subsection{De Riemann-Stieltjesintegraal}
\label{sec:de-riem-stieltj}

\begin{de}
  Zij $f:\ \interval{a}{b} \rightarrow \mathbb{R}$ een begrensde functie.
  Zij $P = \{x_{0},x_{1},\dotsc,x_{n}\}$ een verdeling van het interval $\interval{a}{b}$.
  De \term{$F$-ondersom} $\underline{S}(f,P,F)$ van $f$ bij de verdeling $P$ definieren we als volgt:
\[ \underline{S}(f,P,F) = \sum_{k=1}^{n} \inf\{ f(x) \mid x\in \interval{x_{k-1}}{x_{k}} \left( F(x_{k}) - F(x_{k-1}) \right) \]
  De \term{$F$-bovensom} $\overline{S}(f,P,F)$ van $f$ bij de verdeling $P$ definieren we als volgt:
\[ \overline{S}(f,P,F) = \sum_{k=1}^{n} \sup\{ f(x) \mid x\in \interval{x_{k-1}}{x_{k}} \left( F(x_{k}) - F(x_{k-1}) \right) \]
\end{de}

\begin{de}
  We noemen een begrensde functie $f:\ \interval{a}{b} \rightarrow \mathbb{R}$ \term{Riemann-Stieltjesintegreerbaar} ten opzichte van $F$ als $\underline{S}(f,P,F)$ gelijk is aan $\overline{S}(f,P,F)$.
  In dit geval noemen we deze waarde de \term{Riemann-Stieltjesintegraal} van $f$ over $\interval{a}{b}$ ten opzichte van $F$ en noteren we deze als volgt:
  \[ \int_{a}^{b}f\ dF \]
\end{de}

\begin{bst}
  Zij $F:\ \interval{a}{b}\rightarrow \mathbb{R}$ een stijgende afleidbare functie en $F'$ Riemannintegreerbaar.
  Zij $f:\ \interval{a}{b} \rightarrow \mathbb{R}$ een begrensde functie.
  Als $fF'$ Riemannintegreerbaar is, dan is $f$ Riemann-Stieltjesintegreerbaar ten opzichte van $F$ en geldt volgende ongelijkheid:
  \[ \int_{a}^{b}f(x)\ dF(x) = \int_{a}^{b}f(x)F'(x)\ dx \]
\TODO{bewijs p 40}
\end{bst}


\end{document}

%%% Local Variables:
%%% mode: latex
%%% TeX-master: t
%%% End:
