\documentclass[main.tex]{subfiles}
\begin{document}



\chapter{Continu\"iteit van functies over metrische ruimten}
\label{cha:continuiteit-van-functies-over-metrische-ruimten}

\begin{bpr}
  \label{pr:continuiteit-itv-rijen}
  Zij $X,d_{X}$ en $Y,d_{Y}$ metrische ruimten.
  Zij $f:\ A \subseteq X \rightarrow Y$ een functie en een $a\in A$, dan is $f$ continu in $a$ als en slechts als voor elke rij $(x_{n})_{n}$ in $A$ die naar $a$ convergeert $(f(x_{n}))_{n}$ naar $f(a)$ convergeert.

  \begin{proof}
    Bewijs van een equivalentie
    \begin{itemize}
    \item $\Rightarrow$\\
      Kies een willekeurige rij $(x_{n})_{n}$ in $A$ die naar $a$ convergeert.
      Kies een willekeurige $\epsilon\in\mathbb{R}_{0}^{+}$.
      Omdat $f$ continu is in $a$, bestaat er een $\delta\in \mathbb{R}_{0}^{+}$ als volgt:
      \[ \forall x\in A:\ d_{X}(x,a) < \delta \Rightarrow d_{Y}(f(x),f(a)) \]
      Omdat $(x_{n})_{n}$ convergeert naar $a$, bestaat er een $n_{0}\in \mathbb{N}$ als volgt:
      \[ \forall n\in \mathbb{N}:\ n \ge n_{0} \Rightarrow  d_{X}(x_{n},a) < \delta \]
      Er geldt dan bijgevolg:
      \[ \forall n\in \mathbb{N}:\ n \ge n_{0} \Rightarrow d_{Y}(f(x),f(a)) < \epsilon \]
    \item $\Leftarrow$\\
      Contrapositie: Als $f$ niet continu is in $a$ zal er een rij in $A$ bestaat die in $a$ convergeert waarvoor $(f(x_{n}))_{n}$ niet naar $f(a)$ convergeert.\\
      Kies een willekeurige $\epsilon\in\mathbb{R}_{0}^{+}$.
      Omdat $f$ niet continu is, bestaat er een $\epsilon \in \mathbb{R}_{0}^{+}$ als volgt:
      \[ \forall \delta \in \mathbb{R}_{0}^{+}, \exists x\in A:\ d_{X}(x,a) < \delta \wedge d_{Y}(f(x),f(a)) \ge \epsilon \]
      Kies dan achtereenvolgens voor $n\in \mathbb{N}$ $\delta = \frac{1}{n}$ en vindt de $x$ uit de vorige formule om een rij $(x_{n})_{n}$ op te bouwen.
      \[ \forall n\in \mathbb{N}:\ d_{X}(x_{n},a) < \frac{1}{n} \wedge d_{Y}(f(x_{n}),f(a)) \ge \epsilon \]
      Uit de eerste ongelijkheid volgt dat $(x_{n})_{n}$ naar $a$ convergeert terwijl uit de tweede ongelijkheid volgt dat $(f(x_{n}))_{n}$ niet kan convergeren naar $a$.
    \end{itemize}
  \end{proof}
\end{bpr}

\begin{bpr}
  \label{pr:metrische-ruimte-continu-itv-opens}
  Zij $X,d_{X}$ en $Y,d_{Y}$ metrische ruimten.
  Zij $f:\ A \subseteq X \rightarrow Y$ een functie, dan is $f$ continu op $A$ als en slechts voor elke $V \subseteq Y$, $f^{-1}(V)$ relatief open is in $A$.

  \begin{proof}
    Bewijs van een equivalentie
    \begin{itemize}
    \item $\Rightarrow$\\
      Kies willekeurig een open deel $V$ van $Y$.
      Als $f^{-1}(V)$ leeg is, is $f^{-1}(V)$ triviaal open.
      Stel daarom dat $f^{-1}(V)$ niet leeg is, dan bestaat er een $a\in f^{-1}(V)$.
      Wat volgt geldt voor alle $a\in f^{-1}(V)$.
      $f(a)$ zit dan in $V$ en omdat $V$ open is bestaat er een $\epsilon \in \mathbb{R}_{0}^{+}$ zodat $B_{Y}(f(a),\epsilon)$ een deel is van $V$.
      Omdat $f$ continu is in $a$ kunnen we een $\delta\in\mathbb{R}_{0}^{+}$ vinden als volgt:
      \[ \forall x\in A:\ d_{X}(x,a) < \delta \Rightarrow d_{Y}(f(x),f(a)) < \epsilon \]
      Of handiger opgeschreven:
      \[ f\left(B_{X}(a,\delta) \cap A\right) \subseteq B_{Y}(f(a),\epsilon) \]
      Dit betekent dat $B_{X}(a,\delta) \cap A$ een deel is van $f^{-1}(V)$ en dat $f^{-1}(V)$ dus relatief open is in $A$.
    \item $\Leftarrow$\\
      Stel dat voor elk open deel $V$ van $Y$ geldt dat $f^{-1}(V)$ relatief open is in $A$.
      Kies nu een willekeurige $a\in A$.
      We bewijzen dat $f$ continu is in $a$.
      Kies daartoe een willekeurige $\epsilon \in \mathbb{R}_{0}^{+}$ en beschouw de verzameling $U$ als volgt:
      \[ U = f^{-1}\left(B_{Y}(f(a),\epsilon)\right)\]
      $U$ bevat allesinds $a$ en is bovendien relatief open in $A$.
      Er bestaat dus een $\delta \in \mathbb{R}_{0}^{+}$ zodat het volgende geldt:
      \[ B_{X}(a,\delta) \cap A \subseteq U \]
      Kies een willekeurige $x\in B_{X}(a,\delta) \cap A$, dan behoort $x$ tot $U$ en $f(x)$ dus tot $f(U) = B_{Y}\left(f(a),\epsilon\right)$.
      Er geldt met andere woorden $d_{Y}(f(x),f(a)) < \epsilon$, dus $f$ is continu in $a$.
    \end{itemize}
  \end{proof}
\end{bpr}


\section{Eigenschappen van continue functies}
\label{sec:eigensch-van-cont}

\begin{bpr}
  Beschouw metrische ruimten $X,d_{X}$, $Y,d_{Y}$ en $Z,d_{Z}$ en functies $f:\ X \rightarrow  Y$ en $g:\ Y \rightarrow Z$.
  Stel dat $f$ continu is in een punt $a\in X$ en $g$ in $f(a)$, dan is $g\circ f$ continu in $a$.

  \begin{proof}
    Kies een willekeurige $\epsilon\in \mathbb{R}_{0}^{+}$.
    Omdat $g$ continu is in $f(a)$ kunnen we een $\eta \in \mathbb{R}_{0}^{+}$ vinden als volgt:
    \[ \forall y\in Y:\ d_{Y}(f(a),y) < \eta \Rightarrow d_{Z}(g(f(a)),g(y)) < \epsilon \]
    Omdat $f$ bovendien continu is in $a$ kunnen we een $\delta \in \mathbb{R}_{0}^{+}$ vinden als volgt:
    \[ \forall x\in X:\ d_{X}(a,x) < \delta \Rightarrow d_{Y}(f(a),f(x)) < \eta \]
    Voor elke $x\in X$ met $d_{X}(x,a) < \delta$ geldt bijgevolg $d_{Z}\left(g(f(a)),g(f(x))\right) < \epsilon$.
    $g\circ f$ is dus continu in $a$.
  \end{proof}
\end{bpr}

\begin{bst}
  \label{st:beeld-van-compact-ook-compact}
  Zij $X,d_{X}$ en $Y,d_{Y}$ metrische ruimten. en $f:\ X \rightarrow Y$ een continue functie.
  Als $K \subseteq X$ compact is, is $f(K)$ ook compact.

  \begin{proof}
    Kies een wilekeurige open overdekking $\mathcal{F}$ van $f(K)$.
    We moeten aantonen dat er een eindige deeloverdekking bestaat.
    Omdat $f$ continu is over heel $X$, is het invers beeld van een open in $Y$ open in $X$.\prref{pr:metrische-ruimte-continu-itv-opens}
    Beschouw $\mathcal{F}'$ als volgt:
    \[ \mathcal{F}' = \{ f^{-1}(V) \mid V \in \mathcal{F} \} \]
    $\mathcal{F}'$ is dus een open overdekking van $K$.
    Omdat $K$ compact is, valt $\mathcal{F}'$ te reduceren tot een eindige deeloverdekking van $K$.
    Er bestaan dus $V_{1},V_{2},\dotsc,V_{n} \in \mathcal{F}$ als volgt:
    \[ K = \bigcup_{i=1}^{n}f^{-1}(V_{i}) \]
    Beeldt nu beide leden af onder $f$:
    \[ f(K) = f\left( \bigcup_{i=1}^{n}f^{-1}(V_{i}) \right) = \bigcup_{i=1}^{n}f\left(f^{-1}(V_{i})\right) = \bigcup_{i=1}^{n}V_{i} \]
    $\mathcal{F}$ valt dus te reduceren tot een eindige deeloverdekking $\{V_{1},\dotsc,V_{n}\}$ van $f(K)$.
  \end{proof}
\end{bst}

\begin{bst}
  Zij $X,d_{X}$ en $Y,d_{Y}$ metrische ruimten. en $f:\ X \rightarrow Y$ een continue functie.
  Als $K \subseteq X$ compact is, is $f$ uniform continu op $K$.

  \begin{proof}
    Kies een willekurige $\epsilon \in \mathbb{R}_{0}^{+}$.
    Omdat $f$ continu is, bestaat er voor elke $x\in K$ een $\delta_{x} \in \mathbb{R}_{0}^{+}$ als volgt: 
    \[ \forall y\in Y:\ d_{X}(x,y) < \delta_{x} \Rightarrow d_{Y}\left(f(x),f(y)\right) < \frac{\epsilon}{2} \]
    Met al deze $\delta_{x}$ kunnen we een open overdekking $\mathcal{F}$ van $K$ construeren als volgt:
    \[ K \subseteq \bigcup_{x\in K}B_{X}\left(x,\frac{\delta_{x}}{2}\right) \]
    Omdat $K$ compact is, bestaat er van $\mathcal{F}$ een eindige deeloverdekking en dus een $n\in \mathbb{N}_{0}$ en punten $x_{1},x_{2},\dotsc x_{n}\in K$ als volgt:
    \[ K \subseteq \bigcup_{j=1}^{n}B_{X}\left(x_{j},\frac{\delta_{x}}{2}\right) \]
    Kies $\delta = \min_{i}\left\{\frac{\delta_{x_{i}}}{2} \right\}$
    Dit minimum bestaat zeker omdat het een eindig aantal $\delta_{i}$'s zijn.
    Kies nu willekeurig $x,y\in K$ met $d_{X}(x,y) < \delta$.
    Neem dan de open bol $B_{X}\left(x_{j},\frac{\delta_{x_{j}}}{2}\right)$ waarin $x$ ligt.
    Beschouw nu $d_{X}(x_{j},y)$:
    \[ d_{X}(x_{j},y) \le d_{X}(x_{j},x) + d_{X}(x,y) < \frac{\delta_{x_{j}}}{2} + \delta \le \frac{\delta_{x_{j}}}{2} + \frac{\delta_{x_{j}}}{2} = \delta_{x_{j}} \]
    We besluiten dat $f$ uniforum continu is op $K$ met de volgende afschatting:
    \[ d_{Y}(f(x),f(y)) \le d_{Y}(f(x),f(x_{j})) + d_{Y}(f(x_{j}),f(y)) < \frac{\epsilon}{2} + \frac{\epsilon}{2} = \epsilon \]
  \end{proof}
\end{bst}

\begin{bst}

  Zij $X,d_{X}$ en $Y,d_{Y}$ metrische ruimten. en $f:\ X \rightarrow Y$ een continue injectieve functie.
  Als $X$ compact is, dan is $f^{-1}:\ f(X) \rightarrow X$ continu

  \begin{proof}
    We zullen aantonen dat $f^{-1}$ continu is door aan te tonen dat het invers beeld onder $f^{-1}$ van ee nopen deel van $X$ open is in $f(X)$.\prref{pr:metrische-ruimte-continu-itv-opens}
    Equivalent hiermee tonen we aan dat het inves beeld onder $f^{-1}$ van een gesloten deel van $X$ gesloten is in $f(X)$.
    Kies een willekeurig gesloten deel $G$ van $X$.
    Als gesloten deel van een compacte verzameling is $G$ ook compact.\prref{pr:gesloten-deel-van-compact-is-compact}
    Het beeld van $G$ onder $f$ is dus ook compact:\stref{st:beeld-van-compact-ook-compact}
    \[ f(G) = \left(f^{-1}\right)^{-1}(G) \]
    Omdat $\left(f^{-1}\right)^{-1}(G)$ compact is, is $\left(f^{-1}\right)^{-1}(G)$ gesloten.\stref{st:rijcompact-dan-gesloten}\stref{st:compact-dan-rijcompact}
  \end{proof}
\end{bst}

\begin{bst}
  Zij $X,d_{X}$ en $Y,d_{Y}$ metrische ruimten. en $f:\ X \rightarrow Y$ een continue functie.
  Als $S \subseteq X$ samenhangend is, is $f(S)$ ook samenhangend.

  \begin{proof}
    Bewijs uit het ongerijmde: Stel dat $f(S)$ niet samenhangend is.
    Er bestaan dan niet-lege, onderling gescheiden delen $C$ en $D$ van $Y$ die $f(S)$ opbouwen.
    Noem $A = S \cap f^{-1}(C)$ en $B= S \cap f^{-1}(D)$.
    $S$ is dan de unie van $A$ en $B$.
    $A$ en $B$ zijn bovendien niet leeg.
    We beweren nu dat $A$ en $B$ onderling gescheiden moeten zijn.
    Eerst beweren we dat $\overline{A} \cap B$ leeg is, de andere bewering is analoog.
    Stel immers dat $\overline{A} \cap B$ niet leeg is, dan zit er minstens \'e\'en element $x$ in.
    Kies nu een willekeurige rij $(x_{n})_{n}$ in $A$ met $x$ als limiet.
    Vanwege de continu\"iteit van $f$ moet $(f(x_{n}))_{n}$ dan naar $f(x)$ convergeren.\prref{pr:continuiteit-itv-rijen}
    Omdat $f(x_{n})$ een element is van $f(A)$ en dat op zich een deel van $C$ zal $f(x)$ tot de sluiting van $C$ behoren.\clarify{waarom precies? Het is zo ingebakken nu dat ik niet meer kan wijzen op de stelling die dit beweert}
    Omdat $x$ ook tot $B$ behoort, zal ook $f(x)$ een element zijn van $f(B)$ en dat op zich een deel van $D$.
    $f(x)$ zal dus in de doorsnede van $\overline{C} \cap D$ zitten.
    Dit kan niet omdat de doorsnede leeg is.
    Contradictie.
  \end{proof}
\end{bst}


\section{Convergentie van rijen van continue functies}
\label{sec:conv-van-rijen}

\begin{de}
  Beschouw een rij $(f_{n})_{n}$ van functies op een verzameling $X$ met waarden in een metrische ruimte $Y,d_{Y}$, dan zeggen we dat $(f_{n})_{n}$ \term{puntsgewijs convergeert} op $X$ naar een functie $f:\ X \rightarrow Y$ als en slechts als voor elke $x\in X$ de rij $(f_{n}(x))_{n}$ naar $f(x)$ convergeert:
  \[ \forall x\in X, \forall \epsilon \in \mathbb{R}_{0}^{+}, \exists n_{0} \in \mathbb{N}, \forall n\in \mathbb{N}:\ n \ge n_{0}:\ \Rightarrow d_{Y}(f_{n}(x),f(x)) < \epsilon \]
\end{de}

\begin{de}
  Beschouw een rij $(f_{n})_{n}$ van functies op een verzameling $X$ met waarden in een metrische ruimte $Y,d_{Y}$, dan zeggen we dat $(f_{n})_{n}$ \term{uniform convergeert} op $X$ naar een functie $f:\ X \rightarrow Y$ als en slechts als het volgende geldt:
  \[ \forall \epsilon \in \mathbb{R}_{0}^{+}, \exists n_{0}\in \mathbb{N}, \forall x\in X, \forall n\in \mathbb{N}:\ n \ge n_{0} \Rightarrow d_{y}(f_{n}(x),f_{n}(y)) < \epsilon \]
\end{de}

\begin{bst}
  Zij $X,d_{X}$ en $Y,d_{y}$ metrische ruimten.
  Beschouw een rij $(f_{n})_{n}$ van functies $f_{n}:\ X \rightarrow Y$.
  Veronderstel dat $(f_{n})_{n}$ uniform convergeert op $X$ naar een functie $f:\ X \rightarrow Y$, dan gelden volgende uitspraken:
  \begin{enumerate}
  \item Als alle $f_{n}$ continu zijn in eenzelfde $a\in X$, dan is ook $f$ continu in $a$.
  \item Als alle $f_{n}$ uniform continu zijn op $X$, dan is ook $f$ uniform continu op $X$.
  \end{enumerate}
\TODO{bewijs:oefening}
\end{bst}

\begin{bst}
  \label{st:stelling-van-dini}
  De \term{stelling van Dini}\\
  Zij $X,d$ een compacte metrische ruimte en $(f_{n})_{n}$ een rij van functies van $X$ naar $\mathbb{R}$.
  Als aan de volgende voorwaarden voldaan is zal $f_{n}$ uniform convergeren naar $f$.
  \begin{enumerate}
  \item Alle functies $f_{n}$ zijn continu.
  \item De rij $(f_{n})_{n}$ convergeert puntsgewijs naar een continue functie $X \rightarrow \mathbb{R}$.
  \item Voor elke $x\in X$ is de rij $(f_{n}(x))_{n}$ stijgend.
  \end{enumerate}

  \begin{proof}
    Kies een $\epsilon \in \mathbb{R}_{0}^{+}$.
    Omdat de rij $(f_{n})_{n}$ puntsgewijs convergeert op $f$, bestaat er voor elke $x\in X$ een $n_{x}\in \mathbb{N}$ zodat $\left|f(x)-f_{n_{x}}(x)\right) < \epsilon$ geldt.
    Omdat de functie $f-f_{n_{x}}$ continu is\prref{pr:optelling-continu}, bestaat er dan ook een open bol $B(x,\delta_{x})$ zodat voor $y\in B(x,\delta_{x})$, $\left|f(x)-f_{n_{x}}(y)\right| < \epsilon$ geldt.
    Al deze bollen samen vormen een open overdekking van de compacte verzameling $X$.
    Er bestaat dus een eindige deeloverdekking met bollen rond punten $x_{1},x_{2},\dotsc,x_{k}\in X$:
    \[ X = \cup_{i=1}^{k}B\left(x_{i},\delta_{x_{i}}\right) \]
    Kies $n_{0} = \max\left\{n_{x_{1}},n_{x_{2}},\dotsc,n_{x_{k}}\right\}$.
    Kies nu willekeurig een $n\in \mathbb{N}$, groter dan $n_{0}$ en een willekeurige $y\in X$.
    $y$ zit dan in \'e\'en van de bollen, zeg $B\left(x_{j},\delta_{x_{j}}\right)$.
    Omdat de rij $\left(f_{n}(y)\right)_{n}$ stijgt en $n \ge n_{0} \ge n_{x_{j}}$ geldt, zal het volgende gelden:
    \[ \left|f(x)-f_{n}(y)\right| \le \left| f(y) - f_{n_{x_{j}}}(y) \right| < \epsilon \]
    Dit bewijst de uniforme convergentie.
  \end{proof}
\end{bst}

\TODO{tegenvoorbeelden}


\end{document}

%%% Local Variables:
%%% mode: latex
%%% TeX-master: t
%%% End:
