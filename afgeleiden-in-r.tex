\documentclass[main.tex]{subfiles}
\begin{document}

\chapter{Afgeleiden van functies van $\mathbb{R}$ naar $\mathbb{R}$}
\label{cha:afgel-van-funct}

\section{Basisbegrippen}
\label{sec:basisbegrippen}

\begin{de}
  Beschouw een functie $f: A \subseteq \mathbb{R} \rightarrow \mathbb{R}$ en een punt $a\in A$ dat ook een ophopingspunt is van $A$.
  We zeggen dat $f$ \term{afleidbaar} is in $a$ als de volgende limiet bestaat en eindig is:
  \[ L = \lim_{h\rightarrow 0}\frac{f(a+h)-f(a)}{h} \]
  We noemen $L$ de \term{afgeleide} van $f$ in $a$ en noteren $L$ als $f'(a)$.
  Als alle punten van $A$ ophopingspunten zijn van $A$ noemen we $f$ \term{afleidbaar} over $A$ als $f$ afleidbaar is in elke $x\in A$.
  In dit geval definieren we de \term{afgeleide functie} $f$ als volgt:
  \[ f':\ A \subseteq \mathbb{R} \rightarrow \mathbb{R}:\ x \mapsto f'(x) \]
\end{de}

\begin{st}
  De afgeleide van een functie in een punt is uniek (omdat dat punt een ophopingspunt is).
\extra{bewijs}
\end{st}

\TODO{equivalente definite voor afgeleide zonder $h$}

\subsection{Meetkundige betekenis}
\label{sec:meetk-betek}

\begin{de}
  Een \term{koorde} is een rechte tussen twee punten $(a,f(a))$ en $(a+h,f(a+h))$ op de grafiek van een functie.
  \[ y = \frac{f(a+h) -f(a)}{h} (x-a) + f(a) \]
\end{de}

\begin{de}
  De \term{raaklijn} aan een grafiek in een punt is de limiet van de koorde in dat punt van $h$ gaande naar nul.
  \[ y = \lim_{h\rightarrow 0}\frac{f(a+h) -f(a)}{h} (x-a) + f(a) \]
  We noemen de raaklijn de \term{eerste orde benadering} of de \term{standaard eerstegraadsbenadering} van $f$ rond $a$.
\end{de}

\begin{de}
  Beschouw een functie $f:\ A \subseteq \mathbb{R} \rightarrow \mathbb{R}$ en een punt $a \in A$.
  \begin{itemize}
  \item Als $a$ een ophopingspunt is van $A \cap \interval[open left]{-\infty}{a}$ en de beperking van $f$ tot $A \cap \interval[open left]{-\infty}{a}$ afleidbaar is in $a$, dan noemen we $f$ \term{linksafleidbaar} in $a$.
    We noemen de afgeleide van de beperkte functie in $a$ de \term{linkerafgeleide} $f'(a^{-})$ van $f$ in $a$.
  \item Als $a$ een ophopingspunt is van $A \cap \interval[open right]{a}{+\infty}$en de beperking van $f$ tot $A \cap \interval[open left]{a}{+\infty}$ afleidbaar is in $a$, dan noemen we $f$ \term{rechtsafleidbaar} in $a$.
    We noemen de afgeleide van de beperkte functie in $a$ de \term{rechterafgeleide} $f'(a^{+})$ van $f$ in $a$.
  \end{itemize}
\end{de}

\begin{st}
  $f$ is links- , respectievelijk rechtsafleidbaar in $a$ als en slechts de volgende limiet bestaat en eindig is.
  \[ \lim_{h\overset{<}{\rightarrow} 0}\frac{f(a+h)-f(a)}{h} \quad\text{en}\quad \lim_{h\overset{>}{\rightarrow} 0}\frac{f(a+h)-f(a)}{h} \]
\extra{bewijs}
\end{st}
 

\section{Elementaire eigenschappen en rekenregels}
\label{sec:elem-eigensch-en}

\begin{pr}
  Beschouw een functe $f:\ A \subseteq \mathbb{R} \rightarrow \mathbb{R}$ en een $a\in A$.
  Stel dat $a$ een ophopingspunt is van zowel $\interval[open left]{-\infty}{a} \cap A$ als $\interval[open right]{a}{+\infty} \cap A$.
  $f$ is dan afleidbaar in $a$ als en slechts $f$ zowel links- als rechtsafleidbaar is in $a$ en $f'(a^{-})=f'(a^{+})$.
  \[ f'(a) = f'(a^{-})=f'(a^{+}) \]
\TODO{bewijs: oefening}
\end{pr}

\begin{pr}
  \label{pr:afleidbaar-dan-continu}
  Beschouw een functie $f:\ A \subseteq \mathbb{R} \rightarrow \mathbb{R}$ en een $a\in A$ een ophopingspunt van $A$.
  Als $f$ afleidbaar is in $a$, dan is $f$ continu is $a$.

  \begin{proof}
    Zij $f$ een afleidbare functie.
    Om te bewijzen dat de limiet van $f$ in $a$ $f(a)$ is\prref{pr:functie-continu-asa-limiet-is-beeld} bewijzen we het volgende:
    \[ \lim_{x \rightarrow a}\left(f(x)-f(a)\right)\]
    Deze limiet is eveneens gelijk aan de volgende:
    \[ = \lim_{x \rightarrow a}\left(f(x)-f(a)\right)\frac{x-a}{x-a} \]
    Vanwege de rekenregels voor limieten mogen we deze limiet splitsen.\prref{pr:rekenregels-limieten}
    \[ = \lim_{x \rightarrow a}\frac{f(x)-f(a)}{x-a}\lim_{x \rightarrow a}(x-a) \]
    Omdat $f$ afleidbaar is geldt dan het volgende:
    \[ = f'(a)\cdot 0 = 0 \]
  \end{proof}
\end{pr}

\begin{tvb}
  Het omgekeerde geldt niet.
\extra{tegenvoorbeeld}
\end{tvb}

\begin{st}
  Er bestaan functies van $\mathbb{R}$ naar $\mathbb{R}$ die overal continu maar nergens afleidbaar zijn.
\extra{bewijs p 10}
\end{st}

\begin{pr}
  Beschouw een functie $f: A \subseteq \mathbb{R} \rightarrow \mathbb{R}$ en een punt $a\in A$ dat een ophopingspunt is van $A$.
  Stel dat $f$ afleidbaar is en zij $\lambda \in \mathbb{R}$.
  \[ \lambda f \text{ is afleidbaar in } a \text{ en } (\lambda f)'(a) = \lambda f'(a) \]

  \begin{proof}
    We bewijzen dit in twee delen.
    \begin{itemize}
    \item $\lambda f$ is afleidbaar in $a$.
\extra{bewijs}
    \item $(\lambda f)'(a) = \lambda f'(a)$\\
      Zie hiervoor de rekenregels voor limieten.\prref{pr:rekenregels-limieten}
      \[ \lim_{x \rightarrow a}\frac{\lambda f(x)-\lambda f(a)}{x-a} \lim_{x \rightarrow a}\lambda \frac{\lambda f(x)-\lambda f(a)}{x-a} = \lambda \lim_{x \rightarrow a}\frac{f(x)-f(a)}{x-a} \]
    \end{itemize}
  \end{proof}
\end{pr}


\begin{pr}
  Beschouw functie $f,g: A \subseteq \mathbb{R} \rightarrow \mathbb{R}$ en een punt $a\in A$ dat een ophopingspunt is van $A$.
  Stel dat $f$ en $g$ afleidbaar zijn.
  \[ f+g \text{ is afleidbaar in } a \text{ en } (f+g)'(a) = f'(a) + g'(a) \]

  \begin{proof}
    We bewijzen dit in twee delen.
    \begin{itemize}
    \item $f+g$ is afleidbaar in $a$.
      \extra{bewijs}
    \item $(f+g)'(a) = f'(a) + g'(a)$\\
      Zie hiervoor de rekenregels voor limieten.\prref{pr:rekenregels-limieten}
      \[
      \begin{array}{rl}
        \lim_{x \rightarrow a}\frac{(f+g)(x)-(f+g)(a)}{x-a}
        &= \lim_{x \rightarrow a}\frac{f(x)+g(x)-f(a)-g(a)}{x-a}\\
        &= \lim_{x \rightarrow a}\frac{f(x)-f(a)+g(x)-g(a)}{x-a}\\
        &= \lim_{x \rightarrow a}\frac{f(x)-f(a)}{x-a}+\frac{g(x)-g(a)}{x-a}\\
        &= \lim_{x \rightarrow a}\frac{f(x)-f(a)}{x-a}+\lim_{x \rightarrow a}\frac{g(x)-g(a)}{x-a}\\
        &= f'(a) + g'(a)
      \end{array}
      \]
    \end{itemize}

  \end{proof}
\end{pr}

\begin{pr}
  Beschouw functie $f,g: A \subseteq \mathbb{R} \rightarrow \mathbb{R}$ en een punt $a\in A$ dat een ophopingspunt is van $A$.
  Stel dat $f$ en $g$ afleidbaar zijn.
  \[ fg \text{ is afleidbaar in } a \text{ en } (fg)'(a) = f(a) g'(a) + f'(a)g(a) \]

  \begin{proof}
    We bewijzen dit in twee delen.
    \begin{itemize}
    \item $fg$ is afleidbaar in $a$.
      \extra{bewijs}
    \item $(fg)'(a) = f(a) g'(a) + f'(a)g(a)$\\
      Zie hiervoor de rekenregels voor limieten.\prref{pr:rekenregels-limieten}
      \[
      \begin{array}{rl}
        \lim_{x \rightarrow a}\frac{(fg)(x)-(fg)(a)}{x-a}
        &= \lim_{x \rightarrow a}\frac{f(x)g(x)-f(a)g(a)}{x-a}\\
        &= \lim_{x \rightarrow a}\frac{f(x)g(x)-f(a)g(x)+f(a)g(x) -f(a)g(a)}{x-a}\\
        &= \lim_{x \rightarrow a}g(x)\frac{f(x)-f(a)}{x-a} + f(a)\frac{g(x) -g(a)}{x-a}\\
        &= \lim_{x \rightarrow a}g(x)\lim_{x \rightarrow a}\frac{f(x)-f(a)}{x-a} + f(a)\lim_{x \rightarrow a}\frac{g(x) -g(a)}{x-a}\\
        &= g(a)f'(a) + f(a)g'(a)
      \end{array}
      \]
    \end{itemize}
  \end{proof}
\end{pr}

\begin{pr}
  Beschouw functie $g: A \subseteq \mathbb{R} \rightarrow \mathbb{R}$ en een punt $a\in A$ dat een ophopingspunt is van $A$.
  Stel dat $g$ afleidbaar is, en $g(a) \neq 0$.
  \[ \frac{1}{g}:\ \{ x \in A \mid g(x) \neq 0\} \rightarrow \mathbb{R}: x \mapsto \frac{1}{g(x)} \]
  \[ \left(\frac{1}{g}\right)' \text{ is afleidbaar in } a \text{ en } \left(\frac{1}{g}\right)'(a) = -\frac{g'(a)}{g(a)^{2}} \]

  \begin{proof}
    We bewijzen dit in twee delen.
    \begin{itemize}
    \item $\frac{1}{g}$ is afleidbaar.\\
      Omdat $g$ afleidbaar is in $a$ is $g$ continu in $a$.\prref{pr:afleidbaar-dan-continu}
      Omdat $g(a)$ verschillend is van $0$ bestaat er dus een $\delta > 0$ zodat $g(x) \neq 0$ voor alle $x\in A$, dichter dan $\delta$ bij $a$.\waarom
      We vinden dus dat $A \cap \interval[open]{a-\delta}{a+\delta}$ een deel is van $A_{0} = \{ x \in A \mid g(x) \neq 0\}$.
      Omdat $a$ een ophopingspunt is volgt hier uit dat $a$ ook een ophopingspunt is van $A_{0}$.\waarom
      $0$ is dus een ophopingspunt van $(A_{0}-a) \setminus \{0\}= \{ x-a \mid x \in A, g(x) \neq 0\} \setminus \{0\}$.
      We beschouwen de limiet in $0$ van de functie $\Delta$:
      \[ \Delta:\ (A_{0}-a) \setminus \{0\} \rightarrow \mathbb{R}:\ h \mapsto \frac{1}{h}\left(\frac{1}{g(a+h)}-\frac{1}{g(a)} \right) \]
      Met behulp van de rekenregels voor limieten vinden we dan het volgende:
      \[
      \begin{array}{rl}
        \lim_{h \rightarrow 0}\Delta(h)
        &= \lim_{h \rightarrow 0}\frac{1}{h} \frac{g(a)-g(a+h)}{g(a+h)g(a)}\\
        &= -\lim_{h \rightarrow 0}\frac{g(a+h)-g(a)}{h} \cdot \lim_{h\rightarrow 0}\frac{1}{g(a+h)g(a)}\\
        &= -\frac{g'(a)}{g(a)^{2}}
      \end{array}
      \]
      \extra{bewijs meer uitleggen}
    \item $\left(\frac{1}{g}\right)'(a) = -\frac{g'(a)}{g(a)^{2}}$\\
      \[
      \begin{array}{rl}
        \lim_{x \rightarrow a}\frac{\left(\frac{1}{g}\right)(x)-\left(\frac{1}{g}\right)(a)}{x-a}
        &= \lim_{x \rightarrow a}\frac{\frac{1}{g(x)}-\frac{1}{g(a)} }{x-a}\\
        &= \lim_{x \rightarrow a}\frac{1}{(x-a)g(x)}-\frac{1}{(x-a)g(a)}\\
        &= \lim_{x \rightarrow a}\frac{g(a)-g(x)}{(x-a)g(a)g(x)}\\
        &= \lim_{x \rightarrow a}\frac{g(a)-g(x)}{(x-a)}\lim_{x\rightarrow a}\frac{1}{g(a)g(x)}\\
        &= -\frac{g'(a)}{g(a)^{2}}\\
      \end{array}
      \]
    \end{itemize}
  \end{proof}
\end{pr}

\begin{st}
  \label{st:kettingregel}
  De \term{kettingregel}\\
  Beschouw functies $f:\ A \subseteq \mathbb{R} \rightarrow B \subseteq \mathbb{R}$ en $g:\ B \subseteq \mathbb{R} \rightarrow \mathbb{R}$.
  Veronderstel dat $a\in A$ een ophopingspunt is van $A$ en dat $f$ afleidbaar is in $a$.
  Stel bovendien dat $f(a)$ een ophopingspunt is van $B$ en dat $g$ afleidbaar is in $f(a)$.
  $g\circ f$ is dan afleidbaar in $a$:
  \[ (g \circ f)'(a) = g'(f(a))f'(a) \]

  \begin{proof}
    Noem $h=g\circ f$.
    We moeten aantonen dat de volgende limiet bestaat en gelijk is aan $g'(f(a))f'(a)$.
    \[ \lim_{x\rightarrow a}\frac{h(x)-h(a)}{x-a} \]
    Noem $b=f(a)$ en beschouw de functie $phi: B \rightarrow \mathbb{R}$:
    \[
    \phi(y) =
    \left\{
      \begin{array}{cl}
        \frac{g(y)-g(b)}{y-b} & \text{ als } y\in B, y \neq b\\
        g'(x) &\text{ als } y=b\\
      \end{array}
    \right\}
    \]
    \clarify{waarom deze functie nodig?}
    Omdat $g$ afleidbaar is in $b$ is de limiet van $\phi$ in $b$ gelijk aan de afgeleide van $g$ in $b$, wat het beeld is van $\phi$ in $b$:
    \[ \lim_{y\rightarrow b}\phi(y) = g'(b) = \phi(b) \]
    $\phi$ is dus continu in $b$.
    Omdat $f$ afleidbaar is in $a$, is $f$ continu in $a$.\prref{pr:afleidbaar-dan-continu}
    De functie $\phi \circ f: A \rightarrow \mathbb{R}$ is dus continu in $a$.
    Er geldt dan het volgende:
    \[ \lim_{x\rightarrow a}\phi(f(x)) = \phi(f(a)) = \phi(b) = g'(b) \]
    Voor elke $x\in a$ met een beeld verschillend van $f(a)$ geldt dan het volgende:
    \[ \phi(f(x)) = \frac{g(f(x)) - g(f(a))}{f(x)-f(a)} \]
    Omgevormd:
    \[ g(f(x)) - g(f(a)) = \phi(f(x))(f(x)-f(a)) \]
    Merk nu op dat de bovenstaande vergelijking ook geldt voor de $x\in A$ met $f(x)=f(a)$.
    Bijgevolg geldt ze voor alle $x\in A$.
    Voor alle $x\in A$, verschillend van $a$, geldt nu het volgende:
    \[ 
    \frac{h(x)-h(a)}{x-a} = \frac{g(f(x))-g(f(a))}{x-a} = \frac{g(f(x))-g(f(a))}{f(x)-f(a)}\frac{f(x)-f(a)}{x-a} = \phi(f(x))\left(\frac{f(x)-f(a)}{x-a}\right)
    \]
    We weten nu dat de limieten voor $x$ gaande naar $a$ voor beide factoren bestaan en eindig zijn.\waarom
    De rekenregels van limieten leren ons dat de limiet voor $x$ gaande naar $a$ ook bestaat en er als volgt uitziet:
    \[ \lim_{x\rightarrow a}\frac{h(x)-h(a)}{x-a} = \lim_{x\rightarrow a}\phi(f(x))\lim_{x\rightarrow a}\left(\frac{f(x)-f(a)}{x-a}\right) = g'(f(a))f'(a) \]
    \extra{bewijs extra uitleggen}
  \end{proof}
\end{st}

\begin{st}
  Zij $f: A \subseteq \mathbb{R} \rightarrow \mathbb{R}$ een bijectie.
  Noteer de inverse bijectie met $g$.
  Zij $a \in A$ een ophopingspunt van $A$.
  Veronderstel dat $f$ afleidbaar is in $a$ en dat de afgeleide er niet nul is.
  Stel bovendien dat $g$ continu is in $f(a)$, dan is $f(a)$ een ophopingspunt van $B$ en is $g$ afleidbaar in $f(a)$:
  \[ g'(f(a)) = \frac{1}{f'(a)} \]

  \begin{proof}
    Bewijs in delen
    \begin{itemize}
    \item $f(a)$ is een ophopingspunt is van $B$.\\
      Kies een willekeurige $\eta \in \mathbb{R}_{0}^{+}$.
      Omdat $f$ afleidbaar is in $a$, is $f$ zeker continu in $a$.\prref{pr:afleidbaar-dan-continu}
      We kunnen dus een $\xi\in \mathbb{R}_{0}^{+}$ vinden zodat $f(x) \in \interval[open]{f(a)-\eta}{f(a)+\eta}$ geldt voor alle $x\in A$ die dichter dan $\xi$ bij $a$ liggen.
      Omdat $a$ een ophopingspunt is van $A$, kunnen we een $x_{0}\in A$ vinden, verschillend van $a$, die dichter dan $\xi$ bij $a$ ligt.
      Het beeld van $x_{0}$ onder $f$ zal dan in $\interval[open]{f(a)-\eta}{f(a)+\eta}\cap B$ liggen.
      Omdat $f$ injectief is zal $f(x_{0})$ verschillend zijn van $f(a)$.
      De volgende verzameling is dus niet leeg, wat bewijst dat $f(a)$ een ophopingspunt is van $B$.
      \[ \interval[open]{f(a)-\eta}{f(a)+\eta}\cap (B \setminus \{f(a)\}) \]
    \item $g$ is afleidbaar in $f(a)$.\\
      Kies een $x\in A\setminus \{a\}$ en noteer $y=f(x)$ en $b=f(a)$.
      \[ 
      \begin{array}{rl}
        \left| \frac{g(y)-g(b)}{y-b} - \frac{1}{f'(a)} \right|
        &= \left| \frac{x-a}{f(x)-f(a)} - \frac{1}{f'(a)} \right|\\
        &= \left| \frac{f'(a)(x-a)-(f(x)-f(a))}{(f(x)-f(a))f'(a)} \right|\\
        &= \left| \frac{1}{f'(a)}\right|\left| \frac{f'(a)(x-a)-(f(x)-f(a))}{f(x)-f(a)} \right|\\
        &= \left| \frac{1}{f'(a)}\right|\left|\frac{f'(a)(x-a)}{f(x)-f(a)} -1 \right|\\
        &= \left| \frac{1}{f'(a)} \right| \left| \frac{x-a}{f(x)-f(a)} \right| \left| f'(a)-\frac{(f(x)-f(a))}{x-a} \right|\\
      \end{array}
      \]
      Omdat $f$ afleidbaar is, kunnen we een $eta_{1}\in \mathbb{R}$ nemen zodat uit $0<|x-a|<\eta_{1}$ en $x\in A$ het volgende volgt: ($f'(a)$ is de limiet van de functie $\frac{f(x)-f(a)}{x-a}$.)
      \[ \left| \frac{f(x)-f(a)}{x-a} -f'(a)\right| < \frac{1}{2}|f'(a)| \]
      En dus:
      \[ \left| \frac{f(x)-f(a)}{x-a} \right| > \frac{1}{2}|f'(a)| \]
      Keren we dit om, dan krijgen we de volgende ongelijkheid:
      \[ \left| \frac{x-a}{f(x)-f(a)} \right| < \frac{2}{|f'(a)|} \]
      Kies nu een $\epsilon \in \mathbb{R}_{0}^{+}$. We gaan nu op zoek naar een $\delta\in\mathbb{R}_{0}^{+}$ zodat uit $0 <|y-b|< \delta$ en $y\in B$ het volgende volgt:
      \[ \left|\frac{g(y)-g(b)}{y-b} -\frac{1}{f'(a)} \right| < \epsilon\]
      Hieruit volgt dan de stelling.\waarom
      Kies nu een $\eta_{2}$ zodat uit $0<|x-a|< \eta_{2}$ en $x\in A$ het volgende volgt:
      \[ \left| \frac{f(x)-f(a)}{x-a} -f'(a) \right| < \epsilon\frac{|f'(a)|^{2}}{2} \]
      Omdat $g$ continu is in $b$, kunnen we een $\delta \in \mathbb{R}_{0}^{+}$ vinden zodat $|y-b| < \delta$ en $y\in B$ impliceert dat $|x-a|$ kleiner is dan $\min\{\eta_{1},\eta_{2}\}$.
      Er volgt nu uit $0<|x-a| <\min\{\eta_{1},\eta_{2}\}$ en $x\in A$ het volgende:
      \[
      \begin{array}{rl}
        \left| \frac{g(y)-g(b)}{y-b} - \frac{1}{f'(a)} \right|
        &= \left| \frac{1}{f'(a)} \right| \left| \frac{x-a}{f(x)-f(a)} \right| \left| f'(a)-\frac{(f(x)-f(a))}{x-a} \right|\\
        &< \left| \frac{1}{f'(a)} \right|\frac{2}{|f'(a)|}\epsilon\frac{|f'(a)|^{2}}{2}\\
        &= \epsilon
      \end{array}
      \]
\extra{bewijs verbeteren!!}
    \item $g'(f(a)) = \frac{1}{f'(a)}$\\
    \end{itemize}
  \end{proof}
\end{st}

\section{Middelwaardestelling van Rolle en Lagrange}
\label{sec:midd-van-rolle}

\subsection{Extrema}
\label{sec:extrema}

\begin{de}
  Beschouw een functie $f:\ A \subseteq \mathbb{R} \rightarrow \mathbb{R}$.
  We zeggen dat $f$ een \term{globaal maximum}, respectievelijk \term{globaal minimum} bereikt (over $A$) in een $a\in A$ als het volgende geldt:
  \[ \forall x\in A:\ f(a) \ge f(x) \quad\text{respectievelijk}\quad \forall x\in A:\ f(a) \le f(x)\]
\end{de}

\begin{de}
  Beschouw een functie $f:\ A \subseteq \mathbb{R} \rightarrow \mathbb{R}$.
  We zeggen dat $f$ een \term{lokaal maximum}, respectievelijk \term{lokaal minimum} bereikt (over $A$) in een $a\in A$ als het volgende geldt:
  \[ \exists \delta \in \mathbb{R}_{0}^{+}\forall x\in A:\ |x-a| < \delta \Rightarrow f(a) \ge f(x)\]
  respectievelijk
  \[ \exists \delta \in \mathbb{R}_{0}^{+}\forall x\in A:\ |x-a| < \delta \Rightarrow f(a) \le f(x)\]
\end{de}

\begin{de}
  Een \term{globaal extremum}, respectievelijk \term{lokaal extremum} is ofwel een globaal maximum ofwel een globaal minimum, respectievelijk ofwel een lokaal maximum ofwel een lokaal minimum.
\end{de}

\begin{de}
  Een inwendig punt van een deel $A$ van $\mathbb{R}$ waarvoor $f'(a)=0$ geldt, noemt men een \term{kritiek punt} van $f$.
\end{de}

\begin{pr}
  Beschouw een functie $f:\ A \subseteq \mathbb{R} \rightarrow \mathbb{R}$ en een inwendig punt $a$ van $A$.
  Stel dat $f$ een lokaal extremum bereikt in een $a$ en dat $f$ afleidbaar is in $a$, dan geldt $f'(a) = 0$.

  \begin{proof}
    Gevalsonderscheid:
    \begin{itemize}
    \item $f$ bereikt een lokaal maximum in $x$.
      Omdat $a$ een inwendig punt is van $A$ en een lokaal maximum bereikt in $a$ kunnen we een $\delta \in \mathbb{R}_{0}^{+}$ nemen als volgt:
      \[ \forall x\in \mathbb{R}: |x-a| < \delta \Rightarrow (x\in A \wedge f(a) \ge f(x)) \]
      Voor elke $h\in \mathbb{R}$ met $|h| < \delta$ zal $a+h$ dus tot $A$ behoren en ook het volgende gelden:
      \[ f(a+h) -f(a) \le 0 \]
      Beschouw nu de rij $(h_{n})_{n}$ gedefinieerd als volgt:
      \[ h_{n} = \frac{\delta}{2n} \]
      $(h_{n})_{n}$ convergeert dan naar $0$ maar geen enkele $h_{n}$ is nul.
      Bovendien geldt het volgende voor elke $n\in \mathbb{N}$:
      \[ \frac{f(a+h_{n}) -f(a)}{h_{n}} \le 0 \]
      Omdat de limiet de orde bewaart zal het volgende gelden:
      \[ f'(a) = \lim_{h\rightarrow 0}\frac{f(a+h)-f(a)}{h} = \lim_{n \rightarrow \infty}\frac{f(a+h_{n})-f(a)}{h_{n}} \le 0 \]
      Analoog vinden we dat $f'(a)$ ook groter dan of gelijk aan $0$ moet zijn.
      $f'(a)$ moet dus gelijk zijn aan $0$.
    \item $f$ bereikt een lokaal minimum in $x$.
      \extra{bewijs}
    \end{itemize}
  \end{proof}
\end{pr}

\subsection{Stijgen of dalen}
\label{sec:stijgen-dalen}

\begin{de}
  Beschouw een functie $f: J \subseteq \mathbb{R} \rightarrow \mathbb{R}$ gedefinieerd op een interval $J$.
  Zij $I$ een deelinterval van $J$.
  \begin{itemize}
  \item We zeggen dat $f$ \term{stijgt} als het volgende geldt.
    \[ \forall x,y \in I:\ x \le y \Rightarrow f(x) \le f(y) \]
  \item We zeggen dat $f$ \term{daalt} als het volgende geldt.
    \[ \forall x,y \in I:\ x \le y \Rightarrow f(x) \ge f(y) \]
  \item We zeggen dat $f$ strikt \term{stijgt} als het volgende geldt.
    \[ \forall x,y \in I:\ x < y \Rightarrow f(x) < f(y) \]
  \item We zeggen dat $f$ strikt \term{daalt} als het volgende geldt.
    \[ \forall x,y \in I:\ x < y \Rightarrow f(x) > f(y) \]
  \end{itemize}
\end{de}

\begin{pr}
  Stel dat een functie $f: I \subseteq \mathbb{R} \rightarrow \mathbb{R}$ een afleidbare functie is op een open interval $I$.
  \begin{itemize}
  \item Als $f$ stijgt over $I$, dan geldt $\forall x\in I:\ f'(x) \ge 0$.
  \item Als $f$ daalt over $I$, dan geldt $\forall x\in I:\ f'(x) \ge 0$.
  \end{itemize}

  \begin{proof}
    Bewijs in delen.
    \begin{itemize}
    \item Zij $f$ een functie die stijgt over $I$.
      Kies nu een element $a$ van $I$.
      Omdat $I$ open is bestaat er een $\delta\in \mathbb{R}_{0}^{+}$ zodat voor alle $h<\delta$ $a+h$ tot $I$ behoort.
      Voor alle $h\in \mathbb{R}_{0}^{+}$ kleiner dan $\delta$ geldt nu het volgende:
      \[ \frac{f(a+h)-f(a)}{h} \ge 0 \]
      $a+h$ is immers groter dan, of gelijk aan, $a$ en $f$ is stijgend over $I$.
      De stelling volgt nu uit het feit dat limieten ongelijkheden behouden.\needed
    \item 
      \extra{bewijs}
    \end{itemize}
  \end{proof}
\end{pr}

\extra{het omgekeerde geldt ook} 

\subsection{Klassiekers}
\label{sec:twee-klassiekers}

\subsubsection{Rolle}
\label{sec:rolle}

\begin{st}
  \label{st:rolle}
  De \term{middelwaardestelling van Rolle}\\
  Zij $f:\ \interval{a}{b} \rightarrow \mathbb{R}$ een continue functe op een begrensd gesloten interval $\interval{a}{b}$.
  Veronderstel dat $f$ afleidbaar is in $\interval[open]{a}{b}$ en dat $f(a)$ gelijk is aan $f(b)$.
  Er bestaat dan een $c\in \interval[open]{a}{b}$ zodat $f'(c)=0$ geldt.
  T.t.z. er bestaat dan een kritiek punt $c\in \interval[open]{a}{b}$.

  \begin{proof}
    Omdat $f$ een continue functie is, gedefinieerd op een gesloten begrensd interval, is $f$ begrensd.\needed
    De supremumeigenschap van $\mathbb{R}$ geeft ons dan dat er een supremum $M$ en een infimum $m$.
    \[ M = \sup\{f(x) \mid x \in \interval{a}{b} \} \]
    \[ m = \inf\{f(x) \mid x \in \interval{a}{b} \} \]
    Als $m$ gelijk is aan $M$, dan geldt $f(x)=f(a)$ voor alle $x \in \interval{a}{b}$ en is $f$ dus constant.
    In elke $c \in \interval{a}{b}$ is de afgeleide van $f$ dan nul.
    Beschouw nu het geval dat $m< M$ geldt.
    Er treedt dan minstens \'e\'en van de volgende gevallen op:
    \begin{itemize}
    \item $f(a) < M$\\
      Er bestaat dan een $c$ zodat $f(c)$ gelijk is aan $M$.\needed
      Omdat $f(a)$ gelijk is aan $f(b)$ en kleiner dan $M=f(c)$, kan $c$ niet gelijk zijn aan $a$ of aan $b$ en moet $c$ dus in het interval $\interval[open]{a}{b}$ liggen.
      $c$ is dan een kritiek punt, en dus moet de afgeleide van $f$ in $c$ nul zijn.\needed
    \item $f(a) > m$\\
      \extra{bewijs}
    \end{itemize}
  \end{proof}
\end{st}

\subsubsection{Lagrange}
\label{sec:middelwaardestelling-lagrange}

\begin{st}
  \label{st:middelwaardestelling-lagrange}
  De \term{middelwaardestelling van Lagrange}\\
  Zij $f:\ \interval{a}{b} \rightarrow \mathbb{R}$ een continue functe op een begrensd gesloten interval $\interval{a}{b}$.
  Veronderstel dat $f$ afleidbaar is in $\interval[open]{a}{b}$, dan bestaat er een $c\in \interval[open]{a}{b}$ als volgt.
  \[ f'(c) = \frac{f(b)-f(a)}{b-a} \]

  \begin{proof}
   Beschouw de hulpfunctie $g$:
   \[ g: \interval{a}{b} \rightarrow \mathbb{R}:\ x \mapsto g(x) = f(x)-\frac{f(b)-f(a)}{b-a}(x-a) \]
   \begin{itemize}
   \item Er geldt $g(a) = g(b)$:
     \[ g(a) = f(a)-\frac{f(b)-f(a)}{b-a}(a-a) = f(a) \]
     \[ g(b) = f(b)-\frac{f(b)-f(a)}{b-a}(b-a) = f(a) \]
     Er bestaat dus een $c\in \interval[open]{a}{b}$ zodat $g'(c)$ nul is.\stref{st:rolle}
   \item $g'$ ziet er bovendien als volgt uit:
     \[
     \begin{array}{rl}
       \lim_{x \rightarrow x}\frac{g(x)-g(a)}{x-a}
       &= \lim_{x \rightarrow a}\frac{f(x)-\frac{f(b)-f(a)}{b-a}(x-a) - f(a)}{x-a}\\
       &= \lim_{x \rightarrow a} \frac{f(x)-f(a)}{x-a} - \frac{f(b)-f(a)}{b-a}\\
       &= f'(x)-\frac{f(b)-f(a)}{b-a}\\
     \end{array}
     \]
   \item In $c$ volgt uit $g'(c)$ de stelling:
     \[
     f'(c)-\frac{f(b)-f(a)}{b-a} = 0 
     \]
     \[
     f'(c) = \frac{f(b)-f(a)}{b-a} 
     \]
   \end{itemize}
  \end{proof}
\feed
\end{st}


\subsubsection{Cauchy}
\label{sec:cauchy}

\begin{st}
  \label{st:middelwaardestelling-cauchy}
  De \term{middelwaardestelling van Cauchy}\\
  Zij $f,g:\ \interval{a}{b} \rightarrow \mathbb{R}$ continue functes op een begrensd gesloten interval $\interval{a}{b}$.
  Veronderstel dat $f$ en $g$ afleidbaar zijn in $\interval[open]{a}{b}$, dan bestaat er een $c\in \interval[open]{a}{b}$ als volgt.
  \[ \left( f(b) - f(a) \right) g'(c) = f'(c) \left( g(b) - g(a) \right) \]

  \begin{proof}
    Beschouw de hulpfunctie $h$:
    \[ h:\ \interval{a}{b} \rightarrow \mathbb{R}:\ x \mapsto h(x) = (f(b)-f(a))g(x) -f(x)(g(b)-g(a)) \]
    \begin{itemize}
    \item Er geldt $h(a)=h(b)$:
      \[
      \begin{array}{rl}
        h(a) &= (f(b)-f(a))g(a) -f(a)(g(b)-g(a))\\
             &= f(b)g(a)-f(a)g(a)-f(a)g(b)+f(a)g(a)\\
             &= f(b)g(a)-f(a)g(b)
      \end{array}
      \]
      \[
      \begin{array}{rl}
        h(b) &= (f(b)-f(a))g(b) -f(b)(g(b)-g(a))\\
             &= f(b)g(b)-f(a)g(b) -f(b)g(b)+f(b)g(a)\\
             &= f(b)g(a)-f(a)g(b)
      \end{array}
      \]
     Er bestaat dus een $c\in \interval[open]{a}{b}$ zodat $h'(c)$ nul is.\stref{st:rolle}
    \item $h'$ ziet er bovendien als volgt uit:
      \[
      h'(x) = (f(b)-f(a))g'(x)-f'(x)(g(b)-g(a))
      \]
      \question{hoe moet ik dit verder uitwerken?}
   \item In $c$ volgt uit $h'(c)$ de stelling:
     \[
     (f(b)-f(a))g'(c)-f'(c)(g(b)-g(a)) = 0
     \]
     \[
     (f(b)-f(a))g'(c) = f'(c)(g(b)-g(a))
     \]
    \end{itemize}
  \end{proof}
  \feed
\end{st}

\subsection{Stijgen, dalen of constant zijn}

\begin{pr}
  Zij $f:\ \interval{a}{b} \rightarrow \mathbb{R}$ een continue functe op een begrensd gesloten interval $\interval{a}{b}$ die afleidbaar is in $\interval[open]{a}{b}$.
  \begin{itemize}
  \item Als $f'(c) \ge 0$ geldt voor alle $c \in \interval[open]{a}{b}$, dan is $f$ stijgend in $\interval{a}{b}$
    \[ \forall x,y \in \interval{a}{b}:\ x \le y \Rightarrow f(x) \le f(y) \]
  \item Als $f'(c) \le 0$ geldt voor alle $c \in \interval[open]{a}{b}$, dan is $f$ dalend in $\interval{a}{b}$
    \[ \forall x,y \in \interval{a}{b}:\ x \le y \Rightarrow f(x) \ge f(y) \]
  \end{itemize}

  \begin{proof}
    Elk deel appart.
    \begin{itemize}
    \item Kies willekeurig $x$ en $y \in \interval{a}{b}$ met $x\le y$.
      We passen de middelwaardestelling van lagrange toe op de functie $f$, beperkt tot het interval $\interval{x}{y}$.\stref{st:middelwaardestelling-lagrange}
      Dit levert een $c\in \interval[open]{x}{y}$ als volgt:
      \[ f(y)-f(x) = f'(c)(y-x) \]
      Omdat $f'(c)$ positief is, alsook $(y-x)$, geldt $f(y)-f(x) \ge 0 \Leftrightarrow f(y) \ge f(x)$.
    \item \extra{bewijs}
    \end{itemize}
  \end{proof}
\end{pr}

\begin{pr}
  Zij $f:\ \interval{a}{b} \rightarrow \mathbb{R}$ een continue functe op een begrensd gesloten interval $\interval{a}{b}$ die afleidbaar is in $\interval[open]{a}{b}$.
  Als $\forall c\in \interval[open]{a}{b}: f'(c) = 0$ geldt, dan is $f$ constant.

  \begin{proof}
    Kies willekeurig een $x\in \interval[open left]{a}{b}$.\clarify{waarom halfopen?}
    We passsen de middelwaardestelling van Lagrange toe op de functie $f$, beperkt tot het interval $\interval{a}{x}$.\stref{st:middelwaardestelling-lagrange}
    Dit levert een $c\in \interval[open]{a}{x}$ als volgt:
    \[ f(x)-f(a) = f'(c)(x-a) \]
    Omdat $f'(c)$ nul is, moet $f(x)$ gelijk zijn aan $f(a)$.
  \end{proof}
\extra{rechtstreeks bewijzen vanuit de definitie van een afgeleide}
\end{pr}

\subsection{Limieten van rijen van afleidbare functies}
\label{sec:limieten-van-rijen}

\begin{st}
  Beschouw een rij $(f_{n})_{n}$ van afleidbare functies gedefinieerd op een interval $I \subseteq \mathbb{R}$ met waarden in $\mathbb{R}$.
  Stel dat $(f_{n})_{n}$ puntsgewijs convergeert naar een functie $f:\ I \rightarrow \mathbb{R}$ en dat $(f'_{n})_{n}$ uniform op $I$ convergeert, dan is $f$ afleidbaar:
  \[ \forall a\in I:\ f'(a) = \lim_{n\rightarrow +\infty}f_{n}'(a) \]

  \begin{proof}
    Kies een willekeurige $a\in $ en noteer de limiet van $f_{n}'(a)$ als $L$.
    We tonen het volgende aan, en dan volgt daaruit de stelling:
    \[ \lim_{x\rightarrow a}\frac{f(x)-f(a)}{x-a} = L \]
    \begin{itemize}
    \item Merk eerst op dat voor alle $x\in I$, verschillend van $a$, en alle $m\in \mathbb{N}$ het volgende geldt:
      \[
      \begin{array}{rl}
        \left| \frac{f(x)-f(a)}{x-a} - L \right| 
        &\le \left| \frac{f(x)-f(a)}{x-a} - \frac{f_{m}(x)-f_{m}(a)}{x-a} \right|\\
        &+ \left| \frac{f_{m}(x)-f_{m}(a)}{x-a} - f_{m}'(a) \right|\\
        &+ \left| f'_{m}(a) - L \right|
      \end{array}
      \]
    \item Kies nu een willekeurige $\epsilon \in \mathbb{R}_{0}^{+}$.
      \begin{itemize}
      \item 
        Omdat $(f_{n}')_{n}$ uniform convergeert kunnen we een $n_{0}\in \mathbb{N}$ vinden als volgt:
        \[ \forall m,n \ge n_{0},\forall y\in I:\ |f'_{n}(y)-f'_{m}(y)| < \frac{\epsilon}{3} \]
        Vanwege de middelwaardestelling van lagrange bestaat er voor elke $x\in I$, verschillend van $a$, en voor elke $n,m\in \mathbb{N}$ een $c$ tusen $a$ en $x$ als volgt:\stref{st:middelwaardestelling-lagrange}
        \[ \left| \frac{f_{n}(x)-f_{n}(a)}{x-a} - \frac{f_{m}(x)-f_{m}(a)}{x-a}\right| = \left| f'_{n}(c) -f'_{m}(c) \right| \]
        We vinden dus het volgende:
        \[ \forall m,n \ge n_{0},\forall y\in I:\ \left| \frac{f_{n}(x)-f_{n}(a)}{x-a} - \frac{f_{m}(x)-f_{m}(a)}{x-a}\right| < \frac{\epsilon}{3} \]
        Nemen we hiervan de limiet voor $n$ gaande naar $+\infty$, dan vinden we de volgende ongelijkheid.
        \[ \forall m \ge n_{0},\forall y\in I:\ \left| \frac{f(x)-f(a)}{x-a} - \frac{f_{m}(x)-f_{m}(a)}{x-a}\right| \le \frac{\epsilon}{3} \]
        \clarify{waarom plots $\le$?}
      \item 
        Omdat $f'_{n}(a)$ naar $L$ convergeert, kunnen we een $m_{0} \in \mathbb{N}$ vinden zodat het volgende geldt:
        \[ \forall m\in \mathbb{N}: m \ge m_{0} \Rightarrow \left| f'_{m}(a) - L \right| < \frac{\epsilon}{3}\]
      \item 
        Omdat $f_{m}$ afleidbaar is in $a$, kunnnen we een $\delta \in \mathbb{R}_{0}^{+}$ vinden  als volgt:
        \[ \forall x\in I: 0 < |x-a| < \delta:\ \left| \frac{f_{m}(x)-f_{m}(a)}{x-a} - f_{m}'(a) \right| < \frac{\epsilon}{3} \]
      \end{itemize}
    \item Uit het eerste puntje volgt nu de stelling:
      \[ \forall x\in I:\ 0 < |x-a| < \delta:\ \left| \frac{f(x)-f(a)}{x-a} - L \right|  < \frac{\epsilon}{3}+\frac{\epsilon}{3} +\frac{\epsilon}{3} = \epsilon \]
    \end{itemize}
  \end{proof}
\end{st}




\end{document}
