\documentclass[main.tex]{subfiles}
\begin{document}

\chapter{Algoritmes}
\label{cha:algoritmes}

\section{Continu\"iteit in $\mathbb{R}$}
\label{sec:continuiteit-mathbbr}

\subsection{Bewijzen dat een functie continu is aan de hand van de definitie}
\label{sec:bewijzen-dat-een}

\subsubsection{Abstract}

Zij $f$ een functie:
\[ f:\ A \subseteq \mathbb{R} \rightarrow \mathbb{R}:\ x \mapsto f(x) \]
Bewijs dat $f$ continu is over $A$.

\begin{itemize}
\item Kies een willekeurige $a \in A$ om de continu\"iteit over heel $A$ te bewijzen.\needed
\item Maak een afschatting voor $|f(x)-f(a)|$ voor een willekeurige $x \in A$ op basis van het voorschrift van $f$.
\item Zorg dat in de afschatting ergens $|x-a|$ voorkomt.
\item Optioneel: Maak gepaste afschattingen (naar onder) voor de overblijvende voorkomens van $x$ buiten de uitdrukking $|x-a|$, in het achterhoofd houdende dat $|x-a|$ willekeurig klein mag gekozen worden.
\item Kies een willekeurige $\epsilon$ (uit de definitie van continu\"iteit).
\item Kies dan $\delta$ zodat door $|x-a|$ te vervangen in de afschatting van $|f(x)-f(a)|$ er enkel nog $\epsilon$ overblijft. Zorg dat in de uitdrukking voor $\delta$ geen $x$ meer voorkomt.
$\delta$ is dus een uitdrukking in $a$ en $\epsilon$.
\item Besluit dat voor de gekozen $\delta$ uit $|x-a|$ volgt dat $|f(x)-f(a)|$ kleiner is dan $\epsilon$.
\end{itemize}
\feed
\subsubsection{Voorbeeld}

Beschouw de functie $f_{b}$:
\[ f_{b}:\ \mathbb{R}\setminus \{ b \} \rightarrow \mathbb{R}:\ x \mapsto \frac{1}{x-b} \]
Bewijs dat $f_{b}$ continu is.

\begin{proof}
  We overlopen simpelweg de beschreven stappen.
  \begin{itemize}
  \item Kies een wilekeurige $a \in \mathbb{R}\setminus \{ b \}$.
  \item Beschouw $|f(x)-f(a)|$ voor een willekeurige $x \in \mathbb{R}\setminus \{ b \}$:
    \[ |f(x)-f(a)| = \left|\frac{1}{x-b} - \frac{1}{a-b}\right| = \frac{|(a-b) - (x-b)|}{|x-b||a-b|} = \frac{|a-x|}{|x-b||a-b|} \]
  \item Voor $|x-a|$ kleiner dan $|a-b|$ geldt nu het volgende:
    \[ |x-b| = |x-a+a-b| \le |x-a|+|a-b| \le 2|a-b| \]
  \item Kies een willekeurige $\epsilon \in \mathbb{R}_{0}^{+}$.
  \item Kies voor $\delta$ nu het minimum van $|a-b|$ en $3|a-b|\epsilon$.
  \item Uit $|x-a| < \delta$ geldt nu het volgende:
    \[ |f(x)-f(a)| = \frac{|a-x|}{|x-b||a-b|} \le \frac{|a-x|}{3|a-b|} < \frac{\delta}{3|a-b|} \le \frac{3|a-b|\epsilon}{3|a-b|} = \epsilon \]
  \end{itemize}
\end{proof}
\feed

\subsection{Bewijzen dat een functie continu is aan de hand van rijen}
\TODO{uitschrijven}
\subsubsection{Abstract}
\subsubsection{Voorbeeld}

\subsection{Bewijzen dat een functie continu is aan de hand van open delen}
\TODO{uitschrijven, heel belangrijk!!}
\subsubsection{Abstract}
\subsubsection{Voorbeeld}

% Beschouw de functie $f$:
% \[ f:\ \mathbb{R}^{+} \rightarrow \mathbb{R}:\ x \mapsto \sqrt{x} \]
% Bewijs dat $f$ continu is aan de hand van open delen van $\mathbb{R}$

% \begin{proof}
%   Merk eerst op dat $a \in f^{-1}(V)$ het volgende inhoudt:
%   \[ a \in f^{-1}(V) \Leftrightarrow \exists v \in V: \sqrt{v}=a \]
%   We moeten het volgende bewijzen:
%   \[ \forall a \in f^{-1}(V):\ \exists \epsilon \in \mathbb{R}_{0}^{+}:\ \forall b \in \mathbb{R}^{+}:\ |a-b| < \epsilon \Rightarrow b \in f^{-1}(V) \Leftrightarrow \exists w \in \mathbb{R}: \sqrt{w} = b \]
%   Kies nu een willekeurig open deel $V$ van $\mathbb{R}$:
%   \[ \forall v\in V:\ \exists \delta \in \mathbb{R}_{0}^{+}:\ \forall w\in V:\ |v-w| < \delta \Rightarrow w \in V \]
%   Als $f^{-1}(V)$ leeg is, is $f^{-1}(V)$ trivialerwijze open.
%   Als $f^{-1}(V)$ niet leeg is, is $V$ een deel van $\mathbb{R}^{+}$ (want $\sqrt{\mathbb{R}^{+}} = \mathbb{R}^{+}$).
%   Kies een willekeurige $a\in f^{-1}(V)$.
%   Kies willekeurig $v$ en $w$ uit $V$ en beschouw de $\delta$, gegeven door de definitie van openheid hierboven.
%   Vanuit de $\delta$ voor $v$ construeren we de gevraagde $\epsilon$ voor $a$.
%   \[ |v-w| < \delta \Rightarrow \sqrt{|v-w|} < \sqrt{\delta} \Rightarrow |\sqrt{v}-\sqrt{w}| < \sqrt{\delta} \]
%   Kies nu $\epsilon = \sqrt{\delta}$ en een willekeurige $b\in \mathbb{R}^{+}$.
%   \[ |a-b| = |\sqrt{v}-b| < \sqrt{\delta} \]
% \end{proof}


\subsection{Bewijzen dat een functie niet continu is}
\subsubsection{Abstract}
\subsubsection{Voorbeeld}


\end{document}

%%% Local Variables:
%%% mode: latex
%%% TeX-master: t
%%% End:
