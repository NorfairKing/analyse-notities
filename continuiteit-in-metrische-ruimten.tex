\documentclass[main.tex]{subfiles}
\begin{document}

\section{Continu\"iteit}
\label{sec:continuiteit}

\begin{bpr}
  Zij $X,d_{X}$ en $Y,d_{Y}$ metrische ruimten.
  Zij $f:\ A \subseteq X \rightarrow Y$ een functie en een $a\in A$, dan is $f$ continu in $a$ als en slechts als voor elke rij $(x_{n})_{n}$ in $A$ die naar $a$ convergeert $(f(x_{n}))_{n}$ naar $f(a)$ convergeert.
\TODO{bewijs: oefening}
\end{bpr}

\begin{bpr}
  Zij $X,d_{X}$ en $Y,d_{Y}$ metrische ruimten.
  Zij $f:\ A \subseteq X \rightarrow Y$ een functie, dan is $f$ continu op $A$ als en slechts voor elke $V \subseteq Y$ $f^{-1}(V)$ relatief open is in $A$.
\TODO{bewijs: oefening}
\end{bpr}

\subsection{Eigenschappen van continue functies}
\label{sec:eigensch-van-cont}

\begin{bpr}
  Beschouw metrische ruimten $X,d_{X}$, $Y,d_{Y}$ en $Z,d_{Z}$ en functies $f:\ X \rightarrow Y$ en $g:\ Y \rightarrow Z$.
  Stel dat $f$ continu is in een punt $a\in X$ en $g$ in $f(a)$, dan is $g\circ f$ continu in $a$.
\TODO{bewijs: oefening}
\end{bpr}

\begin{bst}
  Zij $X,d_{X}$ en $Y,d_{Y}$ metrische ruimten. en $f:\ X \rightarrow Y$ een continue functie.
  Als $K \subseteq X$ compact is, is $f(K)$ ook compact.
\TODO{bewijs p 68}
\end{bst}

\begin{bst}
  Zij $X,d_{X}$ en $Y,d_{Y}$ metrische ruimten. en $f:\ X \rightarrow Y$ een continue functie.
  Als $K \subseteq X$ compact is, is $f$ uniform continu op $K$.
\TODO{bewijs p 69}
\end{bst}

\begin{bst}
  Zij $X,d_{X}$ en $Y,d_{Y}$ metrische ruimten. en $f:\ X \rightarrow Y$ een continue injectieve functie.
  Als $X$ compact is, dan is $f^{-1}:\ f(X) \rightarrow X$ continu.
\TODO{bewijs p 69}
\end{bst}

\begin{bst}
  Zij $X,d_{X}$ en $Y,d_{Y}$ metrische ruimten. en $f:\ X \rightarrow Y$ een continue functie.
  Als $S \subseteq X$ samenhangend is, is $f(S)$ ook samenhangend.
\TODO{bewijs p 70}
\end{bst}

\subsection{Convergentie van rijen van continue functies}
\label{sec:conv-van-rijen}

\begin{de}
  Beschouw een rij $(f_{n})_{n}$ van functies op een verzameling $X$ met waarden in een metrische ruimte $Y,d_{Y}$, dan zeggen we dat $(f_{n})_{n}$ \term{puntsgewijs convergeert} op $X$ naar een functie $f:\ X \rightarrow Y$ als en slechts als voor elke $x\in X$ de rij $(f_{n}(x))_{n}$ naar $f(x)$ convergeert:
  \[ \forall x\in X, \forall \epsilon \in \mathbb{R}_{0}^{+}, \exists n_{0} \in \mathbb{N}, \forall n\in \mathbb{N}:\ n \ge n_{0}:\ \Rightarrow d_{Y}(f_{n}(x),f(x)) < \epsilon \]
\end{de}

\begin{de}
  Beschouw een rij $(f_{n})_{n}$ van functies op een verzameling $X$ met waarden in een metrische ruimte $Y,d_{Y}$, dan zeggen we dat $(f_{n})_{n}$ \term{puntsgewijs convergeert} op $X$ naar een functie $f:\ X \rightarrow Y$ als en slechts als het volgende geldt:
  \[ \forall \epsilon \in \mathbb{R}_{0}^{+}, \exists n_{0}\in \mathbb{N}, \forall x\in X, \forall n\in \mathbb{N}:\ n \ge n_{0} \Rightarrow d_{y}(f_{n}(x),f_{n}(y)) < \epsilon \]
\end{de}

\begin{bst}
  Zij $X,d_{X}$ en $Y,d_{y}$ metrische ruimten.
  Beschouw een rij $(f_{n})_{n}$ van functies $f_{n}:\ X \rightarrow Y$.
  Veronderstel dat $(f_{n})_{n}$ uniform convergeert op $X$ naar een functie $f:\ X \rightarrow Y$, dan gelden volgende uitspraken:
  \begin{enumerate}
  \item Als alle $f_{n}$ continu zijn in eenzelfde $a\in X$, dan is ook $f$ continu in $a$.
  \item Als alle $f_{n}$ uniform continu zijn op $X$, dan is ook $f$ uniform continu op $X$.
  \end{enumerate}
\end{bst}

\begin{bst}
  Zij $X,d$ een compacte metrische ruimte en $(f_{n})_{n}$ een rij van functies van $X$ naar $\mathbb{R}$.
  Als aan de volgende voorwaarden voldaan is zal $f_{n}$ uniform convergeren naar $f$.
  \begin{enumerate}
  \item Alle functies $f_{n}$ zijn continu.
  \item De rij $(f_{n})_{n}$ convergeert puntsgewijs naar een continue functie $X \rightarrow \mathbb{R}$.
  \item Voor elke $x\in X$ is de rij $(f_{n}(x))_{n}$ stijgend.
  \end{enumerate}
\TODO{bewijs p 72}
\end{bst}




\end{document}

%%% Local Variables:
%%% mode: latex
%%% TeX-master: t
%%% End:
