\documentclass[main.tex]{subfiles}
\begin{document}



\section{Complexe getallen}
\label{sec:complexe-getallen}

\begin{de}
  De verzameling $\mathbb{C}$ van \term{complexe getallen} defini\"eren we als volgt, samen met de optelling $(+)$ en vermenigvuldiging $(\cdot)$.
  \[ \mathbb{C} = \mathbb{R}^{2} = \{ (a,b) \mid a,b\in \mathbb{R} \} \]
  \[ (+):\ \mathbb{C}^{2} \rightarrow \mathbb{C}:\ (a,b) + (c,d) = (a+c,b+d) \]
  \[ (\cdot):\ \mathbb{C}^{2} \rightarrow \mathbb{C}:\ (a,b) \cdot (c,d) = (ac-bd, ad+bc) \]
\end{de}

\begin{st}
  $\mathbb{C},+,\cdot$ is een veld.
  \extra{bewijs p 81}
\end{st}

\begin{pr}
  We kunnen $\mathbb{R}$ inbedden in $\mathbb{C}$ met de volgende injectieve afbeelding:
  \[ \phi:\ \mathbb{R} \rightarrow \mathbb{C}:\ a \mapsto (a,0) \]
  Deze afbeelding is bovendien een ringmorfisme:
  \[ \forall a,b \in \mathbb{R}:\ \phi(a+b) = \phi(a) + \phi(b) \]
  \[ \forall a,b \in \mathbb{R}:\ \phi(ab) = \phi(a)\phi(b) \] 
  \extra{bewijs: oefening}
\end{pr}

\begin{opm}
  Onder deze injectie beschouwen we $\mathbb{R}$ als een deelveld van $\mathbb{C}$.
\end{opm}

\begin{de}
  We noteren een element $(a,b)$ van $\mathbb{C}$ vaak als $a+bi$.
  We noemen dit de \term{carthesiaanse vorm van een complex getal}.
\end{de}

\begin{opm}
  Deze notatie komt van pas om rekenregels binnen $\mathbb{C}$ eenvoudig te houden als we $i^{2}=1$ als regel in het achterhoofd houden. 
  \extra{bewijs!}
\end{opm}

\begin{de}
  We noemen in een element $a+bi$ van $\mathbb{C}$ $a$ het \term{re\"eel} deel en $b$ het \term{imaginair} deel.
  We voeren daarom twee afbeeldingen in:
  \[ Re:\ \mathbb{C} \rightarrow \mathbb{R}:\ (a+bi) \mapsto a \]
  \[ Im:\ \mathbb{C} \rightarrow \mathbb{R}:\ (a+bi) \mapsto b \]
\end{de}

\begin{st}
  De \term{hoofdstelling van de algebra}\\
  $\mathbb{C}$ is algebra\"isch gesloten: Een $n$-de graadsveelterm over $\mathbb{C}$ heeft precies $n$ wortels in $\mathbb{C}$.
  \zb
\end{st}

\begin{pr}
  Er bestaat geen orde op $\mathbb{C}$ die van $\mathbb{C},+,\cdot$ een totaal geordend veld maakt.

  \begin{proof}
    Bewijs uit het ongerijmde: stel dat er wel zo'n orde $\le$ bestaat.
    Dan moet $i$ ofwel groter dan $0$ zijn, ofwel kleiner dan nul.
    In beide gevallen volgt $-1 > 0$ of $1 < 0$,\prref{pr:geordend-veld-ongelijkheid-vermenigvuldiging} wat niet mogelijk is in een geordende veld.\prref{pr:nul-kleiner-dan-een}
  \end{proof}
\end{pr}

\begin{de}
  Het \term{complex toegevoegde} $\overline{a+bi}$ van een complex getal $a+bi$ definieren we als volgt:
  \[ \overline{a+bi} = a-bi \]
  \[ \overline{\, \cdot\ }:\ \mathbb{C} \rightarrow \mathbb{C}:\ a+bi \mapsto \overline{a+bi} = a-bi \]
\end{de}

\begin{de}
  De \term{modulus} $|a+bi|$ van een complex getal $a+bi$ definieren we als volgt:
  \[ |a+bi| = \sqrt{a^{2}+b^{2}} \]
  \[ |\cdot|:\ \mathbb{C} \rightarrow \mathbb{R}:\ a+bi \mapsto |a+bi| = \sqrt{a^{2}+b^{2}} \]
\end{de}

\begin{ei}
  De modulus van een complex getal is de tegenhanger van de absolute waarde van een re\"eel getal.
  \[ \forall a \in \mathbb{R}:\ |a| = |\phi(a)| \]
  \begin{proof}
    $\forall a\in \mathbb{R}:\ |\phi(a)| = |a+0i| = \sqrt{a^{2} + 0^{2}} = |a|$
  \end{proof}
\end{ei}

\begin{pr}
  \[ \forall z\in \mathbb{C}:\ \bar{\bar{z}} = z \]

  \begin{proof}
    $\forall a+bi\in \mathbb{C}:\ \bar{\bar{a+bi}} = \bar{a-bi} = a-(-bi) = a+bi$
  \end{proof}
\end{pr}

\begin{pr}
  \label{pr:modulus-door-optelling}
  \[ \forall z_{1},z_{2}\in \mathbb{C}:\ \overline{z_{1}+z_{2}} = \overline{z_{1}} + \overline{z_{2}} \]

  \begin{proof}
    $\forall a+bi,c+di \in \mathbb{C}:\ \overline{a+bi + c+di} = \overline{(a+c)+(b+d)i} =(a+c)-(b+d)i = a+c-bi-di = (a-bi) + (c-di)$
  \end{proof}
\end{pr}

\begin{pr}
  \[ \forall z_{1},z_{2}\in \mathbb{C}:\ \overline{z_{1}z_{2}} = \bar{z}_{1}  \bar{z}_{2} \]

  \begin{proof}
    $\forall a+bi,c+di \in \mathbb{C}:\ \overline{a+bi \cdot c+di} = \overline{ac-bd + (ad+bc)i} = ac-bd - adi - bci = ac-(-b)(-d) + (a(-d)+(-b)c)i = \overline{a+bi} \cdot \overline{c+di} $
  \end{proof}
\end{pr}

\begin{pr}
  \[ \forall z\in \mathbb{C}:\ Re(z) = \frac{z+\bar{z}}{2} \]

  \begin{proof}
    $\forall a+bi\in \mathbb{C}:\ \frac{(a+bi)+\overline{a+bi}}{2} = \frac{(a+bi)+(a-bi)}{2} = \frac{2a}{2} = a = Re(a+bi)$
  \end{proof}
\end{pr}

\begin{pr}
  \[ \forall z\in \mathbb{C}:\ Im(z) = \frac{z-\bar{z}}{2i} \]

  \begin{proof}
    $\forall a+bi\in \mathbb{C}:\ \frac{(a+bi)-\overline{a+bi}}{2i} = \frac{(a+bi)-(a-bi)}{2i} = \frac{(a+bi-a+bi)}{2i} = \frac{2bi}{2i} = b = Im(a+bi) $
  \end{proof}
\end{pr}

\begin{pr}
  \[ \forall z\in \mathbb{C}:\ |\bar{z}| = |z| \]

  \begin{proof}
    $\forall a+bi\in \mathbb{C}:\ |\bar{a+bi}| = |a-bi| = \sqrt{a^{2}+(-b)^{2}} = \sqrt{a^{2}+b^{2}} = |a+bi|$
  \end{proof}
\end{pr}

\begin{pr}
  \label{pr:reel-deel-kleiner}
  \[ \forall z\in \mathbb{C}:\ |Re(z)| \le |z| \]

  \begin{proof}
    $\forall a+bi\in \mathbb{C}:\  |Re(a+bi)| = |a| \le \sqrt{a^{2}+b^{2}} = |a+bi|$ 
  \end{proof}
\end{pr}

\begin{pr}
  \label{pr:imaginair-deel-kleiner}
  \[ \forall z\in \mathbb{C}:\ |Im(z)| \le |z|\]

  \begin{proof}
    $\forall a+bi\in \mathbb{C}:\  |Im(a+bi)| = |b| \le \sqrt{a^{2}+b^{2}} = |a+bi|$ 
  \end{proof}
\end{pr}

\begin{pr}
  \label{pr:normaal-maal-toegevoegde-in-r}
  \[ \forall z\in \mathbb{C}:\ \bar{z}z = Im(z)^{2}+Re(z)^{2} \in \mathbb{R} \]

  \begin{proof}
    $\forall a+bi\in \mathbb{C}:\  \overline{a+bi}\cdot(a+bi) = (a-bi)(a+bi) = a^{2} + (-bi)a + a(bi) + (-bi)(bi) = a^{2} + b^{2} \in \mathbb{R}$
  \end{proof}
\end{pr}

\begin{pr}
  \label{pr:modulus-in-termen-van-toegevoegde}
  \[ \forall z\in \mathbb{C}:\ |z| = \sqrt{\bar{z}z} \]

  \begin{proof}
    $\forall a+bi\in \mathbb{C}:\ \sqrt{\overline{a+bi}\cdot(a+bi)} = \sqrt{a^{2}+b^{2}} = |a+bi|$
  \end{proof}
\end{pr}

\begin{pr}
  \[ \forall z\in \mathbb{C}:\ \forall z \in \mathbb{C}_{0}: \frac{1}{z} = \frac{\bar{z}}{|z|^{2}} \]

  \begin{proof}
    $\forall a+bi\in \mathbb{C}:\ \frac{\overline{a+bi}}{|a+bi|^{2}} = \frac{a-bi}{\left(\sqrt{a^{2}+b^{2}}\right)^{2}} = \frac{a-bi}{a^{2}+b^{2}}= \frac{(a-bi)(a+bi)}{(a^{2}+b^{2})(a+bi)} = \frac{a^{2}+b^{2}}{(a^{2}+b^{2})(a+bi)} = \frac{1}{a+bi}$
  \end{proof}
\end{pr}

\begin{pr}
  \[ \forall z_{1},z_{2}\in \mathbb{C}:\ |z_{1}z_{2}| = |z_{1}||z_{2}| \]

  \begin{proof}
    \[
    \forall a+bi,c+di \in \mathbb{C}:\
    \begin{array}{rll}
      |(a+bi)(c+di)| &= |ac-bd + (ad+bc)i|\\
      &= \sqrt{(ac-bd)^{2} + (ad+bc)^{2}}\\
      &= \sqrt{a^{2}c^{2} +b^{2}d^{2}-2abcd + a^{2}d^{2} + b^{2}c^{2} +2abcd}\\
      &= \sqrt{a^{2}c^{2} +b^{2}d^{2} + a^{2}d^{2} + b^{2}c^{2}}\\
      &= \sqrt{a^{2}+b^{2}}\sqrt{c^{2}+d^{2}}
    \end{array}
    \]
  \end{proof}
\end{pr}

\begin{pr}
  \[ \forall z_{1},z_{2}\in \mathbb{C}:\ |z_{1}+z_{2}| \le |z_{1}|+|z_{2}| \]

  \begin{proof}
    \[
    \forall a+bi,c+di \in \mathbb{C}:\
    \begin{array}{rll}
      |(a+bi)+(c+di)|^{2} &= |(a+c)+(b+d)i|^{2}\\
      &= (a+c)^{2}+ (b+d)^{2}\\
      &= a^{2}+c^{2} + b^{2}+d^{2}+2(ac + bd)\\
      &= a^{2}+c^{2} + b^{2}+d^{2}+2Re((a-bi)(c+di))\\
      &\le a^{2}+b^{2} + c^{2}+d^{2} + 2|(a-bi)(c+di)|\\
      &= a^{2}+b^{2} + c^{2}+d^{2} + 2\sqrt{a^{2}c^{2} + a^{2}d^{2} +b^{2}c^{2}+ b^{2}d^{2}}\\
      &= a^{2}+b^{2} + c^{2}+d^{2} + 2\sqrt{(a^{2}+b^{2})(c^{2}+d^{2})}\\
      &= |a+bi|^{2} + |c+di|^{2} + 2|a+bi||c+di|\\
      &= \left(|a+bi|+|c+di|\right)^{2}\\
    \end{array}
    \]
  \end{proof}
\end{pr}

\begin{pr}
  \label{pr:tweede-driehoeksongelijkheid-C}
  \[ \forall x,y,z\in \mathbb{C}:\ |x-z| \le |x-y| + |y-z| \]

  \begin{proof}
    Gebruik de driehoeksongelijkheid op $|(x-y)+(y-z)| \le |x-y| + |y-z|$
  \end{proof}
\end{pr}

\begin{pr}
  \[ \forall z_{1},z_{2}\in \mathbb{C}:\ ||z_{1}|-|z_{2}|| \le |z_{1}-z_{2}| \]

  \begin{proof}
    \[ |x| + |y-x| \ge |x+y-x| = |y| \]
    \[ |y| + |x-y| \ge |y+x-y| = |x| \]
    Verplaats in beide ongelijkheden de linker term naar de rechterkant:
    \[ |y-x| \ge |y| - |x| \]
    \[ |x-y| \ge |x| - |y| \]
    $|y-x|$ is gelijk aan $|x-y|$. Hieruit, samen met $t \ge a \wedge t \ge -a \Rightarrow t \ge |a|$ volgt de stelling:
    \[ |x-y| \ge \left||x|-|y|\right| \]
    \extra{gebruikte stelling afsplitsen}
  \end{proof}
\end{pr}

\begin{de}
  Zij $z = a+bi$ de carthesiaanse co\"ordinaten van een complex getal, dan definieren we de \term{poolcoordinaten} van dat getal als $(r,\theta)$:
  \[ a = r \cos \theta \quad\text{ en }\quad b = r\sin \theta \]
  \[ z = r(\cos \theta + i \sin \theta)  \]
\end{de}

\begin{st}
  \label{st:vermenigvuldiging-poolcoordinaten}
  \[
  \forall z_{1},z_{2} \in \mathbb{C}:\ 
  r_{1}(\cos \theta_{1} + i \sin \theta_{1}) \cdot r_{2}(\cos \theta_{2} + i \sin \theta_{2})
  = r_{1}r_{2}\left(\cos(\theta_{1}+\theta_{2}) + i\sin(\theta_{1}+\theta_{2})\right)
  \]
  
  \[ 
  \begin{array}{l}
    \forall z_{1},z_{2} \in \mathbb{C}:\\
    r_{1}(\cos \theta_{1} + i \sin \theta_{1}) \cdot r_{2}(\cos \theta_{2} + i \sin \theta_{2})\\
    = r_{1}r_{2} (\cos \theta_{1} + i \sin \theta_{1})(\cos \theta_{2} + i \sin \theta_{2})\\
    = r_{1}r_{2} \left( \left(\cos \theta_{1}\cos \theta_{2} -\sin \theta_{1}\sin \theta_{2}\right)+ i\left( \sin \theta_{1}\cos \theta_{2} + \cos \theta_{1}\sin \theta_{2}\right)  \right)\\
    = r_{1}r_{2}\left(\cos(\theta_{1}+\theta_{2}) + i\sin(\theta_{1}+\theta_{2})\right)\\
  \end{array}
  \]
\end{st}

\begin{st}
  De \term{formule van de Moivre}\\
  \[ \forall z = r(\cos \theta + i \sin \theta) \in \mathbb{C}, n\in \mathbb{Z}:\ (r(\cos \theta + i \sin \theta))^{n} = r^{n}(\cos n\theta + i \sin n\theta)\]

  \begin{proof}
    Gevalsonderscheid
    \begin{itemize}
    \item Voor alle $n\in \mathbb{N}$ volgt dit meteen uit stelling
      \ref{st:vermenigvuldiging-poolcoordinaten}.
    \item Voor $n=-1$:
      \[
      \begin{array}{rl}
        (r(\cos \theta + i \sin \theta))^{-1}
        &= \frac{1}{r(\cos \theta + i \sin \theta)}\\
        &= \frac{\cos \theta - i \sin \theta}{r(\cos \theta + i \sin \theta)(\cos \theta - i \sin \theta)}\\
        &= \frac{\cos(\theta) - i\sin(\theta)}{r}\\
        &= r^{-1}\cos(-\theta) + i\sin(-\theta)\\
      \end{array}
      \]
    \item Voor $-n\in \mathbb{Z}^{-}$ komt een $-n$-de macht met een $n$-de macht en een $-1$-e macht na elkaar.
    \end{itemize}
  \end{proof}
\end{st}

\begin{de}
  We noemen $z\in \mathbb{C}$ een $n$-de \term{eenheidswortel} als het volgende geldt:
  \[ z^{n} = 1 \]
  Merk op dat er voor graad $n$ precies $n$ eenheidswortel zijn:
  \[ z = \cos\left(\frac{2k\pi}{n}\right) + i \sin \left( \frac{2k\pi}{n} \right) \]
  \extra{afsplitsen in stelling?}
\end{de}


\end{document}

%%% Local Variables:
%%% mode: latex
%%% TeX-master: t
%%% End:
