% Voor subfiles
\usepackage{subfiles}

% Voor todo's
\usepackage{todonotes}

% Voor wiskunde
\usepackage{amsmath}
\usepackage{amsfonts}
\usepackage{amssymb}
\usepackage{amsthm}

% Een mooier bestand met wiskunde ondersteuning
\usepackage{libertine}
\usepackage[libertine]{newtxmath}

% Om het totaal aantal pagina's te tellen
\usepackage{lastpage}
\usepackage{afterpage}

% Om figuren op de juiste plaats te krijgen
\usepackage{float}

% Voor frames
\usepackage{mdframed}

% Om de marges aan te passen
\usepackage[left=2cm,right=2cm,top=2cm,bottom=2cm,headheight=15pt]{geometry}

% Voor mooiere breuken
\usepackage{nicefrac}

% Voor kleuren, duh
\usepackage{color}

% Voor intervallen
\usepackage{interval}

% Voor mooiere enumerates
\usepackage{enumerate}

% Voor mooie verbatim stukjes
\usepackage{verbatim}

% Voor tekeningen
\usepackage{tikz}
\usetikzlibrary{decorations.pathreplacing}
\usepackage{pgfplots}
\pgfplotsset{soldot/.style={only marks,mark=*}}                                         
\pgfplotsset{holdot/.style={fill=white,only marks,mark=*}}
\usepgfplotslibrary{fillbetween}
\usetikzlibrary{external}
\usetikzlibrary{decorations}
\usetikzlibrary{calc}
\usetikzlibrary{arrows.meta}
\usetikzlibrary{matrix}
\usetikzlibrary{fit}
\usetikzlibrary{shapes}
\usetikzlibrary{topaths}

%\iffalse
\tikzexternalize[prefix=figures/]
\makeatletter
\renewcommand{\todo}[2][]{
    \tikzexternaldisable
    \@todo[#1]{#2}
    \tikzexternalenable
}
\makeatother
%\fi

\usepackage{tikz-3dplot}

% Headers
\usepackage{fancyhdr}
\pagestyle{fancy}
\lhead{Analyse Notities}
\rhead{\thepage}
\cfoot{Tom Sydney Kerckhove}

\renewcommand{\headrulewidth}{0.4pt}
\renewcommand{\footrulewidth}{0.4pt}

% Om te doorstrepen in vergelijkingen
\usepackage{cancel}


% Voor commit in titelpagina
\immediate\write18{git rev-parse --short HEAD > commit.tex}


% Voor samengenomen cellen in tabel
\usepackage{multirow}%
% indices
\usepackage{makeidx}
\makeindex

% Voor urls (dit moet als laatste)
\usepackage{hyperref}
