% Voor subfiles
\usepackage{subfiles}

% Voor todo's
\usepackage{todonotes}

% Voor wiskunde
\usepackage{amsmath}
\usepackage{amsfonts}
\usepackage{amssymb}
\usepackage{amsthm}

% Voor urls
\usepackage{hyperref}

% Een mooier bestand met wiskunde ondersteuning
\usepackage{libertine}
\usepackage[libertine]{newtxmath}

% Om het totaal aantal pagina's te tellen
\usepackage{lastpage}
\usepackage{afterpage}

% Om figuren op de juiste plaats te krijgen
\usepackage{float}

% Voor frames
\usepackage{mdframed}

% Om de marges aan te passen
\usepackage[left=2cm,right=2cm,top=2cm,bottom=2cm]{geometry}

% Voor mooiere breuken
\usepackage{nicefrac}

% Voor intervallen
\usepackage{interval}

% Voor tekeningen
\usepackage{tikz}
\usetikzlibrary{decorations.pathreplacing}
\usepackage{pgfplots}
\pgfplotsset{soldot/.style={only marks,mark=*}}                                         
\pgfplotsset{holdot/.style={fill=white,only marks,mark=*}}
\usetikzlibrary{external}
\tikzexternalize[prefix=figures/]

\makeatletter
\renewcommand{\todo}[2][]{
    \tikzexternaldisable
    \@todo[#1]{#2}
    \tikzexternalenable
}
\makeatother

%indices
\usepackage{makeidx}
\makeindex
